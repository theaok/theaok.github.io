% latexdiff trustCityAug11.tex trustCity-aok-10-Sep2021.tex > trustCityAug11__trustCity-aok-10-Sep2021.tex
% to have line numbers
%\RequirePackage{lineno}
\documentclass[11pt, letterpaper]{article}      
\usepackage[margin=.1cm,font=small,labelfont=bf]{caption}[2007/03/09]
%\usepackage{endnotes}
%\let\footnote=\endnote
\usepackage{setspace}
\usepackage{longtable}                        
\usepackage{anysize}                          
\usepackage{natbib}                           
\bibpunct{(}{)}{;}{a}{,}{,}                   
%\bibpunct{(}{)}{,}{a}{}{,}                   
\usepackage{amsmath}
\usepackage[pdftex]{graphicx} %draft is a way to exclude figures                
\usepackage{epstopdf}
\usepackage[hidelinks]{hyperref}                             % For creating hyperlinks in cross references
\newcommand{\hilite}[1]{\textcolor{black}{#1}}	
\newcommand{\hiliteadam}[1]{\textcolor{blue}{#1}}	
% \usepackage[margins]{trackchanges}
% \note[editor]{The note}
% \annote[editor]{Text to annotate}{The note}
%    \add[editor]{Text to add}
% \remove[editor]{Text to remove}
% \change[editor]{Text to remove}{Text to add}

%TODO make it more standard before submission: \marginsize{2cm}{2cm}{1cm}{1cm}
\marginsize{.85in}{.85in}{.7in}{.7in}%{left}{right}{top}{bottom}   
					          % Helps LaTeX put figures where YOU want
 \renewcommand{\topfraction}{1}	                  % 90% of page top can be a float
 \renewcommand{\bottomfraction}{1}	          % 90% of page bottom can be a float
 \renewcommand{\textfraction}{0.0}	          % only 10% of page must to be text

 \usepackage{float}                               %latex will not complain to include float after float

\usepackage[table]{xcolor}                        %for table shading
\definecolor{gray90}{gray}{0.90}
\definecolor{orange}{RGB}{255,128,0}

\renewcommand\arraystretch{.9}                    %for spacing of arrays like tabular

%-------------------- my commands -----------------------------------------
\newenvironment{ig}[1]{
\begin{center}
 %\includegraphics[height=5.0in]{#1} 
 \includegraphics[height=3.3in]{#1} 
\end{center}}

 \newcommand{\cc}[1]{
\hspace{-.13in}$\bullet$\marginpar{\begin{spacing}{.6}\begin{footnotesize}\color{blue}{#1}\end{footnotesize}\end{spacing}}
\hspace{-.13in} }

%-------------------- END my commands -----------------------------------------



%-------------------- extra options -----------------------------------------

%%%%%%%%%%%%%
% footnotes %
%%%%%%%%%%%%%

%\long\def\symbolfootnote[#1]#2{\begingroup% %these can be used to make footnote  nonnumeric asterick, dagger etc
%\def\thefootnote{\fnsymbol{footnote}}\footnote[#1]{#2}\endgroup}	%see: http://help-csli.stanford.edu/tex/latex-footnotes.shtml

%%%%%%%%%%%
% spacing %
%%%%%%%%%%%

% \abovecaptionskip: space above caption
% \belowcaptionskip: space below caption
%\oddsidemargin 0cm
%\evensidemargin 0cm

%%%%%%%%%
% style %
%%%%%%%%%

%\pagestyle{myheadings}         % Option to put page headers
                               % Needed \documentclass[a4paper,twoside]{article}
%\markboth{{\small\it Politics and Life Satisfaction }}
%{{\small\it A} }

%\headsep 1.5cm
% \pagestyle{empty}			% no page numbers
% \parindent  15.mm			% indent paragraph by this much
% \parskip     2.mm			% space between paragraphs
% \mathindent 20.mm			% indent math equations by this much

%%%%%%%%%%%%%%%%%%
% extra packages %
%%%%%%%%%%%%%%%%%%

\usepackage{datetime}
\usepackage{longtable}
\usepackage{caption}


\usepackage[latin1]{inputenc}
\usepackage{tikz}
\usetikzlibrary{shapes,arrows,backgrounds}


%\usepackage{color}					% For creating coloured text and background
%\usepackage{float}
\usepackage{subfig}                                     % for combined figures

\renewcommand{\ss}[1]{{\colorbox{blue}{\bf \color{white}{#1}}}}
\newcommand{\ee}[1]{\endnote{\vspace{-.10in}\begin{spacing}{1.0}{\normalsize #1}\end{spacing}\vspace{.20in}}}
\newcommand{\emd}[1]{\ExecuteMetaData[/tmp/tex]{#1}} % grab numbers  from stata


% \usepackage[margins]{trackchanges}
% \usepackage{rotating}
% \usepackage{catchfilebetweentags}
%-------------------- END extra options -----------------------------------------
\date{{}\today \hspace{.2in}\xxivtime}

\title{  % remember to have Vistula University!!
\vspace{-.5in}  Misanthropolis: Do Cities Promote Misanthropy?
%
}
\author{
% ANONYMOUS
%   \hfill I thank XXX.  All mistakes are ours.} \\
%}
}



\begin{document}

%%\setpagewiselinenumbers
%\modulolinenumbers[1]
%\linenumbers

\bibliographystyle{/home/aok/papers/root/tex/ecta}
\maketitle
\vspace{-.4in}
\begin{center}

\end{center}


Highlights:
\begin{itemize}
\item Using the US General Social Survey (GSS, 1972-2016) we study the effect of urbanicity on misanthropy (distrust and dislike of humankind).
% \item The effect size of urbanicity is about half of that of
%  income.
\item  Places with a population larger than several hundred thousand people versus places with a population smaller than a few thousand (but not the countryside) are more misanthropic.
\item Misanthropy remained highest in the large cities until around 2005---in large cities ($>$250k) it declined over 2000-2010, and in small places ($<$10k) it increased steeply over 1990-2010.
\end{itemize}


\begin{abstract}
\noindent 
We use pooled US General Social Survey (GSS, 1972-2016) data to study the effect of urbanism on misanthropy (distrust and dislike of humankind). Evolution (small group living),  homophily or ingroup preference, and classic urban sociological theory  suggest that misanthropy should develop in the most dense and
heterogeneous places, such as large cities. Our results mostly agree: misanthropy is highest in cities with a population larger than several hundred thousand people, and the effect size of urbanicity is about half of that of
 income. Yet, the rural advantage is disappearing---from 1990 to  2010, misanthropy has increased fastest in the smallest places ($<10k$). One possible reason is that smaller places have been left behind, and rural resentment has increased. This is only the second quantitative study on the urbanicity-misanthropy nexus and more research is needed. Results may not be generalized outside of the US.

\end{abstract}

\noindent{\sc keywords:  city, urbanism, trust, misanthropy, distrust, fairness,
  helpfulness, misanthropolis, US General Social Survey (GSS)}%, social capital
%\vspace{-.25in} 

\begin{spacing}{1.4} %TODO MAYBE before submission can make it like 2.0
\rowcolors{1}{white}{gray90}
\vspace{.25in}
%  instead \ExecuteMetaData[../out/tex]{ginipov} do \emd{ginipov}


{\small\it \noindent ``Real misanthropes are not found in solitude, but in the world; since it is experience of life, and not philosophy, which produces real hatred of mankind.''} Giacomo Leopardi\\

{\small\it \noindent ``Whenever I tell people I'm a misanthrope they react as
  though that's a bad thing [...] I live in London, for God's sake. Have you
  walked down Oxford Street recently? Misanthropy's the only thing that gets you
  through it. It's not a personality flaw, it's a skill.''} Charlie Brooker\\

%{\small\it \noindent ``The more I learn about people, the more I like my dog.''} \footnote{Indeed, misanthropy can be a justified attitude towards humankind in light of how humans compare with certain animals \citep{cooper2018animals}.} {\small\it Mark Twain}\\[0.3em]

%{\small\it \noindent ``To look at the
% cross-section of any plan of a big city is to look at something like the
% section of a fibrous tumor.'' Frank Lloyd Wright\\}
\newpage

\section*{Introduction}

\noindent As urbanization rampantly adds tens of millions of people to cities every year, it is important to understand how the urban way of life affects the human condition, particularly as it relates to social interactions.
 %
 Urbanism is not only a built environment, but also a way of life with profound social consequences. 
 %
The concern is longstanding---the effect of urbanism on the human condition has
been studied since Aristotle \citep{jowett1920aristotle}, by many intellectuals
such as Thomas Jefferson and Henry David Thoreau \citep{white77}. The classic urban sociology, in particular, has produced significant insight into this subject \citep{wirth38,tonnies57,simmel03}. Yet, the topic of misanthropy  remains largely unexplored. This study is inspired by \citet{amin06} and \citet{thrift05}, whose sharp observation of
the urban way of life suggests the existence of urban misanthropy. 

Misanthropy---from \textit{misos}(\textit{n.}), ``dislike or hate,'' and \textit{anthropos}(\textit{n.}), ``humans''---refers to the lack of faith in others and the dislike of people in general.   
%
Misanthropy is a critical judgment on human life caused by failings that are ``ubiquitous, pronounced, and entrenched'' \citep[p. 7]{cooper2018animals}.
 %
Socrates \citep[cited in][]{melgar13} argued
that misanthropy develops when one puts complete trust in someone, thinking the
person to be absolutely true, sound, and reliable, only to later discover that
the person is deceitful, untrustworthy, and fake---when this happens frequently,
misanthropy develops.
%\noindent
 Notably, \citet[]{thrift05} proposes that ``misanthropy is a natural
condition of cities, one which cannot be avoided and will not go away''
(p. 140). % \footnote{For a detailed discussion please refer to the Supplementary Online Material (SOM)}
 
Thus, using novel data we conduct empirical quantitative analyses over the years of 1972--2016 to test this urban misanthropy thesis. The paper is structured as follows: We start with a brief overview of how urbanicity has impacted different aspects of life. Next we present the underlying theory, the urbanism-misanthropy pathways, by bringing together human evolutionary history (small group living), homophily or ingroup preference, and classic urban sociological theory suggesting that
misanthropy should be observed in the most dense and heterogeneous places, such as large cities. We end the literature review by pointing to gaps in the literature and pro-urban proclivity: remarkably, there is only one substantive quantitative study on urbanicity-misanthropy conducted thirty seven years ago, in a literature that is dominated by pro-urban scholarship. % Our empirical analysis follows, and concludes with the proposition that many of the largest metropolitan areas are in fact misanthropolis, a misanthropic metropolis.
 Our
empirical analysis follows and we conclude with a discussion of results and % a proposition of the misanthropolis,
% a misanthropic metropolis, and
 takeaways for policy and practice.
 %
   

\section*{Literature}

\subsection*{Advantages of City Life (Pro-Urbanism)}    
    
Much of the recent urban scholarship has emphasized the positive aspects of cities \citep{thrift05,amin06,aokCityBook15,peck16}, a case in point being the bestselling book, the ``Triumph of the City: How Our Greatest Invention Makes Us Richer, Smarter, Greener, Healthier,
and Happier'' \citep{glaeser11}.
 %
Studies have highlighted how metropolitan areas facilitate different aspects of life by providing more amenities, freedom, research and innovation, economic growth, higher wages, multiple efficiencies related to density in transportation, and public goods provision \citep{tonnies57,osullivan09,meyer13,rosenthal02,bettencourt10,stansel2019,ahlfeldt2022,liu2020,smith2021}.
 %
According to \cite{park84}, ``City air makes men free (Stadt Luft macht frei)'' \citep[p.12]{park84}---the diversity and heterogeneity found in urban centers can translate into increased tolerance and acceptance of others \citep{tuch87,wirth38,stephan82,aok20}. Similarly, studies point to how urban heterogeneity and diversity can benefit the economy by creating technological innovations, increasing productivity levels, and enhancing the supply and the quality of goods and services \citep{rodriguez2019does,rodriguez2020,mulligan2020}. Returns from education are greater in cities than smaller places, and urbanites have more economic opportunities than rural dwellers \citep{florida13,berry2005,storper2009,roca2017}. Cities are engines of economic growth and development \cite{osullivan09,carvalho2016,glaeser11}.  %

Cities can also contribute to a greater sense of community \citep[]{chavis2002sense,macke2019smart}. 
 Although city life is related to impersonal social relations, particularly when it comes to neighborly relations, studies show that residents of compact urban areas have higher levels of social interaction, participation in religious groups, and volunteering than residents of suburban areas \citep{nguyen10,mazumdar18,anon17-cities-oslo}. Research suggest that cities and metropolitan areas increase chances of interaction, in-person communication, and face-to-face contact \citep{carvalho2016,storper2004}, which in turn may create more opportunities for socialization~\citep{anon17-cities-oslo}. Thus, urbanites can have larger social networks and socialize more frequently than low-density suburban residents, while having more opportunities to meet new friends or partners \citep{mouratidis18,anon17-cities-oslo}.   
 %   
 %Much of the impersonal social relations observed in cities is due to neighbor relations \citep{nguyen10,mazumdar18}. 
 Concurrently, urbanites are able to more easily create their own communities in cities (e.g., shop in a particular bodega, use a specific laundromat, worship in a well-liked church/temple, frequent a preferred gym) and will socialize and trust those in their social bubble. If that trust is broken, it's easier to find another bodega, another laundromat, and so forth. {In rural and small communities, on the other hand, if trust is broken, it is more difficult to find a replacement, and life can become cumbersome.}
 %
 %
 % These are all important benefits of living in a city, as opposed to living in a village, the suburbs, or the countryside.

%While much knowledge has been produced on the benefits of city life, few studies have focused on drawbacks. 

%Given the aforementioned, it may seem incongruous that cities would generate misanthropy.  

%missing connection between these two sections of the literature review

\subsection*{Urbanism-Misanthropy Pathways}

In contrast,
%to the literature documenting the aforementioned benefits of urban life, 
fewer studies have focused on urban misanthropy 
\citep{thrift05,melgar13,keeling13,smith97,bloch87,wilson85,ray81,gibson17,rosenberg57,rosenberg56}.
%So, how can cities produce misanthropy? There are several pathways or mechanisms. 
%In the past few centuries, there has been an exponential and unprecedented surge in the number of people clustering together in cities.
% Humans have not evolved for city living. 
%These outline the mechanisms through which cities can induce misanthropy. 
First, living in large, dense, and heterogeneous settlements (city living) is, at least in some ways, incompatible with human nature \citep{haidt12B}. Throughout our evolutionary
history, for thousands of years, humans have lived in small, low-density
homogeneous groups. As hunter gatherers, humans lived in small bands of 50 to 80
people; later, they formed simple horticultural groups of 100 to 150 people,
finally clustering in groups as large as 5,000-6,000  people as they evolved
into more advanced societies \citep{maryanski92}.
 %
%Human nature is unlike that of other species living at high densities such as bees \citep{haidt12B}
% Hunter gatherer bands of 50-80 or early societies of few thousand are not only
% quantitatively smaller, bul also fundamentally and qualitatively different from
% modern urbanism.
%  %
%  Urbanization has deeply affected multiple  aspects of social, political, and economic life \citep{kleniewski2010cities}. 
% Before industrialization took off, in the early 1800s, only several  percent of
% the world population lived in cities. The proportion more than doubled by 1900, to 13 percent, as people moved to be near factories and industrial sites
% \citep{davis55}. In 1950, a third of the world population inhabited cities, and by 2050 it is estimated that city dwellers will represent approximately two thirds of the global population (\url{https://esa.un.org/unpd/wup}). 
Similarly, humans tend to have ingroup preference or homophily, and
accordingly, usually lack preference for or dislike heterogeneity
\citep{smith14,mcpherson01,bleidorn16,putnam07}, which is a characteristic feature
of cities \citep{wirth38,amin06,thrift05}. Previous research indicates that high diversity is related to lower trust and less social participation \citep{alesina99,alesina00,luttmer01,alesina02,rodriguez2019does}.
%All of these factors are likely to lead to urban misanthropy. 
%Diversity has been considered a strong and persistent barrier to developing trust across racial, ethnic or national origins \citep{glaeser00}. Cultural diversity can affect trust among residents of multi-ethnic and multi-religious places, generating animosity and creating conflict while simultaneously harnessing this multitude of experiences to 


Early sociologists  
% \footnote{these are dated but classic theories are still the best and most illuminating research in the field}
 proposed that urbanization creates malaise due to three core characteristics of cities: size, density, and heterogeneity. Increased population size creates anonymity and
 impersonality, density creates sensory overload and withdrawal from social
 life, and heterogeneity leads to anomie, deviance, and lower trust and wellbeing (\citet{park84,
   simmel03, tonnies57, wirth38,putnam07,aok_brfss_segregation15,herbst14,postmes02,vogt07,smelser99}).
%
%\subsection*{Literature: Urbanism and Distrust/Dislike of Humankind (Misanthropy)}
%
% In this section we underline how the classic urban sociological literature along with other relevant literatures provide a theoretical explanation to urban misanthropy. 
%
City life can cause cognitive overload, stress, and coping \citep{simmel03, milgram70,lederbogen11}. An overloaded system can suppress stimuli resulting in a \textit{blas\'e} attitude
\citep{simmel03}---city life can cause withdrawal, impersonality, alienation, superficiality, transitiveness, and shallowness \citep{wirth38}. Furthermore, city life may intensify cunning and calculated behavior \citep{tonnies57}, estrangement, antagonism, disorder, vice, and crime
\citep{milgram70,park15,park84,bettencourt10b}, which can lead to negative 
responses when interacting with others.
%
Urbanism can also exert a negative influence on the quality of social relationships \citep{wilson85}, and urbanites are often depicted by the social imaginary as ill-mannered and unreliable, which can lead to misanthropy
% \citep[e.g.,][]{aokCityBook15,aok-sizeFetish17}. It is not just city living, studies show that growing up in a city is associated with negative consequences later in life \citep{lederbogen11,aok20}.
\citep[e.g.,][]{aokCityBook15,aok-sizeFetish17}. These attributes of city living can be long lasting---growing up in a city is also associated with negative consequences later in life regardless of where you end up living \citep{lederbogen11,aok20}.

%
Of the many urban challenges, next to crime, crowding may be especially conducive to misanthropy.  
Crowding can be a significant problem in large cities, which forces a large number of people to live in close proximity (household crowding) and in a small amount of space (residential crowding). Crowding is associated not only with higher levels of stress and depression, but also with aggression \citep{regoeczi2008,calhoun62}. 
There are striking examples of crowding in the largest and densest cities around
the world. New York City, for example, offers 250, or even 100 square feet apartments to its residents
\citep{abc,yoneda,dailynews}. Some ``cubbyholes,'' are yet smaller at 40 square feet \citep{newyorktimes}. In other dense cities, like Hong Kong, crowding can be even worse \citep{newyorktimes2}. To be sure, the majority of the urban population does not live in such extreme crowding conditions, and crowding is also an issue in smaller areas---some people crowd in houses in small towns or villages.
  While high density is not the same as crowding, the two concepts are often
  correlated \citep{meyer13}, and urban crowding is probably becoming more
  common  as cities are becoming less affordable  \citep[e.g.,][]{misraCL15oct6,floridaCL18apr11,weinbergCL16aug11,solariMISC19apr24,schuetzMISC19may7,kotkin_db_mar20_13}. 
%    
 Still, urbanization does not need to lead to extreme overcrowding, and attention should be
 paid to the many potential drivers of crowding such as uneven urban
  development, gentrification and displacement, as well as inequality in large
  globalizing cities.
 %
  While urbanization faces many challenges, notably in the global South \cite{hong21}, 
  there are strategies, solutions, and best practices for sustainable urbanization \citep{bauduceau2015towards, world2016urban,ochoa2018learning,tan2016sustainable}.
  %
%  For an overview of best practices in sustainable urbanization see \citet{ochoa2018learning,tan2016sustainable}

  % Density may impact pathology more than crowding \citep{levy1974effects}. Yet, it is not only density and crowding, other factors such as social support and expectations matter as well \citep{cassel2017health,chan78}. However, results are mixed; some studies didn't find negative effects of density or crowding \citep{collette1976urban}. While it seems reasonable to assume that density and crowding are usually positively related, some studies do not find this to be the case \citep{webb1975meaning,rodgers1982density}. 
%   %
%  The literature about density
%   and crowding is mostly dated as well. Current research should address this gap in the literature, especially as crowding is probably becoming more widespread.
% %
% For a discussion and overview of density, crowding and human behavior see \citet{boots1979population,choldin1978urban} and \citet{ramsden09}.} 
% 
% Concurrently,  crime, traffic congestion, and incidence of infectious diseases ({case in point, the current COVID19 crisis}) do increase with population size \citep{bettencourt10,bettencourt10b,bettencourt07}.



In sum, city life can make people become more distant from or hostile toward other human beings. % \footnote{There are, however, multiple advantages of city life as discussed in the next section.}  
For many, urban life is being ``lonely in the midst of a million'' (Twain), ``lonesome together''
(Thoreau), alienated \citep{wirth38,nettler1957measure}, ``awash in a sea of strangers''
\citep[Merry cited in][p. 99]{wilson85} in a ``mosaic of little worlds which touch, but do not interpenetrate'' \citep[][p. 40]{park84}. Thus, we hypothesize: \\
 
{\indent\hspace{1in}\textit{Urbanicity contributes to increased levels of misanthropy.\\}}
     

\subsection*{Gaps in the Literature and Study's Main Contribution} 

The gap in the literature is two-fold. First, the current urban literature tends
to avoid the negative side of  urbanism
\citep{thrift05,amin06,aokCityBook15,peck16}. Second, there is only one
quantitative study  focused on the urbanicity-misanthropy relationship \citep{wilson85}. 
 %
 % Therefore, a major contribution of this study is to build on the largely overlooked literature and  extend it with empirical analysis using novel data.  

Academic thinking about cities has for the most part swung in a pro-urban direction for many decades not only in the US \citep{hansonCityJournalautumn15}, but also in world development generally \citep{lipton77}. The classic sociological urban theory \citep{wirth38,milgram70,park15,park84,simmel03,tonnies57} gave way to
  sub-cultural theory \citep{fischer75,fischer95,wilson85, palisi83}, while debates about the optimal size of a city \citep{richardson72,singell74,alonso60,alonso71,elgin75,capello00} emerged in the-bigger-the-better ideology \citep{glaeser11}. As a result, there is no recent research in the urbanicity-misanthropy relationship---only two studies examined this relationship employing quantitative methods \citep{wilson85,smith97}. \citet{smith97} lists a simple bivariate correlation between urbanicity and misanthropy among dozens of other bivariate correlations in a General Social Survey technical report without discussing the topic. 
%The report is published in a journal, but it is a carbon copy of the ``GSS Topical Report No. 29,'' that is mostly a listing of correlations with annotations as \citet{smith97} was exploring factors relating to misanthropy in American society in general.
Therefore, \citet{wilson85} is the only substantive quantitative study focusing on the urbanicity-misanthropy nexus.% ---such gap in the literature is rare.


\citet{wilson85} used the 1972-1980 GSS dataset in his analysis---he does not show trends over time and only controlled for a handful of variables.  Arguably, like other
contemporary social scientists such as \citet[][]{veenhoven94}, \citet[][]{meyer13} and \citet[][]{fischer82}, \citeauthor{wilson85} has a slight pro-urban
proclivity---under-emphasizing and discounting the negative side of urbanism.  
%Likewise, \citet{wilson85} provides a narrow sociological view on the topic. Therefore, the aim of this paper is precisely to fill this gap in the literature by providing an up to date quantitative analysis of the relationship between urbanicity and misanthropy. We control  for an extensive set of variables, examine trends over the last four decades, and provide a much broader and interdisciplinary perspective. 

The lack of research on the link between urbanicity and misanthropy in urban studies
seems to emerge from an avoidance to focus on the darker and misanthropic side
of cities. As Nigel Thrift aptly observed, there is ``a more deep-seated sense of misanthropy which urban commentators have been loath to acknowledge, a sense of misanthropy which is too often treated as though it were a dirty secret'' \citep[p. 134]{thrift05}. Therefore, the aim of this paper is to fill this gap in the literature by
bringing together a largely overlooked literature from across different fields and
by providing an up to date quantitative analysis of the relationship between
urbanicity and misanthropy. Building on and extending \citet{wilson85}, we control for an extensive set of variables, examine trends over the last four decades, and provide a much broader and interdisciplinary perspective of the relationship between urbanicity and misanthropy. 


% Despite all of the benefits of city life, nonetheless, the question  remains:
% \textit{Could urban areas increase misanthropy?} We explore and attempt to answer this question next. 

\section*{Method} 

\subsection*{Data}

\hilite{We use unique} misanthropy measure from the 1972-2016 US General Social Survey (GSS;
\url{http://gss.norc.org}). The GSS is a cross-sectional, nationally
representative survey, administered annually since 1972 until 1994 when it
became biennial. The unit of analysis is at the individual level and data are collected in face-to-face in-person interviews \citep{davis07}. The full dataset contains about 60 thousand observations pooled over 1972-2016. All variables were recoded in such a way that a higher value means more. 
 % 

\citet{marsden20} provides an useful overview  of the GSS, one of the most
widely used datasets in contemporary social science. The GSS has a wide range of
attitude and behavior data, together with a wide and detailed body of background
information including socioeconomic status, social mobility,
social control, the family, civil liberties, and morality. % Topical
%  modules designed to investigate new issues or to expand the coverage of an existing subject
% have been part of the GSS since 1977.
%

The misanthropy scale items and urbanicity
measures have been part of GSS since its first wave in 1972. 
 The GSS takes care to ensure the over-time comparability of measures for trend analyses \citep{marsden20}, which we utilized in this study to examine the relationship between urbanicity and misanthropy over 4 decades. According to \citet{marsden20}, the GSS prioritizes survey quality, maintaining response rates above the survey industry standard.

\subsection*{Research Design and Model}

The research design is ex post facto \citep{mohr95}. Our study is observational or correlational---data used are secondary, without any experimental manipulation.\footnote{Observational or correlational studies are not
without merit%despite what many economists % , especially so called
% experimentalists,
%think
---many scientific breakthroughs were first
discovered in observational studies---for instance that smoking is
related to cancer \citep[e.g.,][]{blanchflower11,oswald14}.
 %
 Furthermore, experimental data % are now in fashion in economics, and it is often
% overlooked that they 
 suffer from many critical problems that are not inherent in
observational data such as lack of external validity, small sample size, and  
artificial laboratory setting. % , and forced imaginary roles, such as a person
% pretending to be a company or  imagining winning a
% lottery.
For a discussion see \citet{pawson97}.}

As explained in the next subsection, the dependent variable, `misanthropy,' is
continuous. Hence, we  use ordinary least squares (OLS) to analyze the
relationship between urbanicity and  misanthropy.
Multilevel techniques are not applicable  as the GSS is only representative of
large census regions, and we do not have the restricted GSS data with finer
geographical information.
 %
 GSS is a repeated cross-sections dataset with different persons in each wave, hence
 panel data techniques are not applicable either. 

\subsection*{Misanthropy}

We measure misanthropy, the distrust and dislike of humankind, with a three item  Rosenberg's  misanthropy index \citep{rosenberg56,smith97}:\\

\indent\textsc{trust}. ``Generally speaking, would you say that most people can be trusted or that you can't be too
careful in dealing with people?''  $1=$``cannot trust,'' $2=$     ``depends,'' $3=$   ``can trust.''\\
\indent\textsc{fair}. ``Do you think most people would try to take advantage of you if they got a chance, or
would they try to be fair?'' $1=$``take advantage,'' $2=$       ``depends,'' $3=$          ``fair.'' \\
\indent\textsc{helpful}. ``Would you say that most of the time people try to be helpful, or that they are mostly just
looking out for themselves?'' $1=$``lookout for self,'' $2=$       ``depends,'' $3=$        ``helpful.''\\ 

Rosenberg defines misanthropy as a general uneasiness, dislike, and
apprehensiveness towards strangers \citep{rosenberg56}. Using the three items,
we utilized factor analysis with varimax rotation to produce an index, and we
reversed it so that it measures misanthropy. Cronbach's alpha is .67. The
distributions of these items, as well as the descriptive statistics for all other variables, are in the Supplementary Online Material (SOM).

 Although, much controversy about the assessment of misanthropy exists in the
 literature, the Rosenberg scale has become the standard measure for
 self-reported misanthropy and was designed to assess one's degree of confidence
 in the trustworthiness, goodness, honesty, generosity and brotherliness of
 people in general \citep{rosenberg56}. The measurement encompasses ``faith in
 people,'' ``attitudes towards human nature,'' and an ``individual's view of
 humanity.'' The Rosenberg misanthropy scale has been a cornerstone on the GSS since 1972, and the measurement is not contaminated by social desirability bias \citep{ray81}. 
 The Rosenberg misanthropy scale is the most popular and widely cited measurement of misanthropy. Some authors \citep[e.g.,][]{wuensch2002misanthropy} have used other scales, but their approaches are disjoint from the mainstream literature, and there is not much discussion of the concept or measurement that they used in their research.  

Strictly speaking, the Rosenberg scale does not measure the dislike of ``all people,'' but ``most people.'' \citet{wilson85} suggests it is dislike of strangers, specifically. Likewise, \citet{delhey11} have recently argued that ``most people'' predominantly connotes outgroups. Note that this relates to the homophily/ingroup theory---a dislike for an outgroup typically means relative preference for the ingroup. 
 
\subsection*{Urbanicity}

Urbanicity is measured in three ways to show that the
results are robust to the definition. First, it is measured using deciles of population size
(\textsc{size}). Deciles are used to investigate if there are any nonlinear
effects on misanthropy. Two other variables are used to measure urbanism under
their original GSS names: \textsc{xnorcsiz} and \textsc{srcbelt}.

{\citet{wilson85} uses these two variables in his study. One technical problem,
  however, is that he assumes that these variables are
  continuous. \citet{wilson85} explicitly states that \textsc{xnorcsiz} is an ordinal
  variable, and we disagree: one cannot really say whether a suburb is larger
  than an unincorporated large area and smaller than an area of 50 thousand
  people.}

Both \textsc{xnorcsiz} and the \textsc{srcbelt} variables categorize places into metropolitan areas, big cities, suburbs, and  unincorporated areas. The advantage of \textsc{size} is that it allows us to calculate a misanthropy 
 gradient by the exact size of settlement. \textsc{Xnorcsiz} and \textsc{srcbelt} take into account the fact that populations cluster at different densities (e.g., suburbs are less dense than cities). The GSS does not provide a density variable. 

The \textsc{srcbelt} measurement is arguably the best fitting to illustrate the
urban vs. rural divide: the divide is between metropolitan areas vs. smaller areas
\citep{hansonCityJournalautumn15}, and the \textsc{srcbelt} variable identifies the
metropolitan areas (as Metropolitan Statistical Areas) and it classifies metros
by their rank and size: small rur, small urb, 13-100 sub, 1-12 sub, 13-100 msa, 1-12 msa. The GSS detailed codebook descriptions are in the SOM. 
   

\subsection*{Controls}

In the choice of control variables, we follow \citet{welch07} and  \citet{smith97}.
The higher the social standing, the more favorable view of others---we
control for income, education, and race. The social class literature suggests that
individuals' social class should be assessed  using both objective (e.g.,
income and education) and subjective indicators \citep[e.g.,][]{kraus09}. % \footnote{We thank an anonymous reviewer for this important point. Subjective class correlates with education and income moderately at about .4 (either continuous or polychoric). On one hand, subjective class and urbanicity are likely to be confounded. On the other hand, it turns out that correlations of urbanicity measures and subjective class are very small, below .1 (either continuous or polychoric). The social class item in the GSS reads: ``If you were asked to use one of four names for your social class, which would you say you belong in: the lower class, the working class, the middle class, or the upper class?'' and is coded from 1 (lower) to 4 (upper). We will just treat it as a control variable and enter it as a continuous variable without using a set of dummies for simplicity.}
Thus, we control for person's perceived social class as well. 

Negative experiences are likely to increase misanthropy, therefore we control
for fear of crime (there is no adequate measurement of actual victimization in
the GSS). Crime is relevant because the larger the place, the more crime
\citep{bettencourt10b,wirth38,white77}, and the more crime, the more misanthropy
\citep{wilson85}. As explained by \citet{glaeser1999}, cities may create greater
returns to crime because urban areas provide criminals more access to the wealthy and to
a greater range of victims. Likewise, the lower likelihood of
arrest, and the lower probability of recognition are features of urban life that
make crime more frequent  \citep{glaeser1999}. % The higher crime rates in big cities are particularly salient to our research given that
 Fear of crime can result in social problems such as lower interpersonal and institutional trust, change in behavioral patterns and lifestyle, and integration into society (see \citet{krulichova2018life}) being therefore an important control. 

We also control for unemployment, self-reported health, and age. We control for
divorce, a predictor of misanthropy.  Misanthropy should be higher among
cultural groups and minorities that have been discriminated against---we control
for race, being born in the US, and religious denomination. Religious belief may
reduce misanthropy---religions commonly promote philanthropy and altruism. This
is especially true of social religiosity (services attendance, church
membership), but individual religiosity or believing (prayer, closeness, and
belief in God) may actually increase misanthropy \citep{aok20rel}. Misanthropy
may be lower among older people, and there may be a curvilinear relationship,
therefore we control for age and age$^2$. Men tend to be more misanthropic---we
control for gender. Recent movers may be more misanthropic and although there is not an adequate measure in the GSS indicating whether someone moved recently, we use a proxy for international relocation by controlling for being born in the US.

In addition, we control for subjective wellbeing---the goal is to alleviate a potential problem of spuriousness. It may not be the size of a place that causes higher misanthropy, but %a lack of success
poor quality of life or unhappiness \citep{aok21} that correlates with
both urbanicity and misanthropy.  % \footnote{Unhappiness with city life is common in developed countries \citep{aokCityBook15,sorensen14,morrison17,ala18,aok20,aok21}---
% and quality of life/wellbeing may arguably impact misanthropy. % And we included fear of crime as well, one of the most important confounders---crime increases misanthropy and tends to be higher in cities---and another key measure is income, which is controlled for.
 In addition, we control for health which may vary across urbanicity
 \citep[e.g.,][]{chen2019differences}, and possibly, unhealthy persons are more likely to be misanthropic.  
 Concurrently, liberals and immigrants are more likely to live in cities and both groups are less satisfied with their lives \citep{aok11a,aokJap14} and potentially more misanthropic. Thus, we control for political ideology and immigration status.

Data were pooled over 1972-2016, and hence we include year dummies. Also, there
are substantial regional differences across the US---we include a ``South'' dummy variable. All variables are defined along with their survey questions in the SOM.

\section*{Results}

This section reports the empirical results of our hypothesis test:
 \textit{urbanicity contributes to increased levels of misanthropy}.

Tables \ref{regA}, \ref{regB}, and \ref{regC} show the regression results of misanthropy. We use three measures of
urbanicity, one in each table,  and each urbanicity measure is entered as a set of dummy variables to
explore nonlinearities. The base case is the smallest place in the case of
\textsc{size} and \textsc{srcbelt}, and the second smallest category on \textsc{xnorcsiz}:
 ``$<$2.5k, but not countryside.'' Coefficients of interest are those on the
 largest  places such as the second largest category ``192-618k,'' and especially the largest one ``618k-'' in Table
\ref{regA}, and corresponding the second largest and the very largest places in Tables
\ref{regB} and \ref{regC}.

\begin{table}[h!]\centering
\caption{OLS regressions  of misanthropy. Beta (fully standardized) coefficients
  reported. All models include year dummies. \textsc{Size} deciles (base: $<$2k).} \label{regA}
\begin{scriptsize} \begin{tabular}{p{1.8in}p{.45in}p{.45in}p{.45in}p{.45in}p{.45in}p{.45in}p{.45in}p{.45in}p{.45in}p{.45 in}}\hline
                    &   $<.5m$   &     $>.5m$   &   $<.5m$   &     $>.5m$   &   URYrurTow   &     URYcity   \\
2022                &       -0.21** &       -0.41** &       -0.12** &       -0.23   &        0.75***&        0.23+  \\
constant            &        7.54***&        7.65***&        7.50***&        7.38***&        7.54***&        7.69***\\
N                   &        3111   &         521   &        3572   &         373   &        1154   &         836   \\

\hline  *** p$<$0.01, ** p$<$0.05, * p$<$0.1; robust std err
\end{tabular}\end{scriptsize}\end{table}

\begin{table}[h!]\centering
\caption{OLS regressions  of misanthropy. Beta (fully standardized) coefficients
  reported. All models include year dummies.  \textsc{Xnorcsiz} (base: $<$2.5k, but not countryside).} \label{regB}
\begin{scriptsize} \begin{tabular}{p{1.8in}p{.45in}p{.45in}p{.45in}p{.45in}p{.45in}p{.45in}p{.45in}p{.45in}p{.45in}p{.45 in}}\hline
                    &   $<.5m$   &     $>.5m$   &   $<.5m$   &     $>.5m$   &   URYrurTow   &     URYcity   \\
2022                &       -0.18*  &       -0.39+  &       -0.20***&       -0.45** &        0.42***&        0.21   \\
income              &        0.09***&        0.01   &        0.06***&        0.14***&        0.07*  &        0.13***\\
age                 &       -0.03*  &       -0.08** &       -0.02+  &       -0.06+  &        0.00   &       -0.06** \\
age2                &        0.00** &        0.00** &        0.00** &        0.00*  &       -0.00   &        0.00** \\
male                &       -0.18** &       -0.13   &       -0.11*  &       -0.27+  &        0.06   &        0.19   \\
married or living together as married&        0.53***&        0.74***&        0.44***&        0.23   &        0.46** &        0.06   \\
divorced/separated/widowed&        0.07   &        0.15   &       -0.11   &       -0.14   &       -0.37+  &       -0.19   \\
autonomy            &       -0.11*  &       -0.07   &       -0.11** &       -0.01   &       -0.06   &        0.06   \\
freedom             &        0.44***&        0.42***&        0.35***&        0.43***&        0.43***&        0.36***\\
trust               &        0.12+  &        0.42** &        0.43***&        0.28+  &       -0.05   &        0.10   \\
postmaterialist     &       -0.05   &       -0.18   &       -0.11*  &        0.14   &       -0.02   &        0.15   \\
god important       &        0.01   &        0.05*  &        0.02*  &       -0.01   &        0.05** &        0.06** \\
constant            &        4.08***&        5.95***&        4.59***&        4.80***&        3.47***&        4.58***\\
N                   &        1985   &         309   &        2283   &         237   &         736   &         579   \\

\hline  *** p$<$0.01, ** p$<$0.05, * p$<$0.1; robust std err
\end{tabular}\end{scriptsize}\end{table}

\begin{table}[h!]\centering
\caption{OLS regressions  of misanthropy. Beta (fully standardized) coefficients
  reported. All models include year dummies. \textsc{Srcbelt} (base: small rur).} \label{regC}
\begin{scriptsize} \begin{tabular}{p{1.8in}p{.45in}p{.45in}p{.45in}p{.45in}p{.45in}p{.45in}p{.45in}p{.45in}p{.45in}p{.45 in}}\hline
                    &   $<.5m$   &     $>.5m$   &   $<.5m$   &     $>.5m$   &   URYrurTow   &     URYcity   \\
2022                &       -0.12   &       -0.26   &       -0.06   &       -0.24+  &        0.44***&        0.23   \\
health              &        0.48***&        0.67***&        0.62***&        0.77***&        0.56***&        0.32** \\
income              &        0.05** &       -0.01   &        0.04***&        0.08** &        0.05   &        0.12***\\
age                 &       -0.02*  &       -0.07*  &       -0.01   &       -0.03   &        0.01   &       -0.05*  \\
age2                &        0.00** &        0.00** &        0.00** &        0.00+  &       -0.00   &        0.00*  \\
male                &       -0.16*  &       -0.15   &       -0.09+  &       -0.23+  &       -0.01   &        0.14   \\
married or living together as married&        0.49***&        0.60** &        0.38***&        0.21   &        0.41** &        0.04   \\
divorced/separated/widowed&        0.05   &        0.20   &       -0.15   &       -0.27   &       -0.36+  &       -0.16   \\
autonomy            &       -0.12** &       -0.09   &       -0.10** &        0.07   &       -0.09   &        0.04   \\
freedom             &        0.38***&        0.29***&        0.29***&        0.31***&        0.40***&        0.35***\\
trust               &        0.07   &        0.28*  &        0.34***&        0.21   &       -0.07   &        0.01   \\
postmaterialist     &       -0.05   &       -0.26+  &       -0.09*  &        0.06   &        0.01   &        0.12   \\
god important       &        0.01   &        0.02   &        0.02+  &        0.00   &        0.05** &        0.06** \\
constant            &        2.72***&        4.29***&        2.46***&        2.01*  &        1.31+  &        3.31***\\
N                   &        1985   &         309   &        2279   &         236   &         736   &         578   \\

\hline  *** p$<$0.01, ** p$<$0.05, * p$<$0.1; robust std err
\end{tabular}\end{scriptsize}\end{table}

The first column of each table (a1, b1, c1) shows coefficients from a basic
regression of misanthropy on a set of dummy variables for a given urbanicity
measure without any control variables, except for South and year dummies (not shown). 
 %
The largest negative effect of urbanicity on misanthropy is observed for the largest
places, as expected. In the case of \textsc{size} and \textsc{srcbelt}, the second largest effects tend to be on the second largest place, also as expected.  In
the case of \textsc{xnorcsiz}, in addition to the largest cities, the countryside is quite
misanthropic. This is an unexpected result---we had not hypothesized that the countryside would be misanthropic. Perhaps countrymen are not used to swarms of people, or
perhaps they are countrymen because they are misanthropic and distrust and
dislike people. %ADAM: this seems out of place (thoughts?) \citet{keeling13} argues that the links between wilderness and misanthropy are false. 
%rubia ok
The second columns (a2, b2, c2) in the tables add controls following
\citet{welch07} and \citet{smith97}.
 %
 The change in  estimates is substantial across all three urbanicity
 measures---midsize places become much more misanthropic---now they are about
 half or third as misanthropic as the largest place (all urbanicity estimates
 are relative to the base category). 
 %
 In Table \ref{regB}, an interesting result on the
\textsc{xnorcsiz} dummies is that of misanthropic suburbs, the so called ``places of
nowhere'' \citep{kunstler12}. These results seem to support studies documenting the existence of a poor social fabric in American suburbia \citep{duany01,kunstler12,kay97}.
 %
Overall,
 we find that having controlled for a standard set of misanthropy predictors, midsize places are more misanthropic, and still the
 largest places are the most misanthropic in comparison to the smallest
 places (the base case for all estimates).
 %
Thus, the larger the place, the more misanthropy. 

The addition of marital status in Model 3 does not change the estimates, and the addition of extra controls in Model 4 attenuates the slopes only slightly  across
all three measures of urbanicity.
 % 
{While the fullest specifications are the least biased in terms of omitted
variables, the sample size is much smaller than the more basic models due to
missing observations on additional variables. These more elaborate
specifications are rather over-saturated models with collinearity and 
too many non-essential controls. These models rather serve as a robustness check, and are not the most final or appropriate models. %The previous two studies on misanthropy did not examine the
%effect of these variables \citep{wilson85,smith97}.
%Hence, lower statistical significance and smaller effect sizes are somewhat expected.
 Note that  \citet{wilson85} did not control for variables added in Model 4 and beyond.}

%The final most elaborate specifications also show no significant misanthropy difference for the 2nd largest places---these results contradict earlier results where the second largest places were the second most misanthropic. Therefore results for the second largest places should be interpreted with care.

Model 4a adds ``\textsc{afraid to walk at night in neighborhood}'' to Model 4,
and Model 4b adds a ``\textsc{white household}'' dummy to Model 4, and finally
Model 4c adds both variables.  The rationale for the three models 4a, 4b, and 4c is
that the sample size drops by about half due to missing data when adding ``\textsc{afraid to walk at night in neighborhood}'' to the model. Furthermore, race is
likely to play a role not only with respect to urbanicity and misanthropy, but it
may also correlate with being ``\textsc{afraid to walk at night in neighborhood},'' e.g., whites may be more afraid than others. We use the three
models 4a, 4b, and 4c with different combinations of the two variables to test robustness of the results. 
%

In Table \ref{regA}, Model a4c and Table \ref{regB}, Model b4c%(but not c4c)
, the largest places
remain significantly more misanthropic than the base case. Yet, the magnitude of the effect on the largest places is not greater than that for mid-sized places, suburbs, and even the countryside. Such result could be puzzling.
 %
 But as argued earlier, \textsc{srcbelt} is the variable that probably best
 captures the urban-rural divide, and when using \textsc{srcbelt} in Table
 \ref{regC}, we find that even the oversaturated 
 Model c4c shows that it is the largest places (both 1-12 msa, and 1-12 sub) that are markedly more misanthropic than all other places vs. the base case, the smallest places.

 The overall conclusion is that the places housing up to a few thousand people
 (except for the countryside) are the most liking and trusting of humankind (the least
 misanthropic). In other words, there is misanthropy in larger places, especially in the largest places---places that have a population bigger than several hundred thousand people versus the smallest places (up to a few thousand people, and not the countryside).

%\footnote{These most elaborate models have smaller
%  sample sizes and suffer from missing observations and some multicollinearity. 
 % Note, the most elaborate specifications are rather over-saturated models with
%too many controls and  collinearity.

%The conclusion is not
%  that there is no meaningful difference in misanthropy across places of
%  different size. Most evidence points to the largest
%  places being most misanthropic. 



%\citet{wilson85} has argued two key predictors of misanthropy: crime and race. 

%Like Wilson, for lack of a
%better variable, we are using fear of crime as a proxy in our analysis
%(\textsc{afraid to walk at night in neighborhood}), which is thought to increase
%misanthropy and correlate with urbanicity. Therefore, the inclusion of this
%variable should attenuate heavily the urbanicity-misanthropy relationship, and
%it does in model a4a. \citet{wilson85} also argues that urban misanthropy is
%more common among  whites than minorities. Inclusion of \textsc{white household} dummy 
% (without \textsc{afraid to walk at night in neighborhood}) in a4b has a similar effect to \textsc{afraid to walk at night in neighborhood}. 
% Finally in model a4c both variables are entered together, and the urbanicity effect is attenuated more and less significant. Results for the other two measures of urbanicity shown in tables \ref{regB} and \ref{regC} are similar. One difference is that in table \ref{regB}, the smallest areas (``countryside'') are slightly more misanthropic than the base case, ``smaller than 2.5k but not countryside.''

 

%Political ideology, marital status, health, SWB, and notably race and fear of crime explain away much of the city disadvantage, but not all of it. Hence, the conclusion is similar to studies examining SWB in urban areas \citep{aok_brfss_city_cize16}, it is cities, themselves, their core characteristics, and not city problems that are related to misanthropy. 

%Indeed, even if the results were insignificant, they would be still worth reporting---many would think that there is less misanthropy in cities---clearly we are in the midst of a pro-urbanism period, where it is fashionable to argue about city benefits \citep[e.g.,][]{glaeser11}. However, the results show that there is no such benefit with respect to misanthropy---cities are at least slightly more misanthropic than other places.

%Why did several midsize categories score relatively high on misanthropy? We do not have an explanation for this phenomenon. Perhaps, following \citet{aok-ls_fisher16}'s rationale, such places strip people of the naturalness found in the smallest places, and yet do not provide amenities and the benefits found in the largest places.

The effect sizes are considerable---all tables report beta coefficients
and the effect size of the largest place is at least about as large as half of the effect
of income.  To summarize, we find a weak to moderate support for our initial
hypothesis that urbanicity is related to increased misanthropy. The results are
only weak to moderate, and not strong, because the effect sizes are small to
moderate, and not large. In addition, there are caveats to the results as elaborated in the discussion section.

\subsection*{Analysis Over Time}

We complement our pooled data analysis with an investigation of over-time change
in the relationship between urbanicity and misanthropy---again, the advantage of the GSS
is a long time span of 1972-2016. Figure \ref{tim} plots misanthropy by size of
place over time.

\begin{figure}[H]
  \includegraphics[width=3in]{timINK.pdf}\centering
\caption{Misanthropy by size of population over time. Smoothed with moving average filter using 3 lagged, current, and 3 forward terms.}\label{tim}%collapsed categories of \textsc{xnorcsiz}.
\end{figure}

Overall, misanthropy remained highest in the large cities until
about 2005. Around 2000, the trends have changed---misanthropy for the largest
cities ($>$250k) started to decline, and misanthropy for the smallest places
($<$10k) started to increase steeply. Misanthropy for medium
sized places (10-250k) has been mostly increasing over 1972-2016. Hence, the
finding of urban misanthropy for the largest places is due to the pre-2005 period.
%
These patterns are similar when controlling for predictors of
misanthropy. Predicted values from the regression Model a3a in Table \ref{regDbyHand} in
the SOM are plotted in Figure \ref{timPre}. 


\begin{figure}[H]
  \includegraphics[width=3in]{timPreINK-ok.pdf}\centering
\caption{Misanthropy by size of population over time. Predicted values from the regression in column a3a from table \ref{regDbyHand} in the SOM. 95\% CI shown.}\label{timPre}%collapsed categories of \textsc{xnorcsiz}.
\end{figure}

There is convergence in misanthropy across urbanicity over time, with the
smallest places increasing their level of misanthropy the most. Misanthropy has
increased across all urbanicity levels in the US over 1972-2016, but it has
increased the most in the smallest places. % \footnote{In a few years as data
                                % become available, it will be instructive to
                                % find out whether the COVID19 pandemic has
                                % caused these trends to reverse. It's likely
                                % that the largest cities have become more
                                % misanthropic due to the pandemic.} 
 Note that the interactive regression
specification used to produce the predicted values plotted in Figure
\ref{timPre} is a time-linear model, which does not allow for nonlinearities observed for the raw values in Figure \ref{tim}.
% Indeed, if anything, the predicted values graphed show even greater increase in misanthropy and greater convergence for all areas than the raw values in figure \ref{tim}. 

%


\section*{Conclusion and Discussion}

% by bringing to the forefront a much needed discussion on one of the negative consequences of city life, particularly in the largest metropolitan areas.
This study seeks to spark debate on an overlooked area of urban studies. 
 Our results suggest the existence of
\emph{Misanthropolis}---misanthropic metropolis,  where distrust and dislike for humankind
abound.\footnote{The term \emph{misanthropolis} was coined by one of the authors.}


%City living has an enormous effect on humanity---the world is urbanizing at an astonishing pace---each year cities add tens of millions of people. Arguably the biggest divide of all is urban-rural, and it is important to investigate its multiple dimensions. 
In this article we have focused on a  novel area, the urbanicity-misanthropy
nexus. % \footnote{For a long time social scientists have tried to understand how urbanization affects human beings. Yet, the most sharp and critical observations were published decades ago---it is our contribution to connect with the illuminating classical studies amid current pro-urbanism trends. We offer the first up to date quantitative test based on a classic theoretical background.}
Evolution  (small group living),  homophily or ingroup preference, and classic urban sociological theory suggest that human dislike for other humans should be observed in the most dense and heterogeneous places such as metropolitan areas and large cities. Our results mostly agree: misanthropy is highest in cities larger than several hundred thousand people. There are caveats, however. 

First, the effect sizes are small to moderate, about half of the
effect of income. Second, it is only the second study  \citep[after][]{wilson85} on the topic and more data and research are needed to form  reliable conclusions. Third, the urban
misanthropy thesis holds up relatively robustly only for the largest cities or metropolitan areas (larger than several hundred thousand people). Some places in between, such as larger towns or suburbs, are not misanthropic depending on the model specification. Fourth,
the level of misanthropy in smaller areas is now reaching about the same level
as in large cities. In addition, our study uses US data only, and the conclusions may not generalize outside of the United States. Finally, this is a correlational study, and causality may not be present. 
% 

For these reasons, the evidence in support of our urban misanthropy thesis is
 weak to moderate.  We would like to stress, however, that we do find strong evidence that,
 overall,  cities are not less misanthropic than smaller places, and this in
 itself is a counter-intuitive finding worth of reporting and of future investigation.
 In addition, even the small to moderate effect size of urbanicity on misanthropy
 as found in this study, has an enormous practical combined effect size due to the
 sheer scale of urbanism---half of the world population is urban and growing by tens
 of millions every year. Hence, the small to moderate effect size found in the
 present study translate into large or very large effect in the aggregate. 

Our study fills a gap in the urban studies literature by improving and extending the research by \citet{wilson85}. Our analysis uses much more data spanning four decades, a larger set of control variables, and levels of size variables without
forcing untenable assumption of interval/ratio scale and linear effects. Our
results do not necessarily contradict, but rather extend \citet{wilson85}: there
is misanthropy in the largest places % for everyone %he did like black v white
and we find more robust evidence than \citet{wilson85} in this regard. Concurrently, we confirm the finding by \citet{fischer81} of a relatively strong relationship between community size and distrust. Notably, we find that rural misanthropy is on the rise.

% The magnitude of the effect of urbanicity is important to discuss. There is
% evidence of a large magnitude effect of urbanicity on trusting behavior. In one experiment,
% trust differed several-folds between city and town, a larger difference than
% across gender---the trust benefit of being female over male is smaller than the
% benefit of town over city \citep{milgram70}. While our results do not indicate a
%  strong effect of urbanicity on misanthropy, we do find a substantial
% effect---about half of the effect of income in our analysis%\footnote{One explanation is that people's trust is low in cities mostly because there are  simply too many people, not necessarily because they dislike  people.}
% ---contraposing \citet{wilson85}, who argued that there is only a small effect.

As in any correlational study, we cannot claim causality. There are, however,
reasons to believe that urbanicity can cause misanthropy. Size, density, and
heterogeneity are theoretically linked to many negative emotions
\citep{wirth38}, and make general dislike for humankind likely. Homophily and
evolutionary arguments discussed earlier also support this reasoning. {Furthermore, there is neurological and experimental evidence that city living is unhealthy to the human brain \citep{lederbogen11} and causes lower trust \citep{milgram70}.}

Reverse causality would not make sense: misanthropy or distrust/dislike of people, should
not lead someone to live in close proximity to many people, in a city. This rationale should also exclude self-selection---if anything, people who
 love to be among  people, not misanthropes, would choose to move to and/or stay in cities. Perhaps, this reasoning can explain the results showing that while misanthropy is high in the largest
cities, it is also high in the countryside. Arguably, many people tired of urban crowds move to the countryside \citep[e.g.,][]{deweyWP17nov23}.
%
On the other hand, a potential reason for a misanthrope, or any  non-conformist type, to live in a city (or wilderness; but not in a village or small town), is anonymity.

Can the relationship between urbanicity and misanthropy be spurious? Cities have many problems: notably urban poverty and urban crime which can intensify misanthropy. We cannot control for all urban problems, but we have
controlled for the key urban problem leading to misanthropy: fear of crime, and we also accounted for poverty by controlling for family income. 
 %
Still, would there be urban misanthropy if there were no urban problems? Should we expect misanthropy in a city with low crime rates, low levels of inequality, plentiful affordable amenities, parks, public spaces, and so forth? It's possible that urban areas devoid of urban problems may not experience misanthropy. However, urban misanthropy could still be present even in the absence of urban problems because, at least to some degree, it is the city itself, its core characteristics that can lead to misanthropy: all large cities have large population, moderate-high or high density, and usually moderate or high heterogeneity as compared to smaller
places. Some degree of misanthropy is arguably a natural state of urban life---we concur with Thrift that: ``misanthropy is a natural condition of cities, one which cannot be avoided and will not go away'' \citep{thrift05}.
%

Two apparently important missing variables are measures of discontent and
inequality. However, both inequality \citep[e.g.,][]{daleyMISCNYT20apr14} % \footnote{While inequality is rising fastest in
  % urban areas, it was still higher in rural areas over the period of the study.}
and arguably discontent, especially recently 
\citep[e.g.,][]{case15,hansonCityJournalautumn15,fullerNYT17monD} 
%
% \footnote{One may debate where the level of discontent is higher
%   \citep{florida21}, but much research points to rural areas:
%   \citep[e.g.,][]{case15,hansonCityJournalautumn15,fullerNYT17monD}. Likewise,
%   one may argue that both inequality and discontent are making Americans blame
%   others and therefore become more misanthropic. Again, if anything this should
%   be observed even more in rural areas. And Americans are actually quite
%   resilient to inequality, at least as compared to Europeans
%   \citep{alesina04al}.}
 are higher in rural areas. Therefore, potential left out variable bias actually makes our results conservative---our pooled results would have been stronger, had we controlled for these variables. 
 %
 And our over-time analysis would possibly have indicated a smaller increase (if any) in rural misanthropy, had we controlled for inequality and especially discontent.  
 %
 In addition, Americans are quite resilient to inequality, at least as compared to Europeans \citep{alesina04al}, and hence inequality may not matter much for misanthropy in the US.
 % 
Still, future research should test whether inequality and discontent affect these results. 

Future research should also control for numerous urban amenities (e.g., parks,
public spaces) affecting quality of life in cities, and examine the
urbanity-misanthropy nexus of specific metropolitan areas in the United
States. The GSS public version of the dataset used here does not allow for
identification of municipalities. Another venue for future research is to examine the effect of urbanicity during one's childhood: does urban upbringing affect one's misanthropy later in life? We know that urban upbringing has negative consequences on neural processing and subjective wellbeing (SWB) later in life \citep{lederbogen11,aok20}. 

Why are smaller places becoming more misanthropic? One possible explanation is that rural folks and smaller places are being left behind \citep{fullerNYT17monD,hansonCityJournalautumn15,aok-misanthropy-trustCity,aok-swbGenYcity18,aokCityBook15}---rural areas are economically disadvantaged \citep{glaeser11,osullivan09,florida21}---economic and educational opportunities, as well as other social benefits seem to abound in cities as previously discussed. It's possible that rural resentment could lead to increasing rural misanthropy, which we observed in this study, particularly as rural folks feel that they are being governed by an urbanized elite \citep{wuthnow18,fullerNYT17monD}. % As stated by a Californian farmer \citep[][p. 2]{fullerNYT17monD}, 
  % ``They've devastated ag jobs, timber jobs, mining jobs with their environmental regulations, so yes, we have a harder time sustaining the economy, and therefore there's more people that are in a poorer situation.''
  
%\citet{smith97} argued that the more subordinate a group is, and the more isolated the members of the group are, the greater the misanthropy; and that urbanicity has no direct impact on negativism.  %p12,13
%We disagree: while cities have never been subordinate, but always dominating \citep[e.g.,][]{aok-sizeFetish17},
%\footnote{In some specific cases this is not   true---there are always exceptions to any social scientific rule. For instance, after the urban white flight and before the recent urban renaissance, at least in some ways, suburbs were dominating \citep[e.g.,][]{adams14}.} 
% there are multiple theoretical reasons to believe that cities in fact do increase negativism---for a recent review see \citet{aokCityBook15}. 
%Hence, our conclusions are congruent to those of \citet{schilke15} with respect to trust---misanthropy can be higher in dominating places. Yet, at the same
% time, rural America has clearly increasingly become subordinated, and this is perhaps another reason why misanthropy is growing there.
%\footnote{We speculate that the main reason is that rural areas have been left behind economically and socially, with very little opportunities, investment and development---being left behind is not necessarily the same as being subordinated, yet, this is perhaps another reason why misanthropy is growing there.}  

{
\citet{smith97} argued that the more subordinate a group is, and the more
isolated the members of the group are, the greater the level of misanthropy. % ; and that urbanicity has no direct impact on negativism.  %p12,13
% We disagree: while cities (or suburbs \citep[e.g.,][]{adams14}) have never been
% subordinate, but always dominating
% \citep[e.g.,][]{aok-sizeFetish17,aokCityBook15}, there are multiple theoretical
% reasons to believe that cities in fact do increase negativism---for a recent
% review see \citet{aokCityBook15}.  
%  %   
%  Hence, our conclusions are congruent to those of \citet{schilke15} with respect to trust---misanthropy can be higher in dominating places. Yet, at the same
%  time, rural America has clearly increasingly become subordinated, and this is one reason why misanthropy is growing there.\footnote{We speculate that the main reason is that rural areas have been left behind \citep{hansonCityJournalautumn15,hansonCJ17winter17,fullerNYT17monD}---being left behind is not necessarily the same as being subordinated.}  
 This could help explain rural misanthropy.  Although, the rural resentment may be more against cities or urbanites, rather than people in general.\footnote{We thank an anonymous reviewer for this point. More discussion is available in the Supplementary Online Material (SOM).}
  More research is needed to better understand this phenomenon in rural areas. % As a sidenote, our results confirm the findings of research examining subjective wellbeing (SWB) in cities---rural folks have also always been at an advantage when it comes to SWB (at least since the U.S. GSS
% started collecting data in 1972), but very recently this advantage has disappeared \citep{aok-swbGenYcity18}. We interpret this as evidence of a rural-urban divide and the fact that rural areas have been left behind.
}

\section*{Takeaway for Policy and Practice}

It is undeniable that there are multiple economic, environmental, and social
advantages to cities. 
 Cities are largely necessary, and so is perhaps urban misanthropy---to survive and function in a city. This echoes Simmel's blase attitude comment when describing an urbanite---in order to survive and function in a city, one must withdraw \citep{simmel03}. Or as put commonsensically by Charlie Brooker:
 ``I live in London [...] Misanthropy's the only thing that gets you
  through it. It's not a personality flaw, it's a skill.''
 Neurological \citep{lederbogen11} and experimental \citep{milgram70} evidence  confirms Simmel's observations. 
 There are serious disadvantages to urban life, and they should be taken into account by planners and practitioners. 
 
 More consideration should be given to smaller areas that have been left behind, as lamented by some
\citep[e.g.,][]{fullerNYT17monD,hansonCityJournalautumn15}, but not heard by
most.
 %
 An alarming emergency is the so called ``deaths of despair''---Americans killing themselves out of despair---and the problem is more rural than urban or suburban \citep{case15,case20}.
 % https://jamanetwork.com/journals/jamanetworkopen/fullarticle/2782212
 % https://www.wfae.org/health/2021-08-31/in-places-like-rural-nc-deaths-of-despair-and-education-level-lead-to-decline-in-life-expectancy
 % https://www.americancommunities.org/chapter/american-communities-experience-deaths-of-despair-at-uneven-rates/
 % https://www.upi.com/Health_News/2021/06/09/rural-areas-death-rates-up-study/3661623187038/
Denying resources to smaller places should be given more thought and consideration.
 
Although heterogeneity can contribute to misanthropy in cities, if mechanisms
are in place to facilitate dialogue across different groups and if people are
encouraged to interact with each other, that is, if the ``melting pot'' really
happens, and the ``other'' becomes a fellow human being, then diversity can
yield important social and economic benefits \citep{rodriguez2019does}.  
% In places where it is not possible to build dialogue between different groups of people, where connection and meaningful exchange does not occur, and groups and communities remain in their own spaces, living side by side and yet miles apart, misanthropy can thrive and undermine any social and economic benefits from a diverse environment \citep{rodriguez2019does}. 
Thus, there is a case to be made in favor of more recreational opportunities and events, community services, and social spaces in the largest cities to promote social connections and create a sense of community.  Future research should determine whether these recommendations can curtail misanthropy in
cities. Auxiliary evidence already exists---distrust and dislike are largely about strangers and outgroups \citep{wilson85, delhey11}, and interventions can turn outgroups into ingroups, e.g., a new group such as a sports team can be formed to turn strangers into an ingroup  \citep[e.g.,][]{smith10}.
 
Misanthropy may not seem tangible or meaningful for urban planners and practitioners at a first glance.  When consideration is given to how misanthropy can cause negative outcomes, however, there are reasons to be concerned. Misanthropy reduces people's desire to invest and to be involved in their communities and may remove social bonds that deter people from harming others \citep{weaver2006,hirschi1993,fafchamps2006,walters2013}. Furthermore, misanthropy is correlated with dysfunctional and animus behaviors such as
 homophobia, sexism, racism, and ageism \citep{cattacin2006}. Overall, misanthropy can arguably contribute to isolation and loneliness---urban problems with serious consequences that city planners have to grapple with. 
 
Given our findings, it is impossible to overlook the current COVID-19 pandemic effects---large cities in general experience the worst infectious diseases spread \citep{bettencourt10}. This health crisis will arguably further exacerbate misanthropy in the largest metropolitan areas, as fear and suspicion of the `other' increases---many people have fled New York City, for example, to stay away from other people. The avoidance and distrust of `others' due to fear of infection, particularly in the largest and densest cities, may have intensified misanthropy and should be considered as well. 


% The takeaway for policy and practice is that misanthropy should be of concern as
%it leads to tangible consequences---dissolution of the social fabric and
%dysfunction. While some degree of misanthropy may be inherent to urbanism
% \citep{thrift05}, some of it may be arguably mitigated by policies designed to bring people together. At the same time, planners and practitioners must start paying attention to rural areas, which have been largely left behind with little resources---misanthropy has been growing there most steeply. 

%This study focuses solely on the U.S. and the results and takeaways for practice may
%not be generalized to other countries. 
%There is a reason to believe that future research in other developed countries
%will find similar results, especially in Western countries where people are
%unhappier in the largest metropolitan areas, and therefore more likely to be
%misanthropic \citep{aokCityBook15}. 
%In developing countries, however, cities may not be more misanthropic for one simple
%reason---life is simply often unbearable outside of the city, without necessities such as access to healthcare and basic consumer goods. Misanthropy is arguably less likely if cities, and only cities, provide basic needs. This is, however, an speculation and cross-country research is needed.

\bibliography{trustCity,/home/aok/papers/root/tex/ebib}


\clearpage

\section*{\LARGE SOM-R (Supplementary Online Material-for Review)}

The literature exploring the nexus between urbanicity and misanthropy is relatively small as discussed in the paper. In the following subsections we provide more detail of this relationship by exploring how scholars from different fields have portrayed cities and discussed urban misanthropy. 

\subsection{Amin's and Thrift's insights motivating the present study}

Our study is largely inspired by \citet{amin06} and \citet{thrift05}, whose sharp observation of
the urban way of life suggest the existence of urban misanthropy: 

\begin{quote}
  cities are polluted,
  unhealthy, tiring, overwhelming, confusing, alienating. They are places of
  low-wage work, insecurity, poor living conditions and dejected isolation for
  the many at the bottom of the social ladder daily sucked into them. They hum
  with the fear and anxiety linked to crime, helplessness and the close
  juxtaposition of strangers. They symbolize the isolation of people trapped in
  ghettos, segregated areas and distant dormitories, and they express the
  frustration and ill-temper of those locked into long hours of work or travel \citep[][p. 1011]{amin06}.
\end{quote}
\begin{quote}
%  \textit{The misanthropic city}\\
%  Cities bring people and things together in manifold combinations. Indeed, that is probably the most basic
%definition of a city that is possible. But it is not the case that these combinations sit comfortably with one
%another. Indeed, they often sit very uncomfortably together. 
 Many key urban experiences are the result of
juxtapositions which are, in some sense, dysfunctional, which jar and scrape and
rend. [...]  % What do surveys
% show contemporary urban dwellers are most concerned by in cities? Why crime, noisy neighbors, a whole
% raft of intrusions by unwelcome others.
There is, in other words, a {misanthropic} thread that runs through
the modern city, a distrust and avoidance of precisely the others that many writers feel we ought to be
welcoming in a world increasingly premised on the mixing which the city first
brought into existence \citep[][p. 140]{thrift05}.
\end{quote}


\subsection{Auxiliary Writings On Urbanism and Misanthropy}

{\small\it \noindent ``Here is the great city: here have you nothing to seek and
  everything to lose.''} Nietzsche\\

Steve Pile in his colorful writings about cities often invokes
urban folklore characters that prey on humans in cities, e.g., vampires, werewolves, ghosts  \citep{pile05,pile05B,pile99}.
%
Specifically, old cities carry melancholia \citep{pile05B}, which can arguably translate into misanthropy.
%

Nietzsche, one of the greatest observers of the human condition suggested urban
misanthropy by referring % , expressed misanthropic views
% himself \citep[e.g.,][]{avramenko2004zarathustra} % \footnote{He expressed dislike for the masses in the city and accordingly left the more densely populated areas for solitude in the mountains. See for example \citep{nietzsche05}.}
%  and made a powerful analogy using one the most iconic and crowded places in a city, the marketplace, while referring
 to urbanites as ``the flies in the market-place'' \citep{nietzsche05}. He expressed dislike for the masses in the city and expressed misanthropic views himself; accordingly, he left the more densely populated areas for solitude in the mountains. See for example \citep{nietzsche05}.
%  and made a powerful analogy using one the most iconic and crowded places in a city, the marketplace
 % Another philosoper, Giacomo Leopardi,  argued along similar ``Real misanthropes are not found in solitude, but in the world; since it is experience of life, and not philosophy, which produces real hatred of mankind.'' 


\subsection{Engels' Description of Industrial City}   

\citet{gibson17} offered a misanthropic interpretation of urbanism saying: ``Houellebecq matches this vision of hell with an insistent evocation of the anomic urban and metropolitan cityscapes (p. 220),'' and previously on page 153 said:

\begin{quote}
  Sitwell's city is the citta infernale [hell city], and the city is where one confronts essential truth; nature, by contrast, is incidental, exists as nooks and byways. In the urban `circles of hell,' Sitwell writes, all the forms of misery congregate together.  Here one learns all one needs about the `old tyrannies and cruelties,' `the rankness of all human nature,' `this muddle and waste that we have made of the world.' Cities are places where `men have created and known fear' as a consequence of `the man-made chasms' between them.
\end{quote}

Such description of urbanism reminds of Engels' classic vivid description of the industrial city:

\begin{quote}
  In a rather deep hole, in a curve of the Medlock and surrounded on all four
  sides by tall factories and high embankments, covered with buildings, stand
  two groups of about two hundred cottages, built chiefly back to back, in which
  live about four thousand human beings, most of them Irish. The cottages are
  old, dirty, and of the smallest sort, the streets uneven, fallen into ruts and
  in part without drains or pavement; masses of refuse, offal and sickening
  filth lie among standing pools in all directions; the atmosphere is poisoned
  by the effluvia from these, and laden and darkened by the smoke of a dozen
  tall factory chimneys. A horde of ragged women and children swarm about here,
  as filthy as the swine that thrive upon the garbage heaps and in the
  puddles. In short, the whole rookery furnishes such a hateful and repulsive
  spectacle as can hardly be equalled in the worst court on the Irk. The race
  that lives in these ruinous cottages, behind broken windows, mended with
  oilskin, sprung doors, and rotten doorposts, or in dark, wet cellars, in
  measureless filth and stench, in this atmosphere penned in as if with a
  purpose, this race must really have reached the lowest stage of humanity.
\end{quote}

This quote is from
 \url{https://www.marxists.org/archive/marx/works/1845/condition-working-class/ch04.htm},
 where there is even more elaboration and description.
  
This quote is particularly relevant since urbanization started with the industrial revolution. Arguably, the main rationale for urbanism has been capitalistic and economic \citep{osullivan09,glaeser11}. See for instance \citet{harvey12,aokCityBook15,molotch76}.

\subsection{Homophobia and Transphobia As Misanthropy}

As previously discussed, a notable advantage of cities is that they are more welcoming and tolerant than other places \citep{park84,tuch87,wirth38,stephan82,aok20}. Manifestations of homophobia and transphobia are arguably not an opinion, but rather an expression of misanthropy \citep{lehmannMISC22jun6}. 
It is somewhat surprising, or paradoxical, that misanthropy is higher in cities as
argued here.  What reconciles this apparent conflict is that as LGBTQ awareness increased over the four decades studied here, the smallest places have become more misanthropic. Another point of consideration is that misanthropy might be conditional (e.g., someone might hate or dislike just a specific sect of the population) on the existing biases of people dwelling in cities, thus future research should consider how misanthropy varies across different groups of people taking an intersectional approach. 

\subsection{Google Scholar's citation counts of  \citet{wilson85}}

Remarkably, according to Google Scholar, \citet{wilson85} is only cited by four studies thus far
 % (\url{https://scholar.google.com/scholar?cites=5520865883768238779&as_sdt=5,31&sciodt=0,31&hl=en})
---\citet{smith97} and 3 others---and none of these studies focus on
misanthropy. Thus, aside from \citet{wilson85}, there is simply no literature on
this topic. Given this thirty seven year gap in the literature, the present study is pioneering ground breaking research in the current generation of urban scholarship.

\section{GSS Codebook Descriptions of Urbanicity Measures.}   

\textsc{Size}. This code is the population to the nearest 1,000 of the smallest civil
division listed by the US Census (city, town, other incorporated
area over 1,000 in population, township, division, etc.) which
encompasses the segment. If a segment falls into more than one
locality, the following rules apply in determining the locality for
which the rounded population figure is coded.
If the predominance of the listings for any segment are in one of the
localities, the rounded population of that locality is coded.
If the listings are distributed equally over localities in the
segment, and the localities are all cities, towns, or villages, the
rounded population of the larger city or town is coded. The same is
true if the localities are all rural townships or divisions.
If the listings are distributed equally over localities in the segment
and the localities include a town or village and a rural township or
division, the rounded population of the town or village is coded.

\textsc{Xnorcsiz}. Expanded N.O.R.C. size code. 
a. A suburb is defined as any incorporated area or unincorporated area
of 1,000+ (or listed as such in the US Census PC (1)-A books) within
the boundaries of an SMSA but not within the limits of a central city
of the SMSA. Some SMSAs have more than one central city, e.g.,
Minneapolis-St. Paul. In these cases, both cities are coded as central
cities.
b. If such an instance were to arise, a city of 50,000 or over which is
not part of an SMSA would be coded `7'.
c. Unincorporated areas of over 2,499 are treated as incorporated areas
of the same size. Unincorporated areas under 1,000 are not listed by
the Census and are treated here as part of the next larger civil
division, usually the township.

\textsc{Srcbelt}. SRC beltcode. The SRC belt code (a coding system originally devised to describe
rings around a metropolitan area and to categorize places by size
and type simultaneously) first appeared in an article written by
Bernard Laserwitz (American Sociological Review, v. 25, no. 2, 1960),
and has been used subsequently in several SRC surveys.
Its use was discontinued in 1971 because of difficulties particularly
evident in the operationalization of ``adjacent and outlying areas.''
For this study, however, we have revised the SRC belt code for users
who might find such a variable useful. The new SRC belt code utilizes
``name of place'' information contained in the sampling units
of the NORC Field Department.
    
\section{Variable Definitions. Descriptive Statistics, and Additional Results.}    
    
Below we show the variable definitions, basic descriptive statistics, and additional regression results.
%Adam, please fix the tables, you need to use `` instead of '' at the begining
%of the quotes, e.g, ``What is your religious preference?'' Also, for family
%income, I think the comma should be next to INCOME91,
%rubia they'll do it at the typsetting stage
{\footnotesize
\input{varDes1.tex}
\input{varDes2.tex}
}
\input{gss_h0.tex} 
\input{gss_h1.tex} 
\input{gss_h2.tex} 
\input{gss_h3.tex} 

\clearpage
In the manuscript, we have plotted results from the simple specification Model a3a
from Table \ref{regDbyHand}, but note that more elaborate specifications with
more variables and a dummy variable for time are similar.

 \begin{spacing}{.67}
\begin{table}[H]\centering
\caption{OLS regressions  of misanthropy. Beta (fully standardized) coefficients
  reported. All models include year dummies.} \label{regDbyHand}
\begin{tiny} \begin{tabular}{p{1.2in}p{.45in}p{.45in}p{.45in}p{.45in}p{.45in}p{.45in}p{.45in}p{.45in}p{.45in}p{.45 in}}\hline
                    &          a1   &          a2   &          a3   &          a4   &          a5   \\
post pandemic            &       -0.20** &       -0.13+  &       -0.10   &       -0.02   &       -0.18*  \\
city lg500k&        0.05   &        0.19*  &        0.20*  &        0.11   &        0.07   \\
post pandemic $\times$ city lg500k&       -0.26*  &       -0.26*  &       -0.26*  &       -0.21+  &       -0.15   \\
United Kingdom      &       -0.04   &        0.03   &        0.08   &       -0.01   &       -0.04   \\
Uruguay             &        0.82***&        0.92***&        0.95***&        0.68***&        0.43***\\
2011                &       -0.82***&       -0.72***&       -0.54***&       -0.47***&       -0.44***\\
2012                &       -0.10   &        0.15+  &        0.11   &        0.02   &        0.05   \\
income              &               &        0.14***&        0.13***&        0.08***&        0.08***\\
age                 &               &       -0.05***&       -0.04***&       -0.03***&       -0.03***\\
age2                &               &        0.00***&        0.00***&        0.00***&        0.00***\\
male                &               &       -0.16***&       -0.17***&       -0.16***&       -0.11** \\
married or living together as married&               &        0.46***&        0.46***&        0.39***&        0.44***\\
divorced/separated/widowed&               &        0.01   &        0.01   &       -0.03   &       -0.07   \\
god important       &               &               &        0.03***&        0.03***&        0.02***\\
trust               &               &               &        0.38***&        0.25***&        0.26***\\
postmaterialist     &               &               &       -0.04   &       -0.05+  &       -0.04   \\
autonomy            &               &               &       -0.10***&       -0.10***&       -0.09***\\
health              &               &               &               &        0.71***&               \\
freedom             &               &               &               &               &        0.40***\\
constant            &        7.58***&        7.42***&        7.14***&        4.40***&        4.47***\\
N                   &        9196   &        7746   &        6038   &        6032   &        5970   \\
 %TODO order nicely by hand:regDbyHand.tex
 \hline  *** p$<$0.01, ** p$<$0.05, * p$<$0.1; robust std err
\end{tabular}\end{tiny}\end{table}
 \end{spacing}

 
In Table \ref{regE} the results show that while whites are in general less misanthropic
than minorities, they are more misanthropic in larger places, thus confirming
\citet{wilson85}. Note, the column names correspond with earlier tables.  
 In Model a4c1 we interact urbanicity with the white household dummy---indeed we find confirmation for \citet{wilson85}---clearly whites experience more misanthropy in urban areas. \citet{wilson85} explains this
 pattern using Fischer's sub-cultural theory.

 \begin{spacing}{.67}
   \begin{table}[H]\centering
     \caption{OLS regressions  of misanthropy. All models include year
       dummies. Size deciles (base: $<$2k). Srcbelt (base: small rur). Xnorcsiz (base: $<$2.5k, but not countryside).} \label{regE}
     \begin{tiny} \begin{tabular}{p{1.2in}p{.45in}p{.45in}p{.45in}p{.45in}p{.45in}p{.45in}p{.45in}p{.45in}p{.45in}p{.45 in}}\hline
         \input{regE.tex}
         \hline  *** p$<$0.01, ** p$<$0.05, * p$<$0.1; robust std err
       \end{tabular}\end{tiny}\end{table}
 \end{spacing}



 
\end{spacing}
\end{document}

