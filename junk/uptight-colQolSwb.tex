%to have line numbers
%\RequirePackage{lineno}
\documentclass[10pt, letterpaper]{article}      
\usepackage[margin=.1cm,font=small,labelfont=bf]{caption}[2007/03/09]
%\usepackage{endnotes}
%\let\footnote=\endnote
\usepackage{setspace}
\usepackage{longtable}                        
\usepackage{anysize}                          
\usepackage{natbib}                           
%\bibpunct{(}{)}{,}{a}{,}{,}                   
\bibpunct{(}{)}{,}{a}{}{,}                   
\usepackage{amsmath}
\usepackage[% draft,
pdftex]{graphicx} %draft is a way to exclude figures                
\usepackage{epstopdf}
\usepackage{hyperref}                             % For creating hyperlinks in cross references
\hypersetup{pdfborder={0 0 0.4}} %have nice light boxes on refs

% \usepackage[margins]{trackchanges}

% \note[editor]{The note}
% \annote[editor]{Text to annotate}{The note}
%    \add[editor]{Text to add}
% \remove[editor]{Text to remove}
% \change[editor]{Text to remove}{Text to add}

%TODO make it more standard before submission: \marginsize{2cm}{2cm}{1cm}{1cm}
\marginsize{1cm}{1cm}{.5cm}{.5cm}%{left}{right}{top}{bottom}   
			 		          % Helps LaTeX put figures where YOU want
 \renewcommand{\topfraction}{1}	                  % 90% of page top can be a float
 \renewcommand{\bottomfraction}{1}	          % 90% of page bottom can be a float
 \renewcommand{\textfraction}{0.0}	          % only 10% of page must to be text

 \usepackage{float}                               %latex will not complain to include float after float

\usepackage[table]{xcolor}                        %for table shading
\definecolor{gray90}{gray}{0.90}
\definecolor{orange}{RGB}{255,128,0}

\renewcommand\arraystretch{.9}                    %for spacing of arrays like tabular

%-------------------- my commands -----------------------------------------
\newenvironment{ig}[1]{
\begin{center}
 %\includegraphics[height=5.0in]{#1} 
 \includegraphics[height=3.3in]{#1} 
\end{center}}

 \newcommand{\cc}[1]{
\hspace{-.13in}$\bullet$\marginpar{\begin{spacing}{.6}\begin{footnotesize}\color{blue}{#1}\end{footnotesize}\end{spacing}}
\hspace{-.13in} }

%-------------------- END my commands -----------------------------------------

\usepackage{pdfpages}


%-------------------- extra options -----------------------------------------

%%%%%%%%%%%%%
% footnotes %
%%%%%%%%%%%%%

%\long\def\symbolfootnote[#1]#2{\begingroup% %these can be used to make footnote  nonnumeric asterick, dagger etc
%\def\thefootnote{\fnsymbol{footnote}}\footnote[#1]{#2}\endgroup}	%see: http://help-csli.stanford.edu/tex/latex-footnotes.shtml

%%%%%%%%%%%
% spacing %
%%%%%%%%%%%

% \abovecaptionskip: space above caption
% \belowcaptionskip: space below caption
%\oddsidemargin 0cm
%\evensidemargin 0cm

%%%%%%%%%
% style %
%%%%%%%%%

%\pagestyle{myheadings}         % Option to put page headers
                               % Needed \documentclass[a4paper,twoside]{article}
%\markboth{{\small\it Politics and Life Satisfaction }}
%{{\small\it Adam Okulicz-Kozaryn} }

%\headsep 1.5cm
% \pagestyle{empty}			% no page numbers
% \parindent  15.mm			% indent paragraph by this much
% \parskip     2.mm			% space between paragraphs
% \mathindent 20.mm			% indent math equations by this much

%%%%%%%%%%%%%%%%%%
% extra packages %
%%%%%%%%%%%%%%%%%%

\usepackage{datetime}


\usepackage[latin1]{inputenc}
\usepackage{tikz}
\usetikzlibrary{shapes,arrows,backgrounds}


%\usepackage{color}					% For creating coloured text and background
%\usepackage{float}
\usepackage{subfig}                                     % for combined figures

\renewcommand{\ss}[1]{{\colorbox{blue}{\bf \color{white}{#1}}}}
\newcommand{\ee}[1]{\endnote{\vspace{-.10in}\begin{spacing}{1.0}{\normalsize #1}\end{spacing}\vspace{.20in}}}
\newcommand{\emd}[1]{\ExecuteMetaData[/tmp/tex]{#1}} % grab numbers  from stata

%TODO before submitting comment this out to get 'regular fornt'
\usepackage{sectsty}
\allsectionsfont{\normalfont\sffamily}
\usepackage{sectsty}
\allsectionsfont{\normalfont\sffamily}
\renewcommand\familydefault{\sfdefault}

\usepackage[margins]{trackchanges}
\usepackage{rotating}
\usepackage{catchfilebetweentags}

\usepackage{abstract}
\renewcommand{\abstractname}{}    % clear the title
\renewcommand{\absnamepos}{empty} % originally center

\RequirePackage{etex}
%-------------------- END extra options -----------------------------------------
\date{{}\today \hspace{.2in}\xxivtime}
\title{Colombia: Unlivable but Happy.} %\\Quality of Life
  %(QOL) and Subjective Wellbeing (SWB) 2000-2020   % remember to have Vistula University!!
  
\author{
% Adam Okulicz-Kozaryn\thanks{EMAIL: adam.okulicz.kozaryn@gmail.com
%   \hfill I thank XXX.  All mistakes are mine.} \\
% {\small Rutgers - Camden  % and Vistula University
% }
}

\begin{document}

%%\setpagewiselinenumbers
%\modulolinenumbers[1]
%\linenumbers

%\bibliographystyle{/home/aok/papers/root/tex/ecta}
\maketitle
\vspace{-.4in}
\begin{center}

\end{center}


\textbf{abstract:}
\begin{abstract}
%   \noindent%There is a Latin American phenomenon: 
%   %Latinos are highly satisfied with their lives despite poor livability of the environment. % such as underdevelopment, social conflict, and poor policy. 
%   % 
%   This mostly theory %conceptual  % theoretical
%   and review article studies apparently unlivable but happy Colombia, ``fool's
%   paradise'', in contrast to apparently  livable but quite unhappy United
%   States, ``fool's hell.''
%  % (dangerous, poor, unequal, etc)
%  % Colombians are happy despite
% % apparently unlivable Colombia--
% % and
% % point to directions for future research. 
% %
% % 
% % Colombia is % the happiest country in the world or
% % one of the several very happiest countries,
% % % depending on measurment. 
% %  and at the same time apparently unlivable--
%  %
%   % of a society measured
%  % usually expressed mostly
%  %
%  We argue that Colombia is not a fool's paradise--genuine
% happiness is possible in an apparently unlivable environment, because
%  livability tends to be measured mostly in terms of material comfort, and 
%   already very basic material comfort is good enough to satisfy human needs and
%   produce happiness. % ; 2)
%  % non-commodities such as %subjective/positive freedom
%  %  personal freedom and social connection not only matter, but also are 
%  %  hampered by excessive pursuit of commodities. % --more of one less of the other
%  %  %
%   %
%   %
%   % Impeccable organization and physical infrastructure such as that
% % in Singapore or richest parts of the US
% % richest parts of the US or Germany or Hong Kong
% % Counterintuitevely, material comfort may actualy produce unhappiness. 
%   On the contrary, it is the excessive pursuit of material comfort,
%   such as that in the US, that arguably actually decreases happiness by: 1) having to focus on what's unimportant, overwork, and alienate oneself;  
% and 2) making an environment and interaction inhuman, sanitized,
% hospital/airport-like, and alienating. 
% %
% %Arguably, the solution of the apparent paradox of unlivability but happiness is that the good outweights the
% %bad--Colombia is very happy because it is actualy quite livable. 
%  %
% %This paper argues that Colombia is one of the very best countries to
% %visit, and it may even be one of best places to live.
% % The world has much to learn from Colombia. % how to be happy
% %--Colombia is a real paradise. 
% %
%  The main limitation is that we cannot fully rule out that Colombia is a fool's paradise% , perhaps ignorance is a bliss
% . We discuss alternative explanations and provide directions for future research.
%  %
% 
% lina uptight ornamented abs:
This theoretical and review-based article examines the paradox of Colombia--a country often perceived as unlivable, yet frequently reporting relatively high levels of subjective well-being--compared to the United States, a nation widely regarded as highly livable but with comparatively lower happiness indicators. We argue that genuine happiness is possible even in contexts with limited material comfort, as conventional measures of livability often prioritize material conditions that may not be required for well-being once basic needs are met. Conversely, the excessive pursuit of material comfort--as exemplified in the US--may diminish happiness by fostering overwork, alienation, and depersonalized, sanitized environments. While we cannot entirely rule out the possibility that Colombia's reported happiness reflects adaptive responses to adversity, we critically examine alternative explanations and propose directions for future research.
  
\end{abstract}
\vspace{.15in} 
\noindent{\sc Life Satisfaction, Happiness, Subjective Well-Being, Quality of
  Life, Livability, Best Places to Live, Colombia, Alienation, % Marx,
  Degrowth
}
\vspace{.25in} 

\begin{spacing}{2} %TODO MAYBE before submission can make it like 2.0
\rowcolors{1}{white}{gray90}

%  instead \ExecuteMetaData[../out/tex]{ginipov} do \emd{ginipov}





% This theoretical/review paper explores one of the great happiness puzzles--Colombia, appraently one of the least
% livable countries is one of the happiest countries. 
% We start by defining terms, then provide brief theoretical background, and move to the core of the paper: the paradox: happy colombia;
% but unlivable; and key sec: reconciling evidence: fools paradise?; we fisinh with
%  summary/conclusion. 


%lina
% During the last few decades, academics, governments, and multilateral organizations have produced significant evidence to
% conclude that 
 Traditional
progress/development 
 measures like
income, production, and consumption cannot fully account for people's life
experiences % , happiness,
 or emotions \citep{diener09,stiglitz09al,lambert2020towards}.
% % To bridge this
% % gap, the measurement of
%  Subjective Wellbeing (SWB) has become a new relevant
%  metric % component of policymaking and academic research,
% % informing new directions of policy interventions and promoting broad debates about a better life
% \citep{diener09}.
% 
% Most of the research and broad debates about SWB concern developed countries
% (aka WEIRD: Western, Educated, Industrialized, Rich and Democratic \citep{henrich10}). % They also reflect the spirit of policies in countries where
% % problems like widespread violence, rampant corruption, or abject poverty are not the primary policy concern \citep{van2020s}. The dominance of
% % developed countries in the perspective and practices of wellbeing research and
% % policymaking limit the extrapolation of theories, methodologies, and policy recommendations
% % to developing countries. 
%  %aok
%  Worse yet,
 While acknowledged in theory as inadequate%  that economic measures poorly capture
 % overall progress/development
, the predominant focus in world development is
 still on economy and material comfort. The economic and material comfort
 champion, the US, offers progress/development lessons to developing countries.\footnote{With recent backlash from the BRICS (Brazil, Russia,
   India, China, South Africa, Iran, Egypt, Ethiopia, and the United Arab
   Emirates) and others, the de-dollarization and trade re-orientation away from
   the West, and so forth. Still, the West, and especially the US typically sees itself as
 the righteous world leader with the lessons to teach to others--the point well
 made by world development veteran Jeffrey Sachs, see for instance: \url{https://www.youtube.com/watch?v=rgMPIhBLp1I}.} 
 At the same time, it is typically overlooked that there
 are wellbeing lessons from the developing countries. Notably Latin American
 countries such as Colombia tend to be happier than many developed countries. %What can we learn from Colombians? 

 Our study follows the classic happiness theorizing by Veenhoven
(\citeyear{veenhoven00b,veenhoven95,veenhoven14b}) and Michalos
(\citeyear{michalos14B})% , and Cummins \citep{sirgy02}
; with Latin focus following \citet{rojas15} and \citet{yamamoto16}.
 Further, we complement this traditional line of inquiry with rarely used
perspectives in happiness research: folklore theory and
Marx's alienation theory.   
 %
 % to understand conflicting arguments
 We conclude that happiness and material underdevelopment can coexist, and even
 that underdevelopment may promote happiness in some ways.  
% We frame our analysis in Latin America,
% zooming on Colombia  to explore how personal life satisfaction
% can co-exist in an apparently unlivable environment.\footnote{The analysis is situated before the
% pandemic. COVID-19 has significantly increased poverty in the region \citep{wb22}, which deserves an analysis on its
% own.}

 
% The objective here is not only to document the the Latin American phenomenon,
% an apparent contradiction of low livability and high happiness. We notably propose 
% that the apparent contradiction can be resolved at least to some degree if we
% measure livability differently: assume only basic material comfort is necessary and
% non-material aspects such as personal freedom and social connection are given more weight than usual
% in the West. We cover a broad spectrum of issues including alienation to provide
% a new perspective and spark future research. 

The article is structured as follows. We start by documenting the Latin American
phenomenon 
\citep{rojas15,rojas2019well} in Colombia: Colombians (like other Latinos) are
satisfied with life (sec. 1) despite low livability (high poverty, insecurity,
inequality, etc) (sec. 2). Having established an apparent paradox of high happiness despite low
livability we move to the theoretical framework to understand it (sec. 3):  Veenhoven's 4 qualities of life
(\citeyear{veenhoven00b}) are used to map happiness to livability, and Michalos 2 variable
theory \citep{michalos14B} to describe the apparent contradiction as ``fool's paradise,'' which is then explored with livability (sec. 3.1), folklore
(sec. 3.2), and Marx's alienation (sec. 3.3) theories. As a complement to fool's paradise in Colombia % , and a look
% at the opposite side
 we turn to
fool's hell (very livable but only moderately happy) in Singapore (sec. 3.4) and
the US (sec. 3.5). We finish with discussion and conclusion (sec. 4).
% ,alternative explanations and counterarguments (sec. 4.1) and takeaway for policy
% and practice and future research (sec. 4.2.). 
 %
 Subjective Wellbeing (SWB)/happiness is defined in first paragraph of section
``1 Happy Colombia.''
Livability is defined in first paragraph of section ``2 Unlivable Colombia.''



\section{Happy Colombia}

Subjective Wellbeing (SWB), happiness, wellbeing, and life satisfaction are used 
interchangeably. Technically, we mean evaluative/cognitive life satisfaction as
opposed to affective happiness, positive and negative affects/emotions, or
Eudaimonia/flourishing/functioning. Evaluative/cognitive life satisfaction is measured in this section, the only section that uses happiness
data% in table \ref{t1}
.    
 For elaboration and auxiliary points see online appendix.

A revered (or despised) US politician, Newt Gingrich, aptly observes that the US Declaration of Independence doesn't
guarantee happiness, neither it binds the government to produce it, but 
merely guarantees its pursuit (\url{https://www.youtube.com/watch?v=PWA4EmeaLgc}). Accordingly, some Americans succeed and some fail
this pursuit, and the US as a country is below the top quartile
of world happiness rankings (46/160 in World Database of Happiness; table \ref{t1}).

Colombia's constitution, on the other hand, states that happiness is the
government's business--government should work to increase it: ``The general wellbeing and
improvement of the population's quality of life are social purposes of the
State.'' (art 366, \url{https://www.constituteproject.org/constitution/Colombia_2015}). 
 And Colombians are indeed very happy, happier than Americans, but
arguably not because of the Colombian government, rather despite it--in multiple ways Colombian government and outcomes it produces
are a failure--see next section: ``Unlivable Colombia.''   

Like other Latinos, Colombians are very happy, about 8.5
on 0-10  evaluative/cognitive life satisfaction scale, a score much higher than expected given
economic, social, and institutional indicators \citep{PNUD}. 
 %
Colombia is % the happiest country 
% in the World  \citep[e.g.,][]{roosHUFF19mar26},  or at
% least
one of  the happiest countries--both World Values Surveys and
World Database of Happiness\footnote{We do not
  use Gallup data because Gallup data are designed for commerce, rather than research--Gallup charges \$30,000 (per year) for  access
  (authors' inquiry). It is rather ``happiness industry'' \citep{davies15} than research.
  %
In general, an argument can be made that there is a corporatization of academia,
which has some negative consequences \citep{mills2012corporatization,cox2013corporatization,millsNYT12fa,CatropaNYT20feb8,schmidlinNYT15oct10}.
 %
 % We advocate use of free World Values Survey (WVS) at 
 % \url{https://www.worldvaluessurvey.org}. 
} rank it top 3 in table \ref{t1}.
 A World Values Survey (WVS) SWB item reads: ``All things considered, how
satisfied are you with your life as a whole these days?'' on scale
1='dissatisfied' to 10='satisfied'. World Database of Happiness (WDH) is an
aggregate of various surveys, and SWB survey items are very similar to that in WVS. 
% \footnote{but then it went from the happiest to 66th \url{https://www.infobae.com/en/2022/03/18/even-happiness-was-lost-in-colombia-survey-places-it-among-the-least-happy-countries-in-the-world}.}
% but again research should ignore gallup--and its fishy just like they
% manipulate to show urban is happier bet they manipulate latin america to show
% less happy than stiupid usa wtf:
% https://www.larepublica.co/globoeconomia/los-paises-mas-felices-del-mundo-en-2024-3825512
% https://www.visualcapitalist.com/a-map-of-global-happiness-by-country-in-2024/
  % , a trully outstanding score.
% top 10 countries just costa rica and colombia from latin america, and colombia
% poorer than costa rica;
 Colombia is outstanding at achieving  high happiness at low economic development. 
 %
 Colombia is happier than all other Latin countries in table \ref{t1} and about
as happy as Mexico, but Colombia is significantly poorer than Mexico, at least
25\% poorer either in nominal or Purchasing Power Parity (PPP) terms.\footnote{See 
  \url{https://data.worldbank.org/indicator/NY.GDP.PCAP.CD}
  and
  \url{https://data.worldbank.org/indicator/NY.GDP.PCAP.PP.CD}.
%
%  \url{https://en.wikipedia.org/wiki/List_of_countries_by_GDP_(nominal)_per_capita}
%  and
%  \url{https://en.wikipedia.org/wiki/List_of_countries_by_GDP_(PPP)_per_capita}
}                                 
 %Colombia shines.
 

%as a side note uzbekistan and taijkistan may seem surprosing
\begin{table}[H]\centering\footnotesize
\caption{\label{t1}  10 happiest countries in the world. Data from World
  Database of Happiness (WDH) 2010-2019 out of 160 countries at
 \url{worlddatabaseofhappiness.eur.nl/rank-reports/satisfaction-with-life}; 
 and World Values Surveys (WVS) 2005-2022 [waves 5-7] out of 88 countries at  
  % \url{https://worlddatabaseofhappiness-archive.eur.nl/hap_nat/desc_na_genpublic.php?cntry=122};
 \url{https://www.worldvaluessurvey.org}. Technical information for WDH is at
 \url{https://worlddatabaseofhappiness.eur.nl/reports/finding-reports-on-happiness-in-nations/technical-details-to-rank-reports-of-happiness-in-nations/technical-details-to-rank-report-average-happiness-in-nations-2010-2019}
and WVS documentation is at \url{https://www.worldvaluessurvey.org/WVSDocumentationWV7.jsp}.}
\begin{tabular} {@{} lll|lll @{}}   \hline
 \multicolumn{3}{c}{WDH}&\multicolumn{3}{c}{WVS}\\
  rank& country&  happiness (1-10)&  rank& country&  happiness (1-10) \\ \hline
 1 	&	Denmark	        &	8.2   & 1 	&     Puerto Rico &  8.4   \\                          
 2 	&	Mexico	        &	8.1   & 2 	&          Mexico &  8.3   \\                          
 3 	&	Colombia	&	8.1   & 3 	&         Colombia&  8.3   \\                  
 4 	&	Switzerland	&	8     & 4 	&           Qatar &  8.0   \\                  
 5 	&	Finland	        &	8     & 5 	&          Norway &  7.9   \\                          
 6 	&	Iceland	        &	8     & 6 	&       Nicaragua &  7.9   \\                          
 7 	&	Costa Rica	&	7.9   & 7 	&      Tajikistan &  7.9   \\                  
 8 	&	Norway	        &	7.9   & 8 	&     Switzerland &  7.9   \\                          
 9 	&	Canada	        &	7.9   & 9 	&      Uzbekistan &  7.9   \\                          
 10	&	Qatar         	&	7.8   & 10	&         Ecuador &  7.8   \\\hline                          
 % 11	&	Sweden	        &	7.8      \\                          
 % 12	&	Austria    	&	7.7      \\                          
 % 13	&	Nicaragua	&	7.7      \\                  
 % 14	&	Uzbekistan	&	7.7      \\                  
 % 15	&	Netherlands	&	7.6      \\                  
 % 16	&	Ecuador  	&	7.6      \\                          
 % 17	&	Israel  	&	7.6      \\                          
 % 18	&	Luxembourg	&	7.6      \\                  
 % 19	&	Belgium 	&	7.5      \\                          
 % 20	&	United Arab Emirate&	7.5      \\               
 % 21	&	Bosnia Herzegovina &	7.5      \\               
 % 22	&	Panama  	&	7.5      \\                          
 % 23	&	Brazil  	&	7.4      \\\hline                    
\end{tabular}\end{table}       





%ya typo in wdh--they translated wrong lationo barometero 1-4 or 5 to 1-10 
% In first panel of table \ref{t1} Colombian happiness skyrocketed from about 2.5 in 1997-2000 to about 3.3-3.4 only few years later in 2004 and remained mostly within 3.3-3.4 range ever since. An increase of about 1 on 1-4 SWB scale for a country within just few years if not unprecedented, is clearly an outstanding feat. Likewise, in second panel of table \ref{t1}
%                              
% over time big improvement from around 2.5 in late 90s to around 3.35 in late
% 2010s, 34 percent improvemnet, perhaps can be explained that in 80s and early 90
% colombia experienced much violence, and by late 90s became much safer, so
% Colombians became happier.
%  %
% massive increase in happiness since 90s is arguably because conditions improve
% so much! Even though they are still relatively bad as compared to other
% countries, they are really much better than in 90s. So rather MDT, Colombians
% may be very happy because compare to what it used to be.
% As per livability
% theory it is a paradox, relatively bad conditions as compared to other countries
% and happy.\footnote{Also note that neighboring countries, Ecuador and Panama saw
% similar staggering increases over the same period of time: \url{https://worlddatabaseofhappiness-archive.eur.nl/hap_nat/nat_fp.php?cntry=127&mode=3&subjects=29&publics=5} and \url{https://worlddatabaseofhappiness-archive.eur.nl/hap_nat/nat_fp.php?cntry=86&mode=3&subjects=28&publics=1}--we leave for the future research as we focus on Colombia, and we are less familiar with Panama and Ecuador. } 
% %ya, 2.5 on 1-4 scale is .6 of the way, >8 on 1-10 is >.8 of the way, indeed huge improvement

\section{Unlivable Colombia} %ya make sure both tables on the same page for contrast

Livability is  % ``the degree to which a living
% environment fits the adaptive repertoire of a species,'' i.e.,
 the degree of fit between the environment and human needs 
% Applied to human society,
% it denotes the fit of institutional arrangements with human needs and
% capacities. Livability theory explains observed differences in happiness in
% terms of need-environment fit.''
 \citep{veenhoven14b}. Livability is measured in
table \ref{tab1} in this section, with further discussion in
section ``3.1 Livability Theory: Human Needs''


In sharp contrast to high % (subjective)
 Colombian happiness, 
Colombia is not livable or has low objective
quality of life (QOL),
 as measured with 
objective indicators in table \ref{tab1}.\footnote{Colombia scores mediocre or low on
all indicators in table \ref{tab1}. Still, there are many other 
  ways to measure QOL.  UsNews, for instance, % is typical example of materialist approach to quality of life--it
  ranks Colombia 68/78
  (\url{https://www.usnews.com/news/best-countries/colombia}). World Economic
  Forum provides indicators,  too \citep{world2017travel}.   
}

 
%LATER LEAVE IT AS IT IS AND THEN HAVE GRAPHS OVER TIME
\begin{table}[H]\centering\tiny
\caption{\label{tab1} Livability/Quality Of Life (QOL): objective indicators% v subjective wellbeing or life satisfaction
  . For details on each indicator click the link under ``Source'' column.}
\begin{tabular} {@{} lrrrr @{}}   \hline 
Indicator&Value  & Source   \\ \hline
2019 poverty (national benchmark)&42\%&   {\tiny\url{https://data.worldbank.org/indicator/SI.POV.NAHC?locations=CO}}\\
2011 median daily income/cap PPP USD&\$7&{\tiny\url{https://www.pewresearch.org/fact-tank/2015/09/23/seven-in-ten-people-globally-live-on-10-or-less-per-day/}}        \\
2019 percent on $<$\$5.5/day&30\%&{\tiny\url{https://data.worldbank.org/indicator/SI.POV.UMIC?locations=CO}}
% https://en.wikipedia.org/wiki/List_of_countries_by_percentage_of_population_living_in_poverty
\\
% 2019 gini rank&   &        {\tiny\url{https://data.worldbank.org/indicator/SI.POV.GINI}}
% https://en.wikipedia.org/wiki/List_of_countries_by_income_equality\\
2017 R/P 10\%&    40&
                      {\tiny\url{https://data.worldbank.org/indicator/SI.DST.10TH.10}}%or %they seem to changed on wb; see %https://academic.oup.com/book/1917/chapter/141696965 similar
                      %https://en.wikipedia.org/wiki/List_of_countries_by_income_equality
  \\
2020 unemployment rate&    15\%&        {\tiny\url{https://data.worldbank.org/indicator/SL.UEM.TOTL.ZS?locations=CO}}\\
2020 freedom rank&   96/210% 65/100
                 &        {\tiny\url{https://freedomhouse.org/countries/freedom-world/scores}}\\
2021 corruption rank&   87/180 % 39/100
                 &{\tiny\url{https://www.transparency.org/en/cpi/2021/index/col}}        \\
2020 political stability, no violence/terrorism  pctile&  20th& {\tiny\url{https://info.worldbank.org/governance/wgi/Home/Reports}}\\
2020 rule of law pctile&  34th&    {\tiny\url{https://info.worldbank.org/governance/wgi/Home/Reports}}   \\ 
2021 working conditions decile&   bottom decile&       {\tiny\url{https://www.globalrightsindex.org/en/2021/countries/col}}\\
2018 quality of roads rank& 110/137& {\tiny\url{https://reports.weforum.org/pdf/gci-2017-2018-scorecard/WEF_GCI_2017_2018_Scorecard_EOSQ057.pdf}} \\
2021 victims of intentional homicide rank&top decile& {\tiny\url{https://dataunodc.un.org/dp-intentional-homicide-victims}}\\
2023 criminality rank&2/193& {\tiny\url{https://ocindex.net/rankings?f=rankings&view=List}}\\\hline 
%{\bf 2010-2019 average life satisfaction rank}&  {\bf 3/160}&  {\tiny\url{https://worlddatabaseofhappiness.eur.nl/rank-reports/satisfaction-with-life/}}\\\hline
\end{tabular}\end{table}



 %
  Some of the objective
 indicators of quality of life from table \ref{tab1} are remarkably deficient:   
 %
  About a third of Colombians live on less than \$5.50 a day (2019).
 {Poverty (national benchmark) is at 42\%--the whole nation has a
   higher poverty rate than one of the poorest cities in the US, Camden NJ, at
   36\% (also national benchmark, \url{census.gov/quickfacts/camdencitynewjersey}).
    Median daily PPP per capita income in 2011 was at \$7 (the US was at \$56).}
  %
  Colombia is in the bottom decile of working conditions: there are 
murders and impunity, union-busting and dismissals. %This is one of the most troubling statistics as it involves clear violations of human rights.
 %
%
  %NO, would need to double check
  % 16th most unequal country in the world out of 160 countries in terms of
  % gini--yet gini is quite abstract and meaningless like many economic measures
  % such as utility--gini simply goes on scale from 0 (everyone makes same income)
  % to 1 (one person makes all the income), and a number in between, say .3 or .5
  % is impossible to interpret, what that actually exactly means.
  % Better are other
  % measures of inequality such as  
  R/P 10\% is the ratio of the average income of the richest
  10\% to the poorest 10\%--Colombia ranks 3rd out of 70 at a remarkable 40--top
  decile of Colombians makes on average 40x the average of the poorest decile--even greater disparity than in the unequal US at 30. 
%
 Unemployment rate is at 15\%, 
% , in 2010, poverty rate  was at 34.1\% with the median income
% per capita of \$US 6,000, Gini index of 0.56, and unemployment  rate at 11 \%,
with informal labor at about 50\% of the workforce
\citep[][]{hurtado2016socioeconomic}.

All these deficiencies--insecurity, precarious labor, poverty,
and inequality are expected to result in  unhappiness.  
%  Notably inequality is associated with a multitude of
% negative outcomes \citep{wilkinson09}.\footnote{But see criticism by
%   \citet{snowdon10}.}  Specifically in Latin America,
 Inequality in Latin America was found to
have negative effects on happiness as it signals persistent
unfairness \citep{graham05}--unfairness seems to be more detrimental to
happiness than inequality \citep{starmans17}.  
Inequality is a stark feature of Colombian life, and it is inequality that has sparked recent mass protests \citep{icg21}.
% \footnote{ 
% In Colombia, inequality is related to gaps in rural and urban areas. The perceived wellbeing in the country is associated with economic
% circumstances and access to education. However, those in the top and bottom quantiles of SWB value the
% domains that matter most for their SWB differently. For those in the bottom 20\% of life satisfaction, education,
% income, and employment are the most relevant factors, whereas for those in the upper 20\%, standard of living, housing
% affordability, and civic engagement matter the most \citep{burger2021happy}.}

Colombia is still being haunted by violence and conflict,
much of which is rural (\citet{turkewitzNYT21sep26}, \url{hrw.org/world-report/2020/country-chapters/colombia}).
% https://www.britannica.com/place/Colombia/Colombia-in-the-21st-century
% https://en.wikipedia.org/wiki/Colombian_conflict
 %
 Colombia is less
stable and more violent than 80 percent of the countries--in table \ref{tab1} metric ``political
stability and absence of violence and terrorism'' is only at 20th percentile. 
 %
Colombia is only partly free, ranking 96/210, %65 out of 100.
  and quite corrupt  87/180.  % 39/100
  %
% Voice and accountability and government effectiveness and control of corruption
% are around median rank. Colombia scores best on Regulatory Quality, better than
% about 2/3 of countries, but
Rule of law is also problematic below about 2/3 of
countries.  
 %
 Crime rates are high: in terms of homicides Colombia ranks
in top decile, and by one ``criminality'' index it ranks as the 2nd most
``criminal'' country in the world (after Myanmar; out of 193
countries.)\footnote{Although it is probably an overstatement in terms of how
  crime affects an average person--this ``criminality'' index weighs heavily
  organized crime networks: \url{https://ocindex.net/report/2023/02-about-the-index.html}.}


In terms of quality of roads Colombia ranks 110/137--part of the problem is
mountains, yet, for example, equally mountainous Ecuador has relatively succeeded in road
building. %
% yet roads are precarious as one of the authors of the study found out
% when his bus crashed
Transport is the blood of the society \citep[e.g.,][]{devos13}--roads are a
basis for travel, commerce, and trade, especially that Colombia has no rail. 
% quality (Singapore best at 6.5), Colombia scores 3.4 \url{https://www.theglobaleconomy.com/rankings/roads_quality}. 


\section{% QOL v SWB: 
  Unlivable but Happy--``Fool's Paradise''?% --explaining the contradictions
}

In the previous two sections we have established that Colombia is happy, but
unlivable. Hence, it could be labeled a ``Fool's Paradise,'' a happy paradise,
yet a fool's paradise, because it is unlivable and so there is no reason to be
happy. 
 %
 Now we will examine this apparent contradiction.  The goal of this study is to try to explain
 this apparent massive mismatch or paradox, and spark the debate/future research.  % of
% apparently low or mediocre livability or Quality Of Life (QOL)  and top life satisfaction.
 %
%An apparent conclusion so far  % from table \ref{tab1}
%is that SWB doesn't doepend much on QOL. % money or other indicators of objective quality
% of life
 
It is instructive to start with Veenhoven's 4 qualities of life
(\citeyear{veenhoven00b}) in table \ref{t4}. Life chances as an outer quality in
first cell (livability of environment) should in theory correspond with life
results as an inner quality in the last cell (satisfaction of the person). That is, a livable place should be happy. 


\begin{table}[h!]
  \centering
  \begin{tabular}{l|llllll}
\hline          &outer qualities&inner qualities\\\hline
    life-chances&livability of environment&life-ability of the person\\
    life-results&utility of life&satisfaction of the person\\\hline
  \end{tabular}
  \caption{Veenhoven's 4 qualities of life
    (\citeyear{veenhoven00b}) }
  \label{t4}
\end{table}



 Colombia's % an (objective)
low livability/QOL  % (subjective) Life Satisfaction and
 should result in low satisfaction/happiness--it is unexpected for Colombia to
 be a happy country.  
 %
%
 Colombia scores mediocre or low on most livability/QOL indicators, 
but tops rankings of satisfaction/happiness. % both global overal cognitive life satisfaction
% and momentary affective happiness.
 In other words, it appears to be unlivable but happy,
a so called ``Fool's Paradise,'' a place where people are subjectively happy,
despite objective misery \citep{michalos14B}.
 %
An intersection of QOL and SWB can be visualized in a 2x2 matrix in table \ref{tab:2}--expected
outcomes are low-low or  high-high, but there can also be unexpected low-high ``fool's
paradise'' or high-low ``fool's hell.''


\begin{table}[h!]
  \centering
  \begin{tabular}{l|llllll}
\hline          &low livability&high livability\\\hline
    low SWB&real hell [deprivation, unhappy poor]&fool's hell [dissonance, unhappy rich]\\
    high SWB&fool's paradise [adaptation, happy poor]&real paradise [wellbeing, happy rich]\\\hline
  \end{tabular}
  \caption{Michalos 2 variable theory: fool's paradise and fool's hell
    \citep{michalos14B}. Cummins classification is shown in square brackets
    \citep[][p.61]{sirgy02}. % (Veenhoven's 4 qualities of life
    % \citeyear{veenhoven00b} are somewhat similar, too.)
    For other examples of fool's
paradise and fool's hell see \citet{aok-swbLivability18}.}
  \label{tab:2}
\end{table}
% CUMMINS: i guess this:  https://www.jstor.org/stable/27522495 TODO read at some point
% 1. Good objective quality of life and good subjective quality of life. This condition
% is referred to as "well-being" and coined as the "happy rich."
% 2. Good objective quality of life and bad subjective quality of life. This condition is
% referred to as "dissonance" and is coined as the "unhappy rich."
% 3. Bad objective quality of life and good subjective quality of life. This condition is
% referred to as "adaptation" and is coined as the "happy poor."
% 4. Bad objective quality of life and bad subjective quality of life. This condition is
% referred to as "deprivation" and is coined as the "unhappy poor."

There are a number of theories and explanations for high happiness despite low
livability. The remainder of this paper is devoted to these: first two happiness
theories (livability and folklore), second Marx theory of alienation, and
finally a complementary analysis from the opposite side (very livable but only
moderately happy): fool's hell in Singapore and the US.  
% In explaining the Colombian mismatch or paradox of low QOL but high SWB, aka
% ``fool's paradise,'' it is useful to dedicate a subsection to livability
% theory. 
% Next, we turn to livability theory.

\subsection{% Theoretical Background: 
  Livability Theory: Human Needs}


Veenhoven's livability/needs theory is a major and ideally
fitting happiness theory, specifically about the link between
livability and SWB, as conceptualized in the last section in tables \ref{t4} and
\ref{tab:2} \citep{veenhoven95,veenhoven14b}. Humans, like all animals, have needs, as those
on the Maslow's Hierarchy of Needs \citep{maslow87}--the more the needs are
satisfied, the more happiness--places or societies that satisfy human needs well
are livable or have high QOL: 

\begin{quote}
  Societies are systems for meeting human needs, but not all societies do that
  job equally well. Consequently, people are not equally happy in all
  societies. Improvement of the fit between social institutions and human needs will result
  in greater happiness. \citep[p. 3645][]{veenhoven14b}
\end{quote}


The apparent Colombian chasm between livability and happiness may point to the
limitations of livability theory.
 %
%Or perhaps, alternatively,
 But we argue that, counter-intuitively, the livability theory may mostly hold
 true  because:
 1) Mediocre or even moderately poor development (physical and
 institutional infrastructure) is already good enough to satisfy most basic human
 needs and make a place livable;
 2) % there are other non-commodities
  Physical and institutional infrastructure mostly serves only first two steps of
  Maslow's pyramid (physiological and safety) \citep{maslow87}. {Human physiological needs are simple
   and easily satisfied without much economic or institutional development.}
  3) Higher needs such as personal
  freedom and social connection that are critically important for
  livability are rarely properly captured by livability metrics.
  % subjective/positive freedom
  4) Given always limited resources and attention, % the more of one, the
  % less of the other one
  there is an opportunity cost. 
  Excessive pursuit of money or consumption at person level, or economic growth
  at community or society level sacrifices non-commodities such as personal freedom and social connection
  notably through overwork and alienation {as elaborated in later
    section ``Theory of Alienation.''} This is an important mechanism that we cannot overemphasize,
and it is often overlooked.    

Next we provide examples of human needs that are overlooked by livability/QOL
indices. These are human needs--they do count towards livability. 
 In the examples we focus on Colombia, and also provide contrasts to the
 developed countries, and indicate trade-offs/opportunity costs. 

 Biophilia \citep{fromm64,wilson21}, a need for contact with nature is a fundamental human need, yet usually forgotten. ''Nature is not a place to visit. It is home'' (Gary Snyder).
 There is a clear tradeoff between economic growth and nature abundance and preservation,
 for instance, the more urbanization, the less natural the human habitat. Or
 next door in Brazil--the more economic expansion and growth, the less Amazon rain forest.
 %
 Climate change is a critical challenge for human needs as it endangers the very
 habitat of homo sapiens \citep{pachauri14}, and again, the more economic growth, the worse the
 environmental degradation  \citep[e.g.,][]{klein14}. %and ya the green growth
                                %or decoupling aint that great in practice
 A reasonable course of action is to
 de-grow the economies \citep{hickel20,kallis11}, especially the rich and carbon
 intensive ones such as the US. {Per Colombian natural resources use and economic growth see discussion in \citet{rubianoNYT22nov16}.}  
  

Related to biophilia and climate change is biodiversity. Biodiversity
improves  happiness \citep{adjei15,prescott17}. 
 %
 Nature  is extraordinary in Colombia. Colombia has 2nd largest biodiversity
after Brazil, despite being about 7x smaller in area. Colombia has just about
any type of natural amenities. Exposure to  nature (as opposed to urbanism) is
the key ingredient for happiness \citep{pretty12,tesson13,thoreau95}.

 Social connection is a human need, and a key to happiness 
\citep{tonnies57,lane00,mcmahon06,putnam01}, and there is plenty of social connection in
Colombia. Colombians are extraordinarily social, friendly, outgoing and
spontaneous--social gatherings, events and festivals are widespread, frequent
and long-lasting. Again, there is a tradeoff, the more focus on money, the longer
the work hours, and the less social connection.

And it is not just wide spread but also deep social connection. 
% Latin America is one region with low GDP per capita (compared to developed countries), but subjective wellbeing
% assessments are as high as or higher than those in developed countries. % In measuring subjective wellbeing based on the
% % emotional state of its inhabitants, Latin America shows one of the highest
% % levels worldwide \citep{beytia2016singularity}.
%  % 
The high level of SWB in Latin America, including Colombia,  is supported
by a key SWB predictor: strong affective relationships. In Latin America, the strength of the affective relationships of its people and the ties they
build in the communities are one of the  primary sources of happiness--SWB in this region has social and affective foundations
\citep{yamamoto16,rojas15}. High SWB is also predicted by satisfaction with family relationships and a higher frequency of positive
emotions. The quantity and quality of interpersonal relationships are pivotal
for the region's high levels of SWB. {\it Relational wealth}, which encompasses the strength and abundance of close and warm interpersonal relations with
family, friends, neighbors, or colleagues, is a strong cultural characteristic in the region. On average, 85\% of Latinos
 report having someone to count on in times of trouble. In some countries like Venezuela, Panama, Argentina,
and Costa Rica, more than 90\% of people report having a good social support
network \citep{rojas2019well}.
 %
 This is in contrast to developed countries where people report
spending only six hours weekly with friends and family, almost half an hour less
than in the previous decade % ; and 1 in 11 people report not having close friends or relatives to count on for help
\citep{van2020s}.
%  %
% High SWB in the region somehow co-exists with poor policy outcomes. People are affected by factors that have been shown to reduce life
% satisfaction, such as government corruption, widespread violence, and economic hardship
% \citep{rojas15}.



Freedom is a great human need, perhaps worth dying for.{
As a movie character Scottish warrior William Wallace put it:
\begin{quote}
  Fight and you may die. Run, and you'll live...at least a while. And dying in
  your beds, many years from now, would you be willing to trade all the days,
  from this day to that, for one chance--just one chance--to come back here and
  tell our enemies that they may take our lives but they'll never take our
  freedom! (\url{imdb.com/title/tt0112573})
\end{quote}
}
 Colombia scores average on freedom listed earlier as a QOL metric in table
\ref{tab1}, but that is one kind of freedom: ``freedom from''
(negative, objective): be no slave, live in a free country,  have no coercion, be free from restrictions/impediments, lack obstacles.
 But there is another kind of freedom: 
 ``freedom to'' (positive, subjective):  be able to choose,  control and direct one's own
 life, and be in charge.
  %
  On scale 1-10, world's average is about 7; the legendary land of the free, the
  US, scores higher at 7.7, but Colombia scores higher yet at 8
  \citep{aok_free_from_to,ee_ls15}. The US does have many freedoms, but
  Colombians % (and Mexicans, too)
  actually do feel more free than people in the US. 
  %world avg from wvs 1981-2020: 6.9 .25
 Arguably one reason for lower feeling of freedom in the US is too much
 focus on work, money, and consumption. The idea is further elaborated in
 ``Theory of Alienation'' section.




% LATER MAYBE %    comparison/discrepancies \citep{michalos85} yes! as compared
% to the past it is much better now!!!!


%  genes/set point
% \citep[eg][]{schnittker08}, 
%    adaptation/adjustment; hedonic treadmill \citep{brickman78cj},
%    happiness just a motivator \citep{carver90} % [rather momentary affective
%   % happiness than global cognitive life satisfaction]



%Finally, there is folklore theory \citep{veenhoven95}.

% it is important so shoul;d be here i guess
%THESE HERE I GUESS SUBSECTIONS IN DISCUSSION/CONCLUSION OR SOM
\subsection{Folklore's Theory: Colombia's ``Good Energy''
  % Popular culture and popular media explanations for
  % happiness--Extremely Good Energy (popular media observations, anectdal
  % evidence authors observations)
}

Folklore theory is an attractive explanation of fool's paradise (table
\ref{tab:2}). It is a theory put forth by Veenhoven as a competing explanation
to his livability theory as discussed in the last section. 
The folklore theory states that happiness is a product of culture--tradition,
national character and widely held notions about life determine one's
happiness \citep{veenhoven95}. Happiness is the reflection of broadly held
perceptions about life which are rooted in traditions and the culture of a
 society \citep{veenhoven95}.  % if a society has a pessimistic outlook on life,
% generations to come might hold the same beliefs even if the situation in their
% country has improved. Nevertheless,
If a culture has an optimistic outlook on
life regardless of  circumstances, future generations will remain
positive.  Thus, %through the folklore theory one can predict that Paraguyans
a society may be happy, regardless of the socioeconomic situation, because of cultural influences \citep{veenhoven95}.
 In other words, one can wear pink glasses and be
happy no matter the circumstances (or see through dark lenses, have a bleak
outlook and be unhappy no matter the circumstances). {This is in sharp contrast to
livability theory, where happiness is a result of a person's experience,
satisfaction of her needs.} 

% Resilient etc but also spontaneous! They laugh at themselves and joke, like
% Angie of her pants economico, lauren about shoes: only in Colombia etc etc

% FROM MY PRESENTATION OF EUR LIVE TO LIVE: 
% tears of god: have to suffer here in this word to be blessed in heaven; people
% believe in karma: if you do something bad you will suffer; colombians have low
% expectaatons, even like getting a high shool degree is success; they live day by
% day, have fun, they are in the moment! latinos dont complain, they deal with it

A Peruvian social psychologist specializing in happiness argues that the origins
of Latin happiness can be traced to:
\begin{quote}
  the minimalist well-being lessons of Andean and Amazonian small traditional
  communities which constitute the grounds of Latin American happiness, a life
  style that mimics the ancestral environment, the deep nature where the
  happiness brain wiring occurred; a physical and social environment that
  naturally activates the brain pleasure circuits. Culture resembles evolutionary
  needs; resources to achieve needs are available for everyone; positive,
  interdependent collectivistic interaction is ingrained in behavior,
  supporting, working, competing, and sharing. \citep[][p.45]{yamamoto16}
\end{quote}
Then Colombian happiness is due to tradition and culture (folklore theory), but
notably at the same time happiness stems from satisfying one's needs (livability theory) as
tradition and culture fit human needs.
 This is a key point: while in principle folklore theory is in contrast to
 livability theory \citep{veenhoven95}, in practice, in specific cases, the two
 theories do not have to be contradictory. If tradition and culture help
 satisfy human needs, as is the case in Latin America, the two theories are not contradictory.  

There is an indigenous concept of "Buen Vivir" (Good Living) \citep{hidalgo17},
similar to Aristotelian "Eudamonia"--both emphasize harmony and community. Buen
Vivir is also about environment and food sovereignty, and it arguably
contributes to SWB in Colombia, as it does in Ecuador
\citep{guardiola14}. Notably, Buen Vivir helps to explain an apparent fool's
paradise--economic poverty is relative--it depends on a specific way of
life. For instance, households that grow their own food and are in an indigenous
community depend less on money to be happy \citep{garcia18}. 
% are less likely to report to be subjective well-being poor 
 {Likewise, kibbutzniks \citep{morawetz77} and 
   Amish \citep{surowiecki2005technology} are able to be happy despite being poor. 
%surowiecki2005technology: \url{http://www.technologyreview.com/review/403558/technology-and-happiness/} or
%   \url{http://scienceblogs.com/cortex/2007/03/16/happiness-wealth-and-the-amish/}.
 }
%
Francis Marquez, an indigenous Colombian vice president, proposed a similar concept to Buen Vivir, ``Vivir Sabroso,'' to live in harmony with
nature, traditions, and community: "vivir sabroso no es vivir con
plata, vivir sabroso es vivir sin miedos" ("living tasty is not living with
money, living tasty is living without fear.")
% \footnote{Vivir Sabroso also has an
% activist/political angle to free and emancipate blacks and indigenous folks to
% live life fully with dignity and without fear: .}

The folklore theory is about % in terms of
national disposition/trait/character. It does appear that
Colombians have a slow-paced familial/social cheerful/happy disposition, which is conducive for happiness. 
 %
% The theory does appear to have explanatroy power for Colombian happiness--it
% does appear that Colombia's tradition/culture is slow-paced familial/social,
% similar to Paraguay's ``Todo Tranquilo'' \citep{lunde18}, which could be
% conducive for happiness. 
 %
 %
% real individual grief and joy
Colombian happiness is arguably real, however, rather than just being due
to cognitive cultural norms, wearing pink glasses. %bias/illusion  
%HOW DO WE ASSESS HOW HAPPY WE ARE? Tenets, implications and tenability of three theories Ruut Veenhoven
% https://pure.uva.nl/ws/files/832269/77387_310125.pdf
 And again, it is not just due to culture on its own, rather due to culture that
 is aligned with human needs and satisfies them, so that the environment is livable
 (livability theory). 

Colombians celebrate over 3,000 festivals or carnivals in
small towns or large cities and have about 20 holidays per year. Folklore 
and carnivals are part of the festive character of the country. The government uses them as tools to promote reconciliation
and build social fabric and identity \citep{gutierrez2006fiestas}. In Colombia, following a robust pattern in Latin America,
collectivist values are ingrained in the culture \citep{mensing2002collectivism}. Family is at the center of collective values, close friends
follow, and in the exterior layer are neighbors and the community. This interdependent collectivism is at the core of Latin
American happiness \citep{yamamoto16}.

Proper treatment of the folklore theory is left for  future research as
authors' anthropological and historical expertise is
limited--but we discuss below %some anectodal evidence, speculation, and common wisdom.
% To have fully explored all the possible explanations for the unexpected
% Colombian happiness, we present here common wisdom and speculations.
% difficult to measure and operationalize nd capture more scientifically hence we
% allow these speculations here that could be a start to form the hypotheses to be
% tested in the future
%"good energy" is diffcult to quantify and not studied much scientifically.
%thsi section lists anectodal evidence and
  popular explanations for Colombia's happiness--future research can test them properly and systematically. 
 %
 %
 %
%Let's focus instead on  popular media/popular culture explanations. % \footnote{As a sidenote, perhaps start with \url{colombia.co} which appears an official governemnt
% website adverstising Colombia to the world, producing the Colombia brand, it
% does talk about great positive energy, eg "Boundless Enthusiasm", but then also
% blends in usual Western business talk about entrepreneurship and so forth, and
% this is good, colombia does need investments. See
% \url{https://www.colombia.co/en/colombia-country/what-are-colombian-people-like}
% }
 %
 Two accounts are informative:
\citet{bargentWP16jan15} and \citet{wallaceBBC17mar30}.

According to an Englishman living in Medellin these are the things that make it
happy in Colombia \citep{bargentWP16jan15}: putting most importance on family, friends,  {and fun}--bravado and blind optimism may help, too% \footnote{Ignornce is a bliss?}
; having less entitlement and appreciating what one has, having joy in small things,
e.g., cheap coffee/alcohol are just fine; not worrying and not expecting much
("tranquilo," "no importa").%myriam also says "no importa";
\footnote{\citet{bargentWP16jan15} observes: "Colombia violence and
cruelty became frighteningly routine"--indeed people can get used to just about
anything \citep{brickman78cj}, and even though the violence is frequent, it
used to be even more prevalent in the 80s and 90s--and happiness can be produced through relative
advantage or improvement \citep{michalos85}.} %MDT
% !!!:
%"Trauma and grief are stitched into the collective consciousness. But so is
%resilience and the drive to enjoy life despite it all [...] Colombians being
%happy is not such a contradiction after all."

\citet{bargentWP16jan15} wonders further that in Colombia emotions change seamlessly and
effortlessly between shame and pride, despair and hope, sorrow and
happiness--but shouldn't they? Isn't being natural, simple, and easy-going a good thing?  
As opposed to the US, where one is supposed to pretend to just be perfect,
happy, and busy working as in ``American Beauty'' movie. {Surface acting
  (faking emotions that are deemed appropriate) is emotionally draining
  \citep{brooksATL22nov17}.} 
% There is good and bad as everywhere, but in Colombia it's all more easy-going and simple, and people are remarkably cheerful and helpful. 
% , warm, open and humorous \citep{bargentWP16jan15}.


\citet{wallaceBBC17mar30} offers many illustrative quotes by Colombians and
about Colombians:

\begin{quote}
  Money is nice but it's not the most important thing. In general we are a
  culture that values what you have. [...] % ,'' and ``We love people and music''
   %
  Colombians are innocent. They're curious. [...] %--marcuse, nietzche, fromm
  Colombians have become indifferent to situations of war. In other words, if
  the problem does not touch me directly, I must feel grateful, satisfied,
  optimistic, lucky. [...] Colombians have always demonstrated incredible,
  Herculean and powerful resilience to war, death and to the harsh history of
  violence and diplomatic failures [...] Colombians feed this resilience through
  human connections and the communal
  experience. [...] %--can get used to just about anything;
 % 
 %
  % And there is dance, indeed it is as if Colombians have music in them--
  We live for parties, holidays % , and fill the void with a fanaticism for
  % sporting events and beauty pageants and entertainment
  [...] The dance frees
  you. It is a way of expression and feeling. Here the music is carried in the
  blood, in the veins, in our heart. It's a great passion we carry throughout
  our lives.% \footnote{ \citet{wallaceBBC17mar30} further explains: 
%    ``Colombian salsa, as opposed to other forms, is denoted by
%    faster syncopations that match the people's natural energy [...] % It's an
%    % egalitarian genre, accessible to everyone, and it seems to make the entire
%    % country happy. But is it a different experience than in other countries,
%    % like, say, with Brazilians and samba?'' ``I think there are several
%    % differences [...] Our dance is much more sociable. It's necessary
%    % to dance salsa as a couple or in a group. There is a more direct
%    % connection. For these reasons, it's been gaining importance in the world.''
% Salsa is a refresher of human dignity. It overshadows inequity and
%    discontent with sharp rhythms and the madness of love. It closes social
%    distances because it requires people to embrace each other in a moment of eye
%    contact, the feel of the skin, bringing them together with movement, helping
%    them to know each other and see the best in each other. It has a long
%    cultural heritage of peace.'' % \citet{motodreamer} adds few happy Colombian traits: warm and polite, deeply affectionate among friends, and kind to strangers; loves to party, reveres, and adores their family, and is so enthusiastic about life.
%  }
\end{quote} 
  

%meh:lists 21 reasons for colombian happiness, skipping many as they are irreleveant (emeralds: biodiversity (nature brings happiness eg pretty, walden, tesson); racial diversity (diversity is related to unhappiness, but racial segregation is related tohappiness  eg herbst, mine, and that sociologist vogt)


\subsection{%Theoretical Background:
  % Marxist
  Theory of Alienation}
%yeah and again like 50% of colombians are in informal economy

Colombians are refreshingly connected, surprisingly so as from Western
point of view. Indeed, there appears to be a chasm between Latin connection/integration and
Western alienation/estrangement. The contrast strikes us as pointing to Marx's
alienation theory.      

%\textbf{see  i  wish i hadnt worked so hard:  on money and Marx!}
%ya one para is enough; can extend in the future!! esp reread horowitz, and
%http://pubs.socialistreviewindex.org.uk/isj79/cox.htm
%https://plato.stanford.edu/entries/alienation/
``Alienation is the transformation of people's own labor into a power which
rules them'' (\url{marxists.org/subject/alienation}). 
 %% as if by a kind of natural or supra-human law. The origin of alienation is commodity fetishism--the belief that inanimate things (commodities) have human powers (i.e., value) able to govern the activity of human beings.''
 %
 Alienation means separation of a person from the conditions of
 meaningful agency--a typical situation is when a person does not own means of
 production--such a person is only an appendage of a machine \citep{horowitzMISC10}. % tiny cog in a machine
 The overall % overreaching
 alienation consists of alienation from: the product of labor,
the activity of labor, one's own specific humanity, and others/society.\footnote{For elaboration see \citet{horowitzMISC10} and
 \url{marxists.org/subject/alienation}. % The Marx's concept seems close to a more modern psychological concept of dissociation--see online appendix.
} 


In Colombia there is a human factor, good energy that we have lost in the West % ern
% civilization plagued with its discontents
 \citep{freud30}.
 %!!!!YES THIS IS IMPORTANT!! IF WATER DOWN OR REMOVE HERE MAKE SURE HAVE IT ELSEWHERE!:
 It arguably does appear that the US (and the West) has given up some humanity to win the
 economic race. Colombia (and Latin America) gave up less humanity, remained 
 % quite human  and
 happy, but lost the economic race (hasn't dominated the world economically).
 %
 %p233 ebshoy found it
\citet[][p.233]{fischer73} made an useful observation on large cities: urbanites pay
``the emotional price for economic wellbeing''--likewise, we argue, so do
Americans, Singaporeans, and other rich societies--as elaborated in ``the US:
Fool's Hell?'' and ``Singapore:
Fool's Hell?''  sections.
 
Colombians' attitude and approach to life is spontaneous unbridled
joy--similar to what Marcuse and Fromm advocated (and what has been lost in the West)
\citep{marcuse15,fromm13,fromm12,fromm64,fromm94}, % \footnote{\citet{marcuse15}
%   contains many valuable insights on freedom, alienation,
% and wellbeing--much of which is a critique of Western or especially the US way
% of life, and much of which seem to materialize in Colombia.
% }
 %
 %
 also reminiscent of Nietzsche's ideal of a child--curious, spontaneous,
creative, and innocent \citep{nietzsche05alt}. 
% so yah its about
% environment ecology like people festive and nature, but maybe even more so about
% attitude and approach to life--kind of like marcuse and fromm ideal
%
%mindful being present; but culture of poverty? banfield?
 %
 It is present time orientation--not living in the past or worrying about future,
happy-go-lucky free spirit without shame or guilt--in sharp
contrast to the West, where anxiety and calculating attitude 
prevail.

{On the other hand, perhaps Latin culture simply could be just culture
  of poverty \citep{banfield67,banfield74} that leads to poor development. Yet, not caring too much can be actually what is needed for much of
  a society in the West--see wonderfully refreshing \citet{manson15}.}
%Free spirit shame guilt free fun living a life. Not being a slave
 
%Not just rapido and trabajo and dinero
The US way of life is unnaturally fast and mostly about money \citep{easterlin73},
 aka ``busyness'' (being busy with work all the time is very desirable)  \citep{gershuny2005busyness, muskIN18nov26}.
 The US way of life is also full of stress, anxiety, and alienation even outside of work and money pursuit.  
%There is a big time anxiety in henwral unrelaywd to trabajo dinero 
  Another source of stress and anxiety may be a need to keep up with the Joneses, and the constant drive to excel in
  everything, be perfect \citep{frank12,manson15}--so well portrayed in 
  ``American Beauty'' and ``Crash'' movies. {
    Excellence is pervasive, e.g., Plano TX, has on its official logo ``City Of
    Excellence'' (\url{planochamber.org/about-plano/}), Rutgers University is a
    place ``where excellence is earned''  (\url{rutgers.edu/excellence}.) 
  %

    But excellence or perfection is not human, on the contrary, to
    err is human. 
 %
 Pursuit of perfection/excellence generates arms race and constantly raises the
 bar, creating even more stress and anxiety in a vicious cycle% , and ultimately
 % by definition creates mass failure
, and
 there can only be one or a handful of winners in any race \citep{frank12,manson15}. In addition to
 stress and anxiety, shame and guilt are arguably created as well. 
}
 
%putnam bowling alone etc
Yet another source of stress and anxiety is quantity and quality of social
relationships--in the US, a highly capitalistic society, social relationships are
about business, not about actual meaningful social contact
\citep{horowitzMISC10,gssLonnieRubia}.  %\textbf{TODO CITE from wish hadnt work so hard}
 %
As compared to %calculating and fake % corrupt
 Americans,  Colombians are  spontaneous, % innocent,
 closer to human nature, and more real. 
 %
%ADD HERE FROM LINA ORG where i have lina in org that colombians dont pay
%attention to politics; here in pasto french guy says they dont pay attention to
%news and bad stuff, life is more about family
Colombians do not pay much attention to economics % bad news
and politics; life is more
about family, friends, and fun \citep{martinez17,martinez20}. 

In Cali, the third largest Colombian city, over 1,200 residents are
surveyed each year since 2014 about what matters for their SWB
\citep{martinez17,martinez20}. What matters the most are family and personal relations. Very few consider  politics, corruption, or other 
adverse external circumstances in their inner wellbeing assessments
\citep{martinez17,martinez20}. Hence, the poor livability by some indicators affects
happiness less, as less attention is being paid to these adverse circumstances. 

 Colombians appear to be  full of agency in contrast to alienated Americans \cite{scitovsky76,duany01,leonard10,chomsky09,wheelerIN15aug6}.  
For instance, in Colombia, pull over from the road and right there on a roadside you get a
friendly personal cup of coffee. In the US, you can  go to a  
Starbucks that feels like a hospital or an airport--robotic and inhuman.\footnote{If
  you are from a developed country, you probably protest the proposition and
  think that Starbucks is friendlier, warmer and cozier than a hospital. Then
  you should go to a developing country, non-tourist destination (no Cartagena,
  no Cancun, etc) and spend time with locals for at least few  months.}
 % 
 Starbucks workers (McDonald's, etc) do seem to be alienated both from the product and the activity--they
have no freedom, autonomy or latitude over the product % (it is given)
and almost none over labor (there are strict procedures that must be followed). 

Same holds for other chains that dominate the US, and could be extended to other
businesses, delivery for instance. US Amazon drivers have cameras and motion
detectors in a truck, and to meet the quota sometimes  have to 
 urinate in the bottle \citep{moyerMISC21apr3}. Similarly in warehouses, workers are
wearing a bracelet with GPS, and to
meet the quota sometimes have to restore to painkillers that are freely
available from dispensaries throughout the warehouse \citep{streitfeldNYT15aug17,guendelsbergerBI19aug9}.
%
What else could be a better example of a loss of humanity or de-humanization than
a human wearing a bracelet with GPS eating painkillers and working alongside
robots in a giant warehouse? {Movie fiction is becoming
  reality--see ''Elysium'' (\url{imdb.com/title/tt1535108}).} 
%
 Of course, not all workers suffer in such dire
conditions, but this is arguably the trend. If innovative and  efficient Amazon does it, others
are likely to follow, or be out of business. % or be forced out of business. 
 %
In Colombia, on the contrary, much of delivery is informal--often a person on a
motorcycle who has plenty of autonomy and freedom over the execution of her job.
 %
%En col much less dinero, eg most work in informal economy, help ech other, real community etc
In fact, about half of the Colombian economy or more is informal % --informal labor is at
% about 50\% or more of the workforce
 \citep[cited in][]{hurtado2016socioeconomic}.\footnote{% Colombia has a high prevalence of
  % informal work above 50\% and
  Workers in the informal sector are less happy
  \citep{hurtado2017hidden}. This would contradict the alienation hypothesis, but it
doesn't take into account confounders such as lower pay, lack of benefits and
instability.}
% A typical set of (CHECK) jobs includes selling juice, bbq,
% and other food or glassess, hats, and other apparell on sidewalk or a roadside.

And there are two other types of alienation: from oneself and society--if one performs
highly specialized tasks in a repetitive fashion for long hours (most of the US workforce), one becomes alienated from herself and the society. This can be easily
observed after working hours--the US workers are like ghosts without much life
in them and without much interaction \citep{putnam01,duany01}.
 %
Colombians, on the other hand, are not alienated--their life is about family, friends, and fun, not
about money--less money orientation, less alienation \citep{gssLonnieRubia}. 
% Totly foundation of all this happiness premium in col is Marx: ppl live here to
% have fun to have life; in us ppl live to make money and spend, consumerism; and
% it fails! They are less happy. Works only for capitalists

% It's real here. Not fake
%  Not anxious. 
% Capitalism alienates. In col much less of it! Much less alienation

 

 As per livability theory \citep{veenhoven14b}, society is a system to satisfy
 human needs, but there are multiple serious discontents of the Western society
 and many human needs are suppressed and not satisfied \citep{freud30}. Notably highly
 capitalistic societies, such as the US,  serve to satisfy the capitalists, rather
 than the working people \citep{marx10}.%just as it earlier did to serve feudal lords; kings, etc etc

 {
Yet so far the alternatives to capitalism such as communism did have failed spectacularly say
as in Soviet Union %j perterson
and continue to fail in Venezuela and
elsewhere. Still, what about fully automated luxury communism \citep{bastani19}?
%  

There are many great things about the West and
especially the US: vibrant international
melting pot--Colombia may be a microcosm of Latin America, but the US may be a
microcosm of the World; %Myriam tia makes a good point that food in us intl, so yeah Colombia mocroscosm of lat am and us of the world 
 relative lack of (crude) corruption, working independent courts,
un-corrupt law enforcement, abundance of goods and services (not that need that
much for livability), and arguably the biggest advantage: some of the highest wages in
the world.

Still, the bottomline is that alienation is not worth the money, unless perhaps
it's millions of US dollars in annual salary. But even well paid jobs, so
called ``6 figure salary jobs'' (100k-1m USD) are arguably not worth it as
 argued by many  who quit such a job and  became much happier
without the good paycheck but without alienation--for example see one such
persuasive account (from Singapore, fool's hell): \url{youtube.com/watch?v=S_D4yJavp8M}.
}

 

% more consumption/physical stuff--less time and energy
%for non-physical stuff; opportunity cost
%
%civ and its discontents--more civilization, less freedom 

   
\subsection{Singapore:  Fool's Hell?}

A useful counterexample to Colombia's ``fool's paradise'' is a ``fool's hell''--a place that has
high objective quality of life or livability, but low subjective wellbeing or
life satisfaction. 

Singapore, Switzerland of Asia, is impeccably % /squeaky
 clean, extremely stable/predictable/disciplined and safe
\citep{singapore22laws,singapore22nanny}. 
%
Singapore, by many standards, is one of the
best, if not the best place in the world. It has the world's 3rd highest (after
Qatar and Luxembourg) Gross Domestic Product per Capita Purchasing Power Parity
adjusted \citep{authorNYT17monD}. It has also the 3rd highest (after Monaco and
Japan) life expectancy \citep{cia},
 2nd highest economic freedom \citep{heritage}. Singaporean children score highest on educational
tests \citep{coughlanBBC17dec6}, it is making greatest progress in health
\citep{fullman2017measuring}, has the world's fastest internet
\citep{mcspaddenNYT17monD}, and 2nd best roads \citep{world2017travel}.
 It even has the world's strongest passport \citep{chandranCNBC17oct24}. In short, one could say that Singapore % has one of                                               
% the top, if not the top quality of life in the World.                                                                        
 is one of most livable places in the world, if not the very most livable. 
 %
 Singapore's life satisfaction rank is 68/160 (WDH). Again, Colombia is ranked 3/160.
% like opposite of singapore, adn yet about the happiest in the world and
% singapore around the middle.


Being impeccably clean, extremely disciplined and safe can hamper positive freedom
(Singapore scores 7 v Colombia's 8 in WVS) and ultimately happiness. For instance,
no smoking in Singapore--but cigars help a great deal in the struggle of life (Freud).\footnote{
``Smoking is one of the greatest and cheapest enjoyments in life, and if you
decide in advance not to smoke, I can only feel sorry for you,''  ``[Cigars have] served me for precisely fifty years as protection and a weapon in the combat of life ... I owe to the cigar a great intensification of my capacity to work and a facilitation of my self-control.'' % (Cohen, Freud on Coke)
  Per Freud see
  \url{https://www.freud.org.uk/2020/04/22/freud-and-his-cigars/} and
  \citet{elkinCA94}. Of course, as with everything, moderation should be
  exercised, for instance, smoking more than 2 cigars a day increases
  probability of health problems much more than smoking 2 cigars or fewer, and while some studies find no health effects of smoking 2 cigars or fewer, some studies  find negative effects \citep[e.g.,][]{chang15}.
  }
  %
 No marijuana, no gum chewing, and a list of prohibitions
continues \citep{singapore22laws,singapore22nanny}.


%LATER/MAYBE  \textbf{really Colombia better than Singapore? discuss more!! }


\subsection{the US:  Fool's Hell?} %(similar to singapore subsection)

The US, the world-renowned ``best country in the world,''\footnote{Many people
  living in the US hold such a view--national pride tends to be high in the US
  (\url{worldpopulationreview.com/country-rankings/most-patriotic-countries}). And
 the US government under Trump also often tends to subscribe to such a view.} 
 %but wvs is smaller https://www.vox.com/2014/5/18/5724552/patriotism-pride-global-world
 home to the American dream, notably the very richest (per capita) country in the world (excluding small countries such as Norway), ranks 
46/160 in WDH--the ``best country in the world'' doesn't even rank in the top
happiness quartile. This indicates fool's hell, the opposite of the Colombian
mismatch of QOL and SWB. Except that, also as in Colombia's case, there is
arguably better match between QOL and SWB than the rankings show, because there are
non-commodity components such as personal freedom and social connection that
need to be taken into account. Colombia is
actually quite livable and so it is happy, and the US actually is not that livable,
and hence not that happy. %
% US and singapore and west rather fool hell--livable but unhappy; but again are
% they really livable? \$ is not everything! ppl think of them as paradise, many
% migate tehre but happiness is not to be found there; you want happiness--come to
% colombia (or mexico)

%US FOOLS APRADIAE AMERICAN DREAM FALSE CORAK
One would
imagine that the US must have top income mobility in the world or surely
somewhere near the top, it is after all the country of the American Dream
where hard work results in success better than elsewhere. But 
 % income inequality correlates negatively with income mobility--more unequal
 % countries have less mobility
 % Thus moderately
% unequal US has only moderate mobility, which
% means that
many other countries actually have ``more realistic dreams''
 than the US--for instance, it is easier to make it in Norway, Denmark, or
 Finland \citep{corak04, corak11, corak13,economist12_oct12,economist12_oct13,economist12_oct13_B}.
 In terms of mobility, the US is somewhat like a Fool's Paradise--people and especially
 immigrants think it is a paradise--you work hard, and you go to the top, but it
 is actually easier in other countries.


In the US, pursuit of money and pursuit of happiness are about the same thing
\citep{easterlin73}.
 %
But we know that a lot of money does not buy much happiness, and if anything, excessive pursuit of it, such as that in the US, may actually decrease happiness \citep{kasser16,dittmar14,brown05,kasser13,schmuck00,kasser93,leonard10}.
 %
Arguably the major culprit is consumerism--in the US one can
make a good living, make a good salary, the problem is people spend it on stuff
they don't need and end up on the hamster wheel. % (again Colombians feel more free
% than Americans 8 v 7.7 in WVS)


 
\section{Discussion and Conclusion}

% the article has a non-standard structure (standard: lit rev, data,
% results)--
The origin of the study has been the apparent paradox in the data of
high happiness despite low livability. Conceptually and theoretically the
article has followed and built on Veenhoven's 4 qualities of life
(\citeyear{veenhoven00b}) and Michalos 2 variable
theory \citep{michalos14B}, livability and folklore theories
\citep{veenhoven95,veenhoven14b}, and Marx's theory of alienation 
 % \citep{horowitzMISC10}
 (\url{marxists.org/subject/alienation}). {In terms of specific happiness literature in Latin America we have especially
built on ``Handbook of Happiness Research in Latin America'' by \citet{rojas15}  
with chapters by \citet{hurtado2016socioeconomic,velasquez2016importance,yamamoto16}.} The
contribution has been to combine the above conceptual
and theoretical approaches to offer new insights about high happiness despite
apparently low livability in Colombia. In the process, we have contrasted
Colombia against the US and West to learn happiness lessons from Colombia for the
US and West. 


% Colombia appears to be a striking paradox in the social indicators
%  field. Colombia scores poor or mediocre on objective quality of life
%  indicators, but is
% an extremely happy country. 
 %
 %
 %
Is Colombia ``unlivable'' but ``happy,'' a so called ``fool's paradise''?
% \citep{aok-swbLivability18}
 Arguably not.
%As per fool paradise question in the title: no, wonderful place, yes, poor, but not only material things matter for livability
 %
%It has wonderful nature, extremely good energy, cheerful welcoming people, and so forth. 
 Colombia is a genuinely % wonderful deeply
 familial/social  
place  as many locals and travelers would attest
\citep[e.g.,][]{roosHUFF19mar26,wallaceBBC17mar30,motodreamer}. % , and so is Cali
 %
 Colombians don't seem to be weighted down by civilization and its discontents
\citep{freud30}, or ghosts of the past \citep{pile05B,pile05}.


The conclusion is that Colombia, like many countries in Latin America, 
 is happy and also quite livable. Surely many QOL metrics point to Colombia's real
and serious problems such as poverty and inequality, but at the same time Colombians enjoy % spontaneous unbirdeled joy
great advantages such as positive freedom and social connection.

Only apparently great happiness is somehow possible in unlivable conditions. 
 %  
 We enumerated earlier Colombia's unlivable conditions (table \ref{tab1}),
 metrics that are mostly economic, some legal/institutional, and physical
 infrastructure. 
 %
 But
 these parameters are not the only ones that matter--there
 are other qualities that matter for human flourishing or happiness. 
 %
Despite all the problems with poverty, roads, corruption, etc, Colombia is a livable place.
 And important and often overlooked point is that given very basic level of
 economic development and material comfort has already been achieved, the
 further focus on money and material comfort can actually be counterproductive
 and result in a less human friendly place. %--marx sec on alienation; and my i wish i hadnt worked so hard
%Then why awesome wealth, impeccable disciplineand organization and physcial infastructure if it is not
%related to human wellbeing? 
 %
 Colombians have a  joyful, stress-less, and spontaneous way of life,
 and they feel free and are socially connected.  
 %
 We can learn from Colombia. % , but before turning to Colombian lessons for
 % happiness, we take a step back and examine counter-arguments.
 


\subsection{Takeaway for Policy and Practice}

The world has much to learn from Colombia and Latin America how to be happy. 
% This paper argues that Colombia is one of the very best countries to
% visit, and it may even be one of the best places to live.

In contrast to Colombia, we have set forth a number of criticisms about the
West, and especially the US, notably alienation. We think that this is a key
lesson and area for improvement. The US and West needs to realize its alienation
and learn from Latin America about what has been lost in the West (alienation is due to
mass industrialization and rampant capitalism that are relatively recent, several generations).

In sharp contrast to Colombia--there is the great American alienation--a key
point is that most Americans don't even seem to realize at all, give it
the slightest thought, and question the status quo and current way of
life--untill perhaps when it's too late and one is on a deathbed wishing to have
been less alienated \citep{ware12}. A trip to a place like Colombia may be necessary to open
one's eyes--when you change place, you change optics.
 %
For instance, figure \ref{terX} suggests that Philadelphia is built for
big business or capitalism, and Cali is built for people. {Figure \ref{terX} contrasts main
  public transit stations. In Cali it is much easier to interact, relax, sit
  down, buy food, find a local bus or taxi.\footnote{It is getting worse in
    Colombia, however--notably the urban landscape is being Westernized with commercial centers "centro commercial" (Western-like malls). Villages are still spared.}
}
 This is a specific lesson for the US from Colombia: to reorient its physical
infrastructure from made-for-business, formal, and awe-inspiring to
made-for-people informal, practical, cozy, friendly, and  relaxed. Do note that
many people would actually say (before reading this article) that
progress/development is to make cities in Colombia more like cities in the US. The mainstream Western ideal is Plano TX, Singapore, Dubai, etc: impeccably
clean, organized, and excellent with perfect, formal, and awe-inspiring physical
infrastructure. It is like proposing a golden cage for a lion, but no matter the amount of
gold, the lion is still happier in its natural environment that fits its needs
(livability theory). 

\begin{figure}[h]\centering\caption{Getting out of the city's main
    transportation hub (Google Maps). (Zoom in to see the details% as much as screen allows
    .)}\label{terX}
\subfloat[][Philadelphia 30th st station]{\includegraphics[width=9cm]{terU.pdf}}
\subfloat[][Cali's terminal de transporte]{\includegraphics[width=9cm]{terC.pdf}}
\end{figure}

Related, there is great Western materialism, consumerism, and obsession with status,
money, and material consumption \citep{leonard10,frank12}. In theory we know that material comfort is not everything \citep{stiglitz09al},  spending on experience such as social
connection contributes more to happiness than spending on material possessions 
\citep{ware12,vanboven05,kumar14,bhattacharjee14}, and indeed what may be needed in the West is economic
degrowth, not growth \citep{kallis12,kallis11}.
 This Western materialism, consumerism, and obsession with status and money is in contrast to and arguably at the expense of social
connection, positive freedom, and unbridled spontaneous joy that characterize
Colombia. This is another lesson from Colombia. 

 
% % Define block styles
% \tikzstyle{block} = [rectangle, draw, fill=black!20, 
%     text width=10em, text centered, rounded corners, minimum height=4em]
% \tikzstyle{b} = [rectangle, draw,  
%     text width=6em, text centered, rounded corners, minimum height=4em]
% \tikzstyle{line} = [draw, -latex']
% \tikzstyle{cloud} = [draw, ellipse,fill=black!20, node distance = 5cm,
%     minimum height=2em]
    
% \begin{tikzpicture}[node distance = 2cm, auto]
%     % Place nodes
%     \node [block] (lib) {liberalism, egalitarianism, welfare};
%     \node [block, below of=lib] (con) {conservatism, competition, individualism};
%     \node [cloud, right of=con] (ls) {well-being};
%     \node [block, below of=ls] (cul) {genes, culture};
%     \node [b, left of =lib, node distance = 4cm] (c) {country-level};
%     \node [b, left of =con,  node distance = 4cm] (c) {person-level};
%     % Draw edges
%     \path [line] (lib) -- (ls);
%     \path [line] (con) -- (ls);
%     \path [line,dashed] (cul) -- (ls);
% \end{tikzpicture}


%PUT THIS NOTE, polish and put to /root/author_what_data --ALWAYS
%stick here stuff as i run it!!! maybe comment out later...

 %\newpage
%\theendnotes
% \bibliography{/home/aok/papers/root/tex/ebib.bib,/home/aok/papers/root/old/2024/linaCommute/tex/linaCommute.bib,/home/aok/papers/root/old/2021/swbCityWorld/tex/swbCityWorld.bib,/home/aok/papers/root/old/2020/swbRes15/tex/swbRes15.bib,/home/aok/papers/root/rr/swbColCitReg/tex/swbColCitReg.bib,/home/aok/papers/root/rr/colQolSwb/tex/colQolSwb.bib}

\begin{thebibliography}{120}
\newcommand{\enquote}[1]{``#1''}
\expandafter\ifx\csname natexlab\endcsname\relax\def\natexlab#1{#1}\fi

\bibitem[\protect\citeauthoryear{Adjei and Agyei}{Adjei and
  Agyei}{2015}]{adjei15}
\textsc{Adjei, P. O.-W. and F.~K. Agyei} (2015): \enquote{Biodiversity,
  environmental health and human well-being: analysis of linkages and
  pathways,} \emph{Environment, development and sustainability}, 17,
  1085--1102.

\bibitem[\protect\citeauthoryear{Banfield}{Banfield}{1967}]{banfield67}
\textsc{Banfield, E.} (1967): \emph{The moral basis of a backward society.},
  Free Press.

\bibitem[\protect\citeauthoryear{Banfield}{Banfield}{1974}]{banfield74}
---\hspace{-.1pt}---\hspace{-.1pt}--- (1974): \emph{The unheavenly city
  revisited}, Little, Brown Boston.

\bibitem[\protect\citeauthoryear{Bargent}{Bargent}{2016}]{bargentWP16jan15}
\textsc{Bargent, J.} (2016): \enquote{Here's what we can learn from
  Colombia--the happiest nation in the world,} \emph{Washingtonpost, 15 January
  2016}.

\bibitem[\protect\citeauthoryear{Bastani}{Bastani}{2019}]{bastani19}
\textsc{Bastani, A.} (2019): \emph{Fully automated luxury communism}, Verso
  Books.

\bibitem[\protect\citeauthoryear{Bhattacharjee and Mogilner}{Bhattacharjee and
  Mogilner}{2014}]{bhattacharjee14}
\textsc{Bhattacharjee, A. and C.~Mogilner} (2014): \enquote{Happiness from
  ordinary and extraordinary experiences,} \emph{Journal of Consumer Research},
  41, 1--17.

\bibitem[\protect\citeauthoryear{Brickman, Coates, and Janoff-Buman}{Brickman
  et~al.}{1978}]{brickman78cj}
\textsc{Brickman, P., D.~Coates, and R.~Janoff-Buman} (1978): \enquote{Lottery
  winners and accident victims: Is happiness relative?} \emph{Journal of
  Personality and Social Psychology}, 36, 917--927.

\bibitem[\protect\citeauthoryear{Brooks}{Brooks}{2022}]{brooksATL22nov17}
\textsc{Brooks, A.~C.} (2022): \enquote{Meetings Are Miserable. One of the most
  straightforward paths to happiness at work is to fight against the scourge of
  time-consuming, unproductive meetings at every opportunity.} \emph{The
  Atlantic}.

\bibitem[\protect\citeauthoryear{Brown and Kasser}{Brown and
  Kasser}{2005}]{brown05}
\textsc{Brown, K.~W. and T.~Kasser} (2005): \enquote{Are psychological and
  ecological well-being compatible? The role of values, mindfulness, and
  lifestyle,} \emph{Social Indicators Research}, 74, 349--368.

\bibitem[\protect\citeauthoryear{Catropa and Andrews}{Catropa and
  Andrews}{2020}]{CatropaNYT20feb8}
\textsc{Catropa, D. and M.~Andrews} (2020): \enquote{Bemoaning the
  Corporatization of Higher Education,} \emph{insidehighered.com, 8 February
  2020}.

\bibitem[\protect\citeauthoryear{{Central Intelligence Agency}}{{Central
  Intelligence Agency}}{2017}]{cia}
\textsc{{Central Intelligence Agency}} (2017): \enquote{Country Comparison:
  Life Expectancy at Birth,} \emph{The World Factbook}.

\bibitem[\protect\citeauthoryear{Chandran}{Chandran}{2017}]{chandranCNBC17oct24}
\textsc{Chandran, N.} (2017): \enquote{Global passport rankings: US loses
  power, tiny Southeast Asian country takes top spot,} \emph{CNBC}.

\bibitem[\protect\citeauthoryear{Chang, Corey, Rostron, and Apelberg}{Chang
  et~al.}{2015}]{chang15}
\textsc{Chang, C.~M., C.~G. Corey, B.~L. Rostron, and B.~J. Apelberg} (2015):
  \enquote{Systematic review of cigar smoking and all cause and smoking related
  mortality,} \emph{BMC Public Health}, 15, 1--20.

\bibitem[\protect\citeauthoryear{Chomsky}{Chomsky}{2009}]{chomsky09}
\textsc{Chomsky, N.} (2009): \enquote{Noam Chomsky discusses Erich Fromm's
  theory of alienation,} \emph{commons.wikimedia.org}.

\bibitem[\protect\citeauthoryear{Clydesdale}{Clydesdale}{2022}]{singapore22nanny}
\textsc{Clydesdale, H.} (2022): \enquote{Singapore: Tough Love in the Nanny
  State,} \emph{asiasociety.org}.

\bibitem[\protect\citeauthoryear{Corak}{Corak}{2004}]{corak04}
\textsc{Corak, M.} (2004): \emph{Generational income mobility in North America
  and Europe}, Cambridge University Press, New York NY.

\bibitem[\protect\citeauthoryear{Corak}{Corak}{2011}]{corak11}
---\hspace{-.1pt}---\hspace{-.1pt}--- (2011): \enquote{Inequality from
  generation to generation: the United States in comparison,} \emph{The
  Economics of Inequality, Poverty, and Discrimination in the 21st Century}.

\bibitem[\protect\citeauthoryear{Corak}{Corak}{2013}]{corak13}
---\hspace{-.1pt}---\hspace{-.1pt}--- (2013): \enquote{Income Inequality,
  Equality of Opportunity, and Intergenerational Mobility,} \emph{The Journal
  of Economic Perspectives}, 79--102.

\bibitem[\protect\citeauthoryear{Coughlan}{Coughlan}{2016}]{coughlanBBC17dec6}
\textsc{Coughlan, S.} (2016): \enquote{Pisa tests: Singapore top in global
  education rankings,} \emph{BBC, 6 December 2016}.

\bibitem[\protect\citeauthoryear{Cox}{Cox}{2013}]{cox2013corporatization}
\textsc{Cox, R.~W.} (2013): \enquote{The corporatization of higher education,}
  \emph{Class, race and corporate power}, 1, 8.

\bibitem[\protect\citeauthoryear{Davies}{Davies}{2022}]{motodreamer}
\textsc{Davies, F.} (2022): \enquote{Colombia--home to the World's happiest
  people,} \emph{motodreamer.com}.

\bibitem[\protect\citeauthoryear{Davies}{Davies}{2015}]{davies15}
\textsc{Davies, W.} (2015): \emph{The Happiness Industry: How the Government
  and Big Business Sold us Well-Being}, Verso Books.

\bibitem[\protect\citeauthoryear{de~Veyra}{de~Veyra}{2022}]{singapore22laws}
\textsc{de~Veyra, D.~M.} (2022): \enquote{Singapore: Laws To Know Before You
  Go,} \emph{goabroad.com}.

\bibitem[\protect\citeauthoryear{De~Vos, Schwanen, Van~Acker, and
  Witlox}{De~Vos et~al.}{2013}]{devos13}
\textsc{De~Vos, J., T.~Schwanen, V.~Van~Acker, and F.~Witlox} (2013):
  \enquote{Travel and subjective well-being: A focus on findings, methods and
  future research needs,} \emph{Transport Reviews}, 33, 421--442.

\bibitem[\protect\citeauthoryear{Diener}{Diener}{2009}]{diener09}
\textsc{Diener, E.} (2009): \emph{Well-being for public policy}, Oxford
  University Press, New York NY.

\bibitem[\protect\citeauthoryear{Dittmar, Bond, Hurst, and Kasser}{Dittmar
  et~al.}{2014}]{dittmar14}
\textsc{Dittmar, H., R.~Bond, M.~Hurst, and T.~Kasser} (2014): \enquote{The
  relationship between materialism and personal well-being: A meta-analysis.}
  \emph{Journal of personality and social psychology}, 107, 879.

\bibitem[\protect\citeauthoryear{Duany, Plater-Zyberk, and Speck}{Duany
  et~al.}{2001}]{duany01}
\textsc{Duany, A., E.~Plater-Zyberk, and J.~Speck} (2001): \emph{Suburban
  nation: The rise of sprawl and the decline of the American dream}, North
  Point Press, New York NY.

\bibitem[\protect\citeauthoryear{Easterlin}{Easterlin}{1973}]{easterlin73}
\textsc{Easterlin, R.~A.} (1973): \enquote{Does money buy happiness?} \emph{The
  public interest}, 30, 3.

\bibitem[\protect\citeauthoryear{Economist}{Economist}{2012{\natexlab{a}}}]{economist12_oct13_B}
\textsc{Economist, T.} (2012{\natexlab{a}}): \enquote{For richer, for poorer,}
  \emph{The Economist, 13 October 2012}.

\bibitem[\protect\citeauthoryear{Economist}{Economist}{2012{\natexlab{b}}}]{economist12_oct13}
---\hspace{-.1pt}---\hspace{-.1pt}--- (2012{\natexlab{b}}): \enquote{True
  Progressivism,} \emph{The Economist, 13 October 2012}.

\bibitem[\protect\citeauthoryear{Economist}{Economist}{2013}]{economist12_oct12}
---\hspace{-.1pt}---\hspace{-.1pt}--- (2013): \enquote{Having your cake; Less
  inequality does not need to mean less efficiency,} \emph{The Economist, 12
  October 2012}.

\bibitem[\protect\citeauthoryear{Elkin}{Elkin}{1994}]{elkinCA94}
\textsc{Elkin, E.~J.} (1994): \enquote{More Than a Cigar,} \emph{Cigar
  Afficionado, 31 December 1994}.

\bibitem[\protect\citeauthoryear{Fischer}{Fischer}{1973}]{fischer73}
\textsc{Fischer, C.~S.} (1973): \enquote{Urban malaise,} \emph{Social Forces},
  52, 221--235.

\bibitem[\protect\citeauthoryear{Frank}{Frank}{2012}]{frank12}
\textsc{Frank, R.} (2012): \emph{The Darwin economy: Liberty, competition, and
  the common good}, Princeton University Press, Princeton NJ.

\bibitem[\protect\citeauthoryear{Freud, Riviere, and Strachey}{Freud
  et~al.}{1930}]{freud30}
\textsc{Freud, S., J.~Riviere, and J.~Strachey} (1930): \emph{Civilization and
  its discontents}, Hogarth Press London.

\bibitem[\protect\citeauthoryear{Fromm}{Fromm}{[1941] 1994}]{fromm94}
\textsc{Fromm, E.} ([1941] 1994): \emph{Escape from freedom}, Holt Paperbacks.

\bibitem[\protect\citeauthoryear{Fromm}{Fromm}{1964}]{fromm64}
---\hspace{-.1pt}---\hspace{-.1pt}--- (1964): \emph{The heart of man: Its
  genius for good and evil}, vol.~12, Taylor \& Francis.

\bibitem[\protect\citeauthoryear{Fromm}{Fromm}{2012}]{fromm12}
---\hspace{-.1pt}---\hspace{-.1pt}--- (2012): \emph{The sane society},
  Routledge, New York NY.

\bibitem[\protect\citeauthoryear{Fromm}{Fromm}{2013}]{fromm13}
---\hspace{-.1pt}---\hspace{-.1pt}--- (2013): \emph{To have or to be?}, A\&C
  Black.

\bibitem[\protect\citeauthoryear{Fullman, Barber, Abajobir, Abate, Abbafati,
  Abbas, Abd-Allah, Abdulkader, Abdulle, Abera et~al.}{Fullman
  et~al.}{2017}]{fullman2017measuring}
\textsc{Fullman, N., R.~M. Barber, A.~A. Abajobir, K.~H. Abate, C.~Abbafati,
  K.~M. Abbas, F.~Abd-Allah, R.~S. Abdulkader, A.~M. Abdulle, S.~F. Abera,
  et~al.} (2017): \enquote{Measuring progress and projecting attainment on the
  basis of past trends of the health-related Sustainable Development Goals in
  188 countries: an analysis from the Global Burden of Disease Study 2016,}
  \emph{The Lancet}, 390, 1423--1459.

\bibitem[\protect\citeauthoryear{Garc{\'\i}a-Quero and
  Guardiola}{Garc{\'\i}a-Quero and Guardiola}{2018}]{garcia18}
\textsc{Garc{\'\i}a-Quero, F. and J.~Guardiola} (2018): \enquote{Economic
  poverty and happiness in rural Ecuador: The importance of Buen Vivir (living
  well),} \emph{Applied Research in Quality of Life}, 13, 909--926.

\bibitem[\protect\citeauthoryear{Gershuny}{Gershuny}{2005}]{gershuny2005busyness}
\textsc{Gershuny, J.} (2005): \enquote{Busyness as the badge of honor for the
  new superordinate working class,} \emph{Social Research: An International
  Quarterly}, 72, 287--314.

\bibitem[\protect\citeauthoryear{Graham and Felton}{Graham and
  Felton}{2006}]{graham05}
\textsc{Graham, C. and A.~Felton} (2006): \enquote{Inequality and happiness:
  Insights from Latin America,} \emph{Journal of Economic Inequality}, 4,
  107--122.

\bibitem[\protect\citeauthoryear{Guardiola and Garc{\'\i}a-Quero}{Guardiola and
  Garc{\'\i}a-Quero}{2014}]{guardiola14}
\textsc{Guardiola, J. and F.~Garc{\'\i}a-Quero} (2014): \enquote{Buen Vivir
  (living well) in Ecuador: Community and environmental satisfaction without
  household material prosperity?} \emph{Ecological Economics}, 107, 177--184.

\bibitem[\protect\citeauthoryear{Guendelsberger}{Guendelsberger}{2019}]{guendelsbergerBI19aug9}
\textsc{Guendelsberger, E.} (2019): \enquote{I'm a writer who went to work at
  an Amazon warehouse, a call center, and a McDonald's. I saw firsthand how
  low-wage work is driving America over the edge.} \emph{Business Insider, 9
  August 2019}.

\bibitem[\protect\citeauthoryear{Guti{\'e}rrez and Cunin}{Guti{\'e}rrez and
  Cunin}{2006}]{gutierrez2006fiestas}
\textsc{Guti{\'e}rrez, E. and E.~Cunin} (2006): \enquote{Fiestas y carnavales
  en Colombia. La puesta en escena de las identidades,} \emph{Medell{\'\i}n:
  Universidad de Cartagena, Institut de Recherche pour le d{\'e}veloppement, la
  Carreta Editores}.

\bibitem[\protect\citeauthoryear{Heritage}{Heritage}{2017}]{heritage}
\textsc{Heritage} (2017): \enquote{2017 Index of Economic Freedom,}
  \emph{Heritage}.

\bibitem[\protect\citeauthoryear{Hickel}{Hickel}{2020}]{hickel20}
\textsc{Hickel, J.} (2020): \emph{Less is more: How degrowth will save the
  world}, Random House.

\bibitem[\protect\citeauthoryear{Hidalgo-Capit{\'a}n and
  Cubillo-Guevara}{Hidalgo-Capit{\'a}n and Cubillo-Guevara}{2017}]{hidalgo17}
\textsc{Hidalgo-Capit{\'a}n, A.~L. and A.~P. Cubillo-Guevara} (2017):
  \enquote{Deconstruction and genealogy of Latin American good living (Buen
  Vivir). The (triune) good living and its diverse intellectual wellsprings,}
  in \emph{Alternative pathways to sustainable development: Lessons from Latin
  America}, 23--50.

\bibitem[\protect\citeauthoryear{Horowitz}{Horowitz}{2022}]{horowitzMISC10}
\textsc{Horowitz, A.} (2022): \enquote{March 1 - Marx's Theory of Alienation,}
  \emph{AS/POLS 2900.6A Perspectives on Politics 2010-11}.

\bibitem[\protect\citeauthoryear{Hurtado}{Hurtado}{2016}]{hurtado2016socioeconomic}
\textsc{Hurtado, D.~A.} (2016): \enquote{Socioeconomic disparities in
  subjective well-being in Colombia,} in \emph{Handbook of happiness research
  in Latin America}, Springer, 343--356.

\bibitem[\protect\citeauthoryear{Hurtado, Hessel, and Avendano}{Hurtado
  et~al.}{2017}]{hurtado2017hidden}
\textsc{Hurtado, D.~A., P.~Hessel, and M.~Avendano} (2017): \enquote{The hidden
  costs of informal work: lack of social protection and subjective well-being
  in Colombia,} \emph{International journal of public health}, 62, 187--196.

\bibitem[\protect\citeauthoryear{IMF}{IMF}{2017}]{authorNYT17monD}
\textsc{IMF} (2017): \enquote{World Economic Outlook Database,}
  \emph{International Monetary Fund}.

\bibitem[\protect\citeauthoryear{{International Crisis Group}}{{International
  Crisis Group}}{2021}]{icg21}
\textsc{{International Crisis Group}} (2021): \emph{The Pandemic Strikes:
  Responding to Colombia's Mass Protests.}, Latin America Report N.90.

\bibitem[\protect\citeauthoryear{Kallis}{Kallis}{2011}]{kallis11}
\textsc{Kallis, G.} (2011): \enquote{In defence of degrowth,} \emph{Ecological
  Economics}, 70, 873--880.

\bibitem[\protect\citeauthoryear{Kallis, Kerschner, and Martinez-Alier}{Kallis
  et~al.}{2012}]{kallis12}
\textsc{Kallis, G., C.~Kerschner, and J.~Martinez-Alier} (2012): \enquote{The
  economics of degrowth,} \emph{Ecological Economics}, 84, 172--180.

\bibitem[\protect\citeauthoryear{Kasser}{Kasser}{2003}]{kasser13}
\textsc{Kasser, T.} (2003): \emph{The high price of materialism}, MIT press.

\bibitem[\protect\citeauthoryear{Kasser}{Kasser}{2016}]{kasser16}
---\hspace{-.1pt}---\hspace{-.1pt}--- (2016): \enquote{Materialistic values and
  goals,} \emph{Annual review of psychology}, 67, 489--514.

\bibitem[\protect\citeauthoryear{Kasser and Ryan}{Kasser and
  Ryan}{1993}]{kasser93}
\textsc{Kasser, T. and R.~Ryan} (1993): \enquote{A dark side of the American
  dream: correlates of financial success as a central life aspiration.}
  \emph{Journal of personality and social psychology}, 65, 410.

\bibitem[\protect\citeauthoryear{Klein}{Klein}{2014}]{klein14}
\textsc{Klein, N.} (2014): \emph{This changes everything: capitalism vs. the
  climate}, Simon and Schuster, New York NY.

\bibitem[\protect\citeauthoryear{Kumar, Killingsworth, and Gilovich}{Kumar
  et~al.}{2014}]{kumar14}
\textsc{Kumar, A., M.~A. Killingsworth, and T.~Gilovich} (2014):
  \enquote{Waiting for Merlot Anticipatory Consumption of Experiential and
  Material Purchases,} \emph{Psychological science}, 1924--1931.

\bibitem[\protect\citeauthoryear{Lambert, Lomas, van~de Weijer, Passmore,
  Joshanloo, Harter, Ishikawa, Lai, Kitagawa, Chen et~al.}{Lambert
  et~al.}{2020}]{lambert2020towards}
\textsc{Lambert, L., T.~Lomas, M.~P. van~de Weijer, H.~A. Passmore,
  M.~Joshanloo, J.~Harter, Y.~Ishikawa, A.~Lai, T.~Kitagawa, D.~Chen, et~al.}
  (2020): \enquote{Towards a greater global understanding of wellbeing: A
  proposal for a more inclusive measure,} \emph{International Journal of
  Wellbeing}, 10.

\bibitem[\protect\citeauthoryear{Lane}{Lane}{2000}]{lane00}
\textsc{Lane, R.~E.} (2000): \emph{The loss of happiness in market
  democracies}, New Haven CT: Yale University Press.

\bibitem[\protect\citeauthoryear{Leonard}{Leonard}{2010}]{leonard10}
\textsc{Leonard, A.} (2010): \emph{The story of stuff: How our obsession with
  stuff is trashing the planet, our communities, and our health-and a vision
  for change}, Simon and Schuster.

\bibitem[\protect\citeauthoryear{Manson}{Manson}{2015}]{manson15}
\textsc{Manson, M.} (2015): \emph{The subtle art of not giving a fuck}, Harper
  One.

\bibitem[\protect\citeauthoryear{Marcuse}{Marcuse}{2015}]{marcuse15}
\textsc{Marcuse, H.} (2015): \emph{Eros and civilization: A philosophical
  inquiry into Freud}, Boston MA: Beacon Press.

\bibitem[\protect\citeauthoryear{Mart{\'\i}nez}{Mart{\'\i}nez}{2017}]{martinez17}
\textsc{Mart{\'\i}nez, L.} (2017): \enquote{Life satisfaction data in a
  developing country: CaliBRANDO measurement system,} \emph{Data in brief}, 13,
  600.

\bibitem[\protect\citeauthoryear{Mart{\'\i}nez and Short}{Mart{\'\i}nez and
  Short}{2020}]{martinez20}
\textsc{Mart{\'\i}nez, L. and J.~R. Short} (2020): \enquote{Life satisfaction
  in the city,} \emph{Scienze Regionali}, 417--438.

\bibitem[\protect\citeauthoryear{Marx}{Marx}{[1867] 2010}]{marx10}
\textsc{Marx, K.} ([1867] 2010): \emph{Capital, vol. 1},
  http://www.marxists.org.

\bibitem[\protect\citeauthoryear{Maslow}{Maslow}{[1954] 1987}]{maslow87}
\textsc{Maslow, A.} ([1954] 1987): \emph{{Motivation and personality}},
  Longman, 3 ed.

\bibitem[\protect\citeauthoryear{McMahon}{McMahon}{2006}]{mcmahon06}
\textsc{McMahon, D.~M.} (2006): \emph{Happiness: A history}, Grove Pr.

\bibitem[\protect\citeauthoryear{McSpadden}{McSpadden}{2015}]{mcspaddenNYT17monD}
\textsc{McSpadden, K.} (2015): \enquote{Singapore has the world's fastest
  Internet: Akamai,} \emph{E27.co, 17 December 2015}.

\bibitem[\protect\citeauthoryear{Mensing}{Mensing}{2002}]{mensing2002collectivism}
\textsc{Mensing, J.~F.} (2002): \emph{Collectivism, individualism and
  interpersonal responsibilities in families: Differences and similarities in
  social reasoning between individuals in poor, urban families in Colombia and
  the United States}, University of California, Berkeley.

\bibitem[\protect\citeauthoryear{Michalos}{Michalos}{1985}]{michalos85}
\textsc{Michalos, A.} (1985): \enquote{Multiple discrepancies theory (MDT),}
  \emph{Social Indicators Research}, 16, 347--413.

\bibitem[\protect\citeauthoryear{Michalos}{Michalos}{2014}]{michalos14B}
\textsc{Michalos, A.~C.} (2014): \enquote{Quality of Life, Two-Variable
  Theory,} in \emph{Encyclopedia of Quality of Life and Well-Being Research},
  Springer, 5307--5309.

\bibitem[\protect\citeauthoryear{Mills}{Mills}{2012{\natexlab{a}}}]{mills2012corporatization}
\textsc{Mills, N.} (2012{\natexlab{a}}): \enquote{The corporatization of higher
  education,} \emph{Dissent}, 59, 6--9.

\bibitem[\protect\citeauthoryear{Mills}{Mills}{2012{\natexlab{b}}}]{millsNYT12fa}
---\hspace{-.1pt}---\hspace{-.1pt}--- (2012{\natexlab{b}}): \enquote{The
  Corporatization of Higher Education.} \emph{dissentmagazine.org, Fall 2012}.

\bibitem[\protect\citeauthoryear{Morawetz, Atia, Bin-Nun, Felous, Gariplerden,
  Harris, Soustiel, Tombros, and Zarfaty}{Morawetz et~al.}{1977}]{morawetz77}
\textsc{Morawetz, D., E.~Atia, G.~Bin-Nun, L.~Felous, Y.~Gariplerden,
  E.~Harris, S.~Soustiel, G.~Tombros, and Y.~Zarfaty} (1977): \enquote{Income
  distribution and self-rated happiness: some empirical evidence,} \emph{The
  economic journal}, 511--522.

\bibitem[\protect\citeauthoryear{Moyer}{Moyer}{2022}]{moyerMISC21apr3}
\textsc{Moyer, E.} (2022): \enquote{Amazon apologizes, says 'peeing in bottles
  thing' is actually a thing for its drivers The company says a tweet that
  suggested otherwise was 'incorrect',} \emph{Cnet, 3 April 2022}.

\bibitem[\protect\citeauthoryear{Musk}{Musk}{2018}]{muskIN18nov26}
\textsc{Musk, E.} (2018): \enquote{A person needs to work 80-100 hours per week
  to 'change the world',} \emph{LinkedIN Pulse}.

\bibitem[\protect\citeauthoryear{Nietzsche}{Nietzsche}{1896}]{nietzsche05alt}
\textsc{Nietzsche, F.~W.} (1896): \emph{Thus spake Zarathustra: a book for all
  and none}, MacMillan and Company, New York NY.

\bibitem[\protect\citeauthoryear{Okulicz-Kozaryn}{Okulicz-Kozaryn}{2014}]{aok_free_from_to}
\textsc{Okulicz-Kozaryn, A.} (2014): \enquote{'Freedom from' and 'freedom to'
  across countries,} \emph{Social Indicators Research}, 118, 1009--1029.

\bibitem[\protect\citeauthoryear{Okulicz-Kozaryn}{Okulicz-Kozaryn}{2015}]{ee_ls15}
---\hspace{-.1pt}---\hspace{-.1pt}--- (2015): \enquote{Freedom and Life
  Satisfaction in Transition,} \emph{Society and Economy in Central and Eastern
  Europe}, 37, 143--164.

\bibitem[\protect\citeauthoryear{Okulicz-Kozaryn}{Okulicz-Kozaryn}{2020}]{gssLonnieRubia}
---\hspace{-.1pt}---\hspace{-.1pt}--- (2020): \enquote{The Top Regrets Of The
  Dying: ``I Wish I Hadn't Worked So Hard.'' (Greed Is Good For Economy, But
  Not For Human Wellbeing),} \emph{Unpublished}.

\bibitem[\protect\citeauthoryear{Okulicz-Kozaryn and Valente}{Okulicz-Kozaryn
  and Valente}{2019}]{aok-swbLivability18}
\textsc{Okulicz-Kozaryn, A. and R.~R. Valente} (2019): \enquote{Livability and
  subjective well-being across European cities,} \emph{Applied Research in
  Quality of Life}, 14, 197--220.

\bibitem[\protect\citeauthoryear{Pachauri, Allen, Barros, Broome, Cramer,
  Christ, Church, Clarke, Dahe, Dasgupta et~al.}{Pachauri
  et~al.}{2014}]{pachauri14}
\textsc{Pachauri, R.~K., M.~Allen, V.~Barros, J.~Broome, W.~Cramer, R.~Christ,
  J.~Church, L.~Clarke, Q.~Dahe, P.~Dasgupta, et~al.} (2014): \emph{Climate
  Change 2014: Synthesis Report. Contribution of Working Groups I, II and III
  to the Fifth Assessment Report of the Intergovernmental Panel on Climate
  Change}, IPCC.

\bibitem[\protect\citeauthoryear{Pile}{Pile}{2005{\natexlab{a}}}]{pile05}
\textsc{Pile, S.} (2005{\natexlab{a}}): \emph{Real cities: modernity, space and
  the phantasmagorias of city life}, Sage, Beverly Hills CA.

\bibitem[\protect\citeauthoryear{Pile}{Pile}{2005{\natexlab{b}}}]{pile05B}
---\hspace{-.1pt}---\hspace{-.1pt}--- (2005{\natexlab{b}}): \enquote{Spectral
  Cities: Where the Repressed Returns and Other Short Stories,} in
  \emph{Habitus: A sense of place}, ed. by J.~Hillier and E.~Rooksby, Ashgate
  Aldershot.

\bibitem[\protect\citeauthoryear{PNUD}{PNUD}{2023}]{PNUD}
\textsc{PNUD} (2023): \emph{Informe sobre Desarrollo Humano para Colombia,
  cuaderno 2. Percepciones y bienestar subjetivo en Colombia: mas alla de
  indicadores tradicionales.},
  https://www.undp.org/es/colombia/publicaciones/informe-desarrollo-humano-colombia-cuaderno-2.

\bibitem[\protect\citeauthoryear{Prescott and Logan}{Prescott and
  Logan}{2017}]{prescott17}
\textsc{Prescott, S.~L. and A.~C. Logan} (2017): \emph{The secret life of your
  microbiome: Why nature and biodiversity are essential to health and
  happiness}, New Society Publishers.

\bibitem[\protect\citeauthoryear{Pretty}{Pretty}{2012}]{pretty12}
\textsc{Pretty, J.} (2012): \emph{The earth only endures: On reconnecting with
  nature and our place in it}, Routledge, New York NY.

\bibitem[\protect\citeauthoryear{Putnam}{Putnam}{2001}]{putnam01}
\textsc{Putnam, R.~D.} (2001): \emph{Bowling Alone: The Collapse and Revival of
  American Community}, New York, NY: Simon \& Schuster.

\bibitem[\protect\citeauthoryear{Rojas}{Rojas}{2015}]{rojas15}
\textsc{Rojas, M.} (2015): \emph{Handbook of Happiness Research in Latin
  America}, Springer.

\bibitem[\protect\citeauthoryear{Rojas}{Rojas}{2019}]{rojas2019well}
---\hspace{-.1pt}---\hspace{-.1pt}--- (2019): \emph{Well-being in Latin
  America: Drivers and policies}, Springer Nature.

\bibitem[\protect\citeauthoryear{Roos}{Roos}{2019}]{roosHUFF19mar26}
\textsc{Roos, D.} (2019): \enquote{Colombia, Not Finland, May Be the Happiest
  Country in the World,} \emph{howstuffworks.com, 26 March 2019}.

\bibitem[\protect\citeauthoryear{Rubiano}{Rubiano}{2022}]{rubianoNYT22nov16}
\textsc{Rubiano, M.~P.} (2022): \enquote{How Colombia plans to keep its oil and
  coal in the ground,} \emph{BBC, 16 November 2022}.

\bibitem[\protect\citeauthoryear{Schmidlin}{Schmidlin}{2015}]{schmidlinNYT15oct10}
\textsc{Schmidlin, K.} (2015): \enquote{The corporatization of higher
  education: With a system that caters to the 1 percent, students and faculty
  get screwed. Low-paid teachers are fighting back against exploitation in
  public \& private colleges. No more poverty wages,} \emph{salon.com, 10
  October 2015}.

\bibitem[\protect\citeauthoryear{Schmuck, Kasser, and Ryan}{Schmuck
  et~al.}{2000}]{schmuck00}
\textsc{Schmuck, P., T.~Kasser, and R.~M. Ryan} (2000): \enquote{Intrinsic and
  extrinsic goals: Their structure and relationship to well-being in German and
  US college students,} \emph{Social Indicators Research}, 50, 225--241.

\bibitem[\protect\citeauthoryear{Scitovsky}{Scitovsky}{1976}]{scitovsky76}
\textsc{Scitovsky, T.} (1976): \emph{The joyless economy: An inquiry into human
  satisfaction and consumer dissatisfaction.}, Oxford U Press, New York NY.

\bibitem[\protect\citeauthoryear{Sirgy}{Sirgy}{2002}]{sirgy02}
\textsc{Sirgy, M.~J.} (2002): \emph{The psychology of quality of life},
  vol.~12, Springer.

\bibitem[\protect\citeauthoryear{Starmans, Sheskin, and Bloom}{Starmans
  et~al.}{2017}]{starmans17}
\textsc{Starmans, C., M.~Sheskin, and P.~Bloom} (2017): \enquote{Why people
  prefer unequal societies,} \emph{Nature Human Behaviour}, 1, 1--7.

\bibitem[\protect\citeauthoryear{Stiglitz, Sen, and Fitoussi}{Stiglitz
  et~al.}{2009}]{stiglitz09al}
\textsc{Stiglitz, J., A.~Sen, and J.~Fitoussi} (2009): \enquote{Report by the
  Commission on the measurement of economic performance and social progress,}
  \emph{Available at www.stiglitz-sen-fitoussi.fr}.

\bibitem[\protect\citeauthoryear{Streitfeld and Kantor}{Streitfeld and
  Kantor}{2015}]{streitfeldNYT15aug17}
\textsc{Streitfeld, D. and J.~Kantor} (2015): \enquote{Inside Amazon: Wrestling
  Big Ideas in a Bruising Workplace,} \emph{The New York Times, 17 August
  2015}.

\bibitem[\protect\citeauthoryear{Surowiecki}{Surowiecki}{2005}]{surowiecki2005technology}
\textsc{Surowiecki, J.} (2005): \enquote{Technology and Happiness: Why getting
  more gadgets won't necessarily increase our well-being,} \emph{Technology
  review}.

\bibitem[\protect\citeauthoryear{Tesson}{Tesson}{2013}]{tesson13}
\textsc{Tesson, S.} (2013): \emph{Consolations of the Forest: Alone in a Cabin
  in the Middle Taiga}, Penguin, London UK.

\bibitem[\protect\citeauthoryear{Thoreau}{Thoreau}{1995 [1854]}]{thoreau95}
\textsc{Thoreau, H.~D.} (1995 [1854]): \emph{Walden}, Dover Publications,
  Mineola NY.

\bibitem[\protect\citeauthoryear{T{\"o}nnies}{T{\"o}nnies}{[1887]
  2002}]{tonnies57}
\textsc{T{\"o}nnies, F.} ([1887] 2002): \emph{Community and society},
  DoverPublications.com, Mineola NY.

\bibitem[\protect\citeauthoryear{Turkewitz}{Turkewitz}{2021}]{turkewitzNYT21sep26}
\textsc{Turkewitz, J.} (2021): \enquote{Five Years After Peace Deal, Colombia
  Is Running Out of Time, Experts Say. A treaty with rebels in 2016 called for
  the end of a decades-long war. But that is not the same as achieving peace,
  and the window for doing so may be closing.} \emph{The New York Times, 26
  September 2021}.

\bibitem[\protect\citeauthoryear{Van~Boven}{Van~Boven}{2005}]{vanboven05}
\textsc{Van~Boven, L.} (2005): \enquote{Experientialism, materialism, and the
  pursuit of happiness,} \emph{Review of general psychology}, 9, 132--142.

\bibitem[\protect\citeauthoryear{van Zanden, Rijpma, Malinowski, Mira~d'Ercole
  et~al.}{van Zanden et~al.}{2020}]{van2020s}
\textsc{van Zanden, J.~L., A.~Rijpma, M.~Malinowski, M.~Mira~d'Ercole, et~al.}
  (2020): \enquote{How's Life? 2020: Measuring Well-being,} .

\bibitem[\protect\citeauthoryear{Veenhoven}{Veenhoven}{2000}]{veenhoven00b}
\textsc{Veenhoven, R.} (2000): \enquote{The four qualities of life,}
  \emph{Journal of happiness studies}, 1, 1--39.

\bibitem[\protect\citeauthoryear{Veenhoven}{Veenhoven}{2014}]{veenhoven14b}
---\hspace{-.1pt}---\hspace{-.1pt}--- (2014): \enquote{Livability Theory,}
  \emph{Encyclopedia of Quality of Life and Well-Being Research}, 3645--3647.

\bibitem[\protect\citeauthoryear{Veenhoven and Ehrhardt}{Veenhoven and
  Ehrhardt}{1995}]{veenhoven95}
\textsc{Veenhoven, R. and J.~Ehrhardt} (1995): \enquote{The Cross-National
  Pattern of Happiness: Test of Predictions Implied in Three Theories of
  Happiness,} \emph{Social Indicators Research}, 34, 33--68.

\bibitem[\protect\citeauthoryear{Vel{\'a}squez}{Vel{\'a}squez}{2016}]{velasquez2016importance}
\textsc{Vel{\'a}squez, L.} (2016): \enquote{The importance of relational goods
  for happiness: Evidence from Manizales, Colombia,} in \emph{Handbook of
  happiness research in Latin America}, Springer, 91--112.

\bibitem[\protect\citeauthoryear{Wallace}{Wallace}{2017}]{wallaceBBC17mar30}
\textsc{Wallace, C.} (2017): \enquote{The world's happiest country?} \emph{BBC,
  30 March 2017}.

\bibitem[\protect\citeauthoryear{Ware}{Ware}{2012}]{ware12}
\textsc{Ware, B.} (2012): \emph{The top five regrets of the dying: A life
  transformed by the dearly departing}, Hay House, Inc.

\bibitem[\protect\citeauthoryear{Wheeler}{Wheeler}{2015}]{wheelerIN15aug6}
\textsc{Wheeler, M.} (2015): \enquote{Why Raising Employee Wages Sometimes
  Backfires,} \emph{LinkedIN Pulse}.

\bibitem[\protect\citeauthoryear{Wilson}{Wilson}{2021}]{wilson21}
\textsc{Wilson, E.~O.} (2021): \enquote{Biophilia,} in \emph{Biophilia},
  Harvard university press.

\bibitem[\protect\citeauthoryear{{World Economic Forum}}{{World Economic
  Forum}}{2017}]{world2017travel}
\textsc{{World Economic Forum}} (2017): \enquote{The travel \& tourism
  competitiveness report,} World Economic Forum,
  \url{https://www3.weforum.org/docs/WEF_TTCR_2017_web_0401.pdf}.

\bibitem[\protect\citeauthoryear{Yamamoto}{Yamamoto}{2016}]{yamamoto16}
\textsc{Yamamoto, J.} (2016): \enquote{The social psychology of Latin American
  happiness,} \emph{Handbook of happiness research in Latin America}, 31--49.

\end{thebibliography}


\includepdf[pages={-}]{som-colQolSwb.pdf}



\end{spacing}
\end{document}




sth wrong my wvs shows ls flat and ruut veenhoven shows huge growth (but his conversion to 1-10 doesn't, so maybe his 1-10 is right lol) and hist first measure (wvs!) on page https://worlddatabaseofhappiness-archive.eur.nl/hap_nat/desc_na_genpublic.php?cntry=122 shows 3.3 in 1998!!
%table centered on decimal points:)
\begin{table}[H]\centering\footnotesize
\caption{\label{t1} Colombia's happiness on scale 1-4 over time and on
  scale 1-10 across space (23 happiest countries in the world) . Data from World Database of Happiness at \url{https://worlddatabaseofhappiness-archive.eur.nl/hap_nat/desc_na_genpublic.php?cntry=122}, originally from \citet{graham02,graham01,graham05} and Latinobarometro.}
\begin{tabular} {@{} lr||rrrrrr @{}}   \hline 
year& Colombian happiness (1-4) &happiness rank& country& happiness (1-10) \\ \hline
1997&2.5    & 1 	&	Denmark	        &	8.2      \\                          
2000&2.4    & 2 	&	Mexico	        &	8.1      \\                          
2001&3.06   & 3 	&	Colombia	&	8.1      \\                  
2003&3.16   & 4 	&	Switzerland	&	8        \\                  
2004&3.14   & 5 	&	Finland	        &	8        \\                          
2004&3.45   & 6 	&	Iceland	        &	8        \\                          
2005&3.17   & 7 	&	Costa Rica	&	7.9      \\                  
2005&3.36   & 8 	&	Norway	        &	7.9      \\                          
2006&3.25   & 9 	&	Canada	        &	7.9      \\                          
2006&3.4    & 10	&	Qatar         	&	7.8      \\                          
2007&3.36   & 11	&	Sweden	        &	7.8      \\                          
2008&3.39   & 12	&	Austria    	&	7.7      \\                          
2009&3.3    & 13	&	Nicaragua	&	7.7      \\                  
2009&3.4    & 14	&	Uzbekistan	&	7.7      \\                  
2010&3.46   & 15	&	Netherlands	&	7.6      \\                  
2010&3.18   & 16	&	Ecuador  	&	7.6      \\                          
2011&3.22   & 17	&	Israel  	&	7.6      \\                          
2013&3.3    & 18	&	Luxembourg	&	7.6      \\                  
2015&3.3    & 19	&	Belgium 	&	7.5      \\                          
2016&3.28   & 20	&	United Arab Emirate&	7.5      \\               
2017&3.34   & 21	&	Bosnia Herzegovina &	7.5      \\               
2018&3.35   & 22	&	Panama  	&	7.5      \\                          
2020&3.37   & 23	&	Brazil  	&	7.4      \\\hline                    
\end{tabular}\end{table}       



big change 50perc change in trust, and small about 3perc in freedom;

in addition to key SWB, we also look at two additional subjective wellbeing
indicators, trust and freedom (SEE MY ERALIER PAPERS ON TRUST AND FREEDOM--what
control for)

rt super interesting, swb flat; but trust went down a lot to a mere .05!! makes
sense all the politics and unrest; but freedom if anything went up, indeed as
some protesters would often say ``we're not afraid anymore''


trust and freedom for later; just swb
. tabstat trust,  stat(count mean sd) by(yr) format(%9.2f)
tabstat trust,  stat(count mean sd) by(yr) format(%9.2f)

Summary for variables: trust
     by categories of: yr (Year survey)

    yr |         N      mean        sd
-------+------------------------------
  1997 |   2995.00      0.10      0.31
  1998 |   2986.00      0.11      0.32
  2005 |   2993.00      0.14      0.35
  2012 |   1501.00      0.04      0.20
  2018 |   1520.00      0.05      0.21
-------+------------------------------
 Total |  11995.00      0.10      0.30
--------------------------------------

. tabstat freedom,  stat(count mean sd) by(yr) format(%9.2f)
tabstat freedom,  stat(count mean sd) by(yr) format(%9.2f)

Summary for variables: freedom
     by categories of: yr (Year survey)

    yr |         N      mean        sd
-------+------------------------------
  1997 |      0.00         .         .
  1998 |   2996.00      7.89      2.09
  2005 |   3002.00      8.04      2.20
  2012 |   1507.00      8.16      1.96
  2018 |   1520.00      8.12      2.31
-------+------------------------------
 Total |   9025.00      8.02      2.15
--------------------------------------
