%to have line numbers
%DIF LATEXDIFF DIFFERENCE FILE
%DIF DEL gssLonnieRubia-aok-oct29.tex   Fri Nov  6 16:09:32 2020
%DIF ADD gssLonnieRubia-aok-nov6.tex    Sun Nov  8 12:15:32 2020
%\RequirePackage{lineno}
\documentclass[10pt, letterpaper]{article}      
\usepackage[margin=.1cm,font=small,labelfont=bf]{caption}[2007/03/09]
%\usepackage{endnotes}
%\let\footnote=\endnote
\usepackage{setspace}
\usepackage{longtable}                        
\usepackage{anysize}                          
\usepackage{natbib}                           
%\bibpunct{(}{)}{,}{a}{,}{,}                   
\bibpunct{(}{)}{,}{a}{}{,}                   
\usepackage{amsmath}
\usepackage[% draft,
pdftex]{graphicx} %draft is a way to exclude figures                
\usepackage{epstopdf}
\usepackage{hyperref}                             % For creating hyperlinks in cross references
%DIF 18a18
\hypersetup{pdfborder={0 0 0.4}} %DIF > 
%DIF -------

%DIF 19d20
%DIF < 
%DIF -------
% \usepackage[margins]{trackchanges}

% \note[editor]{The note}
% \annote[editor]{Text to annotate}{The note}
%    \add[editor]{Text to add}
% \remove[editor]{Text to remove}
% \change[editor]{Text to remove}{Text to add}

%TODO make it more standard before submission: \marginsize{2cm}{2cm}{1cm}{1cm}
%\marginsize{.4cm}{.4cm}{.2cm}{.2cm}%{left}{right}{top}{bottom}   
					          % Helps LaTeX put figures where YOU want
 \renewcommand{\topfraction}{1}	                  % 90% of page top can be a float
 \renewcommand{\bottomfraction}{1}	          % 90% of page bottom can be a float
 \renewcommand{\textfraction}{0.0}	          % only 10% of page must to be text

 \usepackage{float}                               %latex will not complain to include float after float

\usepackage[table]{xcolor}                        %for table shading
\definecolor{gray90}{gray}{0.90}
\definecolor{orange}{RGB}{255,128,0}

\renewcommand\arraystretch{.9}                    %for spacing of arrays like tabular

%-------------------- my commands -----------------------------------------
\newenvironment{ig}[1]{
\begin{center}
 %\includegraphics[height=5.0in]{#1} 
 \includegraphics[height=3.3in]{#1} 
\end{center}}

 \newcommand{\cc}[1]{
\hspace{-.13in}$\bullet$\marginpar{\begin{spacing}{.6}\begin{footnotesize}\color{blue}{#1}\end{footnotesize}\end{spacing}}
\hspace{-.13in} }

%-------------------- END my commands -----------------------------------------



%-------------------- extra options -----------------------------------------

%%%%%%%%%%%%%
% footnotes %
%%%%%%%%%%%%%

%\long\def\symbolfootnote[#1]#2{\begingroup% %these can be used to make footnote  nonnumeric asterick, dagger etc
%\def\thefootnote{\fnsymbol{footnote}}\footnote[#1]{#2}\endgroup}	%see: http://help-csli.stanford.edu/tex/latex-footnotes.shtml

%%%%%%%%%%%
% spacing %
%%%%%%%%%%%

% \abovecaptionskip: space above caption
% \belowcaptionskip: space below caption
%\oddsidemargin 0cm
%\evensidemargin 0cm

%%%%%%%%%
% style %
%%%%%%%%%

%\pagestyle{myheadings}         % Option to put page headers
                               % Needed \documentclass[a4paper,twoside]{article}
%\markboth{{\small\it Politics and Life Satisfaction }}
%{{\small\it Adam Okulicz-Kozaryn} }

%\headsep 1.5cm
% \pagestyle{empty}			% no page numbers
% \parindent  15.mm			% indent paragraph by this much
% \parskip     2.mm			% space between paragraphs
% \mathindent 20.mm			% indent math equations by this much

%%%%%%%%%%%%%%%%%%
% extra packages %
%%%%%%%%%%%%%%%%%%

\usepackage{datetime}


\usepackage[latin1]{inputenc}
\usepackage{tikz}
\usetikzlibrary{shapes,arrows,backgrounds}


%\usepackage{color}					% For creating coloured text and background
%\usepackage{float}
\usepackage{subfig}                                     % for combined figures

\renewcommand{\ss}[1]{{\colorbox{blue}{\bf \color{white}{#1}}}}
\newcommand{\ee}[1]{\endnote{\vspace{-.10in}\begin{spacing}{1.0}{\normalsize #1}\end{spacing}\vspace{.20in}}}
\newcommand{\emd}[1]{\ExecuteMetaData[/tmp/tex]{#1}} % grab numbers  from stata

%TODO before submitting comment this out to get 'regular fornt'
%DIF 112-116c112-116
%DIF < \usepackage{sectsty}
%DIF < \allsectionsfont{\normalfont\sffamily}
%DIF < \usepackage{sectsty}
%DIF < \allsectionsfont{\normalfont\sffamily}
%DIF < \renewcommand\familydefault{\sfdefault}
%DIF -------
% \usepackage{sectsty} %DIF > 
% \allsectionsfont{\normalfont\sffamily} %DIF > 
% \usepackage{sectsty} %DIF > 
% \allsectionsfont{\normalfont\sffamily} %DIF > 
% \renewcommand\familydefault{\sfdefault} %DIF > 
%DIF -------

% \usepackage[margins]{trackchanges}
% \usepackage{rotating}
% \usepackage{catchfilebetweentags}
%-------------------- END extra options -----------------------------------------
\date{{}\today \hspace{.2in}\xxivtime}
\title{\DIFdelbegin \DIFdel{The Top Regrets Of The Dying:}%DIFDELCMD < \\ %%%
\DIFdelend %DIF >  The Top Regrets Of The Dying:\\ 
  ``I Wish I Hadn't Worked So Hard.''\DIFdelbegin %DIFDELCMD < \\
%DIFDELCMD <   %%%
\DIFdel{(Greed Is Good For Economy, But Not For Human Wellbeing) %DIF < \\ 
   }\DIFdelend \DIFaddbegin \DIFadd{---Greed And Life Satisfaction %DIF >  \\
  %DIF >  (Greed Is Good For Economy, But Not For Human Wellbeing)
  %DIF >  \\ 
   }\DIFaddend %(The More Work Hours and Money, The Less Happiness)
}
\author{
% Adam Okulicz-Kozaryn\thanks{EMAIL: adam.okulicz.kozaryn@gmail.com
%   \hfill I thank XXX.  All mistakes are mine.} \\
% {\small Rutgers - Camden}
}
%DIF PREAMBLE EXTENSION ADDED BY LATEXDIFF
%DIF UNDERLINE PREAMBLE %DIF PREAMBLE
\RequirePackage[normalem]{ulem} %DIF PREAMBLE
\RequirePackage{color}\definecolor{RED}{rgb}{1,0,0}\definecolor{BLUE}{rgb}{0,0,1} %DIF PREAMBLE
\providecommand{\DIFaddtex}[1]{{\protect\color{blue}\uwave{#1}}} %DIF PREAMBLE
\providecommand{\DIFdeltex}[1]{{\protect\color{red}\sout{#1}}}                      %DIF PREAMBLE
%DIF SAFE PREAMBLE %DIF PREAMBLE
\providecommand{\DIFaddbegin}{} %DIF PREAMBLE
\providecommand{\DIFaddend}{} %DIF PREAMBLE
\providecommand{\DIFdelbegin}{} %DIF PREAMBLE
\providecommand{\DIFdelend}{} %DIF PREAMBLE
\providecommand{\DIFmodbegin}{} %DIF PREAMBLE
\providecommand{\DIFmodend}{} %DIF PREAMBLE
%DIF FLOATSAFE PREAMBLE %DIF PREAMBLE
\providecommand{\DIFaddFL}[1]{\DIFadd{#1}} %DIF PREAMBLE
\providecommand{\DIFdelFL}[1]{\DIFdel{#1}} %DIF PREAMBLE
\providecommand{\DIFaddbeginFL}{} %DIF PREAMBLE
\providecommand{\DIFaddendFL}{} %DIF PREAMBLE
\providecommand{\DIFdelbeginFL}{} %DIF PREAMBLE
\providecommand{\DIFdelendFL}{} %DIF PREAMBLE
%DIF HYPERREF PREAMBLE %DIF PREAMBLE
\providecommand{\DIFadd}[1]{\texorpdfstring{\DIFaddtex{#1}}{#1}} %DIF PREAMBLE
\providecommand{\DIFdel}[1]{\texorpdfstring{\DIFdeltex{#1}}{}} %DIF PREAMBLE
\newcommand{\DIFscaledelfig}{0.5}
%DIF HIGHLIGHTGRAPHICS PREAMBLE %DIF PREAMBLE
\RequirePackage{settobox} %DIF PREAMBLE
\RequirePackage{letltxmacro} %DIF PREAMBLE
\newsavebox{\DIFdelgraphicsbox} %DIF PREAMBLE
\newlength{\DIFdelgraphicswidth} %DIF PREAMBLE
\newlength{\DIFdelgraphicsheight} %DIF PREAMBLE
% store original definition of \includegraphics %DIF PREAMBLE
\LetLtxMacro{\DIFOincludegraphics}{\includegraphics} %DIF PREAMBLE
\newcommand{\DIFaddincludegraphics}[2][]{{\color{blue}\fbox{\DIFOincludegraphics[#1]{#2}}}} %DIF PREAMBLE
\newcommand{\DIFdelincludegraphics}[2][]{% %DIF PREAMBLE
\sbox{\DIFdelgraphicsbox}{\DIFOincludegraphics[#1]{#2}}% %DIF PREAMBLE
\settoboxwidth{\DIFdelgraphicswidth}{\DIFdelgraphicsbox} %DIF PREAMBLE
\settoboxtotalheight{\DIFdelgraphicsheight}{\DIFdelgraphicsbox} %DIF PREAMBLE
\scalebox{\DIFscaledelfig}{% %DIF PREAMBLE
\parbox[b]{\DIFdelgraphicswidth}{\usebox{\DIFdelgraphicsbox}\\[-\baselineskip] \rule{\DIFdelgraphicswidth}{0em}}\llap{\resizebox{\DIFdelgraphicswidth}{\DIFdelgraphicsheight}{% %DIF PREAMBLE
\setlength{\unitlength}{\DIFdelgraphicswidth}% %DIF PREAMBLE
\begin{picture}(1,1)% %DIF PREAMBLE
\thicklines\linethickness{2pt} %DIF PREAMBLE
{\color[rgb]{1,0,0}\put(0,0){\framebox(1,1){}}}% %DIF PREAMBLE
{\color[rgb]{1,0,0}\put(0,0){\line( 1,1){1}}}% %DIF PREAMBLE
{\color[rgb]{1,0,0}\put(0,1){\line(1,-1){1}}}% %DIF PREAMBLE
\end{picture}% %DIF PREAMBLE
}\hspace*{3pt}}} %DIF PREAMBLE
} %DIF PREAMBLE
\LetLtxMacro{\DIFOaddbegin}{\DIFaddbegin} %DIF PREAMBLE
\LetLtxMacro{\DIFOaddend}{\DIFaddend} %DIF PREAMBLE
\LetLtxMacro{\DIFOdelbegin}{\DIFdelbegin} %DIF PREAMBLE
\LetLtxMacro{\DIFOdelend}{\DIFdelend} %DIF PREAMBLE
\DeclareRobustCommand{\DIFaddbegin}{\DIFOaddbegin \let\includegraphics\DIFaddincludegraphics} %DIF PREAMBLE
\DeclareRobustCommand{\DIFaddend}{\DIFOaddend \let\includegraphics\DIFOincludegraphics} %DIF PREAMBLE
\DeclareRobustCommand{\DIFdelbegin}{\DIFOdelbegin \let\includegraphics\DIFdelincludegraphics} %DIF PREAMBLE
\DeclareRobustCommand{\DIFdelend}{\DIFOaddend \let\includegraphics\DIFOincludegraphics} %DIF PREAMBLE
\LetLtxMacro{\DIFOaddbeginFL}{\DIFaddbeginFL} %DIF PREAMBLE
\LetLtxMacro{\DIFOaddendFL}{\DIFaddendFL} %DIF PREAMBLE
\LetLtxMacro{\DIFOdelbeginFL}{\DIFdelbeginFL} %DIF PREAMBLE
\LetLtxMacro{\DIFOdelendFL}{\DIFdelendFL} %DIF PREAMBLE
\DeclareRobustCommand{\DIFaddbeginFL}{\DIFOaddbeginFL \let\includegraphics\DIFaddincludegraphics} %DIF PREAMBLE
\DeclareRobustCommand{\DIFaddendFL}{\DIFOaddendFL \let\includegraphics\DIFOincludegraphics} %DIF PREAMBLE
\DeclareRobustCommand{\DIFdelbeginFL}{\DIFOdelbeginFL \let\includegraphics\DIFdelincludegraphics} %DIF PREAMBLE
\DeclareRobustCommand{\DIFdelendFL}{\DIFOaddendFL \let\includegraphics\DIFOincludegraphics} %DIF PREAMBLE
%DIF LISTINGS PREAMBLE %DIF PREAMBLE
\RequirePackage{listings} %DIF PREAMBLE
\RequirePackage{color} %DIF PREAMBLE
\lstdefinelanguage{DIFcode}{ %DIF PREAMBLE
%DIF DIFCODE_UNDERLINE %DIF PREAMBLE
  moredelim=[il][\color{red}\sout]{\%DIF\ <\ }, %DIF PREAMBLE
  moredelim=[il][\color{blue}\uwave]{\%DIF\ >\ } %DIF PREAMBLE
} %DIF PREAMBLE
\lstdefinestyle{DIFverbatimstyle}{ %DIF PREAMBLE
	language=DIFcode, %DIF PREAMBLE
	basicstyle=\ttfamily, %DIF PREAMBLE
	columns=fullflexible, %DIF PREAMBLE
	keepspaces=true %DIF PREAMBLE
} %DIF PREAMBLE
\lstnewenvironment{DIFverbatim}{\lstset{style=DIFverbatimstyle}}{} %DIF PREAMBLE
\lstnewenvironment{DIFverbatim*}{\lstset{style=DIFverbatimstyle,showspaces=true}}{} %DIF PREAMBLE
%DIF END PREAMBLE EXTENSION ADDED BY LATEXDIFF

\begin{document}

%%\setpagewiselinenumbers
%\modulolinenumbers[1]
%\linenumbers

\bibliographystyle{/home/aok/papers/root/tex/ecta}
\maketitle
\vspace{-.4in}
\begin{center}

\end{center}


\begin{abstract}
\noindent A palliative nurse listed the most common regrets of the dying in
their last days: ``I wish I hadn't worked so hard'' is among the top, especially
for men. We know from philosophers, social scientists, and religious teachings
that greed and materialism are vices. Yet, neo-classical economic theory, which
dominates current thinking, promotes the maximization of income and consumption as a virtue.  
%
In this paper, we test whether wanting ``more work and more money'' results in
human flourishing measured as life satisfaction. We also use \DIFdelbegin \DIFdel{alternative }\DIFdelend \DIFaddbegin \DIFadd{additional }\DIFaddend measures
of greed\DIFaddbegin \DIFadd{/materialism }\DIFaddend based on whether respondents agreed with the following
statements: ``next to health, money is most important,'' ``no right and wrong
ways to make money,'' and ``a job is just a way to earn money.'' Results \DIFdelbegin \DIFdel{on all
measures concur--greed/materialism is
robustly related to lower life satisfaction.
%DIF < 
The large effect size of greed }\DIFdelend \DIFaddbegin \DIFadd{for all
measures concur---there are large negative effect sizes of these }\DIFaddend measures on
life satisfaction\DIFdelbegin \DIFdel{is remarkable. The negative effect size of greed is }\DIFdelend \DIFaddbegin \DIFadd{, }\DIFaddend on average about half of the positive effect of income.
%
\DIFdelbegin \DIFdel{This study supports policies aiming at improving working conditions and lowering working hours; curbing materialism}\DIFdelend \DIFaddbegin \DIFadd{The findings support policies aiming to curb excessive working hours, materialism, }\DIFaddend and
conspicuous/positional consumption. 
%
This study is associative, not necessarily causal, and results may not generalize beyond the US, especially where people are less obsessed with work and money.  
\end{abstract}
\vspace{.15in} 
\noindent{\sc subjective well-being (swb), happiness, life satisfaction, working
  hours, greed, money, consumerism, conspicuous consumption, materialism 
}
\vspace{.25in} 

\begin{spacing}{1.8} %TODO MAYBE before submission can make it like 2.0
\rowcolors{1}{white}{gray90}


\noindent``Money is therefore not only the object but also the fountainhead of greed.'' Karl Marx, Grundrisse\\


%``Nothing on earth consumes a man more quickly than the passion of resentment.'' Nietzsche\\

\noindent ``I wish I hadn't worked so hard'' is among the the top regrets of the dying \citep{ware12}.
 This is an incredibly useful \DIFdelbegin \DIFdel{resource---to have the wisdom from }\DIFdelend \DIFaddbegin \DIFadd{resource:  the wisdom of }\DIFaddend people who evaluate their life as a whole on their deathbed. Other top resentments among the dying include: ``I wish I'd had the courage to live a life true to myself, not the life others
expected of me,'' ``I wish I'd had the courage to express my feelings,'' ``I wish I had stayed in touch with my friends,'' and ``I wish that I had let myself be happier.''\footnote{See \citet{ware12}. These regrets are all similarly related---live your own life, spend time with loved ones, travel more, etc.---they all
  point to less work; if there is any work involved in these wishes, they are often about being
  more brave and actionable, or taking a different career or investment path, as opposed to
  working harder and getting more money; remarkably,  \DIFdelbegin \DIFdel{apparently }\DIFdelend no one regrets not working harder or making more money\DIFdelbegin \DIFdel{! }\DIFdelend \DIFaddbegin \DIFadd{. }\DIFaddend And yet, this is precisely the most common pursuit during our \DIFdelbegin \DIFdel{lifetime---more income }\DIFdelend \DIFaddbegin \DIFadd{lifetime---income }\DIFaddend and consumption. Still, note that people do not regret some forms of consumption, such as traveling, which relates to extrinsic vs. intrinsic consumerism---buy experience\DIFaddbegin \DIFadd{, }\DIFaddend not material goods. For other studies on deathbed regrets and elaboration of the concept, see SOM (Supplementary Online Material).}\\

There is a clear pattern
in responses---regrets are of a spiritual or social nature, but not materialistic. Research on social indicators, quality of life studies, and subjective well-being should leverage such treasure
trove of information
%, as it provides important cues 
to advance our understanding on how to live a meaningful and happy life, with fewer regrets. 
It  would be prudent to adhere to these laments and to re-evaluate our own life choices. Of particular interest in this study, is one of the top regrets of the dying, wishing not having worked so much.  

In general, philosophers, social scientists, 
%(with the notable exception of economics)
 and religious teachings condemn working excessively and wanting too much money and material possessions---greed is even one of the seven deadly sins in Christian teachings. 
Temperance and restraint from excess are traditionally seen as virtues. Benjamin Franklin, who wrote on moral perfection, includes frugality, temperance, and moderation in his list of virtues.\footnote{''Benjamin Franklin on Moral Perfection''--Practical advice on obtaining a perfectly moral bearing. From his autobiography. \url{https://www.ftrain.com/franklin_improving_self}}

The wisdom of the dying and their honest % (or even  blunt)
evaluation of what really matters in life should be taken seriously, as arguably, there is no one in a better position\DIFdelbegin \footnote{\DIFdel{If anyone is in better position to provide advice on how to live a life, it is philosophers and social scientists. And if they all agree, the dying, philosophers, and social scientists, it is probably a sound advice.}} %DIFAUXCMD
\addtocounter{footnote}{-1}%DIFAUXCMD
\DIFdelend %DIF >  \footnote{If anyone is in better position to provide advice on how to live a life, it is philosophers and social scientists. And if they all agree, the dying, philosophers, and social scientists, it is probably a sound advice.}
 to know what really matters in life than those facing its end. We have a lot to gain from their regrets, particularly \DIFdelbegin \DIFdel{when }\DIFdelend \DIFaddbegin \DIFadd{if }\DIFaddend our own way of life \DIFdelbegin \DIFdel{was }\DIFdelend \DIFaddbegin \DIFadd{is }\DIFaddend like theirs. Yet, most people won't come to the realization that wanting more work and money is indeed a mistake until it is too late.

Greed, materialism, and consumerism \DIFaddbegin \DIFadd{(defined in section \ref{g}) }\DIFaddend became accepted and even celebrated in
American \DIFdelbegin \DIFdel{society---we define these terms and elaborate on them in section \ref{g}}\DIFdelend \DIFaddbegin \DIFadd{society}\DIFaddend .  ``More working hours'' is a badge of courage---``conspicuous exhaustion'' and
``busyness''---especially in Anglo countries \DIFdelbegin \DIFdel{, }\DIFdelend \DIFaddbegin \DIFadd{and }\DIFaddend among professional/managerial jobs
\citep{gershuny05}. 
%
Ellon Musk, for example, proclaims that to be succesful ``a person needs to work 80-100 hours
per week''  \citep{muskIN18nov26}\DIFdelbegin \DIFdel{, which many people might follow given that
Musk is a celebrity}\DIFdelend %DIF >  , which many people might follow given that
%DIF >  Musk is a celebrity
.  Unrestrained income and  consumption maximization (greed) is an integral part of the American
Dream \citep{robinson2009greed}.
%
In popular culture \DIFdelbegin \DIFdel{and popular opinion in the U.S.}\DIFdelend %DIF >  and popular opinion
\DIFaddbegin \DIFadd{in the US}\DIFaddend , wanting to work more hours
and desiring to make more money is typically a virtue {\citep[with some exceptions, e.g.,][]{folbre2000love}.} 
%We live in a deeply materialistic and consumerist society. 
Both hard work and high income are highly desirable as they may signal ambition and desire for success. A person who follows this trajectory, as popular opinion has it, should be happy.\footnote{In one study, students were asked about their feeling related to money, and ``happiness'' was the most frequent emotion they associated with money \citep{mogilner2010pursuit}.
A recent survey found that a third of people define success by their possessions \citep[cited in][]{joye20}.}
%
The American dream is largely based on the capitalistic notion that financial success is what determines who has ``made it'' or ``succeeded'' in society % ---this is what Americans strive for~
\citep{aokditella}: ``Life is a game. Money is how we keep score''  (Ted Turner, attributed\DIFdelbegin \DIFdel{.)}\DIFdelend \DIFaddbegin \DIFadd{).  }\DIFaddend This is the purpose of the free market economy, to satisfy whatever desires and wants there may be\DIFdelbegin \DIFdel{; }\DIFdelend \DIFaddbegin \DIFadd{, }\DIFaddend and to create new ones---arguably, it is only half a joke that marketing is the science of how to make people buy things they don't need for the money they don't have. 
Indeed, money itself creates insatiable wants~\citep{marx1844-human-requirements};
and yet, as this study tests, wanting more work and more money is related to
lowered life satisfaction. If the goal of life is to have a satisfying life% pursue happiness and maximize it
, then our values as a society are incongruous.

In what follows, we document the relationship between indicators of
greed\DIFdelbegin \DIFdel{and
materialism on life satisfaction by drawing on data from }\DIFdelend \DIFaddbegin \DIFadd{/materialism and life satisfaction using }\DIFaddend the US General Social Survey
(GSS). We start by defining our key concepts and variables, present the theory
and literature on how greed\DIFdelbegin \DIFdel{and }\DIFdelend \DIFaddbegin \DIFadd{/}\DIFaddend materialism can affect life satisfaction, and
then proceed to present empirical evidence, discussion of our results, and
conclusions.
\DIFdelbegin %DIFDELCMD < 

%DIFDELCMD < %%%
\DIFdelend %DIF >  these are to clarify as folks may get confused what this is about:
 \DIFaddbegin \DIFadd{Note, we study greed/materialism (an attitude): "I want more hours and more
money" (and similar). 
%DIF > 
 We do not study overwork, overearning, and
overconsumption (behaviors). %DIF >  (just control for hrs and income):
%DIF >  and this is for stiupid americans who think they're not greedy:
As elaborated throughout, in the US, one of the
richest and arguably the most consumerist/materialistic country, wanting more
work and money typically is greed, not need (unless one is in deep poverty).
>>>
}\DIFaddend \section{Subjective Well-being}

% Aristotle proposed eudamonia, good life. Bentham meh 
% , good life; greatest happiness for the greatest number

Happiness is an end in itself. ``What do [men] demand of life and wish to
achieve in it? The answer can hardly be in doubt. They strive after happiness;
they want to become happy and to remain so\DIFdelbegin \DIFdel{.}\DIFdelend '' \citep[][p. 52]{freud30}. 
A brief overview of the concept of happiness is provided in
\citet{mcmahon05}, and a full definition and overview across human history is in \citet{mcmahon06}.

For simplicity, the terms happiness, life satisfaction, and
subjective well-being (SWB) are used interchangeably. But we tend to opt for
``life satisfaction'' as this is what we mostly measure.
Ruut Veenhoven (\citeyear[p. 2]{veenhoven08}) defines happiness as an ``overall judgment of life that draws on two sources of information: cognitive comparison with standards of the good life (contentment) and affective information from how one feels most of the time (hedonic level of affect).''
Some scholars use `life satisfaction' to  refer to cognition and `happiness' to
refer to affect \citep[e.g.,][]{dorahy98etal}. This dichotomy is not pursued here, because there is only one survey item\footnote{This is an inherent limitation of our
study, as the GSS only has one question on life satisfaction. Still, these are the best data for our study--datasets with more precise measures of SWB have inadequate geographical and temporal coverage.} in this study capturing mostly the concept of life satisfaction but also happiness to a lesser degree. Therefore, the \DIFdelbegin \DIFdel{SWB }\DIFdelend definition by \citet{veenhoven08} seems most appropriate.

Even though self-reported and subjective, the life satisfaction measure is reliable
(precision varies), valid, and correlated with similar objective measures of
well-being \citep{myers00,layard05}.


\section{\label{g}Greed}


%\noindent``Greed, envy, sloth, pride and gluttony: these are not vices anymore. No, these are marketing tools. Lust is our way of life. Envy is just a nudge towards another sale. Even in our relationships we consume each other, each of us looking for what we can get out of the other. Our appetites are often satisfied at the expense of those around us. In a dog-eat-dog world we lose part of our humanity.'' Jon Foreman\\


% \noindent``Money is therefore not only the object but also the fountainhead of greed.'' Karl Marx, Grundrisse\\

\noindent `` `Excess and intemperance' are money's true norm.'' \citep{marx1844-human-requirements}\\

The Merriam-Webster's dictionary defines greed as ``a selfish and excessive desire for more of something (as money) than is needed.''  
 \DIFdelbegin \DIFdel{The definition fits our measure, if }\DIFdelend %DIF >  The definition fits our measure, 
 \DIFaddbegin \DIFadd{If }\DIFaddend one doesn't lack necessities (needs), but desires to have more, then it is greed\DIFdelbegin \DIFdel{.}\footnote{\DIFdel{For more definitions see \mbox{%DIFAUXCMD
\citet{seuntjens15b}}\hspace{0pt}%DIFAUXCMD
, and for an useful overview refer to \mbox{%DIFAUXCMD
\citet{wang11b}}\hspace{0pt}%DIFAUXCMD
.}}
%DIFAUXCMD
\addtocounter{footnote}{-1}%DIFAUXCMD
\DIFdelend %DIF >  .\footnote{For more definitions
   \DIFaddbegin \DIFadd{see also \mbox{%DIFAUXCMD
\citet{seuntjens15b}}\hspace{0pt}%DIFAUXCMD
%DIF >  ,
   and %DIF >  for an useful overview refer to
   \mbox{%DIFAUXCMD
\citet{wang11b}}\hspace{0pt}%DIFAUXCMD
).%DIF >  }
}\DIFaddend 

 According to the  livability theory:
 ``Like all animals, humans have innate needs, such as food, safety, and
 companionship---gratification of needs manifests in hedonic
 experience''\citep{veenhoven14b}---for the vast majority of Americans wanting
 more money \DIFdelbegin \DIFdel{, it }\DIFdelend is not to satisfy innate needs.
%
 Thus, desiring to have more money if one can
 satisfy basic needs is greed. In a rich country such as the US, money-orientation is typically greed.
 \footnote{In popular opinion in the US, however,
   greed is more associated with the ENRON scandal and the likes---breaking the
   law to acquire millions. But the definition of a greedy person is a
   person who wants more than is needed, and what is needed are the
   biological/physiological needs that we all share: food, shelter, security,
   etc \citep{veenhoven14b}. It could be argued that to achieve
   self-actualization, or self-esteem as proposed in the Maslow's Hierarchy of Needs \citep{maslow87}, additional money is necessary---but do note that attainment of any of those does not require much money, it is rather that people in a consumerist society wrongly believe that they need money for self-esteem, and self actualization. 
Notably, more work hours to generate more money actually prevents one from socializing and belonging with others, and from doing creative activities that can lead to self actualization (both the human needs on Maslow's Hierarchy). Indeed, the US is in a crisis of alienation and isolation as a result \citep{putnam01,wilkinson09}.}

\citet{bok10} made an useful comparison: today's bottom income decile has a
better quality of life than everyone 100 years ago except for the top
decile. Arguably, a person in the US at the 90th percentile of income 100 years
ago was not critically hampered by the lack of money to satisfy her basic needs,
likewise, \DIFdelbegin \DIFdel{it could be argued }\DIFdelend \DIFaddbegin \DIFadd{we argue }\DIFaddend that the same is true for a person  \DIFdelbegin \DIFdel{today }\DIFdelend at the 10th percentile of
income \DIFaddbegin \DIFadd{today }\DIFaddend in a rich country such as the US.\footnote{In our empirical analysis that follows, we dropped the poorest 10 percent from our sample as a robustness check. We also controlled for income and social class in our models. In addition to
  \citet{bok10}, refer to \citet[e.g.,][]{pinker18} for \DIFdelbegin \DIFdel{more on }\DIFdelend human civilization progress.}
Except for those in deep poverty, wanting more is arguably typically greed due
to materialism \DIFaddbegin \DIFadd{and consumerism}\DIFaddend .
Sure, even in the US, and even for the middle class, more money \DIFdelbegin \DIFdel{would typically }\DIFdelend \DIFaddbegin \DIFadd{could often }\DIFaddend help
with their quality of life, but the point is that by working more one loses
time, which is necessary to satisfy human needs \citep{maslow87}. Humanistic human needs theories of Maslow, Rogers, and Fromm  
 specifically suggest that pursuit of money may distract from fulfillment of
 needs and lead to distress \citep[cited in][]{kasser93}.
 Arguably, for a typical middle class American reducing work hours (income) and
 consumption would result in a better quality of life
 \citep{dittmar14,kasser13,hsee13,leonard10}.
 % lonnie:
 Indeed, if longer work hours \DIFdelbegin \DIFdel{just }\DIFdelend create a ``rat race'' where no one's wellbeing improves, either absolutely or relatively, there is a case for adopting collective policies and practices that mitigate these forces \citep{jauch2020rat,hamermesh2017does,golden2009brief}.

\DIFdelbegin \DIFdel{The }\DIFdelend \DIFaddbegin \DIFadd{In the US the }\DIFaddend problem is not so much the lack of income as conspicuous consumption.  
Even the impoverished in poor developing countries spend as much as 30 percent of income on conspicuous consumption \citep{banerjee11}. \DIFdelbegin \DIFdel{Of course, there is a related problem of income inequality and by all means much more should be redistributed from the rich to the poor. }\DIFdelend %DIF >  Of course, there is a related problem of income inequality and by all means much more should be redistributed from the rich to the poor.
 The upper limit for the 1st decile of usual weekly earnings of full-time wage and salary workers in the US is \$500, or about \$70 daily, which is more than 10
times the amount that half of the world population lives on:
\$5.50.\footnote{The data come from
  \url{https://www.bls.gov/news.release/wkyeng.t05.htm} and
  \url{https://www.worldbank.org/en/news/press-release/2018/10/17/nearly-half-the-world-lives-on-less-than-550-a-day}.}
\DIFaddbegin 

\DIFaddend Typically, the rich are more greedy (and more unethical in general) than the rest of the population \citep{piff17,piff14,piff12,piff10,kraus09}, but it does
not change the fact that the middle class, and even the poor, can be greedy, too. Greed is based on the love for money, not the possession of it. 


Perhaps, according to an US perspective, our measures in this study are not measuring greed per se, but merely capturing a person's money-orientation. \DIFdelbegin \DIFdel{We specifically examine whether respondents preferred having more working hours and more money, the ``more hours and more money,'' variable. Or whether they agreed with the following statements: ``next to health, money is
most important,'' ``there's no right and wrong ways to make money,'' and ``job is just a way
to earn money.'' }\footnote{\DIFdel{There are several greed scales, with items that have
  stronger money orientation than the ones used here. For instance
  \mbox{%DIFAUXCMD
\citet{seuntjens15}}\hspace{0pt}%DIFAUXCMD
: 1. I always want more, 2. Actually, I'm kind of greedy,
  3. One can never have too much money, 4. As soon as I have acquired something,
  5. It doesn't matter how much I have. I'm never completely satisfied, 6. My
  life motto is "more is better," 7. I can't imagine having too many things. \mbox{%DIFAUXCMD
\citet{mussel18} }\hspace{0pt}%DIFAUXCMD
compares different scales.  %DIF <   
}} %DIFAUXCMD
\addtocounter{footnote}{-1}%DIFAUXCMD
\DIFdelend %DIF >  We specifically examine whether respondents preferred having more working hours and more money, the ``more hours and more money,'' variable. Or whether they agreed with the following statements: ``next to health, money is
%DIF >  most important,'' ``there's no right and wrong ways to make money,'' and ``job is just a way
%DIF >  to earn money.''
 But taking international perspective and human
biological needs  \citep[as per][]{veenhoven14b}, our measures
are reasonable and adequate measures of greed.\footnote{\DIFaddbegin {\DIFadd{There are several greed scales, with items that have
  stronger money orientation than the ones used here. For instance
  \mbox{%DIFAUXCMD
\citet{seuntjens15}}\hspace{0pt}%DIFAUXCMD
: 1. I always want more, 2. Actually, I'm kind of greedy,
  3. One can never have too much money, 4. As soon as I have acquired something,
  5. It doesn't matter how much I have. I'm never completely satisfied, 6. My
  life motto is "more is better," 7. I can't imagine having too many things. \mbox{%DIFAUXCMD
\citet{mussel18} }\hspace{0pt}%DIFAUXCMD
compares different scales.  %DIF >   
}} \DIFaddend We are unaware of a large scale nationally representative dataset having such
a proper greed scale that would also contain subjective well-being measures and
its predictors. \DIFdelbegin %DIFDELCMD < \MBLOCKRIGHTBRACE
%DIFDELCMD < 

%DIFDELCMD < %%%
\DIFdelend It\DIFdelbegin \DIFdel{'s rather the US that is an outlier and doesn't fit the international norm in terms of greed, materialism, aggressiveness, dominance, and the like. 
}\DIFdelend \DIFaddbegin \DIFadd{'s rather the US that is an outlier and doesn't fit the international norm in terms of greed, materialism, aggressiveness, dominance, and the like. 
}\DIFaddend %
\DIFdelbegin \DIFdel{The US is perceived as exceptionally and problematically narcissistic \mbox{%DIFAUXCMD
\citep{miller15}}\hspace{0pt}%DIFAUXCMD
.
The US }\DIFdelend \DIFaddbegin \DIFadd{The US is perceived as exceptionally and problematically narcissistic \mbox{%DIFAUXCMD
\citep{miller15}}\hspace{0pt}%DIFAUXCMD
.
The US }\DIFaddend % has managed to developed as a country, and then
\DIFdelbegin \DIFdel{is able to dominate other countries through being (or threatening to be) aggressive and violent \mbox{%DIFAUXCMD
\citep[e.g.,][]{pratto08}}\hspace{0pt}%DIFAUXCMD
. }\DIFdelend \DIFaddbegin \DIFadd{is able to dominate other countries through being (or threatening to be) aggressive and violent \mbox{%DIFAUXCMD
\citep[e.g.,][]{pratto08}}\hspace{0pt}%DIFAUXCMD
. }\DIFaddend %p 203
 \DIFdelbegin \DIFdel{Indeed, the US is considered the leading terrorist organization in the world
 \mbox{%DIFAUXCMD
\citep{truthout14nov3}}\hspace{0pt}%DIFAUXCMD
. 
}\DIFdelend \DIFaddbegin \DIFadd{Indeed, the US is considered the leading terrorist organization in the world
 \mbox{%DIFAUXCMD
\citep{truthout14nov3}}\hspace{0pt}%DIFAUXCMD
. }}
\DIFaddend 

\DIFdelbegin \DIFdel{It is }\DIFdelend \DIFaddbegin \DIFadd{It is }\DIFaddend difficult for people in the US to see that they are greedy, \DIFdelbegin \DIFdel{since }\DIFdelend \DIFaddbegin \DIFadd{because }\DIFaddend the term ``greed''
has negative connotations\DIFdelbegin \DIFdel{, but }\DIFdelend \DIFaddbegin \DIFadd{. But }\DIFaddend at the same time\DIFdelbegin \DIFdel{it }\DIFdelend \DIFaddbegin \DIFadd{, greed }\DIFaddend became the norm, so people don't perceive anything wrong. Indeed, as Jon Foreman put it:
``Greed, envy, sloth, pride and gluttony: these are not vices anymore. No, these
are marketing tools. Lust is our way of life. Envy is just a nudge towards
another sale. Even in our relationships we consume each other, each of us
looking for what we can get out of the other. Our appetites are often satisfied
at the expense of those around us. In a dog-eat-dog world we lose part of our
humanity.''\footnote{Another example: "Most of the time, successful modern life involves [...] working very hard for as much money as possible, and doing what we are told. These elements are almost a conventional prescription for success." \url{theschooloflife.com/thebookoflife/henry-david-thoreau}.}

The intention to work more and make more money, may not seem congruent with
greed, but in an \DIFdelbegin \DIFdel{affluent }\DIFdelend \DIFaddbegin \DIFadd{exceptionally rich, materialistic and consumerist }\DIFaddend society, such as the US, wanting more is usually not a need
but a want or greed. Indeed, an argument can be made that Americans are in general greedy, as they consume most in the world per capita \citep{leonard10,kasser13}.
\DIFdelbegin %DIFDELCMD < 

%DIFDELCMD < %%%
\DIFdelend %DIF > 
If it is not apparent that the US is one of the most greedy or the very most greedy nation, it is clear that the US is the most materialistic/consumerist nation in the world, and greed is close to materialism.

\citet{seuntjens15b} provides \DIFdelbegin \DIFdel{a }\DIFdelend \DIFaddbegin \DIFadd{an }\DIFaddend useful overview of the concept of greed, which
is summarized in this paragraph. In Belk's  definition, greed is one of the core elements of materialism. Although materialistic people can indeed be greedy, greed is broader than just a desire for material possessions. People can be greedy for food, power, or sex, which has nothing to do with materialism. Whereas materialists desire things because they signal success in life, greed can also be felt for things that do not signal success or status (e.g., being greedy for candy).

%meh
% \citet{seuntjens15b} provides a useful overview of the concept of greed:
% In the psychological literature greed is often, and mistakenly, used interchangeably
% with self-interest. In the rational economic model, agents are thought to be self-interested
% and to maximize their outcomes. Self-interest refers to the fact that rational agents only
% care about their own outcomes, and are indifferent concerning the outcomes of others.
% Greed is related to the assumption of maximization, which states that agents always prefer
% to have more rather than less of a good. We believe that greed is an exaggerated form of
% maximizing, in which people not simple prefer to have more, but are also frustrated by not
% having it. While it may be rational to strive for the maximum, striving for more than what is
% possible is not rational. Thus, when people are greedy, they can become so focused on
% what they want or desire that it leads to behaviour that is not rational anymore.
% Another construct used interchangeably with greed is materialism. In Belk's (1984)
% definition, greed is even one of the core elements of materialism. Although materialistic
% people can indeed be greedy, greed is broader than just a desire for material possessions
% (Tickle, 2004). People can be greedy for food, power, or sex, which has nothing to do with
% materialism. Whereas materialists desire things because they signal success in life
% (Richins, 2004), greed can also be felt for things that do not signal success or status (e.g.,
% being greedy for candy).

We will continue with the discussion of greed and \DIFdelbegin \DIFdel{related vices }\DIFdelend \DIFaddbegin \DIFadd{its }\DIFaddend relationship to
human wellbeing in section \ref{rel}, but first we present the theory and define
human wellbeing.

\section{A Theoretical Foundation}

\noindent``Money is therefore not only the object but also the fountainhead of greed.'' Karl Marx, Grundrisse\\

% Marx is one of the greatest thinkers of all times, and his theory is relevant
% here. While his major work ``Capital'' \citep{marx10} may be daunting and
% inaccesbile, there are many excellent intrductions and overviews conveying some
% of the key points, the two relevant here are:
% \url{theschooloflife.com/thebookoflife/the-great-philosophers-karl-marx}
% \url{https://www.google.com/amp/s/www.newyorker.com/magazine/2016/10/10/karl-marx-yesterday-and-today/amp} 
% There is also a free reporsitory of all Marx (and hundreds other Marxists) works: \url{marxists.org}.


% %https://socialistrevolution.org/marxism-vs-modern-monetary-theory/#:~:text=Marx%20explained%20that%20money's%20history,showed%2C%20have%20an%20exchange%20value.&text=required%20for%20its%20production%2C%20and,production%20process%20by%20the%20worker.
% For Marx, value of money comes from commodity exchange value (all commodities
% are produced for exchange, not for usefulness). Because of the key importance of
% exchange, money is a social relation.
%marx on money more here: https://mronline.org/2017/09/18/the-significance-of-marxs-theory-on-money/
%

%Marx saw much negative in money: ``I do not like money, money is the reason we fight.''
Marx wrote a brief paper, ``The Power of Money'' \citep{marx1844-powerOfMoney}. He argues that money is the procurer between a person's needs and the desired object. Money is a powerful and omnipotent being because it can buy anything, and one can appropriate of all objects desired as a result.
%``By possessing the property of buying everything, by possessing the property of
% appropriating all objects, money is thus the object of eminent possession. The
% universality of its property is the omnipotence of its being. It is therefore
% regarded as an omnipotent being. Money is the procurer between man’s need and
% the object.''
But \DIFdelbegin \DIFdel{using Shakespeare and Goethe, Marx also notices that }\DIFdelend %DIF >  using Shakespeare and Goethe, Marx
\DIFaddbegin \DIFadd{also %DIF > notices  that
}\DIFaddend money has a distortive power. It distorts human nature and relations between people:
\begin{quote}
Money, then, appears as this distorting power both against the individual and against the bonds of society, etc., which claim to be entities in themselves. It transforms fidelity into infidelity, love into hate, hate into love, virtue into vice, vice into virtue, servant into master, master into servant, idiocy into intelligence, and intelligence into idiocy. Since money, as the existing and active concept of value, confounds and confuses all things, it is the general confounding and confusing of all things---the world upside-down---the confounding and confusing of all natural and human qualities.

He who can buy bravery is brave, though he be a coward. As money is not exchanged for any one specific quality, for any one specific thing, or for any particular human essential power, but for the entire objective world of man and nature, from the standpoint of its possessor it therefore serves to exchange every quality for every other, even contradictory, quality and object: it is the fraternization of impossibilities. It makes contradictions embrace.

Assume man to be man and his relationship to the world to be a human one: then you can exchange love only for love, trust for trust, etc. If you want to enjoy art, you must be an artistically cultivated person; if you want to exercise influence over other people, you must be a person with a stimulating and encouraging effect on other people. Every one of your relations to man and to nature must be a specific expression, corresponding to the object of your will, of your real individual life.
\end{quote}

Acquiring money is counterproductive---neediness grows as the power of money increases \citep{marx1844-human-requirements}.
% a person is on a hedonic treadmil \citep{brickman78cj}
%
Then, according to Marx, for human flourishing, instead of acquiring more money,
one should rather try to enjoy things without using money,\footnote{For
  instance, in less capitalistic countries such as Eastern Europe or Latin
  America, kids get together spontaneously and play soccer in public space for
  free and are happy. In the US there is a coach, there is special soccer field,
  special gear, etc. It costs money and from anecdotal observation looks less
  happy. \DIFaddbegin \DIFadd{``Money can't buy happiness, but it can make you
awfully comfortable while you're being miserable'' \mbox{%DIFAUXCMD
\citep[][p.26]{munier2004being}}\hspace{0pt}%DIFAUXCMD
.}\DIFaddend } because of the money's distortive property. Hence, we would expect that those who want more money are
not happier, and probably less happy than others\DIFdelbegin \DIFdel{: ``Money can't buy happiness, but it can make you
awfully comfortable while you're being miserable'' \mbox{%DIFAUXCMD
\citep[][p.26]{munier2004being}}\hspace{0pt}%DIFAUXCMD
}\DIFdelend .  
There is little  happiness from money, because happiness is ``subsequent fullfilment of a
  prehistoric wish. That is why wealth brings so little happiness: money was
  not a wish in childhood'' \citep[Freud cited in][p. 203]{marcuse15}.

  
\DIFaddbegin \DIFadd{Just as wanting more money is counterproductive, so is wanting more labor: labor alienates a person from 1) object of her labor, 2) herself and her essence, and 3) from other humans 
\mbox{%DIFAUXCMD
\citep{petrovic63}
  }\hspace{0pt}%DIFAUXCMD
}

\DIFaddend %DONE to res.org PAPER TEST THIS: capitalists v labor using psid effect on happiness :) 
While Marx didn't use directly the terms ``life satisfaction'' or ``happiness\DIFdelbegin \DIFdel{'',}\DIFdelend \DIFaddbegin \DIFadd{,'' }\DIFaddend he had much to say about well-being using different terminology. He was a humanist, inherently interested in human flourishing and well-being.
When arguing for a free classless society, he is essentially advocating for a person's ability to develop her multiple physical and psychological talents and potentials: ``the full development of human mastery over
the forces of nature . . . the absolute working out of [their] creative
potentialities . . . the development of all human powers as an end in
itself''\citep[cited in][p. 91]{struhl16}.
% %like nietzsche become who you are lol
% 

According to Marx, work is a drudgery and toil in capitalism \citep{marx10, lyons07}.
% capitalism produced  wretched living and working conditions
% yeah but thats not about greed!
%
%
Wage slaves are ``hired slaves instead of block slaves. You have to dread the idea of being unemployed and of being compelled to support your masters'' \citep[p. 283][]{goldman03}.
% 
Capitalists largely do not work, their income and
wealth come from capital, not labor. %they are rent seekers duh
Labor under capitalism is a wretched condition. Yet it is necessary, one needs
to make a living and exchange their labor for necessities. But wanting more work
and money through labor (and even capital) than necessary is a futile endeavor and should lead to
more alienation and misery, not human flourishing, which is why one of the top regrets of the dying is, ''I wish I hadn't worked so hard.''

What one should do instead according to Marx is enjoy life freely and spontaneously, ``It will be possible to hunt in the morning, fish in the afternoon, rear cattle in the evening, criticize after dinner . . . without ever becoming hunter, fisherman, herdsman, or critic., and do what one pleases''
%
This agrees with the Frankfurt School, e.g., Marcuse's idea of unrestrained joyful spontaneity \citep{marcuse15}. Even Keynes made similar predictions in his ``Economic Possibilities for Our Grandchildren'' \citep{keynes30}.
% yeah like colombia happier than the us

Instead, under capitalism, as Marx put it  well, ``labor has become not only a
means of life but life's prime want'' \citep[cited in][p. 91]{struhl16}. Indeed,
Americans already live to work, while people in less capitalistic and more
enlightened \DIFdelbegin \DIFdel{nations }\DIFdelend \DIFaddbegin \DIFadd{societies }\DIFaddend work to live \citep{aokditella}. Therefore, wanting even more work and more money seems counterproductive for human flourishing (unless one is in poverty). 

Marx would rather call capitalists ``greedy'' than workers, but of course
workers can be both taken advantage of by greedy capitalists and be ``greedy'' at the same time, especially when they live in a contemporary post-industrial affluent country like the US.
% theSchoolOfLife-marx:
Curiously, Marx thought capitalists are also at least in some ways victims of the capitalist system:
\begin{quote}
  The propertied class and the class of the proletariat present the same human
  self-estrangement. . . . The class of the proletariat feels annihilated in
  estrangement; it sees in it its own powerlessness and the reality of an
  inhuman existence. It is . . . abasement, the indignation at that abasement,
  an indignation to which it is necessarily driven by the contradiction between
  its human nature and its condition of life, which is the outright, resolute
  and comprehensive negation of that nature.
\end{quote} \citep[cited in][p 381]{byron16}.

For example, the idealized bourgeois family was in fact fraught with tension,
oppression, and resentment---the family kept together not because of love but for financial reasons.

%https://en.wikipedia.org/wiki/Marx%27s_theory_of_human_nature
% What is relevant here is Marx's Gattungswesen, species essence, or for simplicity,
Marx agreed that basic human needs must be satisfied (similar to Veenhoven's livability theory \citep{veenhoven14b})\footnote{While some
  argue that Marx had no theory of human nature, a case can be made that he at least in parts of his writing referred to human nature. 
%
  Veenhoven's and Marx's theories are similar in how they both refer to the essential biological/physiological needs we have. But, while Veenhoven emphasizes human similarity to other animals, Marx
emphasizes the differences: ``To know what is useful for a dog, one must study dog-nature. This nature
itself is not deduced from the principle of utility. Applying this to man, he that would criticize all human acts, movements, relations, etc. by the principle
of utility must first deal with human nature in general, and then with human
nature as modified in each historical epoch''\citep[quoted in][p. 83]{struhl16}. 
The varying human nature by historical epoch is counter to evolutionary biology, where genes are relatively
stable over thousands of years; Still Marx does believe in  evolution \citep{heyer82},
and he somewhat acknowledges the problem, where he
worries that some negative human tendencies would still exist after capitalism is abolished.}: ``people cannot be
liberated as long as they are unable to obtain food and drink, housing and clothing of adequate quality and quantity'' \citep[cited in][p. 70]{geras83}.
Marx argues that humans are social beings, and too much focus on individualism
distorts human nature. % \footnote{History shapes human nature, too.}
Humans are not inherently selfish, as economists argue, rather their selfishness results from commodity fetishism. %the fact that people act selfishly is held to be a product of scarcity and capitalism, not an immutable human characteristic.
Humans are alienated from their human nature and other humans under capitalism \DIFdelbegin \DIFdel{\mbox{%DIFAUXCMD
\citep{byron16}}\hspace{0pt}%DIFAUXCMD
}\DIFdelend \DIFaddbegin \DIFadd{\mbox{%DIFAUXCMD
\citep{byron16,petrovic63}}\hspace{0pt}%DIFAUXCMD
}\DIFaddend . %Alienation, for Marx, is the estrangement of humans from aspects of their human nature
Good society should allow full uninhibited spontaneous human expression % as in
% Frankfurt School
\citep{marcuse15}.\footnote{The idea is gratification in the free play of the
  released potentialities of humans, sensousness, liberation of the senses and
  freedom from constraints \citep{marcuse15}. And such freedom, liberation and
  free play are constrained by capitalism, e.g., ``necessary labor is a system of
  essentially inhuman, mechanical, and routine activities''
  \citep[][p. 195]{marcuse15}.} 
%
And this would be one mechanism linking greed  to unhappiness---humans become alienated from their nature, and end up unhappy. 

%The ruling class is capitalists, and the ruling ideas is economics.
%``The ideas of the ruling class are in every epoch the ruling ideas.'' (The German Ideology, 1845)
%The ruling class is capitalists, and the ruling ideas is economics.

Ideology can promote and perpetuate greed. Economics neoclassical school's claim that a \textit{laissez faire} neoliberal free market capitalism is fairest, and people's belief
in this claim certainly contributes to widespread greed. Ironically, the masses supporting capitalism are irrational and acting against
their own \DIFdelbegin \DIFdel{interest---but }\DIFdelend \DIFaddbegin \DIFadd{interest---and }\DIFaddend they do so following the classical economic theory preaching that everyone is rational and self-interested. 
% (sociology is opposite as some others are quite too), of course there are many
% notable economists against it--paul krugman, thomas piketty, bob frank, to name
% the few, but the discipline as a whole is clearly most pro inequality (maybe
% business too) among
% social sciences.
We know that people are not \DIFdelbegin \DIFdel{very }\DIFdelend \DIFaddbegin \DIFadd{perfectly }\DIFaddend rational and they often act against their own
interest \citep{akerlof10,ariely09,shiller15}. Non-capitalists are not
free in capitalism, they are commodities in the market and they work too much and worry too much to enjoy life \citep{aokJap14}. Ironically, we
have capitalism in the first place in order to be free---we justify the very
existence of capitalism with freedom \citep{hayek14,friedman09,glaeser11B}. Free
market provides incentives to embrace capitalism and submit oneself to a capitalist, and economics provides the ``science'' to justify such as system.

%TODO cp here more from becky charlotte evil econs

Economic theory\footnote{Not all of economics is responsible for overwork, \DIFdelbegin \DIFdel{over-earning}\DIFdelend \DIFaddbegin \DIFadd{overearning}\DIFaddend , and
\DIFdelbegin \DIFdel{over-consumption}\DIFdelend \DIFaddbegin \DIFadd{overconsumption}\DIFaddend . It is mostly classical economics like Adam Smith and neoclassical like
Milton Friedman and Gary Becker. And in fairness to  economics, it must
be noted that  virtually all of economics considers work as disutility, and Adam
Smith  even did  condemned dehumanizing effects of repetitive work, and called
it "toil and trouble" \citep[][p. 54]{spencer20}.  
And of course, there are also economists that do
expose false consciousness related to money \citep[e.g,][]{kahneman06c}. % theycall it  Focusing Illusion lol haha
 Also note that in addition to income, leisure is also part of the utility
 function: U = f(Y,L) -- income and leisure time
 \citep{mcconnell2016contemporary}. \DIFaddbegin \DIFadd{One of the most bizzare statements come from
 Nobel prize winning Gary Becker---that happiness is like car in utility function \mbox{%DIFAUXCMD
\citep{becker08}}\hspace{0pt}%DIFAUXCMD
.
}\DIFaddend } states that the more income and consumption, the
more utility or happiness \citep{autor10,becker08}: 
\begin{equation}
%  income = consumption % (\pm investments and savings)
money 
  = utility \approx happiness
\end{equation}

In classical economic theory, self-interest is the key assumption, as rational
people should maximize their personal outcomes \citep{seuntjens15b}. % about adam smith
%
 And by economic theory, profit maximization, not any social responsibility, should be the only concern of businesses \citep{friedman70}.
% Yet pure and unrestrained income and consumption maximization, as economists
% would like it,
Economists advanced a concept of an ideal human being, so called ``homo economicus,'' a perfectly rational homo sapiens who maximizes income and consumption at all times: % , is still a radical idea to most
% humans, even business people.
% Economic ideas
``1) people are self-interested utility-maximizers, 2)
individuals should be unimpeded in their pursuit of their own self-interest
through economic transactions, and 3) virtually all human interactions are
economic transactions'' % create tensions even among business students
\citep[][p. 273]{walker1992greed}.
%
 Indeed, taking economics classes may increase one's greedy behavior \citep{wang11b}.

In addition to maximizing income and consumption, another problem with economics
is the complete, extreme, and unrestrained labor specialization, which according to Marx leads to alienation from human nature
and other humans \DIFaddbegin \DIFadd{\mbox{%DIFAUXCMD
\citep{petrovic63}}\hspace{0pt}%DIFAUXCMD
}\DIFaddend .

%theSchoolOfLife-marx
According to Marx, our work should not be highly specialized in one area, but we should take on
multiple roles: gardening, construction, writing, etc. We should be spontaneous and creative and see ourselves in the product we create: I did that, this is
me. %yeah like nietzshe the goal of life is to become ourselves
% Marx also wants to help us find work that is more meaningful. Work becomes meaningful, Marx says, in one of two ways. Either it helps the worker directly to reduce suffering in someone else or else it helps them in a tangible way to increase delight in others. A very few kinds of work, like being a doctor or an opera star seem to fit this bill perfectly.
Ideally if we could, we should help others decrease their suffering (like nurses do) and increase their delight (like artists do). 

A relevant economics theorist is \citet{keynes30}, who predicted about 100 years
ago that there will be enough wealth for everyone to work less and enjoy
life. The prediction of 15-hour work week was supposed to materialize 100 years later.
It did not happen---we work more, not less. It is forgotten that people
actually worked less before industrialization than they do now
\citep{schor08}. % People tend to over earn, that is, they work to earn more than
% they need \citep{hsee13}. 
 Arguably Marx was correct: in order for the technological progress (which
 did happen) \DIFdelbegin \DIFdel{, }\DIFdelend to liberate workers, there must be communal ownership of means of
 production \citep{spencer20}, otherwhise the toil and drudgery will continue as
 they do. 
In general, however, we don't need much labor anymore to produce what we need. For instance,
%theSchoolOfLife-marx
 in 1700, it took the labor of almost all adults to feed a nation, today hardly
 anyone needs to be employed in farming, making cars needs practically no
 employees, and so forth. Yet, we do not liberate ourselves---Marx is arguably
 more relevant now than in the second half of 20th century % than in the postwar period, the second half of 20th century
\citep{piketty14,peet15,menandMISC16oct3}. %menandMISC16oct3just compares now to 80s

Another economist, Veblen, criticized leisure class and conspicuous consumption
\citep{veblen05a,veblen05b}, but also criticized  primacy of money, which kills
'instinct of workmanship' \citep{spencer20}. His writings are relevant in the sense that overwork
and \DIFdelbegin \DIFdel{over-earning }\DIFdelend \DIFaddbegin \DIFadd{overearning }\DIFaddend is arguably usually for the sake of conspicuous or positional
consumption \citep{haight1997padded},  %lonnie
which in return does not result in happiness, but often creates
unhappiness for a consumer and those around her
\citep{frank12,frank_nyt_mar_20_14,frank08,frank04,kasser13,schmuck00}, e.g., consumption of luxury
  cars decreases satisfaction of others \citep{winkelmann12}.


\section{Subjective Well-being Theory}

There are several SWB theories about how happiness is created. There is the adaptation/adjustment or ''hedonic treadmill'' theory \citep{brickman78cj}: the problem with \DIFaddbegin \DIFadd{greed/}\DIFaddend materialism is that one's goal never gets fulfilled---there is
always a new IPhone or a new model of Lexus, and planned obsolescence \citep{satyro2018planned,agrawal2016limits}. %, which ensures that  objects such as  break often
%
\DIFaddbegin \DIFadd{``The more one has, the more one wants, since satisfactions received only
stimulate instead of filling needs'' \mbox{%DIFAUXCMD
\citep[][p. 248]{durkheim50}}\hspace{0pt}%DIFAUXCMD
. 
%DIF > 
}\DIFaddend The theory of happiness as a motivator \citep{carver90} is also relevant
here. This is one key reason why greed and materialism  work---humans get
momentary bliss or pleasure from making money or spending it, only to find that
it doesn't last and one is back on the hamster wheel. The realization of
what is happening may come when it is too late, at the end of one's life (refer
to the introduction and discussion).

Veenhoven's needs/livability theory is similar to Marx's theory of human nature:
``Like all animals, humans have innate needs, such as for food, safety, and
companionship. Gratification of needs manifests in hedonic
experience''\citep[][p. 3645]{veenhoven14b}.
 One surely needs money to satisfy needs under capitalism. But the vast majority
of people in affluent countries such as the US have already their needs
satisfied, and hence, wanting more is simply greed. Importantly, many poor fail
to satisfy the basic needs, not because they do not have enough money, but because they spend too much, notably on conspicuous or positional consumption. 
%
Likewise, many middle class or rich fail to satisfy their needs because
they \DIFdelbegin \DIFdel{over-consume}\DIFdelend \DIFaddbegin \DIFadd{overconsume}\DIFaddend , overwork, and \DIFdelbegin \DIFdel{over-earn}\DIFdelend \DIFaddbegin \DIFadd{overearn}\DIFaddend . Overwork,
\DIFdelbegin \DIFdel{over-earning, and over-consumption }\DIFdelend \DIFaddbegin \DIFadd{overearning, and overconsumption }\DIFaddend uses limited time and attention that is
necessary to satisfy needs to belong, create, and self-fulfill.\footnote{There
  are many notable exceptions, of course, but in general Americans do overwork \citep{aokditella},
  \DIFdelbegin \DIFdel{over-earn }\DIFdelend \DIFaddbegin \DIFadd{overearn }\DIFaddend \cite{hsee13}, and \DIFdelbegin \DIFdel{over-consume }\DIFdelend \DIFaddbegin \DIFadd{overconsume }\DIFaddend \cite{kasser13}. And paradoxically,
  because most people overwork, \DIFdelbegin \DIFdel{over-earn}\DIFdelend \DIFaddbegin \DIFadd{overearn}\DIFaddend , and \DIFdelbegin \DIFdel{over-consume---one }\DIFdelend \DIFaddbegin \DIFadd{overconsume---one }\DIFaddend in some way
  satisfies her need to belong if one does the same. Yet as discussed \DIFaddbegin \DIFadd{throughout}\DIFaddend , 
  this causes more problems than advantages.}  

Finally, there is the comparison/discrepancies theory \citep{michalos85}. Being
greedy and  materialistic, one not only diminishes her own wellbeing, but also the
wellbeing of others around her. Humans compare with others all of the time, and
a person overworking, \DIFdelbegin \DIFdel{over-earning}\DIFdelend \DIFaddbegin \DIFadd{overearning}\DIFaddend , or overspending makes others the same
way. Working, earning, and spending is like an arms race that can be won only by
minuscule fraction of the population, say the top .001 of a percent of the
population (1 in 100,000), all others lose, especially that in many cases the winner takes it all---Robert Frank provides many examples in his informative ``Darwin's Economy'' \citeyear{frank12}. 

\section{\label{rel}The Relationship Of Greed, Materialism, And Consumerism With  Human Flourishing}


``Does money buy happiness?'' is the title of a classic happiness paper by Easterlin \citeyear{easterlin73} that started the so called ``economics of happiness.'' Fifty
years later, thousands of studies have been produced on the topic and the consensus is that money buys happiness up to a point, or at least that there are diminishing marginal returns (\url{https://worlddatabaseofhappiness.eur.nl}).
%
% While the money--SWB link is the most researched topic in the happiness field,
% most of the thousands of studies about money are about the effect of income on
% SWB \url{https://worlddatabaseofhappiness.eur.nl}. 
%
% 
 In other words, one needs to be able to afford necessities or basic human needs as per Veenhoven's Livability Theory \citep{veenhoven14b}. More
 money than necessary does not buy happiness, and indeed, may actually decrease happiness \DIFdelbegin \DIFdel{as elaborated in this section}\DIFdelend %DIF >  as elaborated in this section
 .

 Already 50 years ago Easterlin has recognized what today is more severe and
 largely unrecognized, that  the pursuit of money and the pursuit of happiness
 are about the same thing in the US. {In one study students were asked about their feeling related to money, and ``happiness'' was the most frequent emotion cited \citep{mogilner2010pursuit}.
A recent survey found that a third of people define success by their possessions
\citep[cited in][]{joye20}. % Aspiring for financial success is an important
% aspect of capitalist cultures \citep{kasser93}.
}
 \DIFaddbegin 

\DIFaddend % 
% lol not sure if this is helpful in any way; and she looks like another one
% b-school happiness bs
% implicitly activating the construct of time motivates individuals to spend more time with friends
%and family and less time working behaviors that are associated with greater happiness. In contrast, implicitly activating money
%motivates individuals to work more and socialize less, which (although productive) does not increase happiness. 
%
% another one by her
% https://www.sciencedirect.com/science/article/pii/S2352250X15300051?casa_token=SZ3vd8qMllMAAAAA:irw62ENvYkLjFrcf8WLzx-Vke_H6aCzI_OmgtypoAswe0F8UrwbLewPSZiXMgJr-LHgNuoabcg8
%
% yeah thats what i find, maybe even more important than income is not wanting it more lol
%Contrary to people's intuitions, happiness may be less contingent on the sheer amount of each resource available and more on how %people both think about and choose to spend them.
%
%Happiness is not having what you want, but wanting what you have.--and there is evidence that both matter independentlly having w%hat you want, and wanting what you have
%
%

Financial success is a central life aspiration in a capitalistic culture and an
integral part of the American Dream \citep{kasser93}.
\DIFaddbegin \DIFadd{Business scholars teach us we need  money to be happy \mbox{%DIFAUXCMD
\cite{whillans17,kushlev15,aknin13,aknin12,norton11,dunn08}}\hspace{0pt}%DIFAUXCMD
.
%DIF > 
}\DIFaddend Money in itself (money
greed)  is an
important part of the dream, but so is materialism and consumption (possessions
greed) \citep{kasser16,dittmar14,kasser13,leonard10}: a suburban large expensive conspicuous ``villa'' (``McMansion'')
\citep{duany01}, a large expensive conspcious car (Cadillac, SUV, etc)
\citep{aok_ls_car15}, an expensive and fashionable computer (Apple), watch
(Rolex), handbag (Louis Vuitton), and the list continues ad infinitum.
%DIF > 
%DIF > brooksATL20oct22:
\DIFaddbegin \DIFadd{Advertisers have promised satisfaction, but have led people instead into a rat
race of joyless production and consumption \mbox{%DIFAUXCMD
\citep{cederstrom18}}\hspace{0pt}%DIFAUXCMD
.
 Even conservatives seem to notice that consumption may not lead to happiness \mbox{%DIFAUXCMD
\citep{brooksATL20oct22}}\hspace{0pt}%DIFAUXCMD
.
}\DIFaddend 

There are closely related and mutually reinforcing forces: greed/money orientation/love of money, materialism, consumerism, conspicuous/positional
consumption---people chase money in order to consume and see that as an end in itself, the goal of life has become to make as much money as possible mostly in
order to acquire as much material possessions as possible.\footnote{Again, like
  with greed and wanting more work and \DIFdelbegin \DIFdel{money; greed }\DIFdelend \DIFaddbegin \DIFadd{money--greed }\DIFaddend is not the same as
  materialism, consumerism  and conspicuous consumption, but in the affluent US
  society it usually is, and again, we will subset our sample to drop the poor to argue this point. 
%
And importantly, even the exaggerated consumption among the so called poor in rich countries is due to wants and not needs. This is the case even in poor countries, where the poor could spend up up to 30 percent more on food than it actually does if it completely cut expenditures on alcohol, tobacco, and festivals \citep{banerjee11}.
It is often men that engage in non-necessary consumption among the poor. 
% RUBIA TODO ADD REFERENCE AND POSSIBLY ELABORATE
% we could elaborate from this paper: http://www.cedlas-er.org/sites/default/files/cer_ien_activity_files/fernandez_y_dasso.pdf

The poor even engage in conspicuous consumption at the expense of proper calorie intake \citep{bellet18}. There is a culture of adornment \citep{cordwell2011fabrics,mascia1992tattoo}. But even in the US, one can see a culture of adornment, also among the poor: Iphones, LV bags, golden chains. %ray %from laeda mentioned golden chains :) 
 One ubiquitous characteristics of the US residential areas, even the poor ones,
 such as Camden NJ, is luxury cars---they are expensive and there is no added    
 wellbeing benefit from owning them \citep{aok_ls_car15}.
}
%More hours and more money typically translates into more consumption. 
%meh maybe dont emphasize this too much and already has key whillans17 above
% \section{time v money}
% the more work, the less free time; and time is often more important for swb than money 

There is also a need to belong mechanism at play: humans have a strong need to
belong and fit, and if they do, they are happier\DIFdelbegin \DIFdel{. For instance, religious people are happier in religious nations
\mbox{%DIFAUXCMD
\citep{aokrel}}\hspace{0pt}%DIFAUXCMD
.}\footnote{\DIFdel{This arguably helps to explain a curious result that
  Americans who overwork in country that overworks (the US) are happy
  \mbox{%DIFAUXCMD
\citep{aokditella}}\hspace{0pt}%DIFAUXCMD
. One need to belong and keep up with Joneses. A related
  explanation is that of false consciousness: work is good and desirable in
  itself as in Protestant Ethics \mbox{%DIFAUXCMD
\citep{weber03}}\hspace{0pt}%DIFAUXCMD
. Or simply it is better and
  happier to toil long hours without much free time than not being able to
  afford necessities, or even not being able to belong and fit in the
  materialistic consumerist society.}}
%DIFAUXCMD
\addtocounter{footnote}{-1}%DIFAUXCMD
\DIFdelend %DIF >  . For instance, religious people are happier in religious nations
 \DIFaddbegin \DIFadd{\mbox{%DIFAUXCMD
\citep{aokrel,aokditella} }\hspace{0pt}%DIFAUXCMD
%DIF >  .\footnote{This arguably helps to explain a curious result that
  %DIF >  Americans who overwork in country that overworks (the US) are happy
  %DIF >  \citep{aokditella}. One need to belong and keep up with Joneses. A related
  %DIF >  explanation is that of false consciousness: work is good and desirable in
  %DIF >  itself as in Protestant Ethics \citep{weber03}. Or simply it is better and
  %DIF >  happier to toil long hours without much free time than not being able to
  %DIF >  afford necessities, or even not being able to belong and fit in the
  %DIF >  materialistic consumerist society.} 
}\DIFaddend Because the US is a deeply materialistic and consumerist society, one may need more money than it would be otherwise necessary to feel comfortable---not
many can be comfortable not keeping up with the Joneses. But such overworking, overearning, and overconsuming has nothing to do with real human needs. It is an artificial
product of capitalism that forces people act that way---all commodities are
produced for exchange, not for usefulness---there will be no production without
consumption \citep{marx1844-human-requirements}.\DIFdelbegin \DIFdel{Therefore, as per one estimate
\$75,000 is the threshold after which more money does not buy more happiness
\mbox{%DIFAUXCMD
\citep{kahneman10}}\hspace{0pt}%DIFAUXCMD
. And importantly, even wanting \$75,000 is still greed,
because according to the greed definition, it is wanting more than necessary (an
artificial ``need'' to belong by overworking, overearning, and overconsuming is not a real need).}\footnote{\DIFdel{Also, as mentioned earlier, there is a related winner-take-all
  mechanism: one needs to overwork and over earn because how the system is constructed---for instance, to succeed, you need a really expensive house
  because these are the typos of houses in really good school districts, where children have the best chance to graduate and go to really good universities, and
  so forth---for elaboration and more examples see \mbox{%DIFAUXCMD
\citet{frank12}}\hspace{0pt}%DIFAUXCMD
.}}
%DIFAUXCMD
\addtocounter{footnote}{-1}%DIFAUXCMD
\DIFdelend %DIF >   Therefore, as per one estimate
%DIF >  \$75,000 is the threshold after which more money does not buy more happiness
%DIF >  \citep{kahneman10}. And importantly, even wanting \$75,000 is still greed,
%DIF >  because according to the greed definition, it is wanting more than necessary (an
%DIF >  artificial ``need'' to belong by overworking, overearning, and overconsuming is not a real need).\footnote{Also, as mentioned earlier, there is a related winner-take-all
%DIF >    mechanism: one needs to overwork and overearn because how the system is constructed---for instance, to succeed, you need a really expensive house
%DIF >    because these are the typos of houses in really good school districts, where children have the best chance to graduate and go to really good universities, and
%DIF >    so forth---for elaboration and more examples see \citet{frank12}.}

This topic is fascinating, because on one hand the majority of the population
accepts or celebrates money orientation (greed) and materialism/consumerism, but on the other hand
we know that it doesn't buy happiness, and that it actually usually leads to
unhappiness. Love of money (greed) and materialism/consumerism can be
exciting and indeed provide a momentary pleasure---this is another reason why we
chase them in addition to being mainstream and fashionable. 
But in the long run greed and materialism/consumerism do not lead to improved SWB, typically they lead to decreased SWB, and often to outright misery. In a
sense that money and consumption do provide momentary excitement and pleasure,
but have typically negative consequences in the long run, money and consumption
are like fatty foods, marijuana, vodka, and gambling \citep{linden11}. People
indeed get addicted because of the momentary pleasure and excitement, and they
often do not realize the negative consequences until it is too late and they
have already lived their life. 
%
% we don't know much about the effect of consumption on happiness \citep[e.g.,][]{wang17,carver16,aok_ls_car15,veenhoven04b,okulicz19}.
%
%
% The quest for possessions, money, image and status can be a costly endeavor; it is
% associated with lower levels of wellbeing, and known to lead to increased compulsive
% consumption, depression, anxiety and risky health behavior
% \citep{dittmar14, kasser16}.

Would you be happier if you were richer? Although, you might think so, it is
actually an illusion \citep{kahneman06c}, or a false consciousness. \DIFdelbegin \footnote{%DIFDELCMD < \url{https://www.marxists.org/glossary/terms/f/a.htm}%%%
}
%DIFAUXCMD
\addtocounter{footnote}{-1}%DIFAUXCMD
\DIFdelend %DIF >  \footnote{\url{https://www.marxists.org/glossary/terms/f/a.htm}}
 We know that materialism \DIFdelbegin \DIFdel{/consumerism /positional goods decreases }\DIFdelend \DIFaddbegin \DIFadd{and consumerism decrease }\DIFaddend happiness 
\citep{kasser16,dittmar14,brown05,kasser13,schmuck00,kasser93,leonard10}, and
related\DIFdelbegin \DIFdel{: }\DIFdelend \DIFaddbegin \DIFadd{, }\DIFaddend extrinsic (v intrinsic) consumption \DIFdelbegin \DIFdel{decrease }\DIFdelend \DIFaddbegin \DIFadd{decreases }\DIFaddend happiness 
\citep{ryan00,ryan99,morrison17}.

People should buy time and experience, not material goods (except bare necessities of course). 
Valuing time and experience over money, not the other way round, predicts happiness \citep{whillans2019valuing}.
Thus, one should buy experience not material goods (e.g., go bowling as opposed to buying more clothes)\citep{putnam01,kasser16,dittmar14}. One should buy time, (e.g., cut commute)% hire a maid lol this is capitalist haha
---time is actually arguably the most important resource \DIFdelbegin \DIFdel{\mbox{%DIFAUXCMD
\citep{whillans17}}\hspace{0pt}%DIFAUXCMD
}\DIFdelend \DIFaddbegin \DIFadd{\mbox{%DIFAUXCMD
\citep{masuda20,williams16,whillans17}}\hspace{0pt}%DIFAUXCMD
}\DIFaddend . Likewise, autonomous and flexible work schedules predict greater happiness \citep{gssLonnie18,aokLead17,farber16sep15,golden06w,golden13}\DIFdelbegin \footnote{\DIFdel{Interestingly, anytime we are paid by the hour, we start thinking of non-work time as money sacrificed...and that opportunity cost view lasts for a lifetime, even when we switch to getting annual salaries \mbox{%DIFAUXCMD
\citep{devoe19}}\hspace{0pt}%DIFAUXCMD
.}}
%DIFAUXCMD
\addtocounter{footnote}{-1}%DIFAUXCMD
\DIFdelend \DIFaddbegin \DIFadd{.%DIF >  \footnote{Interestingly, anytime we are paid by the hour, we start thinking of non-work time as money sacrificed...and that opportunity cost view lasts for a lifetime, even when we switch to getting annual salaries \citep{devoe19}.}
}\DIFaddend 

% Materialism -- Preoccupation with or emphasis on material  objects, comforts, and
% considerations, with a disinterest in or rejection of spiritual, intellectual,
% or cultural values



% some on wealth, notably a recent volume \citep{brule19}.



We know for a long time that money acquisition, materialism, and consumerism do provide at least momentary pleasure: ``a pleasure of gain or a pleasure of
acquisition: at other times of possession'' and buffers against negatives
``immunity from pain'' ``the happening of mischief, pain, evil, or
unhappiness.'' \citep[Bentham cited in ][]{cummins2019jeremy}.\footnote{In original: ``a \DIFdelbegin \DIFdel{pleasure }\DIFdelend \DIFaddbegin \DIFadd{pleafure }\DIFaddend of gain or a \DIFdelbegin \DIFdel{pleasure }\DIFdelend \DIFaddbegin \DIFadd{pleafure }\DIFaddend of \DIFdelbegin \DIFdel{acquisition}\DIFdelend \DIFaddbegin \DIFadd{acquifition}\DIFaddend : at other times a \DIFdelbegin \DIFdel{pleasure }\DIFdelend \DIFaddbegin \DIFadd{pleafure }\DIFaddend of \DIFdelbegin \DIFdel{possession}\DIFdelend \DIFaddbegin \DIFadd{poffeffion}\DIFaddend '' ``the happening of mischief, pain, \DIFdelbegin \DIFdel{evil}\DIFdelend \DIFaddbegin \DIFadd{eveil}\DIFaddend , or \DIFdelbegin \DIFdel{unhappiness}\DIFdelend \DIFaddbegin \DIFadd{unhappinefs}\DIFaddend .''}
Although one needs to remember that Bentham wrote these words before the industrial revolution took off, at a time when deprivation was common, and
indeed more money was necessary for most people to meet basic needs and buffer
against the misfortune. Today, the situation is very different in developed
countries, and certainly in the US---for vast majority of people wanting more money is greed. 


The appeal of greed is not only due to its momentarily excitment and pleasure. 
Greed can be good in many ways as reviewed by \citet{seuntjens15b} and
summarized in this paragraph. Greed has many positive economic consequences. Greed and self-interest are the principal motivators for a flourishing economy: greed motivates the creation of new
products and the development of new industries. 
Some greed may be inherent to human nature---all humans are greedy to some extent. %Being greedy may be vital for human welfare.
Greed may be an evolutionary adaptation promoting self-preservation. Those who are more predisposed to gain and hoard as much resources as possible may have an
evolutionary advantage. \DIFdelbegin \footnote{\DIFdel{But then it also makes things worse for a person and for society, as elaborated throughout, and this may be one reason why
it was considered a vice already by ancient philosophers and religions. There are evolutionary adaptations that are harmful for the species itself \mbox{%DIFAUXCMD
\citep{frank12}}\hspace{0pt}%DIFAUXCMD
.}}
%DIFAUXCMD
\addtocounter{footnote}{-1}%DIFAUXCMD
\DIFdelend %DIF >  \footnote{But then it also makes things worse for a person and for society, as elaborated throughout, and this may be one reason why
%DIF >  it was considered a vice already by ancient philosophers and religions. There are evolutionary adaptations that are harmful for the species itself \citep{frank12}.}
%
 But greed is insatiable. To the greedy, it is never enough. The greedy are permanently on a hedonic treadmill---they may think that they will be happier
with more money, but as soon as they get more\DIFaddbegin \DIFadd{, }\DIFaddend they adapt their desires and expectations and want even more.
%
Greed may result in financial debt. Greed can make  bankers behave recklessly, which in turn can lead to a financial crisis. A classic example of the
negative consequences of greed is the ``Tragedy of the Commons.'' Medieval herders in the UK could let their livestock graze on a common parcel of land besides their own private parcel. There was a clear preference
for herders to let their livestock graze on these ``commons.'' Although rational from an
individual perspective, it led to overgrazing and the common ground becoming infertile and useless to all.  These types of situations occur due to greed.

%\textbf{ADAM: TODO somewhere do talk about mechanisms/causal path, do say why it could be causal! }


Greed is good for business as the Wall Street movie
 character, Gordon Gecko infamously remarked, ``Greed is good.'' And indeed greed is popular among the business elites \citep{robinson2009greed}.
In general, individual differences in entrepreneurial tendencies
and abilities are positively related to primary psychopathy \citep{akhtar2013greed} %adam:TODO\textbf{have a look what this study is about and  rephrase so it fits the story here lol}.
 \citep{feherrelationship}.

While there are studies on materialism, consumerism,
conspicuous/positional consumption and SWB, there are no studies about  %ADAM: A BIT UNCLEAR GREED/MONEY ORIENTATION/LOVE OF MONEY, NEED TO FIND A BETTER WAY TO DESCRIBE WHAT WE ARE DOING HERE. 
greed % /money orientation/love of money %RRV: aok repharsed, is it good now?
      % feel free to rephaze more or even drop this para; we talk about filling
      % the gap in the next one
and SWB, hence this study aims to contribute to the literature by filling this gap. 
There are no studies about the actual pursuit of money, or the intention to work
more and make more money and SWB. This is the first study using ``more hours and
more money'' concept along with other similar measures for this purpose. In what
follows, we describe the dataset used and the analysis we conducted to % examine
% the relationship between greed and SWB, i.e. we
test the hypothesis that: \textit{the more greed, the less SWB}. 

\section{Data and Model}

We use the US General Social Survey (GSS) (\url{gss.norc.org})% cumulative file from
% 1972-2018
. The GSS is collected face-to-face and is nationally
representative. % Since 1994, the GSS is collected every other year (earlier, it was mostly annually).

The outcome of interest, SWB is measured with answers to ``Taken all together, how would you say things are these days---would you say that you are very happy,
pretty happy, or not happy?'' on scale  1$=$not happy, 2$=$happy, and 3$=$very happy. 
%
Note that while the question uses word ``happy,'' it is mostly about overall
cognitive life satisfaction, not momentarily affective happiness.

%as per lonnie:
Two measures of greed, ``\texttt{more hours and more money}'' and  ``\texttt{job is just a way to earn money}'' come from the QWL (Quality of Working Life) module. The QWL module was designed by the National Institute for Occupational Safety and Health (NIOSH) in the Centers for Disease Control and Prevention (CDC) to measure attitudes toward work, workplaces,
safety/health. These two questions  were designed by social \DIFdelbegin \DIFdel{psychology researchers to capture the levels and trends in }\DIFdelend \DIFaddbegin \DIFadd{psychologists to capture }\DIFaddend cultural attitudes, in this case focused on money. 

% Note that while we use the cumulative file  1972-2018, 
The greed/money orientation questions were only asked in a few years:
``\texttt{more hours and more money}'' and ``\texttt{job is just a way to earn money}''
were asked in 1989, 1998, 2006, and 2016. The other two measures, ``\texttt{next
  to  health, money is most  important}'' and ``\texttt{no right and wrong  ways to
  make money}'' were asked in 1973, 1974, and 1976.

Since the years do not overlap, we can not construct a greed scale using these variables. We focus on
showing robustness by using each measure separately to show that no matter how we measure greed, the results are similar. 
%
\DIFdelbegin \DIFdel{Descriptive statistics is in the SOM (Supplementary Online Material).
}\DIFdelend %DIF >  Descriptive statistics is in the SOM (Supplementary Online Material).

Greed/money orientation is arguably confounded with type of work one performs, thus, we include industry dummies for: professional, administrative and managerial, clerical, sales, service, agriculture, production and transport, craft and technical.           
\DIFdelbegin %DIFDELCMD < 

%DIFDELCMD < %%%
\DIFdelend %DIF > 
Likewise, greed/money orientation is possibly confounded with religiosity:
religious people are not supposed to want more money than needed, or to be
greedy. Hence, we include religious dummies: Protestant, Catholic, Jewish, None,
Other, Buddhism, Hinduism, Other Eastern, Muslim/Islam, Orthodox-Christian,
Christian, Native American, and Inter-Nondenominational.

We use household income and not personal income for two reasons: personal income data are missing for a substantial portion of the sample, and  what matters for one's happiness (and greed) is not only her own individual income, but the household income.
\DIFdelbegin %DIFDELCMD < 

%DIFDELCMD < %%%
\DIFdelend %DIF > 
We control for number of people in the household---if one has a family and
children (and possibly elderly in the household), wanting more money may be
necessary, and a need, not greed.

Finally, we control for predictors of SWB. What makes people happy?
\citet{myers00} suggests that age, race, gender, income, education\DIFaddbegin \DIFadd{, }\DIFaddend and marriage are all sources of interpersonal variations in happiness. Young and old people
are happy  \citep[e.g.,][]{teksoz}. Men are less happy than women, the difference being small \citep{blanchflower04o}. At least some income is necessary for happiness and unemployment decreases it
    \citep[e.g.,][]{ditella01moa,ditella01mob,ditella06m}. Being married boosts happiness \citep[e.g.,][]{myers00,diener04s}.
     Blacks are less happy than Whites
    \citep[e.g.,][]{aokcities,aok11a,blanchflower04o}.   
     A key predictor of SWB is health, thus, we control for subjective self-report of health, which is a reasonable measure of objective health \citep{subramanian09b}.

We also control for regional  differences by including dummies for census regions:  New England, Middle Atlantic, E. Nor. Central, W. Nor. Central, South
Atlantic, E. Sou. Central, W. Sou. Central, Mountain, and Pacific. And since we use pooled GSS data, we include year dummies.

We use ordinary least squares (OLS) to analyze the data. Although OLS assumes cardinality of the outcome variable, and happiness is clearly an ordinal variable, 
OLS is an appropriate estimation method to use in this case. \citet{carbonell04} showed that results are substantially the same to those from discrete models, 
and OLS has become the default method in happiness research \citep{blanchflower11}.
Theoretically, while there is still debate about the cardinality of SWB, there are strong arguments to treat it as a cardinal variable \citep{ng96,ng97,ng11}. 


\section{Results}

There are four tables, each for one of the 4  measures of
greed, and each table has 5 models that sequentially elaborate the relationship between each measure of greed and SWB. Model 1 only includes a measure of greed (plus year and region dummies as all models do (not shown). Then  two
alternative models  explore separately the addition of working hours dummies (2a)
and income (2b); model 3 includes all three variables together, model 4 adds the occupational dummies (not shown), and model 5 adds a set of socio-demographic
controls and religion dummies (not shown). All models include year and region dummies.

In table \ref{betaa}, what is notable is that in model a2b the effect of ``\texttt{more and more}'' is half the effect of income. This is a substantial and unexpected effect size: greed cuts happiness received from income by half. Even more remarkably in the full model a5, the effect of ``\texttt{more and more}'' is about as large as that of income.

Controlling for working hours and income doesn't remove the effect of greed---if you want to work more and make more money, it makes you unhappy regardless of your current working hours and
income. {We tried interactions of the greed measures with income and working hours, but we didn't find very clear robust patterns, so we do not report them.}


\begin{spacing}{.9} \begin{table}[H]\centering   \begin{scriptsize}  \begin{tabular}{p{2.3in}p{.5in}p{.5in}p{.5in}p{.5in}p{.5in}p{.5in}p{.5in}p{.5in}p{.5in}p{.5 in}p{.5in}p{.5 in}} \hline \input{../out/betaaMOD.tex} \hline + 0.10 * 0.05 ** 0.01 *** 0.001; robust std err \end{tabular}\end{scriptsize}\caption{\label{betaa}OLS regressions of SWB, fully standardized beta coefficients: \texttt{more hours and more money}}\end{table} \end{spacing}

In table \ref{betab}, the effect of ``\texttt{next to  health, money is most
  important}'' is about half to a third of that of income. In models \DIFdelbegin \DIFdel{with all controls, }\DIFdelend \DIFaddbegin \DIFadd{b4 and b5 }\DIFaddend it looses statistical significance, but remains negative. 

In table \ref{betac}, the effect of \DIFaddbegin \DIFadd{``}\DIFaddend \texttt{no right  and wrong ways to make money}\DIFaddbegin \DIFadd{'' }\DIFaddend is about half to two thirds of income. In table \ref{betad} the effect of ``\texttt{job is just a way to earn money}'' is about a third to a half of the effect of income.

\begin{spacing}{.9} \begin{table}[H]\centering   \begin{scriptsize} \begin{tabular}{p{1.8in}p{.5in}p{.5in}p{.5in}p{.5in}p{.5in}p{.5in}p{.5in}p{.5in}p{.5in}p{.5 in}p{.5in}p{.5 in}}\hline \input{../out/betab.tex} \hline + 0.10 * 0.05 ** 0.01 *** 0.001; robust std err \end{tabular}\end{scriptsize}\caption{\label{betab}OLS regressions of SWB, fully standardized beta coefficients: \texttt{next to  health, money is most  important}}\end{table} \end{spacing}

\begin{spacing}{.9} \begin{table}[H]\centering   \begin{scriptsize} \begin{tabular}{p{1.8in}p{.5in}p{.5in}p{.5in}p{.5in}p{.5in}p{.5in}p{.5in}p{.5in}p{.5in}p{.5 in}p{.5in}p{.5 in}}\hline \input{../out/betac.tex} \hline + 0.10 * 0.05 ** 0.01 *** 0.001; robust std err \end{tabular}\end{scriptsize}\caption{\label{betac}OLS regressions of SWB, fully standardized beta coefficients: \texttt{no right  and wrong ways to  make money}}\end{table} \end{spacing}

\begin{spacing}{.9} \begin{table}[H]\centering   \begin{scriptsize} \begin{tabular}{p{1.8in}p{.5in}p{.5in}p{.5in}p{.5in}p{.5in}p{.5in}p{.5in}p{.5in}p{.5in}p{.5 in}p{.5in}p{.5 in}}\hline \input{../out/betad.tex} \hline + 0.10 * 0.05 ** 0.01 *** 0.001; robust std err \end{tabular}\end{scriptsize}\caption{\label{betad}OLS regressions of SWB, fully standardized beta coefficients: \texttt{job is just a way to earn money}}\end{table} \end{spacing}


% LATER: meh: maybe later if reviewer asks have it in app
% do indicate these interactions with income in the body prominently! or even have
% that in the body; wanting more work and money is not vice for poor;
Controlling for income and unemployment/working hours is critical: wanting more work and money is not a vice for the poor or unemployed (or some underemployed). 
Controlling for income (and social class) responds to potential criticism  that it is low
income or deprivation, and not greed that affects negatively SWB. In the
supplemental material we also subset the sample by either excluding the bottom or the top decile of the income distribution, and the results remain robust.  

The large greed effect sizes % of the greed measures on SWB is
 are remarkable. The negative
effect size of greed is on average about half of the positive effect of
income. Depending on the specification, the effect size of greed is as small as
a third and as large as that of  \DIFdelbegin \DIFdel{the effect size of }\DIFdelend income on SWB.
%
If greed measures were combined into an index, effect size would have probably
been even stronger, but measures come from different years.

The effect size is quite persistent: both income and  hours worked have only moderate confounding effect on the negative effect of greed measures on SWB:
controlling for either of them cuts the effect size by as little as about 10 percent for up to about 40 percent. Income has more confounding effect than working hours. 

\section{Conclusion and Discussion}
%\textbf{i guess remove these subsections when done reorganizing and cleaning}
%\textbf{to literature or conclusion on overwork from first 2 papers with lonnie}
%\textbf{if not disussion then literatureL  do refr to my paper: johs: yeah we live to work, and yeah happier working more, but the real interpretation  (after comments from from readers) is that it is better to be unhappy working a lot, than be even more unhappy not being able to afford necessities such as education and healthcare--so yeah add that to the section where i have conCon among the poor}
%could also cite valente and berry JOHS, Married men in the US are the ones happier working more, not women or single people. 
\DIFaddbegin 

%DIF > from lonnie; maybe as last sentence as he suggests
\DIFadd{``See what your greed for money has done.'' Woody Guthrie%DIF > , 1913 Massacre
}


\DIFaddend \subsection{Conclusion}

The regrets of the dying \citep{ware12} provide a sobering lesson on what is
truly important in life. A person on her deathbed has a unique perspective to
honestly evaluate life as a whole. Overwork, overearning, materialism, and consumerism are scourge of our times.

Our empirical tests agree---greed \DIFdelbegin \DIFdel{/love of money }\DIFdelend is
robustly related to lower life satisfaction. The large effect size of greed measures on SWB is remarkable. The negative effect size of greed is on average about half of the positive effect of income.

The paradox is that popular culture and neoclassical economics promote greed and overwork, and few people question them as a way of life, and yet they lead to deep and painful regrets at the end of life.
%
 It is not new that greed and materialism are vices, we have known this since ancient times, and yet today, greed and materialism are accepted, promoted, and often celebrated in our society---it has become a normal state of affairs. 


\subsection{Discussion}

``There's class warfare, all right, but it's my class, the rich class, that's making war, and we're winning.'' Warren Buffett\\

What Buffet noticed, goes almost unnoticed in the US---one of the exceptionalisms of the US is the very low class consciousness \citep{lipset97, lipset00}.
%
%discussion  and policy: from earlier papers with lonnie; and keynes dream of our grandchildren; yeah as pe veenhoven evidence based pursuit of happiness: humans are irrational so we need scienc to nudge them in the right direction :)
%way higher taxes on wealthy! possibly tax on consumption!
%
``I wish I hadn't worked so hard''  is the opposite of  what is promoted by
capitalists, economists, and politicians \citep{wang11b,wight2005adam}: ``people need to work longer hours'' % Jeb Bush, 2016 presidential candidate
\citep{smithABC15jul8}. \footnote{Not all economists, capitalists, and politicians agree of course,
  for instance see \citet{wight2005adam} and \url{https://www.epi.org/}.}



% About two thrids of US employees are disengaged
% \citep{thompsonMISC20sep20}. %meh maybeCITE more sciency source and add more;;;
Seventy percent of American workers are ``not engaged'' or ``actively disengaged'' at work\citep[][]{harvey14}. 
If the majority of people don't like their job, the extra money may not be worth
the extra time spent at work. \DIFaddbegin \DIFadd{Indeed, many suffer from time poverty
\mbox{%DIFAUXCMD
\cite{williams16}}\hspace{0pt}%DIFAUXCMD
. }\DIFaddend This is consistent with a Marxian perspective that labor
under capitalism is drudgery and toil. Indeed it is ``wage slavery,'' where
labor is commodity---we are like commodities on the market trying to sell our labor.\footnote{Yet in communism, according to Marx, work could be
  liberating, creative, enhancing, and self-fulfilling \citep{spencer20}. Thus
  it would satisfy Maslow's needs of esteem and self-actualization
  \citep{maslow87}. 
%
   To be fair, there are liberating, creative, enhancing, and
  self-fulfilling jobs even in capitalism. The few such jobs include tenured professors---much
of our work is pleasure and self actualization. % ; some
% to ameliorate this problem of varying degrees of alientaion across occupations,
% we control for occupational dummies. Yet, there are few such jobs, e.g.,
Artists, writers, and actors are other such notable jobs. Yet vast majority of
artists, writers, and actors survive on minimum wage from unrelated jobs to support
themselves, because these types of jobs are ``winner take it all''
\citep{frank12}. So are research jobs increasingly ``winner take it all,'' with
proportion of tenure track or tenured faculty declining and competition
increasing (see reports and discussions at \url{aaup.org}).
 And even us, tenured professors, find ourselves alientated and overworked with
 increasing administrative and service loads as even public education is being rampantly corporatized \citep{mills2012corporatization,cox2013corporatization,millsNYT12fa,CatropaNYT20feb8,schmidlinNYT15oct10}.
 % . Even autonomous chose by us research is often
% alienating as amount of work needed in increasingly competetive publishing is
% around what Musk remarked to be nessesar for success around 80-100 hours per
% week CITE AGAIN MUSK.

% while this is arguably true for most of the population that work under
% capitalism alienates
} 

Greed, materialism and overconsumption do not lead to happiness, but to unhappiness, and it can cause pollution and climate change \citep{leonard10,pachauri14}.
 It could be argued that if greed is good for the economy, then  it may be
 good for human wellbeing indirectly---the better the economy, the higher the
 standard of living, and the happier the people. Except that we don't need more
 economic growth. A reasonable case has been made for degrowth by
 \cite{kallis12,kallis11,bergh11} (including reduced work hours \citep{fitzgerald2018working}).

There is a notable paradigm shift under way in terms of what persons and
societies should maximize. The second half of the twentieth century was marked
by maximization of income and consumption and rebuilding of the world after the
wars. It was the goal behind the establishment of international institutions,
e.g., World Bank, IMF, and WTO. Now, even some economists are noticing that the maximization of income or consumption is not the only goal worth pursuing. For instance, Amartya Sen proposed subjective well-being as a measure to maximize \citep{stiglitz09al}. 
\DIFdelbegin \DIFdel{Recently, }\DIFdelend \DIFaddbegin 

\DIFaddend \citet{diener09} has provided an authoritative  
discussion of why potential problems with happiness are not serious enough to make it unusable for interventions, planning, and public policy.  
%
Our findings support policies aiming at improving working conditions and lowering working hours; curbing materialism and conspicuous/positional consumption. 

This study is observational, not causal, and our results may not generalize beyond the US, especially where people are less obsessed with work and money. 
%lonnie: reverse causality:, maybe--can't ruler out--sure unhappy people can be more greedy, but also less so--greed is linked to economic sussess, so they could be relatively happy 
%
Future studies should examine other countries and how greed affects happiness
elsewhere. We speculate that \DIFdelbegin \DIFdel{if anything the results should be stronger }\DIFdelend \DIFaddbegin \DIFadd{the results may hold }\DIFaddend in places like Europe or Latin America where people are happier working less and spending more time with family \DIFdelbegin \DIFdel{, }\DIFdelend and friends \citep{valente16,valente15,aokditella}.  

Although, greed is central to the human existence and contributes to many problems, notably climate change \citep[e.g.,][]{okulicz19}, empirical research on greed is rare. Thus, future research will provide valuable contributions to the literature. 

% \subsection{Takeaway}

American corporate capitalism---the highly competitive economic system embraced by the United States as well as England, Australia and Canada---encourages
materialism more than other forms of capitalism.
%
As expected, citizens who live in more competitive free market systems care more about money, power and achievement than people who live under more cooperative systems. Research also supports the notion that the more people care about money and power, the less they care about community and relationships.
%DIF >  These countries rely more on strategic cooperation among the various players in the economy and society to solve their economic problems, such as unemployment, labor and trade issues, rather than relying mostly on free-market competition as the United States does.
\footnote{\url{https://www.apa.org/monitor/2009/01/consumerism}}\DIFaddbegin \DIFadd{.
 }\DIFaddend As social welfare institutions and labor market policies have some positive
 effect on work conditions \citep{inanc20}, we do support such regulation as a step in the
 right direction. However, without a change in ownership relations,
 as Marx argued, there is unlikely any dramatic change in human wellbeing. %\citep{spencer20} 


Greed, materialism and consumerism are widespread and rampant in the US. They
are accepted and even celebrated. The underlining point is that they are not even
noticed, discussed, or questioned. They became the norm to the point
that they are not much recognized. It's just the way things are, the way of life
as usual---but this was not always the case. The US actually has a rich
tradition of simple, social, and spiritual life, dating back to the settlers
\cite{fischer91}. Benjamin Franklin pointed to frugality, temperance, and
moderation in his list of virtues.\footnote{''Benjamin Franklin on Moral
  Perfection''--Practical advice on obtaining a perfectly moral bearing. From
  his autobiography: \url{https://www.ftrain.com/franklin_improving_self}.}
Likewise,  the US is home of two great transcendentalists: Emerson and Thoreau.
%DIF >  buddhism, nirvana--in order to be free need to renounce material posessions
Similarly, great humanists like Marcuse and Maslow lived in the US. But somehow, arguably with the help of neoclassical economists, notably Hayek, Becker, and Friedman, the US has become a nation where greed is ``good.''

What makes it hard to \DIFdelbegin \DIFdel{discard the tools we have objectified }\DIFdelend \DIFaddbegin \DIFadd{change the current way of life }\DIFaddend is the persistence
of the ideologies that justify \DIFdelbegin \DIFdel{them}\DIFdelend \DIFaddbegin \DIFadd{it}\DIFaddend , and which make what is only a human
invention seem like ``the way things are'' \citep{menandMISC16oct3}.  \DIFdelbegin \DIFdel{Undoing
ideologies is the task of philosophy. }\DIFdelend %DIF >  Undoing
%DIF >  ideologies is the task of philosophy.
Notably, Marx \DIFdelbegin \DIFdel{was a philosopher---the
subtitle of ``Capital'' is ``Critique of Political Economy.'' The uncompleted
book was intended to be a criticism of }\DIFdelend %DIF >  was a philosopher---the
%DIF >  subtitle of ``Capital'' is ``Critique of Political Economy.'' The uncompleted
%DIF >  book was intended to be a
\DIFaddbegin \DIFadd{criticized }\DIFaddend the economic concepts that make social
relations in a free-market economy seem natural and inevitable, in the same way
that concepts like the great chain of being and the divine right of kings once
made the social relations of feudalism seem natural and inevitable \citep{menandMISC16oct3}. 
%
In his 1845 work ``The German Ideology,'' he wrote, ``the ideas of the ruling class are in every epoch the ruling ideas.''
%theSchoolOfLife-marx
In a capitalist society most people, rich and poor, believe all sorts of things
that are really just value judgments that relate back to the economic system,
for example: that a person who doesn't work is practically worthless, that if we
simply work hard enough we will get ahead, and that more belongings will make us
happier.
\DIFaddbegin 

\DIFaddend Our results, provide evidence to the contrary: \DIFdelbegin \DIFdel{working }\DIFdelend \DIFaddbegin \DIFadd{wanting to work }\DIFaddend more for more
money, will result in lower life satisfaction. Inevitably it will be for many people their top regret when facing the end of life. 

%  instead \ExecuteMetaData[../out/tex]{ginipov} do \emd{ginipov}

% \begin{figure}[H]
%  \includegraphics[height=3in]{../out/gov_res_trust.pdf}\centering\label{gov_res_trust}
% \caption{woo}
% \end{figure}





% %table centered on decimal points:)
% \begin{table}[H]\centering\footnotesize
% \caption{\label{freq_im_god} importance of God}
% \begin{tabular} {@{} lrrrr @{}}   \hline 
% Item& Number & Per cent   \\ \hline
% 1(not at all)&    9,285&  9\\
% 2&    3,555&        3\\
% 3&    3,937&        4\\
% 4&    2,888&        3\\
% 5&    7,519&        7\\
% 6&    5,175&        5\\
% 7&    6,050&        6\\
% 8&    8,067&        8\\
% 9&    8,463&        8\\
% 10&   52,385&       49\\
% Total&  107,324&      100\\ \hline
% \end{tabular}\end{table}


% % Define block styles
% \tikzstyle{block} = [rectangle, draw, fill=black!20, 
%     text width=10em, text centered, rounded corners, minimum height=4em]
% \tikzstyle{b} = [rectangle, draw,  
%     text width=6em, text centered, rounded corners, minimum height=4em]
% \tikzstyle{line} = [draw, -latex']
% \tikzstyle{cloud} = [draw, ellipse,fill=black!20, node distance = 5cm,
%     minimum height=2em]

% \begin{tikzpicture}[node distance = 2cm, auto]
%     % Place nodes
%     \node [block] (lib) {liberalism, egalitarianism, welfare};
%     \node [block, below of=lib] (con) {conservatism, competition, individualism};
%     \node [cloud, right of=con] (ls) {well-being};
%     \node [block, below of=ls] (cul) {genes, culture};
%     \node [b, left of =lib, node distance = 4cm] (c) {country-level};
%     \node [b, left of =con,  node distance = 4cm] (c) {person-level};
%     % Draw edges
%     \path [line] (lib) -- (ls);
%     \path [line] (con) -- (ls);
%     \path [line,dashed] (cul) -- (ls);
% \end{tikzpicture}


%PUT THIS NOTE, polish and put to /root/author_what_data --ALWAYS
%stick here stuff as i run it!!! maybe comment out later...

\bibliography{/home/aok/papers/root/tex/ebib.bib,gssLonnieRubia.bib,/home/aok/papers/swbCityWorld/tex/swbCityWorld.bib}


\section*{\Huge \DIFdelbegin \DIFdel{SOM-supplementray online material}\DIFdelend \DIFaddbegin \DIFadd{SOM-Supplementray Online Material}\DIFaddend ; \DIFdelbegin \DIFdel{ONLINE APPENDIX}\DIFdelend \DIFaddbegin \DIFadd{Online Appendix}\DIFaddend }
% \textbf{[note: this section will NOT be a part of the final version of
%   the manuscript, but will be available online instead]} %hence everything below
%                                 %is organized byu section, not subsection
% !!!
% have most of the stuff outputted to online appendix:)--start with that and then
% select stuff to paper--have brief narrative describng patterns in online app too
% !!!

% \section*{Variables' definitions, coding, and distributions}
% \label{app_var_des}

\section{Greed is Good}

Timothy 6:10\\
For the love of money is a root of all kinds of evil, for which some have strayed from the faith in their greediness, and pierced themselves through with many sorrows.\\

Timothy 6:9\\
But those who desire to be rich fall into temptation and a snare, and into many foolish and harmful lusts which drown men in destruction and perdition.\\

%How many people desire to be rich?\\
%Who does NOT desire to be rich?

\DIFdelbegin \DIFdel{And there are more here }\DIFdelend \DIFaddbegin \DIFadd{There are more biblical quotes at }\DIFaddend \url{https://www.biblemoneymatters.com/bible-verses-about-money-what-does-the-bible-have-to-say-about-our-financial-lives/#bible-verses-about-greed}\DIFaddbegin \DIFadd{.
}\DIFaddend 

\DIFaddbegin {\DIFadd{A Confucius successor saw the root of all evil in selfishness \mbox{%DIFAUXCMD
\citep[][p. 332]{hirst34}}\hspace{0pt}%DIFAUXCMD
:
 }\begin{quote}
 \DIFadd{The source of disorder in a State lies in the lack of mutual
 love.... A thief loves his own family, but because he has not a
 similar love for the families of others, he proceeds to steal from
 their homes to add to his own .... Rulers of States love their own
 territory, but having no love for other States, they proceed to
 attack them in order to increase their own possessions. What is
 the remedy for this state of things? . . . If we were to regard the
 property of others as we regard our own, who should steal?
 If we were to have the same regard for the territory and people of
 another State as we have for our own, who would conduct
 aggressive warfare? . . . If we were to have the same regard for
 others as we have for ourselves, who would do anyone an injustice   
 }\end{quote}
}

\DIFaddend The Wall Street Movie, Gordon Gecko\DIFdelbegin \DIFdel{quote about }\DIFdelend \DIFaddbegin \DIFadd{'s full quote ``}\DIFaddend greed is good\DIFaddbegin \DIFadd{''}\DIFaddend :
\begin{quote}
\begin{verbatim}                                                                          
The point is, ladies and gentleman, that greed -- for lack of a better word -- is good.   
                                                                                          
Greed is right.                                                                           
                                                                                          
Greed works.                                                                              
                                                                                          
Greed clarifies, cuts through, and captures the essence of the evolutionary spirit.       
                                                                                          
Greed, in all of its forms -- greed for life, for money, for love, knowledge -- has marke\
d the upward surge of mankind.                                                            
                                                                                          
And greed -- you mark my words -- will not only save Teldar Paper, but that other malfunc\
tioning corporation called the USA.                                                       
\end{verbatim}
\end{quote}

% from beth
Also see: 
Network 1976 "The World is a business" GOD Speech scene:
\url{https://www.youtube.com/watch?v=8jIw22XXSso&feature=youtu.be}

\section{regrets}

\DIFdelbegin \DIFdel{Per the }\DIFdelend \DIFaddbegin \DIFadd{The }\DIFaddend most major regret from \cite{ware12} \DIFaddbegin \DIFadd{is}\DIFaddend :\\

``I wish I'd had the courage to live a life true to myself, not the life others
expected of me.''\\

\DIFaddbegin \noindent \DIFadd{To this point, }\DIFaddend {\DIFdelbegin \DIFdel{There }\DIFdelend \DIFaddbegin \DIFadd{there }\DIFaddend is a Frank Sinatra's song ``My Way'':\\
And now, the end is near\\
And so I face the final curtain\\
My friends, I'll say it clear\\
I'll state my case of which I'm certain\\
I've lived a life that's full\\
I traveled each and every highway\\
But more, much more than this\\
I did it my way\\
Regrets, I've had a few\\
But then again, too few to mention\\
I did what I had to do\\
And saw it through without exemption\\
I planned each chartered course\\
Each careful step along the byway\\
But more, much more than this\\
I did it my way\\
Yes, there were times, I'm sure you knew\\
When I bit off more than I could chew\\
But through it all, when there was doubt\\
I ate it up and spit it out\\
I faced it all and I stood tall\\
And did it my way\\
I've loved, laughed and cried\\
I've had my fill, my share of loosing\\
And now, as tears subside\\
I find it all so amusing\\
To think I did all that\\
And may I say, not in a shy way\\
Oh no, no, not me\\
I did it my way\\
For what is a man, what has he got\\
If not himself then he has not\\
To say all the things he truly feels\\
And not the words of one who kneels\\
The record shows, I took the blows\\
But I did it my way''}\\


\DIFdelbegin \footnote{\DIFdel{And there are websites with more regrets, e.g.,:
I wish I wouldn't have compared myself to others.
I wish I'd taken action and dove in head first.
I wish I didn't wait to ``start it tomorrow.''
I wish I'd taken more chances.
I wish I was content with what I have.
I wish I'd have traveled more.
I wish I'd have laughed it off.
I wish I'd left work at work (for only 40 hours per week).
}%DIFDELCMD < \url{https://www.lifehack.org/articles/communication/these-20-regrets-from-people-their-deathbeds-will-change-your-life.html}
%DIFDELCMD < %%%
}
%DIFAUXCMD
\addtocounter{footnote}{-1}%DIFAUXCMD
\DIFdelend \DIFaddbegin {\DIFadd{And there are websites with more regrets, e.g.,:
I wish I wouldn't have compared myself to others.
I wish I'd taken action and dove in head first.
I wish I didn't wait to ``start it tomorrow.''
I wish I'd taken more chances.
I wish I was content with what I have.
I wish I'd have traveled more.
I wish I'd have laughed it off.
I wish I'd left work at work (for only 40 hours per week).
}\url{https://www.lifehack.org/articles/communication/these-20-regrets-from-people-their-deathbeds-will-change-your-life.html}
}
\DIFaddend 

Apart from palliative nurse diaries, there are  academic studies on
the topic. \citet{morrison11b} lists these regrets:

\begin{verbatim}
Romance, lost love -- 18.1%
Family -- 15.9%
Education -- 13.1%
Career -- 12.2%
Finance -- 9.9%
Parenting -- 9.0%
Health -- 6.3%
Other -- 5.6%
Friends -- 3.6%
Spirituality -- 2.3%
\end{verbatim}

\citet{roese05}  \DIFdelbegin \DIFdel{which }\DIFdelend is a meta \DIFdelbegin \DIFdel{aalysis }\DIFdelend \DIFaddbegin \DIFadd{analysis }\DIFaddend of earlier work on the topic:
\begin{verbatim}
Twelve Life Domains

Career: jobs, employment, earning a living (e.g., "If only I were a dentist")

Community: volunteer work, political activism (e.g., "I should have volunteered more")

Education: school, studying, getting good grades (e.g., "If only I had studied harder in college")

Parenting: interactions with offspring (e.g., "If only I'd spent more time with my kids")

Family: interactions with parents and siblings (e.g., "I wish I'd called my mom more often")

Finance: decisions about money (e.g., "I wish I'd never invested in Enron")

Friends: interactions with close others (e.g., "I shouldn't have told Susan that she'd gained weight")

Health: exercise, diet, avoiding or treating illness (e.g., "If only I could stick to my diet")

Leisure: sports, recreation, hobbies (e.g., "I should have visited Europe when I had the chance")

Romance: love, sex, dating, marriage (e.g., "I wish I'd married Jake instead of Edward")

Spirituality: religion, philosophy, the meaning of life (e.g., "I wish I'd found religion sooner")

Self: improving oneself in terms of abilities, attitudes, behaviors (e.g., "If only I had more self-control")
\end{verbatim}


\begin{verbatim}
Rankings of Life Regrets Within Life Domains (Studies 1 and 2a)

Study 1 (Meta-Analysis)
Study 2a (College Student Sample)
Rank	Domain	Proportion (%)	Rank	Domain	Frequency (%)
1	Education	32.2	1	Romance	26.7
2	Career	22.3	2	Friends	20.3
3	Romance	14.8	3	Education	16.7
4	Parenting	10.2	4	Leisure	10
5	Self	5.5	5	Self	10
6	Leisure	2.5	6	Career	6.7
7	Finance	2.5	7	Family	3.3
8	Family	2.3	8	Health	3.3
9	Health	1.5	9	Spirituality	3.3
10	Friends	1.5	10	Community	0
11	Spirituality	1.3	11	Finance	0
12	Community	0.95	12	Parenting	0
\end{verbatim}

\section{LOMS: \DIFdelbegin \DIFdel{LOVE OFMoney }\DIFdelend \DIFaddbegin \DIFadd{Love OfMoney }\DIFaddend Scale}

see \citet{tang2003income}:

\begin{verbatim}
Items of the Love of Money Scale (LOMS)
Factor 1: Importance
01. Money is important.
02. Money is valuable.
03. Money is good.
04. Money is an important factor in the lives of all
of us.
05. Money is attractive.
Factor 2: Success
06. Money represents my achievement.
07. Money is a symbol of my success.
08. Money reflects my accomplishments.
09. Money is how we compare each other.
Factor 3: Motivator
10. I am motivated to work hard for money.
11. Money reinforces me to work harder.
12. I am highly motivated by money.
13. Money is a motivator.
Factor 4: Rich
14. Having a lot of money (being rich) is good.
15. It would be nice to be rich.
16. I want to be rich.
17. My life will be more enjoyable, if I am rich and
have more money.
\end{verbatim}

\section{Descriptive Statistics}

%\input{../out/var_des}

%\input{../out/hist1}
%\input{../out/hist2}
%\input{../out/hist3}

Pairwise correlations (not shown) of greed variables with SWB are small, about
-.1, but so are pairwise correlations of other variables, e.g.,income is only correlated with SWB at  .2---one needs to remember that about half of SWB is explained by genes \DIFdelbegin \DIFdel{\mbox{%DIFAUXCMD
\citep{lykken96}}\hspace{0pt}%DIFAUXCMD
}\DIFdelend \DIFaddbegin \DIFadd{\mbox{%DIFAUXCMD
\citep{lykken96t}}\hspace{0pt}%DIFAUXCMD
}\DIFaddend .


\section{\DIFdelbegin \DIFdel{paper }\DIFdelend \DIFaddbegin \DIFadd{Paper }\DIFaddend body \DIFdelbegin \DIFdel{results; no beta coefficient but regular one}\DIFdelend \DIFaddbegin \DIFadd{Results:  Regular Coefficients (Not Beta)}\DIFaddend }

\begin{spacing}{.9} \begin{table}[H]\centering  \label{a} \begin{scriptsize} \begin{tabular}{p{1.8in}p{.5in}p{.5in}p{.5in}p{.5in}p{.5in}p{.5in}p{.5in}p{.5in}p{.5in}p{.5 in}p{.5in}p{.5 in}}\hline \input{../out/a.tex} \hline + 0.10 * 0.05 ** 0.01 *** 0.001; robust std err \end{tabular}\end{scriptsize}\caption{OLS regressions of SWB: \texttt{more hours and more money}}\end{table} \end{spacing}

\begin{spacing}{.9} \begin{table}[H]\centering  \label{b} \begin{scriptsize} \begin{tabular}{p{1.8in}p{.5in}p{.5in}p{.5in}p{.5in}p{.5in}p{.5in}p{.5in}p{.5in}p{.5in}p{.5 in}p{.5in}p{.5 in}}\hline \input{../out/b.tex} \hline + 0.10 * 0.05 ** 0.01 *** 0.001; robust std err \end{tabular}\end{scriptsize}\caption{OLS regressions of SWB: \texttt{next to  health, money is most  important}}\end{table} \end{spacing}

\begin{spacing}{.9} \begin{table}[H]\centering  \label{c} \begin{scriptsize} \begin{tabular}{p{1.8in}p{.5in}p{.5in}p{.5in}p{.5in}p{.5in}p{.5in}p{.5in}p{.5in}p{.5in}p{.5 in}p{.5in}p{.5 in}}\hline \input{../out/c.tex} \hline + 0.10 * 0.05 ** 0.01 *** 0.001; robust std err \end{tabular}\end{scriptsize}\caption{OLS regressions of SWB: \texttt{no right  and wrong ways to  make money}}\end{table} \end{spacing}

\begin{spacing}{.9} \begin{table}[H]\centering  \label{d} \begin{scriptsize} \begin{tabular}{p{1.8in}p{.5in}p{.5in}p{.5in}p{.5in}p{.5in}p{.5in}p{.5in}p{.5in}p{.5in}p{.5 in}p{.5in}p{.5 in}}\hline \input{../out/d.tex} \hline + 0.10 * 0.05 ** 0.01 *** 0.001; robust std err \end{tabular}\end{scriptsize}\caption{OLS regressions of SWB:  \texttt{job is just a way to earn money}}\end{table} \end{spacing}


\section{\DIFdelbegin \DIFdel{excluding poor}\DIFdelend \DIFaddbegin \DIFadd{Excluding Poor}\DIFaddend , \DIFdelbegin \DIFdel{bottom }\DIFdelend \DIFaddbegin \DIFadd{Bottom }\DIFaddend 10 \DIFdelbegin \DIFdel{percent}\DIFdelend \DIFaddbegin \DIFadd{Percent}\DIFaddend }

\DIFdelbegin \DIFdel{we }\DIFdelend \DIFaddbegin \DIFadd{We }\DIFaddend exclude those that are needy\DIFaddbegin \DIFadd{, }\DIFaddend not greedy
Interestingly (not shown) greed variables correlate with income at about -.2--meaning that
poorer people are more greedy \DIFdelbegin \DIFdel{, and to some }\DIFdelend \DIFaddbegin \DIFadd{(and to large }\DIFaddend degree needy, \DIFdelbegin \DIFdel{but again as }\DIFdelend \DIFaddbegin \DIFadd{of course). As }\DIFaddend a
robustness check we exlude bottom 10\% of income distribution \DIFdelbegin \DIFdel{as a robustness
check }\DIFdelend to make sure that we capture greed and not need--arguably being in bottom
10\% of income distribution and wanting more money may mean need rather than greed. 


%DIF < \begin{spacing}{.9} \begin{table}[H]\centering  \label{a} \begin{scriptsize} \begin{tabular}{p{1.8in}p{.5in}p{.5in}p{.5in}p{.5in}p{.5in}p{.5in}p{.5in}p{.5in}p{.5in}p{.5 in}p{.5in}p{.5 in}}\hline \input{../out/poora.tex} \hline + 0.10 * 0.05 ** 0.01 *** 0.001; robust std err \end{tabular}\end{scriptsize}\caption{OLS regressions of SWB: \texttt{more hours and more money}}\end{table} \end{spacing}
\DIFaddbegin \begin{spacing}{.9} \begin{table}[H]\centering  \label{a} \begin{scriptsize} \begin{tabular}{p{1.8in}p{.5in}p{.5in}p{.5in}p{.5in}p{.5in}p{.5in}p{.5in}p{.5in}p{.5in}p{.5 in}p{.5in}p{.5 in}}\hline \input{../out/poora.tex} \hline \DIFaddFL{+ 0.10 * 0.05 ** 0.01 *** 0.001; robust std err }\end{tabular}\end{scriptsize}\caption{\DIFaddFL{OLS regressions of SWB: }\texttt{\DIFaddFL{more hours and more money}}}\end{table} \end{spacing}
\DIFaddend 

%DIF < \begin{spacing}{.9} \begin{table}[H]\centering  \label{b} \begin{scriptsize} \begin{tabular}{p{1.8in}p{.5in}p{.5in}p{.5in}p{.5in}p{.5in}p{.5in}p{.5in}p{.5in}p{.5in}p{.5 in}p{.5in}p{.5 in}}\hline \input{../out/poorb.tex} \hline + 0.10 * 0.05 ** 0.01 *** 0.001; robust std err \end{tabular}\end{scriptsize}\caption{OLS regressions of SWB:  \texttt{next to  health, money is most  important}}\end{table} \end{spacing}
\DIFaddbegin \begin{spacing}{.9} \begin{table}[H]\centering  \label{b} \begin{scriptsize} \begin{tabular}{p{1.8in}p{.5in}p{.5in}p{.5in}p{.5in}p{.5in}p{.5in}p{.5in}p{.5in}p{.5in}p{.5 in}p{.5in}p{.5 in}}\hline \input{../out/poorb.tex} \hline \DIFaddFL{+ 0.10 * 0.05 ** 0.01 *** 0.001; robust std err }\end{tabular}\end{scriptsize}\caption{\DIFaddFL{OLS regressions of SWB:  }\texttt{\DIFaddFL{next to  health, money is most  important}}}\end{table} \end{spacing}
\DIFaddend 

%DIF < \begin{spacing}{.9} \begin{table}[H]\centering  \label{c} \begin{scriptsize} \begin{tabular}{p{1.8in}p{.5in}p{.5in}p{.5in}p{.5in}p{.5in}p{.5in}p{.5in}p{.5in}p{.5in}p{.5 in}p{.5in}p{.5 in}}\hline \input{../out/poorc.tex} \hline + 0.10 * 0.05 ** 0.01 *** 0.001; robust std err \end{tabular}\end{scriptsize}\caption{OLS regressions of SWB:  \texttt{no right  and wrong ways to  make money}}\end{table} \end{spacing}
\DIFaddbegin \begin{spacing}{.9} \begin{table}[H]\centering  \label{c} \begin{scriptsize} \begin{tabular}{p{1.8in}p{.5in}p{.5in}p{.5in}p{.5in}p{.5in}p{.5in}p{.5in}p{.5in}p{.5in}p{.5 in}p{.5in}p{.5 in}}\hline \input{../out/poorc.tex} \hline \DIFaddFL{+ 0.10 * 0.05 ** 0.01 *** 0.001; robust std err }\end{tabular}\end{scriptsize}\caption{\DIFaddFL{OLS regressions of SWB:  }\texttt{\DIFaddFL{no right  and wrong ways to  make money}}}\end{table} \end{spacing}
\DIFaddend 

%DIF < \begin{spacing}{.9} \begin{table}[H]\centering  \label{d} \begin{scriptsize} \begin{tabular}{p{1.8in}p{.5in}p{.5in}p{.5in}p{.5in}p{.5in}p{.5in}p{.5in}p{.5in}p{.5in}p{.5 in}p{.5in}p{.5 in}}\hline \input{../out/poord.tex} \hline + 0.10 * 0.05 ** 0.01 *** 0.001; robust std err \end{tabular}\end{scriptsize}\caption{OLS regressions of SWB:  \texttt{job is just a way to earn money}}\end{table} \end{spacing}
\DIFaddbegin \begin{spacing}{.9} \begin{table}[H]\centering  \label{d} \begin{scriptsize} \begin{tabular}{p{1.8in}p{.5in}p{.5in}p{.5in}p{.5in}p{.5in}p{.5in}p{.5in}p{.5in}p{.5in}p{.5 in}p{.5in}p{.5 in}}\hline \input{../out/poord.tex} \hline \DIFaddFL{+ 0.10 * 0.05 ** 0.01 *** 0.001; robust std err }\end{tabular}\end{scriptsize}\caption{\DIFaddFL{OLS regressions of SWB:  }\texttt{\DIFaddFL{job is just a way to earn money}}}\end{table} \end{spacing}
\DIFaddend 


\section{\DIFdelbegin \DIFdel{excluding rich}\DIFdelend \DIFaddbegin \DIFadd{Excluding Rich}\DIFaddend , \DIFdelbegin \DIFdel{top }\DIFdelend \DIFaddbegin \DIFadd{Top }\DIFaddend 10 \DIFdelbegin \DIFdel{perc}\DIFdelend \DIFaddbegin \DIFadd{Percent}\DIFaddend }
\DIFdelbegin \DIFdel{they may actually be capitalists or quasi capitalists
}\DIFdelend %DIF > they may actually be capitalists or quasi capitalists

%DIF < \begin{spacing}{.9} \begin{table}[H]\centering  \label{a} \begin{scriptsize} \begin{tabular}{p{1.8in}p{.5in}p{.5in}p{.5in}p{.5in}p{.5in}p{.5in}p{.5in}p{.5in}p{.5in}p{.5 in}p{.5in}p{.5 in}}\hline \input{../out/richa.tex} \hline + 0.10 * 0.05 ** 0.01 *** 0.001; robust std err \end{tabular}\end{scriptsize}\caption{OLS regressions of SWB: \texttt{more hours and more money}}\end{table} \end{spacing}
\DIFaddbegin \begin{spacing}{.9} \begin{table}[H]\centering  \label{a} \begin{scriptsize} \begin{tabular}{p{1.8in}p{.5in}p{.5in}p{.5in}p{.5in}p{.5in}p{.5in}p{.5in}p{.5in}p{.5in}p{.5 in}p{.5in}p{.5 in}}\hline \input{../out/richa.tex} \hline \DIFaddFL{+ 0.10 * 0.05 ** 0.01 *** 0.001; robust std err }\end{tabular}\end{scriptsize}\caption{\DIFaddFL{OLS regressions of SWB: }\texttt{\DIFaddFL{more hours and more money}}}\end{table} \end{spacing}
\DIFaddend 

%DIF < \begin{spacing}{.9} \begin{table}[H]\centering  \label{b} \begin{scriptsize} \begin{tabular}{p{1.8in}p{.5in}p{.5in}p{.5in}p{.5in}p{.5in}p{.5in}p{.5in}p{.5in}p{.5in}p{.5 in}p{.5in}p{.5 in}}\hline \input{../out/richb.tex} \hline + 0.10 * 0.05 ** 0.01 *** 0.001; robust std err \end{tabular}\end{scriptsize}\caption{OLS regressions of SWB:  \texttt{next to  health, money is most  important}}\end{table} \end{spacing}
\DIFaddbegin \begin{spacing}{.9} \begin{table}[H]\centering  \label{b} \begin{scriptsize} \begin{tabular}{p{1.8in}p{.5in}p{.5in}p{.5in}p{.5in}p{.5in}p{.5in}p{.5in}p{.5in}p{.5in}p{.5 in}p{.5in}p{.5 in}}\hline \input{../out/richb.tex} \hline \DIFaddFL{+ 0.10 * 0.05 ** 0.01 *** 0.001; robust std err }\end{tabular}\end{scriptsize}\caption{\DIFaddFL{OLS regressions of SWB:  }\texttt{\DIFaddFL{next to  health, money is most  important}}}\end{table} \end{spacing}
\DIFaddend 

%DIF < \begin{spacing}{.9} \begin{table}[H]\centering  \label{c} \begin{scriptsize} \begin{tabular}{p{1.8in}p{.5in}p{.5in}p{.5in}p{.5in}p{.5in}p{.5in}p{.5in}p{.5in}p{.5in}p{.5 in}p{.5in}p{.5 in}}\hline \input{../out/richc.tex} \hline + 0.10 * 0.05 ** 0.01 *** 0.001; robust std err \end{tabular}\end{scriptsize}\caption{OLS regressions of SWB:  \texttt{no right  and wrong ways to  make money}}\end{table} \end{spacing}
\DIFaddbegin \begin{spacing}{.9} \begin{table}[H]\centering  \label{c} \begin{scriptsize} \begin{tabular}{p{1.8in}p{.5in}p{.5in}p{.5in}p{.5in}p{.5in}p{.5in}p{.5in}p{.5in}p{.5in}p{.5 in}p{.5in}p{.5 in}}\hline \input{../out/richc.tex} \hline \DIFaddFL{+ 0.10 * 0.05 ** 0.01 *** 0.001; robust std err }\end{tabular}\end{scriptsize}\caption{\DIFaddFL{OLS regressions of SWB:  }\texttt{\DIFaddFL{no right  and wrong ways to  make money}}}\end{table} \end{spacing}
\DIFaddend 

%DIF < \begin{spacing}{.9} \begin{table}[H]\centering  \label{d} \begin{scriptsize} \begin{tabular}{p{1.8in}p{.5in}p{.5in}p{.5in}p{.5in}p{.5in}p{.5in}p{.5in}p{.5in}p{.5in}p{.5 in}p{.5in}p{.5 in}}\hline \input{../out/richd.tex} \hline + 0.10 * 0.05 ** 0.01 *** 0.001; robust std err \end{tabular}\end{scriptsize}\caption{OLS regressions of SWB:  \texttt{job is just a way to earn money}}\end{table} \end{spacing}
 \DIFaddbegin \begin{spacing}{.9} \begin{table}[H]\centering  \label{d} \begin{scriptsize} \begin{tabular}{p{1.8in}p{.5in}p{.5in}p{.5in}p{.5in}p{.5in}p{.5in}p{.5in}p{.5in}p{.5in}p{.5 in}p{.5in}p{.5 in}}\hline \input{../out/richd.tex} \hline \DIFaddFL{+ 0.10 * 0.05 ** 0.01 *** 0.001; robust std err }\end{tabular}\end{scriptsize}\caption{\DIFaddFL{OLS regressions of SWB:  }\texttt{\DIFaddFL{job is just a way to earn money}}}\end{table} \end{spacing}
 \DIFaddend 


% %\input{/tmp/a.tex} %aok_var_des

% % \begin{spacing}{.9}
% %   \begin{table}[H]\centering \caption{Summary statistics.} \label{sumSta} \begin{scriptsize} \begin{tabular}{p{1.8in}p{.5in}p{.5in}p{.5in}p{.5in}p{.5in}p{.5in}p{.5in}p{.5in}p{.5in}p{.5
% %             in}p{.5in}p{.5 in}}\hline
% %         \input{/tmp/aha2.tex}
% %          \end{tabular}\end{scriptsize}\end{table}
% % \end{spacing}

% % \begin{spacing}{.9}
% %   \begin{table}[H]\centering \caption{Correlation matrix.} \label{sumSta} \begin{scriptsize} \begin{tabular}{@{}
% %           p{1.2in} rrrrrrrrrrrrr @{}}\hline
% %         \input{/tmp/ahb2.tex}\hline
% %          \end{tabular}\end{scriptsize}\end{table}
% % \end{spacing}



% Table XXX shows variable distributions. If a variable has more than
% 10 categories it is classified into bins...

% %\input .... %TODO !!!! have input here histograms

% \section*{Additional Descriptive Statistics}
% \label{app_des_sta}

% %make sure i have [H] or h! ???
% % \begin{table}[H]
% % \caption{}
% % \centering
% % \label{}
% % \begin{scriptsize}
% % \input{../out/reg_c.tex}
% % \end{scriptsize}
% % \end{table}

%\newpage
%\theendnotes

\end{spacing}
\end{document}
