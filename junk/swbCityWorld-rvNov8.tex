
%\RequirePackage{lineno}
\documentclass[10pt, letterpaper]{article}      
\usepackage[margin=.1cm,font=small,labelfont=bf]{caption}[2007/03/09]
%\usepackage{endnotes}
%\let\footnote=\endnote
\usepackage{setspace}
\usepackage{longtable}                        
\usepackage{anysize}                          
\usepackage{natbib}                           
%\bibpunct{(}{)}{,}{a}{,}{,}                   
\bibpunct{(}{)}{,}{a}{}{,}                   
\usepackage{amsmath}
\usepackage[% draft,
pdftex]{graphicx} %draft is a way to exclude figures                
\usepackage{epstopdf}
\usepackage{hyperref}                             % For creating hyperlinks in cross references

 
% \usepackage[margins]{trackchanges}

% \note[editor]{The note}
% \annote[editor]{Text to annotate}{The note}
%    \add[editor]{Text to add}
% \remove[editor]{Text to remove}
% \change[editor]{Text to remove}{Text to add}

%TODO make it more standard before submission: \marginsize{2cm}{2cm}{1cm}{1cm}
\marginsize{1cm}{1cm}{.5cm}{.5cm}%{left}{right}{top}{bottom}   
					          % Helps LaTeX put figures where YOU want
 \renewcommand{\topfraction}{1}	                  % 90% of page top can be a float
 \renewcommand{\bottomfraction}{1}	          % 90% of page bottom can be a float
 \renewcommand{\textfraction}{0.0}	          % only 10% of page must to be text

 \usepackage{float}                               %latex will not complain to include float after float

\usepackage[table]{xcolor}                        %for table shading
\definecolor{gray90}{gray}{0.90}
\definecolor{orange}{RGB}{255,128,0}

\renewcommand\arraystretch{.9}                    %for spacing of arrays like tabular

%-------------------- my commands -----------------------------------------
\newenvironment{ig}[1]{
\begin{center}
 %\includegraphics[height=5.0in]{#1} 
 \includegraphics[height=3.3in]{#1} 
\end{center}}

 \newcommand{\cc}[1]{
\hspace{-.13in}$\bullet$\marginpar{\begin{spacing}{.6}\begin{footnotesize}\color{blue}{#1}\end{footnotesize}\end{spacing}}
\hspace{-.13in} }

%-------------------- END my commands -----------------------------------------



%-------------------- extra options -----------------------------------------

%%%%%%%%%%%%%
% footnotes %
%%%%%%%%%%%%%

%\long\def\symbolfootnote[#1]#2{\begingroup% %these can be used to make footnote  nonnumeric asterick, dagger etc
%\def\thefootnote{\fnsymbol{footnote}}\footnote[#1]{#2}\endgroup}	%see: http://help-csli.stanford.edu/tex/latex-footnotes.shtml

%%%%%%%%%%%
% spacing %
%%%%%%%%%%%

% \abovecaptionskip: space above caption
% \belowcaptionskip: space below caption
%\oddsidemargin 0cm
%\evensidemargin 0cm

%%%%%%%%%
% style %
%%%%%%%%%

%\pagestyle{myheadings}         % Option to put page headers
                               % Needed \documentclass[a4paper,twoside]{article}
%\markboth{{\small\it Politics and Life Satisfaction }}
%{{\small\it Adam Okulicz-Kozaryn\footnote{I thank \textbf{TODO}. All mistakes
%are mine.}} }

%\headsep 1.5cm
% \pagestyle{empty}			% no page numbers
% \parindent  15.mm			% indent paragraph by this much
% \parskip     2.mm			% space between paragraphs
% \mathindent 20.mm			% indent math equations by this much

%%%%%%%%%%%%%%%%%%
% extra packages %
%%%%%%%%%%%%%%%%%%

\usepackage{datetime}


\usepackage[latin1]{inputenc}
\usepackage{tikz}
\usetikzlibrary{shapes,arrows,backgrounds}


%\usepackage{color}					% For creating coloured text and background
%\usepackage{float}
\usepackage{subfig}                                     % for combined figures

\renewcommand{\ss}[1]{{\colorbox{blue}{\bf \color{white}{#1}}}}
\newcommand{\ee}[1]{\endnote{\vspace{-.10in}\begin{spacing}{1.0}{\normalsize #1}\end{spacing}\vspace{.20in}}}
\newcommand{\emd}[1]{\ExecuteMetaData[/tmp/tex]{#1}} % grab numbers  from stata

%TODO before submitting comment this out to get 'regular fornt'
\usepackage{sectsty}
\allsectionsfont{\normalfont\sffamily}
\usepackage{sectsty}
\allsectionsfont{\normalfont\sffamily}
\renewcommand\familydefault{\sfdefault}

%\usepackage[margins]{trackchanges}
\usepackage{rotating}
\usepackage{catchfilebetweentags}

\usepackage{abstract}
\renewcommand{\abstractname}{}    % clear the title
\renewcommand{\absnamepos}{empty} % originally center

\usepackage{pgfplotstable}
% -------------------- END extra options -----------------------------------------
\date{{}\today  \hspace{.2in}\xxivtime}
\title{  % remember to have Vistula University!!
%  Happiness is Flextime, part 2: the opposite of flextime, unpredictability
%\textbf{out of date--aper done in goog doc :(}
%city unhappiness is universal across the deveoped world--no cant say that bc
%they are not significant
The Urban--Rural Happiness Gradient Across Countries\\
\\ 
%Claims to the Contrary by
}
\author{
% Adam Okulicz-Kozaryn\thanks{EMAIL: adam.okulicz.kozaryn@gmail.com
%   \hfill I thank XXX.  All mistakes are mine.} \\
% {\small Rutgers - Camden  and Vistula University}
}

\begin{document}

%%\setpagewiselinenumbers
%\modulolinenumbers[1]
%\linenumbers

\bibliographystyle{/home/aok/papers/root/tex/ecta}
\maketitle
\vspace{-.4in}
\begin{center}

\end{center}


\begin{abstract}
This study shows, for the first time, that city unhappiness is 
common across the world. In all developed countries, people are happier in smaller places than in large places. Without exception, we find that city dwellers are not happier than rural residents. This finding is important because it contravenes a common belief that emerged claiming that urban areas are happier, arguably for ideological reasons \citep[e.g.,][]{glaeser11,glaeser14,burger20}. The effort to contravene the findings that cities tend to be less happy than smaller areas, is arguably due to utilitarian axioms: money is centered in cities (production, productivity, income, and consumption increase with population size), and therefore, cities have greater utility, so they must be happier. Yet, empirical evidence says otherwise. 
\end{abstract}
\vspace{.15in} 
\noindent{\sc % Happiness, Life Satisfaction,  Subjective Wellbeing  (SWB),
              % Cities, Urbaniciy, Urban-Rural, Urban-Rural Gradient, World Vaues Survey (WVS)
}
\vspace{.25in} 

Research by \cite{aok11a} provided evidence of a ``urban-rural gradient'' in most countries, where happiness levels rises from lowest in largest cities to highest in the smallest places. The gradient is non-linear---the very largest cities are markedly less happy than all other areas in a country, for example: New York City \citep{aok_brfss_city_cize16,senior_ny_sep16_14}, London \citep{ons11,ibt13}, Helsinki \citep{morrison15}, Bucharest \citep{lenzi16D}, and Sydney \citep[cited in][]{morrison11}.
The goal of this paper is to test the gradient across countries using one dataset with uniform variables. This study shows, for the first time, that city unhappiness is common across the world \footnote{Most extant research about the urban-rural happiness gradient is about the United States, Western Europe, recently China, and a handful of other countries. These studies were conducted in single countries, but not using a uniform dataset across countries. The three apparent exceptions \citep{aokcities,burger20,easterlin10al} do not actually examine the gradient---they all use binary urban-rural operationalizations and present simple mean differences for each country and aggregate results to group of countries in regressions. Similarly, Gallup data used by \citet{burger20} and \citet{easterlin10al} are problematic as elaborated later on in this paper.}
 
The intersection of Quality of Life (QOL), or Subjective Wellbeing (SWB)\footnote{SWB and QOL overlap, but there are important differences, notably QOL is more of an index/aggregate of domains and more subjective, while SWB is subjective mostly an evaluation of one's life as a whole--for a discussion see \citet{aok-swbLivability18}. For simplicity, we use the terms interchangeably throughout the paper.}, and Urban Studies is an exciting new area of research.
Academics, policymakers, administrators, and people in general, have started to pay more
attention to quality of life indicators, and not just to monetary measures such as GDP or income. It is now commonly accepted, even among utilitarians
\citep{stiglitz09al}, that money does not capture the full dimension of subjective wellbeing, being therefore timely to observe human flourishing and examine determinants of happiness for improving quality of life. This occurs at a time when the world is experiencing massive
urbanization---arguably, the most dramatic change to our way of life \cite{wirth38,hansonCityJournalautumn15}. Hence, it raises the question, how do cities affect the human condition? Specifically, do cities affect quality of life and subjective wellbeing?

% TODO once RR add:
% For review of urbanicitysee \citet{aokCityBook15}. This study is an extension
% and follow-up on \citet{aokcities}. For review of the literature on urbanicity
% and happiness see \citet{aokCityBook15}.

Modern research on the effect of cities on human wellbeing should be rooted on
extensive classic urban sociological research \citep{tonnies57,wirth38,simmel03,park15,park84}, 
which contended about the negative effect of cities on humans.
%
Quantitative research on the urban-rural happiness gradient dates back to
\cite{gurin60} and \cite{campbell76etal}, who found a negative effect of urbanicity on humans. Over the past several decades, several dozen studies mostly found a negative effect of urbanicity on human wellbeing as well
\texttt{[blind for peer-review]}% --for review see \citet{aokCityBook15}
. 
%ADAM, I think it's okay to cite yourself here, and several of the papers we got, they won't know we are the authors, so I would remove the blind for peer review reference. 

Yet, most research in the area examines the United States, Western Europe, and recently China and handful of other countries. Most studies are conducted focusing on a single country. Hence, we contribute to the literature by using a uniform dataset across countries.

In what follows, we investigate the relationship between urbanicity and happiness across the world. We begin by defining SWB and the mechanisms likely to link the size of a place to SWB, then discuss the evolving literature on urbanicity and wellbeing, and provide a critical review of economic theory. Then, we present our model, documenting how we used the received literature to control for sources of individual variation, discuss results, and conclude by discussing our the findings that urban dwellers are unhappier across the world, when compared to rural residents. 

   
\section*{Subjective Wellbeing}

Subjective wellbeing is an umbrella term for various subjective measures
of wellbeing, notably positive and negative affects, happiness, and life
satisfaction. Most of the SWB research, including this study, uses the life satisfaction measure, which is a global self evaluation of one's life as a whole. This measure is mostly cognitive and not affective---a respondent evaluates her life as whole globally (everything, including professional, personal, family, community, etc). The measure captures everything that is going on in one's life---that's a major advantage of the SWB measure over other social and economic indicators aiming at measuring the human condition, progress, and development. The SWB measure is simply the most comprehensive measure possible dwarfing earlier measures such as income, education, or life expectancy. For a review see \citet{diener09}.
%TODO once RR add \cite{aok_lsPol16}
Following usual practice, for simplicity, we use these terms interchangeably: SWB, happiness, and life satisfaction, but specifically we mostly mean life satisfaction as defined above.

The SWB measure is also at least adequately reliable and valid and considered good
enough for public policy making and public administration \citep{diener09,stiglitz09al}. And it has been used multiple times in urban research \citep[e.g.,][]{moeinaddini20,mouratidis19,wang19,anon17-cities-oslo,ma17,wkeziak16,valente16,chen15}.

There are cross-cultural comparability caveats, however, and SWB may not be
adequately comparable across countries \citep{kahneman99,diener09}. This limitation should be kept in mind when comparing results across countries in the present study. More focus should be on within-country differences, and this is what this study is mostly about---the difference between smaller and larger places in terms of SWB
within countries. We treat each country separately and do not pull the data together. In short, one should focus on within-country differences across urbanicity and exercise caution when comparing effects across countries. 

\section*{Definition, Theory, and Potential Causal Mechanism}

This is an observational study, not an experiment, and we don't test causality, nevertheless it is important to discuss the potential causal mechanisms driving unhappiness in the largest places.

It is useful to start with the theory that defines urbanicity and predicts how it would affect SWB. 
We start with the classic urban sociological theory of urban malaise
 \citep{tonnies57,wirth38,simmel03,park15,park84}: cities produce superficiality,
transitoriness, withdrawal, impersonality, superficiality, deviance,
shallowness, anomie, alienation, and cognitive overload.\footnote{The classics
  argued that poor social ties existed in cities, but refer to later arguments by Fischer and his subcultural theory \citep{fischer95,fischer75,fischer72}.} 
Sociological theory does not specify at which point urban malaise arises, there is clearly no hard
cutoff point, rather, the more urban, the more malaise. There may be a certain
threshold though, at which malaise intensifies as hinted at by
\citet{fischer73}: in the largest cities% , $>$several hundred thousand
.
In classical urban sociology, a city is defined by a large population size, its density,
and heterogeneity \citep{wirth38}, and clearly it is not a binary distinction,
but a gradient:''we should not expect to find abrupt and discontinuous variation
between urban and rural'' \citep[][p. 2]{wirth38}. To sum up, urbanicity has mostly negative
effect on humans, and it is rather a continuum than a binary, although a threshold
at a population of several hundred thousand may exist where malaise intensifies.

Another indication of continuity in the effect of the size of a place on the human condition
comes from physics% , and clearly points to a continuum
. There is a physical city constant of 1.15: if you double the area's population size, many phenomena (crime, GDP, income, patents) increase by 15\% \citep{blissCL_nov4_14,bettencourt10,bettencourt10b,bettencourt07}.\footnote{For example, supposed there's a city with a population of 1 million and a murder rate of 10 per 100k, for a city with a population of 2 million, the murder rate would be 11.5 per 100k, and so on.} 

We would like to especially highlight that for over 95\% of our evolutionary history\footnote{% Already Simmel observed
   % that old cities had a character of today's small town--for instance see
   % \citet[p. 333]{simmel03}.
Per human species evolutionary history, for instance, see encyclopedia Britannica
   \url{http://www.britannica.com/EBchecked/topic/277071/hunting-and-gathering-culture}. 
    % Wikipedia
   % \url{http://en.wikipedia.org/wiki/List_of_largest_cities_throughout_history},
   % and  also see .
   For post-medieval history see \citet{white77}.%  World population percent living in cities larger than 100k
   % is from , table 1. 
 } we have lived outside of cities as hunter-gatherers usually in small bands of 50--80 people \citep{maryanski92}.
 This way of life, only started to slowly change in about 10,000 years B.C.  with the domestication of animals
 and agriculture. The first large cities (larger than several hundred thousand% again,$>$250k
 ) only emerged after 500 B.C. and there were just handful of them. 
 It was only after industrialization that large cities started to house noticeable proportion of the population, and only in the 20th century that we saw an urbanization explosion---in 1800 a mere 1.7\% of the world population lived in cities larger than 100k, it slowly increased to 2.3\% in 1850, by another 50 years it doubled to 5.5\% in 1900,
 and then it doubled again to 13\% in 1950 \citep{davis55}.

The larger the place, the more the environment differs from the habitat in which we have evolved: dense and crowded,\footnote{There are striking examples of crowding in largest cities.  To be sure, the majority of urban population does not live in such extreme crowding, the trend however is in that direction as cities are becoming larger and less affordable. And, even without extreme crowding, the usual population density is related to crime \citep{bettencourt10b}. There is also evidence that density relates to negative consequences: interestingly, there is evidence that density impacts pathology more than crowding \citep{levy1974effects}. Yet, it is not only density and crowding, other factors such as social support matter as well \citep{cassel2017health}. Some studies didn't find a negative effect of density or crowding and the results were mixed \citep{collette1976urban}. While it seems to be reasonable to assume that density and crowding are positively related, some studies do not find this to be the case \citep{webb1975meaning,rodgers1982density}. Crowding probably has become more common in recent years as cities are becoming less affordable. \citet{misraCL15oct6,floridaCL18apr11,weinbergCL16aug11,solariMISC19apr24,schuetzMISC19may7,kotkin_db_mar20_13}
 For a nice discussion and overview of density, crowding and human behavior see \citet{boots1979population,choldin1978urban}.}
airports, subway or rapid transit, tall buildings in downtown, etc. And while urbanness is a continuum, there is a threshold, likely around several hundred thousands of people, when the built environment changes significantly.
%
 There are at least several significantly different stages of urbanness on the
 urbanness continuum: wilderness, open country, villages, small towns, large towns, cities, large
 cities, and very large cities. Surely, it is difficult to capture urbanness in
 its entirety---most datasets only allow us to analyze a few stages, including the data used
 here. But the point is that treating urbanness as an urban-rural dichotomy \citep[][]{glaeser11,burger20} is an
 oversimplification without much theory to support it.

The biological/evolutionary perspective can be complemented by recent neurological evidence. Urban living is unhealthy to the human brain \citep{lederbogen11} and urban living contributes to the development of psychosis \citep{abrahamyan20}.
 
\section*{Utilitarianism and Happiness} % OR COMBINE WITH NEXT ONE
% \section{the three studies on urban rural happiness gradient across countries}

%  , because biased and
% flawed research is produced.
%
% like one by stiupd wolfers too about income and happiness, that it is ot
%quadratic and just take the log lol
%
The discipline of Economics is largely driven by ``axioms'' (the self-evident truths)
or ``laws.''\footnote{Note that no other social science discipline has axioms, and for a good
reason---they do not exist in the social world, and so they should not appear in
social science. See \citet{feynman81} and \cite{davies18} for elaboration.} 
One axiom is that the more money (income or consumption), the greater the utility \citep[e.g.,][]{autor10}:
\begin{equation}
%  income = consumption % (\pm investments and savings)
money = utility \approx happiness
\end{equation}%

Utility, however, cannot be measured, thus, it is often operationalized as ``happiness'' in the discipline.\footnote{Curiously, some economists who do happiness research are skeptical about it at the same time, and do not consider happiness worthy investigation \citep[e.g.,][]{deaton13c,glaeser14B,glaeser14}.} Although, Easterlin (\citeyear{easterlin15B,easterlin10B}) (and many others) found that over time, in the long run at country level, income is unrelated to happiness: the so called Easterlin Paradox, some researchers have argued the contrary.  \citet{stevenson13}, for example, challenged the Easterlin Paradox by claiming to have conflicting ``evidence.''Except, that they study something different---they examine a different unit of analysis (data at the household level, or across countries at one point in time), and log transform the data. 

The effort to contravene the findings that cities tend to be less happy than smaller areas, is arguably due to utilitarian axioms: money is centered in cities\footnote{Production, productivity, income, and consumption increase with population size \cite{glaeser11C,glaeser07,glaeser01,rosenthal02,rosenthal03,rosenthal08}.}, and therefore, cities have greater utility, so they must be happier. 

Another case in point, is \citet{glaeser11} who examines only poor countries for his urbanicity--happiness analysis, and argues that the relationship holds in general. He basically argues that the positive relationship and the effect is ``driven primarily by poorer countries''---which yields the impression that this overall relationship is
positive for all countries and stronger for poorer countries. Empirical evidence however is incongruent: for most countries the relationship is negative and it is only positive in a few cases, typically in the very poorest countries. Concurrently, \citep{burger20}, who states, ``In line with earlier research, we found that
urban populations are, on average, happier than rural populations in that they return higher levels of happiness,'' builds his case by focusing on exceptional outliers, mostly poor African countries. 

Similarly, \cite{glaeser14} analyses US counties, but retains only cities and drops all other areas. In additional, the analysis is saturated with many controls, and by adding state-fixed effects, which correlates with population size, the relationship flips from a negative to a positive correlation with urbanicity. In contrast, \cite{aok_brfss_city_csize16} using the very same data finds a negative relationship by examining all areas. 
These studies have contributed to the notion that cities are happier than smaller places, yet, the overwhelming evidence points to the contrary %Adam: need to cite here our study, and others to provide more weight to our argument\citep{}. 

\section{What We Know So Far: The Literature}

Most research on the urbanicity--happiness relationship points to an urban-rural happiness gradient, where happiness raises from its lowest level in largest cities, to the highest level in smallest rural areas
\citep[e.g.,][]{campbell76etal,aok11a, ons11, morrison11,
  aok_brfss_city_cize16,senior_ny_sep16_14,ibt13,
  morrison15,lenzi16D,aok20}. \texttt{[blind for peer review]} %For review see \cite{aokCityBook15}.

Yet, most research has been conducted in the US or Western Europe, and there are only three cross-country investigations using a common dataset: \citet{aokcities,easterlin10al,burger20}.

\citet{easterlin10al} focuses on the effect of economic growth by urban--rural and only a small part of the study is about urban--rural differences in SWB, and their results are similar to \citet{aokcities}, who found that in developed countries people are less happy in cities. All three studies, however, are limited. 
%
First, there is no urban--rural gradient in these studies---they all use binary operationalizations, urban v. rural (or three categories). Also, they mostly present simple mean differences for each country and aggregate results to groups of countries in regressions and fail to control for necessary predictors of SWB. 

Most critically, there are multiple problems with the Gallup data used by \citet{easterlin10al,burger20}.\footnote{\citet{easterlin10al} acknowledge Gallup's limitations and attempts
to address them. \citet{burger20}, on the other hand, does not.}  First, it is not meant for
research but for commerce---Gallup charges \$30,000 for data access (per one
year!).\footnote{Gallup charges \$30,000 per year for the use of their happiness data (author's
email inquiry)--private corporations are making a fortune from tax dollars and students tuition--scholars should resist corporatization of academia \citep{mills2012corporatization,cox2013corporatization,millsNYT12fa,CatropaNYT20feb8,schmidlinNYT15oct10}, and
corporatization of happiness research \citep{davies15}.} Second, the urbanicity classification is twofold less precise than in the World Value Survey (WVS), which we used in the present study: 4
versus 8 categories. Third, while the WVS uses precise population size with numeric cutoffs,
Gallup uses fuzzy concepts such as ``rural area'', ``small town or village'', ``large city''.
Fourth, (and this compounds with the third problem), Gallup uses self-reports of urbanicity, which is highly
subjective and problematic in this case---many, if not most people, would likely
classify themselves completely arbitrarily into ``rural area'' versus ``village'' and so forth. The WVS uses interviewer's information about the place. Fifth, apparently much of the data are missing---\citet{easterlin10al} notes that in 14 countries ``rural area'' responses were exceptionally low.
Also, about half of the world population is rural, but \citet{burger20} reports that in their
dataset only about a quarter of respondents report rural residence.

Urbanness or urbanicity is a degree, not a dichotomy.\footnote{Strikingly, \citet{burger20} argues that there is a uniform way to measure urbanicity, which is a mere 3 categories: 1)
Cities, 2) Towns and semi-dense areas and 3) Rural areas; yet, uses a dichotomy in their study.} 

This study is the first to study the urban--rural happiness gradient across countries using a more robust and accurate dataset. 

\section*{Data And Model}

We use the \url{www.worldvaluessurvey.org}, which is representative of about 90\% of
the world population,\footnote{While the WVS is conducted in about 100 countries
  that represent about 90\% of the world population, due to missing data for
  the particular variables of interest, the present's study coverage is slightly
smaller covering about 70 countries (depending on the model and specification).} and as elaborated in the previous section, is much better suited for the study than an inadequate and poorly designed Gallup data. The variables are listed in table \ref{var_des}. 
Country codes and descriptive statistics are in SOM (Supplementary Online Material).  % in this study for the models reported in the body of the paper we use XX countries.

The SWB question reads, "All things considered, how  satisfied are you with your life as a whole these days? Using this card on which 1 means you are "completely dissatisfied" and 10 means you are "completely satisfied" where would you put your satisfaction with your life as a whole?"

Urbanicity is operationalized with the WVS variable ''\texttt{X049}''--note that it is objective and recorded by reviewer, not respondent.
%adam: what do you mean that it's recoded by reviewer?!
There are eight categories ranging from '$<2k$' to '$>500k$.' This is an important advantage, because as elaborated earlier, urbanicity or urbanness is a continuum, not a binary urban v rural dichotomy. We conduct the analysis using a set of dummy variables for all eight categories (leaving out the base case) in the SOM. However, for simplicity and ease of exposition we present simplified results in the body of the paper using three categories only.
In other words, this study will use 8 categories of urbanicity, and summarize results for ease of presentation with 3 categories. Thus, please refer to the Supplemental Material for all the results of all categories.

Because in many countries, there are either no observations or few observations
in the first two bottom categories \texttt{-2k} and \texttt{2-5k}, we combined
them together for the analyses in the main body of the paper. These two
categories together proxy a city-free natural environment most closely resembling the natural human habitat where we have evolved, and it includes: wilderness, open country, and small villages. The other critical category that must be measured based on earlier review of theory is large cities, there is likely to be a threshold at several hundred thousand, hence we use the top category on the WVS variable ``\texttt{X049}'' which is '\texttt{>500k}' as a proxy of large cities. Such places, are the least resembling of the natural human habitat and are
mostly consisting of man--made objects such as asphalt, concrete, glass, etc., and
as per theory, are likely to be least happy. The third category in our main analyses are the places in between, \texttt{5-500k}.
%
% In other word, we are exploring the urbanness gradient fully in the SOM, and
% only then in the second step, for ease of exposition, but preserving original
% findings and patterns, in the body of the paper we simplify it by looking at two extremes
%  \texttt{-5k} v \texttt{500k-}.
The cutoffs for the two extremes are important and must be driven by theory, it cannot be, say, everything up to 100k versus everything more than 100k (or any other values) as in some of the other research. A place never changes abruptly from rural to urban at some cutoff, it is a continuum, it can be simplified to carefully chosen extreme categories, but one must always start with the continuum.
%
And because this aggregation or simplification into 3 categories is still
somewhat arbitrary, we present our alternative specifications in the supplemental material in addition to the full 8-step urbanness gradient. 

% Again the rationale as per theory above is to explore the gradient, the simplest way to do it is to
% contrast smallest v medium and largest places
% this is the key!!! it is not everything is urban or rural as in
%   easterlin or burger! it is rather very rural, say <2k or say <10k v very big
%   >couple hundred thousand; stuff in the middle is mixed; it is a gradient! we
%   use two as gradient illustrative extremes not cassifying everything in
%   between! again gradient is non-linear it is very largest cities v everything
%   else, that dichotomy make sense too, but neither thsi is what easterlin and
%   burger do
% The rationale, as per theory is to 

% It is a gradient raisng from smallest to greatest as argued in
% \citet{aokcities}, hence presentation must be ordinal with multiple categories,
% we have 8 in appendix and can be summarized as smallest v largest; why smallest
% v largest

% \textbf{this is the key!!! it is not everything is urban or rural as in
%   easterlin or burger! it is rather very rural, say <2k or say <10k v very big
%   >couple hundred thousand; stuff in the middle is mixed; it is a gradient! we
%   use two as gradient illustrative extremes not cassifying everything in
%   between! again gradient is non-linear it is very largest cities v everything
%   else, that dichotomy make sense too, but neither thsi is what easterlin and
%   burger do}

Table \ref{var_des} lists the control variables used in the body of the
paper.% \footnote{There are models with additional controls in SOM (Supplementray
  % Online Material).}

\input{../out/varDes.tex}
 
In the choice of controls we generally follow \citep{aok20}. There are specific
controls worth discussing.
%
 Young, single and childless persons and young men with tertiary education are
 relatively more satisfied with urban areas as a place of residence \citep{anon-regional-studies-19}.
Income, class,  and education are important controls---they not only predict greater
 SWB, but are also confounded and higher in cities.\footnote{As shown later comparing unadjusted means results, research reporting that people are happier in cities (e.g. \citep{burger20}) is notably due to the confounding of higher income education and class--see appendix for tables with and without controls.} 
 %ideallly woiuld like cost iof living but missing duh

One great advantage of city living that is often forgotten is freedom, ``City
 air makes men free (Stadt Luft macht frei)'' \citet[p. 12]{park84}\footnote{It
   originated in the Middle Ages, and it meant freedom from feudalism,
   non-feudal islands in a sea of feudalism \citep{harvey12}.}, hence we control
 for freedom. 
 
Likewise, trust is important, as it predicts SWB, and it is lower in cities
 \citep{milgram70}.
 %when RR add my misanthropolis

Health is a key predictor of SWB, and the subjective health measure used here is a reasonable measure of actual health \citep{subramanian09b}.

We use a standard OLS regression with robust standard errors.  We treat the 10-step
happiness variable as continuous. An ordinal happiness variable can be treated as a
continuous variable \citep{carbonell04}.
%
OLS has become the default method in happiness research
\citep{blanchflower11}. Theoretically, while there is still debate about the
cardinality of SWB, there are strong arguments to treat it as a cardinal
variable \citep{ng96,ng97}. 

% \pgfplotstableread{
% N    Ans
% 1   -36
% 2    33
% 3   -52
% 4   -22
% 5    33
% 6    38
% 7    48
% 8  -100
% }\mytable

% \pgfplotstabletranspose[string type , colnames from=N, input colnames to=N]\mytablenew{\mytable}
% \pgfplotstabletypeset[string type]{\mytablenew}


%BUG this corrupts the table!!!
% \pgfplotstableread{/tmp/a.txt}{\a}
% \pgfplotstableread{/tmp/b.txt}{\b}
% \pgfplotstableread{/tmp/c.txt}{\c}

% {\scriptsize
% \pgfplotstabletranspose[string type, unbounded coords=jump]\anew{\a}
% \pgfplotstabletypeset[string type]{\anew}
% }

% {\scriptsize
% \pgfplotstabletranspose[string type , colnames from=country, input colnames to=country]\bnew{\b}
% \pgfplotstabletypeset[string type]{\bnew}
% }

% {\scriptsize
% \pgfplotstabletranspose[string type , colnames from=country, input colnames to=country]\cnew{\c}
% \pgfplotstabletypeset[string type]{\cnew}
% }

\section*{Results}

% While, again, we do concern ourselves with the coninuum, the urban-rural happiness
% gradient, results are in the SOM, here for ease of exposition, we only preent
% the contrast between smallest, medium, and largest places.
%
There is a considerable tradeoff in this study between ease of presentation and
elaboration as there are dozens of countries and presenting different
specifications would result in unwieldy presentation---additional specifications
are in SOM. Here we just present one model that is full including all necessary
and some additional controls (yet, not over saturated where too many variables
result in too many missing observations)---we use here models with controls listed in table
\ref{var_des} 
also the model presented here uses just 3 categories, $<5k$ (base), $5k-500k$, and $500k>$. Results are set in table \ref{d1}. We are interested in the comparison between $<5$ versus $500k>$ because this is according to theory: evolution/ingroup versus most unnatural environment (as this data allows).

%NOTE MANUALLY DELETED -5 column with zeroes and comas for last col
\begin{spacing}{.9} \begin{table}[H]\centering   \begin{scriptsize} \begin{tabular}{p{.5in}p{.5in}p{.5in}p{.5in}p{.5in}p{.5in}p{.5in}p{.5in}p{.5in}p{.5in}p{.5 in}p{.5in}p{.5 in}}\hline \input{../out/exT3-3-manual.tex} \hline * p$<$0.05, $+$ p$<$0.1; robust std err \end{tabular}\end{scriptsize}\caption{\label{d1}OLS regressions of SWB on place size for each country separately including year
dummies (not shown).}\end{table} \end{spacing}


The results in table \ref{d1} % for 24 countires are significant, and vast majority, 19 countires, or
show that out of countries with significant happiness differences across
urbanicity, in 80\% of countries, people are less happy in cities than in
smaller areas. The only exceptions are in the East European Post Soviet countries (ALB, ROU, RUS), and in South-Asian countries (BGD and IND). Notably, these are all poor or developing
countries. In all developed countries, people are happier in smaller places than in large places. Without exception, we find that city dwellers are not happier than rural residents. 

% Results in table \ref{d1} are remarkable. In most countries large cities are less
% happy than small settlements. Remarkably, 
% and the point is that the only ones that are sig and positive are wiers small
% poor and most miserable countries except india and russia which is a big puzzle
%
The conclusion is that in all developed countries studied here, AUS, CAN, DEU, ESP, ITA,
NLD,\footnote{results only in appendix for NLD} NZL, SWE, USA,  the largest areas
are less happy than smaller areas. \footnote{At least in less elaborate specifications shown in the appendix, but even in the most elaborate specifications, even when the coefficient on larger places is insignificant, it is still negative.}

%\textbf{MAYBE LATER}
% make the biggest SWB gaps by country key focus of paper :) 
The urban-rural gradient is greatest in EGY, VEN,\footnote{Note: result for VEN should be interpreted with caution this is the main difference with table \ref{exT4-3} and probably has to do with the fact that there are only 60 obs on the base case category. Other results are
  similar between the two tables.} and VNM have effect sizes larger than one, while effect sizes for most other places are small to moderate, around .3--.5 (on the 1--10 SWB scale).   
 %
Yet, as indicated earlier, because of the limited cross-cultural comparability of the SWB measure, when interpreting our results, the focus should be on within--country SWB differences across urbanicity, and not on comparing cross--country effect sizes.
% We call for retraction of these fake papers.

It is worth noting that in the first column, the majority of the results are
negative with only 5 countries yielding a positive result: GHA, MDA, PER, RUS, and ZAF---again, what is remarkable is that none of these countries are considered ``developed.''


\section*{Conclusion And Discussion}

Throughout most of our evolutionary history, humans have lived in small homogeneous groups with low density. As hunters gatherers humans lived in small bands of 50 to 80 people, later on in simple horticultural society in groups of 100 to 150 people, and in more advanced society these groups reached five to six thousand people \citep{maryanski92}. Hence, unlike other species %\footnote{Human nature is unlike that of bees: by one estimate we're 90\% chimp and only 10\% bee  \citep{haidt12B}.} 
living in heterogeneous, dense, and large settlements (city living) is simply unnatural to human beings.  
%
It is not city problems but the city itself that results in lower wellbeing
\citep{aok_brfss_city_cize16}--the mechanism is important how and why would
urbanness affect human wellbeing? As discussed throughout, contrary to economics, classic urban
sociological theory, biological/evolutionary mechanism, neurological evidence
point to lower human wellbeing in cities.

In vast majority of countries effect in negative, only positive in these:
East European Post Soviet ALB, ROU, RUS, and South-Asian BGD and IND.
East European Post Soviet countries are still quite centralized where power, opportunity, and resources are located in
large cities. %So largest places still do have most of the benefits of the central place as per central place theory
India and Bangladesh are curious outliers. \texttt{[blind for
  peer-review]} %For discussion of India see \{citep schourjua paper}
%

Also note that in about a third or even half of the countries (depending on the
model), there is no SWB difference across urbanicity. This is also a finding
worth reporting as it runs counter to common pro-urbanism and city triumphalism
\citep[e.g.,][]{glaeser11}.
%
One would think cities are the best places to live as people flock there in doves
% could Show graph from UN, and the common narrative is that city is the place to be, and so one would think people are happier there.
So finding of no difference for many cases is already surprising.

Even as coefficient estimates are small to moderate, the practical significance
of the results is very strong because of the sheer size of urbanization. 
%
Say, even a minuscule negative effect of .1 (on scale 1-10) on larger place v
smaller place for a small country
of 10m translates into an effect equivalent to making 100k people from most
miserable to most happy on SWB scale 1-10 if everyone lived in smaller v larger place. % Small or moderate effect sizes translate into large effects because of the sheets size of urbanization in billions of people.
 Globally, for billions of people living in cities, there is a massive amount of
 human misery produced. 

Why in developed world people are less happy in large cities, but in some
developing countries they are happier? There is at least one reason. In many
developing countries life is simply unbearable outside of the city lacking
necessities such as shelter, foot, water, sanitation, and healthcare. In
developed countries, even smallest places have reasonable access to necessities,
and they do not suffer from urban disamenities. %burger will save you in addis
                                %abeba and kill yuo in nyc
% Also, it
% may be so that the grass is always greener on the other side--people idealize
% what is missing or what is rare. 
 
As per Maslow's pyramid of needs \citep{maslow87}, survival and opportunity
come first, and this arguably can explain much of the paradox found in this
paper--despite the city being biologically, neurologically, and socially negative
development for humans, cities can be helpful for human wellbeing at an early stage of development. 


We would like to finish by raising an important issue that is related to the topic. 
An alarming trend in higher education, and in research in general is
corporatization of higher education and research \citep{mills2012corporatization,cox2013corporatization,millsNYT12fa,CatropaNYT20feb8,schmidlinNYT15oct10}. This includes happiness
research \citep{davies15}. ``World Happiness Report'' \citep{helliwell20} and its chapter about urban-rural gap in
happiness \citep{burger20} uses data from a private corporation. Indeed, the
report % , and its presentation (e.g., that new zealand happiness conference)
 is
 largely an advertisement for Gallup. Gallup then sells the happiness data at
\$30,000 (per year!)\footnote{Author's inquiry with Gallup to use their data.}--arguably this is
not meant for research (most researchers cannot afford it). The goal here
appears not to produce knowledge, but to make money--after all the sole
responsibility of a business is  profit \citep{friedman70}.

\section{Takeaway for Practice}
Humans are worse off in cities (in terms of happiness). But not always what
makes us happy is the right thing to do \citep{linden11,haybron08,nussbaum05}. Notably,
climate change is more important than human happiness, and cities are most
environmentally friendly type of settle mt \citep{meyer13}. and there are some limited
things that can be done to make cities less miserable--we know what makes city
happy \citep{ballas13}.

Perhaps the clearest takeaway for practice is that we suffer from overpopulation.
Again, we only need cities because of overpopulation and climate change \citep{pachauri14}, not
because of production or productivity or consumption premium of cities. In fact,
we have too much consumption and we need less consumption
\cite{dittmar14,kasser13,leonard10}. In fact we arguably also need less
production and less economic growth \cite{kallis12,kallis11,bergh11}. While
again cities are most environmentally friendly way to squeeze human
overpopulation (MEYER) to deal better with climate change, cities directly cause
climate change by being centers of production and consumption that drives
climate change. 

but we would need cities less if we had fewer people--contraception,
sterilization etc; once there are fewer people, then we can have a meaningful
discussion about right city size as we used to have couple decades ago
\texttt{[blind for peer review]} %see
                                %discussion in my city book
--it is remarkable that there is
no discussion about it! how could we have gone so wrong to think that the bigger
the better and that there is no limit--cities are ballooning--Tokyo has about 40m
people, and there are many 20m cities; The greatest and largest cities of
antiquity, the Ancient Athens were 140k and Rome was 450k. 

% % \begin{spacing}{.9}
% %   \begin{table}[H]\centering \caption{Correlation matrix.} \label{sumSta} \begin{scriptsize} \begin{tabular}{@{}
% %           p{1.2in} rrrrrrrrrrrrr @{}}\hline
% %         \input{/tmp/gssLonnie//tmp/ahb2.tex}\hline
% %          \end{tabular}\end{scriptsize}\end{table}
% % \end{spacing}



% Table XXX shows variable distributions. If a variable has more than
% 10 categories it is classified into bins...

% %\input .... %TODO !!!! have input here histograms

% \section*{Additional Descriptive Statistics}
% \label{app_des_sta}

% %make sure i have [H] or h! ???
% % \begin{table}[H]
% % \caption{}
% % \centering
% % \label{}
% % \begin{scriptsize}
% % \input{/tmp/gssLonnie/../out/reg_c.tex}
% % \end{scriptsize}
% % \end{table}

%\newpage
%\theendnotes
\bibliography{/home/aok/papers/root/tex/ebib.bib,swbCityWorld.bib,/home/aok/papers/root/rr/misanthropy-trustCity/tex/trustCity.bib}

\section{SOM: ONLINE APPENDIX (THIS WILL NOT BE A PART OF THE PAPER}

\subsection{Country Codes}

note for ease of presentation numbers rounded to full digits
\begin{spacing}{.9} \begin{table}[H]\centering \caption{.} \label{d1} \begin{scriptsize} \begin{tabular}{lllll}\hline \input{/home/aok/papers/swbCityWorld/out/lctry.tex} \hline   * p$<$0.05, $+$ p$<$0.1; robust std err \end{tabular}\end{scriptsize}\caption{lctry}\end{table} \end{spacing}
these were dropped as data were missing on major categories if there less than
30 obs on both collectively 2 smallest categories or on top category: enumerate here:
\textbf{TODO}

note for ease of presentation numbers rounded to full digits
\textbf{TODO} discuss in depth interesting differences lol

\begin{spacing}{.9} \begin{table}[H]\centering \caption{.} \label{d1} \begin{scriptsize} \begin{tabular}{llllllllll}\hline \input{/home/aok/papers/swbCityWorld/out/lcount.tex} \hline   * p$<$0.05, $+$ p$<$0.1; robust std err \end{tabular}\end{scriptsize}\caption{lcount}\end{table} \end{spacing}


\begin{spacing}{1.4} %TODO MAYBE before submission can make it like 2.0
\rowcolors{1}{white}{gray90}


\section{descriptive stats}

like in eb paper maybe also like min and max and everything
in app but order somehow anyway but yeah can see diff in mean across categories
from bivariate but still median and sd by cat would be useful!


\subsection{Limitations}

We do not use Gallup data. Some may argue it is a limitation becuase these data
cover more countries than WVS. However, apparently Gallup data cost tens of
thousands of dollars and we cannot afford it. In fact we'd discourage scientists
from paying from tax money to private corporations to do research. Therefore we
actually consider it our advantage not to use Gallup data.

Many world countries are missing, using more WVS data in the future as they
become available. 

right there are limitations, many countries dropped out as they dont habe many
people in smallest or biggest areas

Cross cultural comparbility is a caveat, we run separate for each country and
ont pool data but still, it should be kept in mind that happiness can mean
soemtig different in different countries. likewise world cities are very
different, bradth of the study is accomanied by oversimplifcation. 

"There is research in this area which claims that urban-rural differentials might be country-specific and not be generalisable at all (Rees, Tonon, Mikkelsen, \& Rodriguez de la Vega, 2017)."

Again we would like
to have more gradation at the top of the distruibution, but 500k is a reasonable
and adequate cutoff to disinguish a large city from other places. there are no
other data better suited for this purpose and we do best we can. And results are
conservative--had we have cutoff at 750k or 1m they'd be stronger MY BOOK AND
CITIES WHEN CITY IS TOOO BIG. 


\subsection{Cities can be actually useful for human wellbeing at early stage of
  developmebt}

The graphs below elaborate the Maslow's pyramid mentioned in the body of the paper.
At first one need to focus on necessities such as survival and cities do help;
again, it is remarkable that in all developed countries studied here, cities are less happy!

\begin{figure}[h!]
\begin{centering}
%\fbox{
 \includegraphics[height=1.2in]{/home/aok/papers/root/old/2011/eurostat_cities/tex/pyramid.pdf}
%} 
 \caption{Place Pyramid, \citep[p 294]{florida08}.} \label{pyramid}
  \end{centering}
\end{figure}

\begin{figure}[h!]
\begin{centering}
%\fbox{
 \includegraphics[height=2.5in]{/home/aok/papers/root/old/2011/eurostat_cities/tex/ingle.pdf}
%} 
 \caption{Well-being and income, \citep{inglehart97}.} \label{ingle}
  \end{centering}
\end{figure}


\subsection{Urbanicity Definition and results by different definition and
  sequentail elaboration}

we have 3 different operationalizatios of urbanicity: origianl 8 cat, collapse
one way and collapse the other way; and 3 sets of models: bivariate (iwth yr
dummies), esentially mean diff betwenn cat; basic set of controls;
necessary/important ones; full//extened (one in the body); and there is 4th one
over saturated but has most missing obs and hence postponed to the next section. 

where i dicuss controls in data and to literature where i slam
  burger and indeed as shown later comparing unadjusted means results in cities
   being happier notably due to confounding of higher income education and class


\begin{spacing}{.9} \begin{table}[H]\centering \caption{.} \label{d1} \begin{scriptsize} \begin{tabular}{p{1.8in}p{.5in}p{.5in}p{.5in}p{.5in}p{.5in}p{.5in}p{.5in}p{.5in}p{.5in}p{.5 in}p{.5in}p{.5 in}}\hline \input{../out/exT4-1.tex} \hline   * p$<$0.05, $+$ p$<$0.1; robust std err \end{tabular}\end{scriptsize}\caption{exT4-1}\end{table} \end{spacing}

\begin{spacing}{.9} \begin{table}[H]\centering \caption{.} \label{d1} \begin{scriptsize} \begin{tabular}{p{1.8in}p{.5in}p{.5in}p{.5in}p{.5in}p{.5in}p{.5in}p{.5in}p{.5in}p{.5in}p{.5 in}p{.5in}p{.5 in}}\hline \input{../out/exT3-1.tex} \hline   * p$<$0.05, $+$ p$<$0.1; robust std err \end{tabular}\end{scriptsize}\caption{exT3-1}\end{table} \end{spacing}

\begin{spacing}{.9} \begin{table}[H]\centering \caption{.} \label{d1} \begin{scriptsize} \begin{tabular}{p{1.8in}p{.5in}p{.5in}p{.5in}p{.5in}p{.5in}p{.5in}p{.5in}p{.5in}p{.5in}p{.5 in}p{.5in}p{.5 in}}\hline \input{../out/exT-1.tex} \hline   * p$<$0.05, $+$ p$<$0.1; robust std err \end{tabular}\end{scriptsize}\caption{exT-1}\end{table} \end{spacing}


\begin{spacing}{.9} \begin{table}[H]\centering \caption{.} \label{d1} \begin{scriptsize} \begin{tabular}{p{1.8in}p{.5in}p{.5in}p{.5in}p{.5in}p{.5in}p{.5in}p{.5in}p{.5in}p{.5in}p{.5 in}p{.5in}p{.5 in}}\hline \input{../out/exT4-2.tex} \hline   * p$<$0.05, $+$ p$<$0.1; robust std err \end{tabular}\end{scriptsize}\caption{exT4-2}\end{table} \end{spacing}

\begin{spacing}{.9} \begin{table}[H]\centering \caption{.} \label{d1} \begin{scriptsize} \begin{tabular}{p{1.8in}p{.5in}p{.5in}p{.5in}p{.5in}p{.5in}p{.5in}p{.5in}p{.5in}p{.5in}p{.5 in}p{.5in}p{.5 in}}\hline \input{../out/exT3-2.tex} \hline   * p$<$0.05, $+$ p$<$0.1; robust std err \end{tabular}\end{scriptsize}\caption{exT3-2}\end{table} \end{spacing}

\begin{spacing}{.9} \begin{table}[H]\centering \caption{.} \label{d1} \begin{scriptsize} \begin{tabular}{p{1.8in}p{.5in}p{.5in}p{.5in}p{.5in}p{.5in}p{.5in}p{.5in}p{.5in}p{.5in}p{.5 in}p{.5in}p{.5 in}}\hline \input{../out/exT-2.tex} \hline   * p$<$0.05, $+$ p$<$0.1; robust std err \end{tabular}\end{scriptsize}\caption{exT-2}\end{table} \end{spacing}


\begin{spacing}{.9} \begin{table}[H]\centering \caption{.} \label{d1} \begin{scriptsize} \begin{tabular}{p{1.8in}p{.5in}p{.5in}p{.5in}p{.5in}p{.5in}p{.5in}p{.5in}p{.5in}p{.5in}p{.5 in}p{.5in}p{.5 in}}\hline \input{../out/exT4-3.tex} \hline   * p$<$0.05, $+$ p$<$0.1; robust std err \end{tabular}\end{scriptsize}\caption{exT4-3}\end{table} \end{spacing}

\begin{spacing}{.9} \begin{table}[H]\centering \caption{.} \label{d1} \begin{scriptsize} \begin{tabular}{p{1.8in}p{.5in}p{.5in}p{.5in}p{.5in}p{.5in}p{.5in}p{.5in}p{.5in}p{.5in}p{.5 in}p{.5in}p{.5 in}}\hline \input{../out/exT3-3.tex} \hline   * p$<$0.05, $+$ p$<$0.1; robust std err \end{tabular}\end{scriptsize}\caption{exT3-3}\end{table} \end{spacing}

\begin{spacing}{.9} \begin{table}[H]\centering \caption{.} \label{d1} \begin{scriptsize} \begin{tabular}{p{1.8in}p{.5in}p{.5in}p{.5in}p{.5in}p{.5in}p{.5in}p{.5in}p{.5in}p{.5in}p{.5 in}p{.5in}p{.5 in}}\hline \input{../out/exT-3.tex} \hline   * p$<$0.05, $+$ p$<$0.1; robust std err \end{tabular}\end{scriptsize}\caption{exT-3}\end{table} \end{spacing}



\textbf{this one should be in appendix: 2 out of 10 again, but not rporting this
this is oversaturated and missing most countries}
\begin{spacing}{.9} \begin{table}[H]\centering \caption{.} \label{d1} \begin{scriptsize} \begin{tabular}{p{1.8in}p{.5in}p{.5in}p{.5in}p{.5in}p{.5in}p{.5in}p{.5in}p{.5in}p{.5in}p{.5 in}p{.5in}p{.5 in}}\hline \input{../out/exT3-4.tex} \hline   * p$<$0.05, $+$ p$<$0.1; robust std err \end{tabular}\end{scriptsize}\caption{exT3-4}\end{table} \end{spacing}



\textbf{so i think start with exT4-2 clean and easy and simple; then exT-3 to
  show detail and robustness}

\textbf{TODO meh yeah i guess drop this first table!!!}
note that all developed countres are less happy in cities, AUS (insiginifacnt
but sig in next table (\textbf{todo}check!)

\begin{spacing}{.9}
  \begin{table}[H]\centering \caption{.} \label{d1} \begin{scriptsize} \begin{tabular}{p{1.8in}p{.5in}p{.5in}p{.5in}p{.5in}p{.5in}p{.5in}p{.5in}p{.5in}p{.5in}p{.5
            in}p{.5in}p{.5 in}}\hline
        \input{../out/exT4-2.tex}
\hline  *** p$<$0.001, ** p$<$0.01, * p$<$0.05, $+$ p$<$0.1; robust std err
         \end{tabular}\end{scriptsize}\caption{exT4-2 OLS regressions of
         \texttt{swb} on \texttt{place size}, controls (not shown) are: enumerate}\end{table}
\end{spacing}

in atble \ref{exT4-2} several appear happier like BGD, IND, LTU, PAK, ROU, and
RUS, when adding more controls and full town cat that disappers except for 4 ctrioes


Results in table \ref{exT-3} are remarkable. In most countries large cities are less
happy than small settlements. Remarkably, without exception, in no developed
country city is happier than smallest areas. The only four countries where
people ar ehappier in large cities are: 

\begin{spacing}{.9}
  \begin{table}[H]\centering \caption{.} \label{d1} \begin{scriptsize} \begin{tabular}{p{1.8in}p{.5in}p{.5in}p{.5in}p{.5in}p{.5in}p{.5in}p{.5in}p{.5in}p{.5in}p{.5
            in}p{.5in}p{.5 in}}\hline
        \input{../out/exT-3.tex}
\hline  *** p$<$0.001, ** p$<$0.01, * p$<$0.05, $+$ p$<$0.1; robust std err
         \end{tabular}\end{scriptsize}\caption{exT-3; note robustness chech
         results are in SOM.}\end{table}
\end{spacing}
and tehre is even one more elaborate model\#4 in app with satfin and crime
like about \textbf{todo count lol} 20/70 neg of all and nly 4/18 of sig are pos
about 80 perc are neg :) or 5/21 in exT4-3

and the point is that the only ones that are sig and positive are wiers small
poor and most miserable countries except india which is a big puzzle

\textbf{TODO repreat it multiple times!} \textbf{TODO add NOR NLD}
so the conclusion is that in all develped countries AUS, CAN, DEU, ESP, ITA,
NZL, SWE, USA,  cities are less happy \footnote{at least in less elaborate
  specs, but even in most elaborate even if insig, still neg} in vast majority
of countries effect in ngeative, only positive in these 4:
russia, moldova and albania are all post soviet countries, they are likely to
still be very centralized where  power opportunity and resources are located in
large cities; india is clearly an outlier here and we dont have a good
explanation \texttt{[blind for peer review]} %\textbf{cite chourjua paper}



//----------------------------OLD




The limitation of \texttt{X049} is not only a low top bin for largest cities
(500k+), but also about a thrird of values missing. Future research can focus on
specific countries using other data or WVS data using \texttt{X049CS} variable,
which has country specific sizes of places, which however are not directly or
easily comparable--bins differ across countries and in some cases place is names
``major city'', ``Farm / Mountain / Fishing village,'' etc). 

show distributuion of place size by country!

\subsubsection{original 8 categories}
\subsubsection{0-5k v 500k+}
yeah this is following berry, but better have 0-5 so that more obs lol

\subsection{Crime and Cost of Living/financial satisfaction}
missing obs but here as a robustness check

TO WHERE I HAVE WE NEED TO CONTROL FOR CRIME:
Urban unhappiness is not only due
to urban problems such as crime and poverty.  Cities themselves, their core
defining characteristics, size and density, are related to unhappiness
\citep{aok_brfss_city_cize16}. 


yeh so one limitation is lack of crime; so bias on results cities wiuld be happier otherwhise

\begin{spacing}{.9} \begin{table}[H]\centering \caption{.} \label{d1} \begin{scriptsize} \begin{tabular}{p{1.8in}p{.5in}p{.5in}p{.5in}p{.5in}p{.5in}p{.5in}p{.5in}p{.5in}p{.5in}p{.5 in}p{.5in}p{.5 in}}\hline \input{../out/exT4-4.tex} \hline   * p$<$0.05, $+$ p$<$0.1; robust std err \end{tabular}\end{scriptsize}\caption{exT4-4}\end{table} \end{spacing}

\begin{spacing}{.9} \begin{table}[H]\centering \caption{.} \label{d1} \begin{scriptsize} \begin{tabular}{p{1.8in}p{.5in}p{.5in}p{.5in}p{.5in}p{.5in}p{.5in}p{.5in}p{.5in}p{.5in}p{.5 in}p{.5in}p{.5 in}}\hline \input{../out/exT3-4.tex} \hline   * p$<$0.05, $+$ p$<$0.1; robust std err \end{tabular}\end{scriptsize}\caption{exT3-4}\end{table} \end{spacing}

\begin{spacing}{.9} \begin{table}[H]\centering \caption{.} \label{d1} \begin{scriptsize} \begin{tabular}{p{1.8in}p{.5in}p{.5in}p{.5in}p{.5in}p{.5in}p{.5in}p{.5in}p{.5in}p{.5in}p{.5 in}p{.5in}p{.5 in}}\hline \input{../out/exT-4.tex} \hline   * p$<$0.05, $+$ p$<$0.1; robust std err \end{tabular}\end{scriptsize}\caption{exT-4}\end{table} \end{spacing}

\section{!!!PLAYING DROP THIS LATER}


\begin{spacing}{.9} \begin{table}[H]\centering \caption{.} \label{d1} \begin{scriptsize} \begin{tabular}{p{1.8in}p{.5in}p{.5in}p{.5in}p{.5in}p{.5in}p{.5in}p{.5in}p{.5in}p{.5in}p{.5 in}p{.5in}p{.5 in}}\hline \input{../out/ex1.tex} \hline   * p$<$0.05, $+$ p$<$0.1; robust std err \end{tabular}\end{scriptsize}\caption{ex1}\end{table} \end{spacing}

\begin{spacing}{.9} \begin{table}[H]\centering \caption{.} \label{d1} \begin{scriptsize} \begin{tabular}{p{1.8in}p{.5in}p{.5in}p{.5in}p{.5in}p{.5in}p{.5in}p{.5in}p{.5in}p{.5in}p{.5 in}p{.5in}p{.5 in}}\hline \input{../out/ex2.tex} \hline   * p$<$0.05, $+$ p$<$0.1; robust std err \end{tabular}\end{scriptsize}\caption{ex1}\end{table} \end{spacing}



\subsection{very first results}

Table \ref{a0} shows resuls of regression of \texttt{SWB} on place size dummies
(controlling for year dummies), which essentially differences in means for each
size category. Results are mixed, but large cities (500k-) and even medium sized
(100-500k) are often happier than the smallest category (the base case or
reference category, -2k). Usually diferences are small to moderate, about .5 (on
1-10 SWB scale), but sometimes large, larger than 1. Do mention the extremes and
think about why--but we do not have explanation for those. 

This is what the literature reports, tht results are mixed, in some cases cities
are happier, in some cases they are not. A key finding of this study is that
once we properly control for key predictors of SWB, almost uniformly large
cities (500k-) are less happy than the smallest settlements (-2k). Results are
shown in the body in table \ref{d1}.


%yah cannot use these 2--didnt drop properly these tiny categories so results
%dont make sense

% \begin{spacing}{.9}
%   \begin{table}[H]\centering \caption{.} \label{d1} \begin{scriptsize} \begin{tabular}{p{1.8in}p{.5in}p{.5in}p{.5in}p{.5in}p{.5in}p{.5in}p{.5in}p{.5in}p{.5in}p{.5
%             in}p{.5in}p{.5 in}}\hline
%         \input{/tmp/a0.tex}
% \hline  *** p$<$0.001, ** p$<$0.01, * p$<$0.05, $+$ p$<$0.1; robust std err
%          \end{tabular}\end{scriptsize}\caption{a0 OLS regressions of
%          \texttt{swb} on \texttt{place size}, controls (not shown) are: enumerate}\end{table}
% \end{spacing}

Results in table \ref{a} are remarkable. In most countries large cities are less
happy than small settlements. Remarkably, in no developed country city is
happier than smallest areas (with exception of KWT and SAU--these are
middle eastern and oil rich, where cities are glorious indeed)--and they are not
developed countries according to IMF or UN anyway neither have very high HDI.

% \begin{spacing}{.9}
%   \begin{table}[H]\centering \caption{.} \label{d1} \begin{scriptsize} \begin{tabular}{p{1.8in}p{.5in}p{.5in}p{.5in}p{.5in}p{.5in}p{.5in}p{.5in}p{.5in}p{.5in}p{.5
%             in}p{.5in}p{.5 in}}\hline
%         \input{/tmp/a.tex}
% \hline  *** p$<$0.001, ** p$<$0.01, * p$<$0.05, $+$ p$<$0.1; robust std err
%          \end{tabular}\end{scriptsize}\caption{a }\end{table}
% \end{spacing}


\end{spacing}
\end{document}
