%to have line numbers
%\RequirePackage{lineno}
\documentclass[10pt, letterpaper]{article}      
\usepackage[margin=.1cm,font=small,labelfont=bf]{caption}[2007/03/09]
%\usepackage{endnotes}
%\let\footnote=\endnote
\usepackage{setspace}
\usepackage{longtable}                        
\usepackage{anysize}                          
\usepackage{natbib}                           
%\bibpunct{(}{)}{,}{a}{,}{,}                   
\bibpunct{(}{)}{,}{a}{}{,}                   
\usepackage{amsmath}
\usepackage[% draft,
pdftex]{graphicx} %draft is a way to exclude figures                
\usepackage{epstopdf}
\usepackage{hyperref}                             % For creating hyperlinks in cross references


% \usepackage[margins]{trackchanges}

% \note[editor]{The note}
% \annote[editor]{Text to annotate}{The note}
%    \add[editor]{Text to add}
% \remove[editor]{Text to remove}
% \change[editor]{Text to remove}{Text to add}

%TODO make it more standard before submission: \marginsize{2cm}{2cm}{1cm}{1cm}
\marginsize{2.5cm}{2.5cm}{2.5cm}{2.5cm}%{left}{right}{top}{bottom}   
					          % Helps LaTeX put figures where YOU want
 \renewcommand{\topfraction}{1}	                  % 90% of page top can be a float
 \renewcommand{\bottomfraction}{1}	          % 90% of page bottom can be a float
 \renewcommand{\textfraction}{0.0}	          % only 10% of page must to be text

 \usepackage{float}                               %latex will not complain to include float after float

\usepackage[table]{xcolor}                        %for table shading
\definecolor{gray90}{gray}{0.90}
\definecolor{orange}{RGB}{255,128,0}

\renewcommand\arraystretch{.9}                    %for spacing of arrays like tabular

%-------------------- my commands -----------------------------------------
\newenvironment{ig}[1]{
\begin{center}
 %\includegraphics[height=5.0in]{#1} 
 \includegraphics[height=3.3in]{#1} 
\end{center}}

 \newcommand{\cc}[1]{
\hspace{-.13in}$\bullet$\marginpar{\begin{spacing}{.6}\begin{footnotesize}{#1}\end{footnotesize}\end{spacing}}
\hspace{-.13in} }

%-------------------- END my commands -----------------------------------------



%-------------------- extra options -----------------------------------------

%%%%%%%%%%%%%
% footnotes %
%%%%%%%%%%%%%

%\long\def\symbolfootnote[#1]#2{\begingroup% %these can be used to make footnote  nonnumeric asterick, dagger etc
%\def\thefootnote{\fnsymbol{footnote}}\footnote[#1]{#2}\endgroup}	%see: http://help-csli.stanford.edu/tex/latex-footnotes.shtml

%%%%%%%%%%%
% spacing %
%%%%%%%%%%%

% \abovecaptionskip: space above caption
% \belowcaptionskip: space below caption
%\oddsidemargin 0cm
%\evensidemargin 0cm

%%%%%%%%%
% style %
%%%%%%%%%

%\pagestyle{myheadings}         % Option to put page headers
                               % Needed \documentclass[a4paper,twoside]{article}
%\markboth{{\small\it Politics and Life Satisfaction }}
%{{\small\it Adam Okulicz-Kozaryn} }

%\headsep 1.5cm
% \pagestyle{empty}			% no page numbers
% \parindent  15.mm			% indent paragraph by this much
% \parskip     2.mm			% space between paragraphs
% \mathindent 20.mm			% indent math equations by this much

%%%%%%%%%%%%%%%%%%
% extra packages %
%%%%%%%%%%%%%%%%%%

\usepackage{datetime}


\usepackage[latin1]{inputenc}
\usepackage{tikz}
\usetikzlibrary{shapes,arrows,backgrounds}


%\usepackage{color}					% For creating coloured text and background
%\usepackage{float}
\usepackage{subfig}                                     % for combined figures

\renewcommand{\ss}[1]{{\colorbox{blue}{\bf \color{white}{#1}}}}
\newcommand{\ee}[1]{\endnote{\vspace{-.10in}\begin{spacing}{1.0}{\normalsize #1}\end{spacing}\vspace{.20in}}}
\newcommand{\emd}[1]{\ExecuteMetaData[/tmp/tex]{#1}} % grab numbers  from stata

%TODO before submitting comment this out to get 'regular fornt'
% \usepackage{sectsty}
% \allsectionsfont{\normalfont\sffamily}
% \usepackage{sectsty}
% \allsectionsfont{\normalfont\sffamily}
% \renewcommand\familydefault{\sfdefault}

\usepackage[margins]{trackchanges}
\usepackage{rotating}
\usepackage{catchfilebetweentags}
%-------------------- END extra options -----------------------------------------
\date{Draft: {}\today}
\title{
Growing Up in a City Will Make You Unhappy For The Rest of Your Life
%Growing up in a city will make your life unhappy\\
%   (New Evidence of Urban Malaise)
%growing up in a city will make you miserable for the rest of your life
}
\author{
% Adam Okulicz-Kozaryn\thanks{EMAIL: adam.okulicz.kozaryn@gmail.com
%   \hfill I thank XXX.  All mistakes are mine.} \\
% {\small Rutgers - Camden}
}

\begin{document}

%%\setpagewiselinenumbers
%\modulolinenumbers[1]
%\linenumbers

\bibliographystyle{ecta}
\maketitle
\vspace{-.4in}
\begin{center}

\end{center}


\begin{abstract}
\noindent This study adds new evidence % in favor
of urban malaise (unhappiness)
hypothesis. % (\citeyear{wirth38})
 A new key finding is added: people who
grew up in cities are less happy later in their lives above and beyond
  unhappiness associated with currently living in a city. 
%
 Strikingly, the negative effect of urbanicity in one's youth is about as
 strong statistically and practically (effect size) as effect of urbanicity of
 current place. 
%
% YEAH THATS FOR CONTINUOUS MEASURES ONLY
% Strinkingly, the negative effect of urbanicity in one's youth is stronger statistically
% than current urbanicity. Effect sizes, in most cases, are stronger, too 
%
 We add one more new finding: there may be a happiness benefit to growing up at a farm.
%
The present study is inspired by \citet{lederbogen11} who
showed that growing up in a city has a negative lasting effect later in person's
life. This study concerns the US, and results  may not generalize to other,
especially developing, countries. 
\end{abstract}
\vspace{.15in} 
\noindent{\sc keywords: subjective wellbeing, happiness, life satisfaction,
  city, urbanism, migration, general social survey (GSS)}
%\vspace{-.25in} 

\begin{spacing}{2.0} %TODO MAYBE before submission can make it like 2.0
\rowcolors{1}{white}{gray90}

\section*{What we know so far: urban malaise} %boilerplate from brfss city cize

There is a longstanding thesis of urban misery or malaise. % This debate, as
% most other debates, can be traced to ancient philosophers
Early sociologists theorized and observed urban malaise
\citep{simmel03,tonnies57,park15,wirth38}. The thesis of urban malaise is well
articulated in classic ``Urbanism as a Way of Life'' \citep{wirth38}. Sociology
interest in urbanism and wellbeing, however, ended early, too. The topic was
abandoned in 1970s with a series of works by Fischer (\citeyear{fischer72,fischer73,fischer75, fischer82}).  
While sociology has abandoned this line of research, other disciplines continued
taking various perspectives on the topic and mostly focusing on size of a
place. By now a consensus has emerged confirming early sociological research:
people are unhappy in cities \citep{balducci09, aokcities,
  aok11a,aokCityBook15,aok-ls_fisher16,aok_brfss_city_cize16,morrison15,morrison17}. 
 There is also a consensus that the opposite of urbanicity, naturalness, is related
 to happiness \citep{pretty12, frumkin01b, wheeler12, white13b, white13, tesson13, maller06, berman08,berman12}.  

%MAYBE about nature ~/papers/root/rr/suburb-nature/tex/suburb-nature.tex

Urban malaise is universal in the developed world.\footnote{Some social groups,
  however, are not least happy in the largest cities--American Millenials is one
example \citep{aok-swbGenYcity18}.} 
The largest American city, New York City, is the least happy or one of the least
happy places in America \citep{aok_brfss_city_cize16,
  senior_ny_sep16_14}. London is the largest and least happy
place in the UK \citep{ons11,ibt13}. Toronto, the largest metropolitan area in
Canada,  is second least happy in Canada, only Vancouver (third metropolitan area)
is less happy \citep{lu15}. Helsinki is the largest and  least happy place in
Finland \citep{morrison15}. Bucharest is the largest and least happy place in
Romania \citep{lenzi16D}.  Australia's largest city, Sydney, is least happy
\citep[cited in][]{morrison11}, and so forth.
%
Arguably it is not self-selection--it is not that unhappy people move to
cities. If anything, people with higher ability \citep{jokela14}  move to cities
(for education and jobs) and then out
of cities to raise family, and it may be happier people that are willing to
migrate in general \citep{bartram13}.
%
Urban unhappiness is not only due
to urban problems such as crime and poverty.  Cities themselves, their core
defining characteristics, size and density, are related to unhappiness
\citep{aok_brfss_city_cize16}. 

Perhaps, the best example of urban malaise is Singapore, by many standards, one of the
best, if not the best place in the World. It has World's third highest (after
Qatar and Luxembourg) Gross Domestic Product per Capita Purchasing Power Parity
adjusted \citep{authorNYT17monD}. It has also third highest (after Monaco and
Japan) life expectancy in the World \citep{cia},
second highest economic freedom \citep{heritage}. Its kids score best on educational
tests \citep{coughlanBBC17dec6}, it is making greatest progress in health
\citep{fullman2017measuring}, has the World's fastest internet \citep{mcspaddenNYT17monD}.
 It even has world's strongest passport
\citep{chandranCNBC17oct24}. In short, one could say that Singapore % has one of
% the top, if not the top quality of life in the World.
 is one of the best, if not the best place in the World.  

Of course, a distinctive feature of Singapore is that it is a nation-city: it is
 almost only  urban, a third % \footnote
{(after Monaco and Macao)} most dense country in the World
 (\url{https://esa.un.org/unpd/wpp}). And according to \citet{glaeser11}, it is triumphant:
``Triumph of the City: How Our Greatest Invention Makes Us Richer, Smarter,
Greener, Healthier, and Happier.'' 

Surely, if Singapore is not the very happiest country in the World, it must be in
top 1 percent or at very least top 5 percent. It doesn't make it to the top
quartile \citep{veenhoven95b}.\footnote{\citet{veenhoven95b} is a generic reference to the World Data Base of Happiness. Actual data are at \url{http://www.worlddatabaseofhappiness.eur.nl/hap_nat/nat_fp.php?mode=8}.}

\section*{Urban migrants}

There are  studies tracing person's Subjective Wellbeing (SWB) as she moves.
 And there is a substantial economic literature on rural-urban migration % --for instance as reviewed
% in
 \citep[e.g.,][]{lall06al}--but it doesn't concern itself with SWB.  
 Most SWB-across-urbanicity-migration studies studies look at rural-urban migration, and only one study at urban-rural
migration \citep{alcock14}. %guess these are panel but not specifically about
                           %moving! white13,white13b
% We diregard studies examing urbanicity or natureness of place degree of chaing
% location such as recent \citet{james16}.
%
There are also other SWB-across-urbanicity-migration studies  but
  they are not systematic% or large scale
. For instance, \citet{tesson13}
  describes his own story--he moved from urban world to wilderness and became
  happy. \citet{pretty12} discusses similar  cases, e.g., people who first
  suffered from an illness and then recovered faster in greener area. 


There are % even
 many  movies telling a story of a usually young, energetic and  ambitious person, who was
born and raised ``in the middle of nowhere,'' in a village or small town, or
perhaps in a larger town but in a ``rural state'' such as Wyoming or Nebraska,
and then  she wants to be better than that and do something with her
life, achieve the American Dream, and what she does? She moves to a
big city, typically New York, Los Angeles, Chicago or one of the handful of
others, and then she faces some hardship, usually makes it, but then 
realizes she's unhappy there and moves back to the middle of nowhere where she
came from and leads a happy life there.

City dwellers may appear happy and
small area dwellers may appear unhappy--yet city happiness
is fake--it is just a smile \citep{aok-sizeFetish17}, pride \citep{balducci09},
or a similar appearance, slapped to a face, but urbanite typically is miserable
\citep{aokCityBook15}. The problem is that a poor rural person may think that
urban flourishing 
is real% , the city is the greatest place
, and then she migrates to a city, and then 
often ends up unhappy, too. Worse, she may falsely think she is better off in a
city, while she is not--it may be false consciousness. % (\url{https://www.marxists.org/glossary/terms/f/a.htm}.)
 Indeed, there is evidence supporting such bias \citep{aok-sizeFetish17,balducci09}.
Rural-urban migrants think
they are happier after the move, and yet their mean happiness scores are
lower than those of rural dwellers (at least in China)
\citep{knight10}. Other Chinese evidence suggests that rural-urban migration may
work for one's SWB as long as the city is not large \citep{chen15}. 
%
US urban immigrants are also often unhappy--indeed, they were labelled ``the
marginal man'' \citep{park28}.
%
Immigrants move largely into cities, and overall they tend not to do well
\citep{malangaCJ06summer, hymowitzCJ17autumn}.% , and are accordingly, arguably,
% they end up unhappy.
% yeah im unhappy too!
%
One problem with largest cities, especially recently, is that most people cannot afraid to live there comfortably \citep{floridaCL16apr28}.
% %THESE ARE JUST MIGRATION NOT URBANICITY, AND ACROSS COUNTRIES, but guess
% %argunment is tht migrants typicallty  go to cities
%
%  Central and Eastern Europeans migrants report lower life satisfaction \citep{baltatescu07}
% (and again, migrants usually migrate to urban areas).
% \citet{bartram13} presents mixed evidence: migrants from Russia, Turkey, and
% Romania are happier than stayers, while  Polish migrants are less happy than stayers.

%  On the other hand, there are some success stories when
% people move to less dense areas \citep[e.g.,][]{massey13,alcock14}.

Migrants are usually arguably irrational--because people who have migrated,  tend to be unhappy \citep{knight10, bartram14, baltatescu07, hendriks14,jong02}--perhaps transitoriness,\footnote{Per turnover/stability: in poor areas turnover is good; in rich it is bad for happiness \citep{ross00}. And there are many studies focusing on the role of relative deprivation for migration \citep[e.g.,][]{stark91t}, but they are beyond the scope.}
detachment, too much Gesellschaft \citep{tonnies57} kill happiness; migrant is a just a marginal man \citep{park28}.
  
%
In general, locals tend to be happier than migrants--locals spend time on more happiness generating activities than internal migrants
\citep{hendriks14}. Migrants often increase heterogeneity and heterogeneity has
many negative consequences in general \citep{alesina99al,alesina00, putnam07},
 including lowered SWB \citep{aokrel,aokRel2,herbst14,vogt07,postmes02}.

There is also a handful of studies using the same data and variable that we use and studying 
how upbringing might have affected tolerance \citep{stephan82, tuch87}. Following these studies,  size of place
where one grew up may affect one's happiness later in life just as it affects tolerance. The
logic is as follows: size of a place is not only a situational but also a
socialization variable. That is, people are not only unhappy in cities because
they live there but also because they learned certain ways of life by growing up
there. 
% 
Socialization (social learning) is extensive in humans, because humans have
long juvenile period during which childhood play and socialization prepares them
for adult roles in society \citep{eagly10}. ``You can take the boy out of the country, but you can't
take the country out of the boy.'' %\citep[cited in ][p. 414]{stephan82}.



\section*{What we do not know yet}

There are important gaps in our knowledge--again most studies take static
contemporary view--how urbanicity now affects SWB now. Line of research about
migration across places of differing urbanicity mostly does not take into
account SWB, and studies that do, almost only consider rural-urban
migration and only one study considers urban-rural migration
\citep{alcock14}. 
%
Hence, there are either studies about current urbanicity or about migration
across degrees of urbanicity.
%
Our study is not about migration. % ; neither on
% urbanicity as is growing literature on it;
It is about urbanicity in
childhood, where a person grew up, and how it affects her SWB later in life. %; and findings as striking--it actually as strong as effect the effect of current urbanicity.
No study so far has considered jointly urbanicity of a place where a
person has grown up and current urbanicity: does urbanicity of a place where a
person has
grown up affect her SWB above and beyond urbanicity of a place where she
currently resides?
%
 This study will try to answer this question. 
%
 This study is inspired by \citet{lederbogen11} who
showed that growing up in a city has a lasting negative effect later in person's life.

\section*{Data and model}

We use the US General Social Survey (GSS) cumulative dataset containing about
60,000 observations from 1972 to 2016 from\\ 
\url{gssdataexplorer.norc.org}. GSS is collected face-to-face and is nationally
representative. Since 1994 GSS is collected every other year (earlier mostly annually). 
%
The advantage of GSS is that it contains a question about person's residence
where she grew up,  variable \textsc{res16}: "Which of the categories on this
card comes closest to the type of place you were living in when you were 16
years old?": 

%papers using it: http://scholar.google.com/scholar?q=%22Which+of+the+categories+on+this+card+comes+closest+to+the+type+of+place+you+were+living+in+when+you+were+16+years+old%3F%22&btnG=&hl=en&as_sdt=0%2C44

\begin{spacing}{.9}
  \begin{enumerate}
  \item nonfarm (or country)
  \item farm
  \item town $<$ 50,000
  \item 50,000 - 250,000
  \item big-city suburb
  \item city $>$ 250,000
  \end{enumerate}
\end{spacing}
One obvious caveat is that a person could have moved during her childhood.
We make an assumption that a place where a person lived when she was 16, is a
place where she grew up, which sometimes is not the case. Still, the
variable arguably has adequate precision: it captures urbanicity for at least a
significant part, and usually majority of childhood--it is unlikely that
a person lived in a place of very different size for most of her childhood than
a place where she lived when she was 16. 
%
Other studies using this variable also make this  assumption \citep{stephan82, tuch87}.

Urbanicity is measured using a set of dummies for
\textsc{xnorcsiz} variable which provides a fine classification according to
 density and size. {Additional results using alternative urbanicity
   measures are in appendix.}

 % and crosstabulations of all size variables are
 % in tables \ref{cc1}, \ref{cc2}, and \ref{cc3} in appendix
 
 All variables
are defined in table \ref{var_des} below.  Table \ref{var_des}  also lists  typical
controls used in the SWB  literature  \citep{aok-ls_fisher16,aok11a}.
Distributions of all variables are shown in appendix in figure \ref{hist}. 


\begin{spacing}{.9}
\input{var_des}
{\footnotesize }
\end{spacing}

Also, there are regional or cultural
differences in just about anything, hence, we include dummies for census
 regions: New England, Middle Atlantic, E. Nor. Central, W. Nor. Central, South
Atlantic, E. Sou. Central, W. Sou. Central, Mountain, and Pacific. Since we use
pooled GSS data, we include year dummies.  


% For simplicity and following \citep{stephan82}, categories (1) and (2) were
% combined together. Likewise, categories (5) and (6) were combined together--it
% is \underline{big-city} suburb and hence it belongs with big city. 

We use ordinary least squares (OLS) to analyze our data. Although OLS assumes cardinality of the
outcome variable, and happiness is clearly an ordinal variable, OLS is an appropriate estimation method to use in this case.
\citet{carbonell04} showed that results are substantially the same to those from discrete models, and  OLS has become the default method in happiness research \citep{blanchflower11}. Theoretically, while there is still debate about the cardinality of SWB, there are strong arguments to treat it as a cardinal variable \citep{ng96,ng97,ng11}. Nonetheless, as a robustness check we also ran multinomial logit regressions, and included the results, which are substantially the same, in the appendix.
% Following practice in leading
% interdisciplinary journals, such as Science or PNAS, only the key results are presented in the body of the paper, and details are postponed to the appendix/supplementary materials.


\section*{Results}


% we proceed as follows:  res16 first and then size now and then elaborate and
% just like xnorcsize cause similar to res16--suburb etc--focus on them; the otehr 2 in appendix :)

All  results are in table \ref{regAB}. 
%
First column is a simple regression of SWB on \textsc{place when 16 yo}. Base
case is ``country, nonfarm'' Only two extreme categories are significant: ``farm'' is
positive and ``250k-'' is negative. Addition of income in column a2 
makes all the categories negative except ``farm''--growing up in any larger place
than ``nonfarm'' or ``farm'' is associated with lower SWB. Also, all
coefficients are now larger. Addition of other sociodemographic controls in a3
diminishes effect sizes only slightly. So does controlling for region and year
dummies in a4. Last addition is \textsc{xnorcsiz} in a5: now, again, as in the
beginning, only extreme categories  ``farm'' and  ``250k-'' remain
significant. Effect of current place, \textsc{xnorcsiz}, is as expected \citep{aok-ls_fisher16}, the larger places are significantly less happy.

What is worth highlighting, and what is arguably unexpected, is that the statistical significance  and effect size
 of \textsc{place when 16 yo} is about as large as that of the current urbanicity \textsc{xnorcsiz}.

It is instructive to focus on an interplay between
\textsc{place when 16 yo} and \textsc{xnorcsiz}--we will approach model elaboration differently--we start with
\textsc{xnorcsiz} and then see how it changes when adding \textsc{place when 16
  yo} in columns  b1 and b2. Comparison of a1
and b1 reveals that the largest places now, ``gt 250k,'' have about twice as
strong effect on SWB as largest places ``250k-'' when one was 16. 
%
Controlling for \textsc{place when 16 yo} in b2 somewhat attenuates estimates on \textsc{xnorcsiz} as compared to b1. 
%
And again, in full specification a5, the effect
sizes become about the same.
 Robustness checks and supplementary results are in appendix.


\begin{spacing}{1}
  \begin{table}[H]\centering
    \caption{OLS regressions  of SWB. Beta (fully standardized) coefficients.} \label{regAB}
    \begin{scriptsize} \begin{tabular}{p{2.1in}p{.49in}p{.49in}p{.49in}p{.49in}p{.49in}p{.49in}p{.49in}p{.49in}p{.49in}p{.49 in}}\hline
        \input{regAB.tex}
        \hline *** p$<$0.001 ** p$<$0.01, * p$<$0.05, + p$<$0.1; robust std err
      \end{tabular}\end{scriptsize}\end{table}
\end{spacing}





\section*{Conclusion and discussion}


This study is not about migration; % , like large literature on it; neither on
% urbanicity as is growing literature on it;
it is about urbanicity in
childhood, where a person grew up, and how it affects her SWB later in
life. Findings are striking: the effect of urbanicity of a place where a person grew up is as strong as the effect of current urbanicity.

The study has found that size of a place where one grew up affects her SWB in
addition to urbanicity of the current place. But also there is a flipside: this
study also found that current size of a place affects our happiness net of where we
grew up and what we may understand as normal based on childhood experience. 
%
This is an important point: people in largest cities are unhappy regardless of
where they gre up. 

Urbanization is arguably the most significant change of human
habitat in our species history. It is happening now at an unprecedented scale:
each year cities balloon by tens of millions of people
(\url{https://esa.un.org/unpd/wup}). It is probably easier to imagine and
understand by giving some thought to cement use in China, country that currently urbanizes most
people: between 1900 and 1999 the US consumed 4,500 million tones of cement;
between 2011 and 2014 China consumed  6,500 million tones of cement \citep{harvey16}.

Perhaps even more worrying is academic support for urbanization. One Harvard
professor has recently written a book titled ``Triumph of the City: How Our
Greatest Invention Makes Us Richer, Smarter, Greener, Healthier, and Happier''
 \citep{glaeser11}.
It turns out that not only living in a city is associated with unhappiness, but
even worse: growing up in a city will make you unhappy for the rest of your life.

% Given rampant urbanization, popular and fashionable pro-urbanism, it is
% important to highlight the many problems of urbanization. Several points are
% worth starting with, highlighting and reiterating.  

Changes in one's neural processing are one pathway between urban upbringing and
adult wellbeing \citep{lederbogen11}. Another pathway may simply have to do
with negative consequences of urbanism: isolation, anomie, deviance, vice,
crime, conspicuous consumption, pollution, noise, crowding, and
poverty\footnote{For a classic statement of urban problems see
  \citet{wirth38,park15,park84,tonnies57,simmel03}, and for modern
  statements see \citet{white77} and \citet{aokCityBook15}.} take the
toll on children who grow up there. Still, it is probably largely unexpected
that  the urban upbringing disadvantage is felt later in adult life--it should
be a sobering finding especially amidst current pro-urban fashion.

Size of a place is clearly an ecological and situational influence on SWB as shown by
previous research. This study's results are consistent with hypothesis that size
of a place has some  socialization influence on SWB, if such socialization results in lowered SWB.
Results are against socialization % /adjuetment
hypothesis that urban upbringing   
 %prepare
makes  one  happier in today's urban world, say by preparing one better to live
in cities. There is no such adjustment. Neither
farm upbringing decreases one's happiness in today's urban world. If anything,
it actually makes one happier later in life.
% As pointed out by \citet{stephan82} the advantage of using early place of
% residence is the ability to test whether size of a place variable is is
% situational (ecological) or socialization (socio-psychological) variable. That is, are urbanites unhappy because
% they live in cities or because they grew up in cities. The mechanism is clear in
% the fist case as elaborated in first section. In the latter case, perhaps
% urbanites develop unhappy city habits.% such as? 
%  Humans are, of course, but what sociologists forget, biological organisms,
%  hence they should be happier in natural setting than in a city. But also, as
%  sociologists are obsessed with, humans are quite social \citep{fiske09}, and more than that,
%  realioty is socially constructed \citep{berger66l}. So then, what is
%  ``natural'' is a social construct: specifically if a person grew up in a city,
%  this may seem to her a ``natural'' environment where she would feel happy.    
The positive effect of growing up on a farm is
not robust to inclusion of additional controls in appendix, however. Hence,
caution in interpretation is needed.

Another area to be cautious is over-time changes--the rural advantage may be
slipping away and rural misery coming in as the smaller places are being
forgotten and left behind
\citep{aok-misanthropy-trustCity,aok-swbGenYcity18,hansonCityJournalautumn15}. It
is hard to imagine flourishing happy rural life amids increasing urban assault
not only grabbing more land but also forcing urban way of life and killing rural
jobs. It is perhaps easiest to understand the sentiment in words of a rural person--one rural
Californian complains: 

\begin{quote}
  We run this state like it's one size fits all. You can't do that [...]\\
In the rural parts of the state we drive more miles, we drive older cars, our
economy is an agriculture- and resource-based economy that relies on tractors.
 You cant move an 80,000-pound load in an electric truck.\\
They've devastated ag jobs, timber jobs, mining jobs with their environmental
regulations, so, yes, we have a harder time sustaining the economy, and
therefore there's more people that are in a poorer situation. \citep{fullerNYT17monD}
\end{quote}

There is a curious result of greater happiness later in life for people who grew
up at a farm. It is not necessarily unexpected--farm kids after learn real life
skills and are ``tougher'' than suburban kids.\footnote{For instance, see a series of
simple and short but arguably to the point articles:
 \citet{FFP14aug13,FFP14dec16,FFP14oct20}.}  At the same time, farm life does not corrupt and make one deviant as cities do
 \citep{wirth38,park84}. Yet, these days farming not only does not make much money but
 typically looses money and New York Times is accordingly warning ``Don't Let
 Your Children Grow Up to Be Farmers'' (\citeyear{smithNYT17aug9})--not only urbanization is
 rampant, but also farming is struggling--unfortunately, capitalism and
 market economy often, if not usually, promote ways of life that do not lead to
 happiness \citep{kasser13,marcuse15,lane00,scitovsky76,klein14}.\footnote{Also see \citet{kasser93,schmuck00,okulicz17B,harvey14,harvey16,stefanSS10may,vohs06,schor08,engels87,lamothe16}. Also note that in
 American past, when there was less commodification, commerce, market economy
 and capitalism, farming was a more well-regraded pursuit \citep{deCrevecoeur81,fischer91}.}

It is worth reciting that while urbanites are less happy than rural folks, they
think that they are more happy (at least in China) \citep{knight10}.
Part of the explanation may be money illusion--tendency to think in nominal
rather than relative terms \citep{shafir97al}--people are lured to cities with
higher earnings, but they do not realize the cost of living. In a similar way,
many people want to live in California thinking about life style, climate and
other amenities, but not appreciating living expenses, and accordingly
California is one of the lest happy states in the US \citep{oswald09w,schkade98k}.

Urbanicity and SWB research is important--it helps a person to make a decision--what would happen if I
 lived in urban or rural area. You can call it evidence based pursuit of happiness. % (but
% also see a recent critique of happiness industry \citep{davies15}). And rather
% than pursuing happiness we should do other things that will make us happy as a
% byproduct or externality \citep{gilbert09}. Yet,
 If you care about your own
happiness, and happiness of your children, you should avoid cities. Of course,
city avoidance is not a new strategy--despite rampant urbanism, Americans
 have preferred less dense areas \citep{fuguitt90,fuguitt75}
 and they still do \citep{YouGov12-place}.
  They tend to settle outside of cities, but close to them, and hence,
 suburbanization, which in effect, at the end, just enlarges existing
 cities. Perhaps, the best solution to fight ever-present and ever-increasing
 urbanization is to consume less and degrow economy--for discussion see
 \citep{aokCityBook15,kallis12}.




%  It may be so that ``difficult and deviant'' people choose to live in
% the cities--and they may be less happy than others and so that may
% depress overall city happiness--that may be to some degree picked up
% person-level controls: smoking, income,
% education and being unemployed, and county-level: crime rate index.
%  But otherwise there is room for improvement and  future research could help
% in this area. In general, however, deviance
% and being difficult are difficult to measure. There is also a
% possibility that unhappy people are lured to cities--for instance
% people who feel they don't belong in smaller areas and people who are
% unhappy where they live may migrate to cities. But it is
% equally possible, and if anything more plausible, that it is actually
% happy and energetic extroverts that move to cities to realize their
% full potential.

% %for this i guess would need a panel which i do not have
% The alternative explanation/Z variable. It may be so 
% urbanism is na way of life and it is reasonable that it changes a person quite
% permanantly--eg pecuniary orientatoon, capiralistic outlook etc etc!!


%  instead \ExecuteMetaData[../out/tex]{ginipov} do \emd{ginipov}

% \begin{figure}[H]
%  \includegraphics[height=3in]{../out/gov_res_trust.pdf}\centering\label{gov_res_trust}
% \caption{woo}
% \end{figure}


%TODO !!!! have input here aok_var_des



% %table centered on decimal points:)
% \begin{table}[H]\centering\footnotesize
% \caption{\label{freq_im_god} importance of God}
% \begin{tabular} {@{} lrrrr @{}}   \hline 
% Item& Number & Per cent   \\ \hline
% 1(not at all)&    9,285&  9\\
% 2&    3,555&        3\\
% 3&    3,937&        4\\
% 4&    2,888&        3\\
% 5&    7,519&        7\\
% 6&    5,175&        5\\
% 7&    6,050&        6\\
% 8&    8,067&        8\\
% 9&    8,463&        8\\
% 10&   52,385&       49\\
% Total&  107,324&      100\\ \hline
% \end{tabular}\end{table}


% % Define block styles
% \tikzstyle{block} = [rectangle, draw, fill=black!20, 
%     text width=10em, text centered, rounded corners, minimum height=4em]
% \tikzstyle{b} = [rectangle, draw,  
%     text width=6em, text centered, rounded corners, minimum height=4em]
% \tikzstyle{line} = [draw, -latex']
% \tikzstyle{cloud} = [draw, ellipse,fill=black!20, node distance = 5cm,
%     minimum height=2em]
    
% \begin{tikzpicture}[node distance = 2cm, auto]
%     % Place nodes
%     \node [block] (lib) {liberalism, egalitarianism, welfare};
%     \node [block, below of=lib] (con) {conservatism, competition, individualism};
%     \node [cloud, right of=con] (ls) {well-being};
%     \node [block, below of=ls] (cul) {genes, culture};
%     \node [b, left of =lib, node distance = 4cm] (c) {country-level};
%     \node [b, left of =con,  node distance = 4cm] (c) {person-level};
%     % Draw edges
%     \path [line] (lib) -- (ls);
%     \path [line] (con) -- (ls);
%     \path [line,dashed] (cul) -- (ls);
% \end{tikzpicture}




%PUT THIS NOTE, polish and put to /root/author_what_data --ALWAYS
%stick here stuff as i run it!!! maybe comment out later...
\section*{\Huge ONLINE APPENDIX}
\textbf{[note: this section will NOT be a part of the final version of
  the manuscript, but will be available online instead]} %hence everything below
                                %is organized byu section, not subsection
\section*{Variables' definitions, coding and distributions}
\label{app_var_des}

{\scriptsize
\begin{verbatim}
Variable xnorcsiz : EXPANDED N.O.R.C. SIZE CODE
Literal Question
NORC SIZE OF PLACE
PostQuestion Text
a A suburb is defined as any incorporated area or unincorporated area
of 1,000+ (or listed as such in the U.S. Census PC (1)-A books) within
the boundaries of an SMSA but not within the limits of a central city
of the SMSA. Some SMSAs have more than one central city, e.g.,
Minneapolis-St. Paul. In these cases, both cities are coded as central
cities.
b If such an instance were to arise, a city of 50,000 or over which is
not part of an SMSA would be coded '7'.
c Unincorporated areas of over 2,499 are treated as incorporated areas
of the same size. Unincorporated areas under 1,000 are not listed by
the Census and are treated here as part of the next larger civil
division, usually the township.
The source of the data is the 1970 U.S. Census population figures
published in the PC (1) -A series, Tables 6 and 10. Practically, the
codes '6' and '10' are localities not listed in Table 6 (Population of
Incorporated Places and Unincorporated Places over 1,000). For the
1980 frame cases analogous tables from the 1980 Census were used.

Descriptive Text
See Appendix T, GSS Methodological Report No. 4.
\end{verbatim}
}

The following figures show variable distributions. If a variable has more than
10 categories it is classified into bins.

\input{hist.tex}


\section*{Endogeneity, Causality, Self-Selection, Robustness}

This study is relatively immune to many of the internal validity threats due to
clear temporal precedence of cause before the effect:
no reverse causality--growing up precedes adulthood; no self-selection either.
%
Even relatively mobile Americans,  virtually never select themselves into a
place at age of 16.  
% Reverse causaliity is excluded by definition: happiness now cannot predict place of residence in the past.

Unobserved heterogeneity or left out variable bias can always be a problem in
data that is not an experiment or a strong quasi-experiment. No amount of
statistics can completely remove the problem. However, we tried to include at
least the most important variables and we controlled for a number of happiness
predictors. 

The key for solving the puzzle of endogeneity is to be able to argue that
variation in main variable of interest, \textsc{place when 16 yo}, is random \citep{sorensen12}, at
least random or exogenous with respect to other key variables in the
model. Again, we feel quite confident that it is the case here--a 16 year old
person has virtually no influence over her place of settlement, and hence, place
of settlement is virtually unrelated to person's characteristics--the other
variables in the model.

To further boost our inference we conduct few robustness checks using
triangulation--using alternative measures of key concepts; and we add self
reported health and occupational dummies. We postponed discussion of these two
variables till the end because they are missing for about half of the sample and
there is a debate about whether health is endogenous
\citep{diener17,diener15,liu16}. Occupational dummies are rarely used in the
literature as predictors of SWB, but we think they can add in robustness, as
occupations clearly correlate with size of a place, and arguably people are
happier in some occupations than other, and hence, omission of this variable may
lead to biased results.  

Two alternative size of a place variables are: \textsc{size deciles}, 
 deciles of population size of a place of residence; and  \textsc{srcbelt},
 which distinguishes between medium and large suburbs and metropolitan
 areas. Their full definitions follow.

{\scriptsize
\begin{verbatim}
Variable size : SIZE OF PLACE IN 1000S (Note, the study uses deciles of the variable)
Literal Question
Size of Place in thousands
A 4-digit number which provides actual size of place of interview
(Cols. 166-169). Remember when using this code to add 3 zeros. Listed
below are the frequencies for gross population categories.

Descriptive Text
This code is the population to the nearest 1,000 of the smallest civil
division listed by the U.S. Census (city, town, other incorporated
area over 1,000 in population, township, division, etc.) which
encompasses the segment. If a segment falls into more than one
locality, the following rules apply in determining the locality for
which the rounded population figure is coded.
If the predominance of the listings for any segment are in one of the
localities, the rounded population of that locality is coded.
If the listings are distributed equally over localities in the
segment, and the localities are all cities, towns, or villages, the
rounded population of the larger city or town is coded. The same is
true if the localities are all rural townships or divisions.
If the listings are distributed equally over localities in the segment
and the localities include a town or village and a rural township or
division, the rounded population of the town or village is coded.
The source of the data is the 1970 U.S. Census population figures
published in the PC (1) -A series, Tables 6 and 10. For cases from the
1980 and 1990 frames analogous tables from the 1980 and 1990 Censuses
were used. See Appendix N for changes across surveys.

\end{verbatim}

\begin{verbatim}
Variable srcbelt : SRC BELTCODE
Literal Question
SRC (SURVEY RESEARCH CENTER, UNIVERSITY OF MICHIGAN) NEW BELT CODE
Descriptive Text
The SRC belt code is described in Appendix D: Recodes. See Appendix N
for changes across surveys. See Appendix T, GSS Methodological Report
No. 4.

Intent of Recode

The SRC belt code (a coding system originally devised to describe
rings around a metropolitan area and to categorize places by size
and type simultaneously) first appeared in an article written by
Bernard Laserwitz (American Sociological Review, v. 25, no. 2, 1960),
and has been used subsequently in several SRC surveys.
Its use was discontinued in 1971 because of difficulties particularly
evident in the operationalization of "adjacent and outlying areas."
For this study, however, I have revised the SRC belt code for users
who might find such a variable useful. The new SRC belt code utilizes
"name of place" information contained in the sampling units
of the NORC Field Department.

Method of Recode

This recode assigns codes to the place of interview. City
characteristics were determined by reference to the
rank ordering of SMSAs in the Statistical Abstract of the United
States, 1972, Table 20. Suburb characteristics
were determined by reference to the urbanized map in the U.S. Bureau
of the Census, 1970 Census ofPopulation, Number of Inhabitants, Series
PC (1) -A. The "other urban" codes were assigned on the basis of
county characteristics found in Table 10 of the 1970 Census of
Population, Number of Inhabitants. For cases
from the 1980, 1990, and 2000 frames analogous tables from the 1980 or
1990 Census were used.
\end{verbatim}

}

The additional variables are defined in a table \ref{var_des2} below, and their
histograms are in the figure \ref{hist2} further below.

\begin{spacing}{.9}
\input{var_des2}
{\footnotesize }
\end{spacing}

\input{hist2.tex}


Finally, we turn to regression results. Table \ref{regCDEF} starts with a full model from the body of the paper and repeats
regressions for the other two alternative measures of urbanicity.
Results are similar in columns c1 through e1. Column f1 is the same model but
uses a continuous measure of size and treats \textsc{place when 16 yo} as continuous as well.
%
What is worth highlighting is that the effect is very strong statistically:
t-value of -6, even in this full specification. And \textsc{place when 16 yo}
has stronger size effect than current size of a place now, about twice as strong. And in f2, the
effect size on \textsc{place when 16 yo} is three times of that on \textsc{size
  of place}. Perhaps, it indicates that there is more continuity on
\textsc{place when 16 yo}, the effect is more monotonic than current size
variables, and hence, it is so much stronger when treating size variables as continuous. 

Subsequent models \#2 repeat previous models but are over-saturated with
self-reported health and occupational dummies controls.
One clear difference is that for the key variable of interest, \textsc{place
  when 16 yo},  ``farm'' is no longer significant. We still report the finding
of ``farm'' as a happy place in the body of the paper, but we caution, that the
result may not be very robust. The results on largest category ``250k-'', on the
other hand, are larger in models \#2, and so are results on each respective
largest place where a person is living currently. 


\begin{spacing}{.9}
  \begin{table}[H]\centering
    \caption{OLS regressions  of SWB. Beta (fully standardized) coefficients. All
      models include year and region dummies.} \label{regCDEF}
    \begin{scriptsize} \begin{tabular}{p{2.1in}p{.50in}p{.50in}p{.50in}p{.50in}p{.50in}p{.50in}p{.50in}p{.50in}p{.50in}p{.50 in}}\hline
        \input{regCDEF.tex}
        \hline *** p$<$0.001 ** p$<$0.01, * p$<$0.05, + p$<$0.1; robust std err
      \end{tabular}\end{scriptsize}\end{table}
\end{spacing}


\section*{Future research}

There is much more to be found: sample could be split by year and cohort--again we know that recently people
are not unhappy in largest cities anymore $<$MASKED FOR PEER REVIEW$>$% \citep{aok-swbGenYcity,aok-misanthropy-trustCity}
--this is very
important--does place of growing up does not affect Millenials?

And one could explore patterns of migrations and interactions between current
place and place earlier. The goal of this study was to show the basic
main relationship; we leave it for the future to look at the details.

Again, a limitation is that a place at age 16 is a proxy for growing up--a person could have moved.
Again, for most people, it is reasonable to assume that a place where a person
lived when she was 16 was a type of a place where a person spend most of her
childhood. Still, future research can improve by using more precise information
and possibly taking into account moving.

Growing up on a farm resulting in later life happiness is a curious result that
would be a fascinating topic for the future research. In particular, research
could investigate an interplay with one's health and occupational choices, as
these variables seem to explain away the effect of farm upbringing as shown in
robustness checks. 

Moving has arguably different effects on different people.
\citet{schoenbaum17}, for instance, makes several convincing points--females
often move due to husband's job, but they benefit less, or indeed,
lose. \citet{kettlewell10} argues almost the opposite: there is no SWB effect
for males, but positive effect for females.\footnote{We find this argument
  somewhat less
  persuasive: there is just a handful of movers for each
gender, Australia has a peculiar geography and rural areas may be unwelcoming,
perhaps especially for females. Also, we do not understant why SWB is
log-transformed, and we note that at least some analyses appear to
be done in excel, while it is well known that excel causes bad science, see for
instance \url{http://www.statisticalengineering.com/Weibull/excel.html}. }
People in low-paid jobs, such as janitors, are better off in smaller
places, even economically--low paid jobs are not much better paid in big cities,
but the cost of living is much bigger \citep{schoenbaum17, irwinNYT17sep3}. 

I have focused on urbanicity, but thinking about place, one can also think about
a region, and US is very diverse across its regions. Indeed, one can have an
impression that US is a collection of different countries, just like European
Union--future research should investigate by region.


%\input .... TODO finish this up

% \section*{Descriptive Statistics}
% \label{app_des_sta}

%make sure i have [H] or h! ???
% \begin{table}[H]
% \caption{}
% \centering
% \label{}
% \begin{scriptsize}
% \input{../out/reg_c.tex}
% \end{scriptsize}
% \end{table}

\newpage
%\theendnotes
%\bibliography{/home/aok/papers/root/tex/ebib,swbRes15}
\begin{thebibliography}{115}
\newcommand{\enquote}[1]{``#1''}
\expandafter\ifx\csname natexlab\endcsname\relax\def\natexlab#1{#1}\fi

\bibitem[\protect\citeauthoryear{Alcock, White, Wheeler, Fleming, and
  Depledge}{Alcock et~al.}{2014}]{alcock14}
\textsc{Alcock, I., M.~P. White, B.~W. Wheeler, L.~E. Fleming, and M.~H.
  Depledge} (2014): \enquote{Longitudinal effects on mental health of moving to
  greener and less green urban areas,} \emph{Environmental science \&
  technology}, 48, 1247--1255.

\bibitem[\protect\citeauthoryear{Alesina, Baqir, and Easterly}{Alesina
  et~al.}{1999}]{alesina99al}
\textsc{Alesina, A., R.~Baqir, and W.~Easterly} (1999): \enquote{Public goods
  and ethnic divisions,} \emph{Quarterly Journal of Economics}, 114,
  1243--1284.

\bibitem[\protect\citeauthoryear{Alesina and Ferrara}{Alesina and
  Ferrara}{2000}]{alesina00}
\textsc{Alesina, A. and E.~L. Ferrara} (2000): \enquote{Participation in
  Heterogeneous Communities,} \emph{National Bureau of Economic Research
  Working Paper}.

\bibitem[\protect\citeauthoryear{Balducci and Checchi}{Balducci and
  Checchi}{2009}]{balducci09}
\textsc{Balducci, A. and D.~Checchi} (2009): \enquote{Happiness and Quality of
  City Life: The Case of Milan, the Richest Italian City.} \emph{International
  Planning Studies}, 14, 25--64.

\bibitem[\protect\citeauthoryear{Baltatescu}{Baltatescu}{2007}]{baltatescu07}
\textsc{Baltatescu, S.} (2007): \enquote{Central and Eastern Europeans
  Migrants, Journal of Identity and Migration Studies of Life. A Comparative
  Study,} \emph{Journal of Identity and Migration Studies}, 1, 67--81.

\bibitem[\protect\citeauthoryear{Bartram}{Bartram}{2014}]{bartram14}
\textsc{Bartram, D.} (2014): \enquote{Inverting the logic of economic
  migration: happiness among migrants moving from wealthier to poorer countries
  in Europe,} \emph{Journal of Happiness Studies}, 1--20.

\bibitem[\protect\citeauthoryear{Bartram}{Bartram}{2013}]{bartram13}
\textsc{Bartram, D.~V.} (2013): \enquote{Happiness and 'economic migration': A
  comparison of Eastern European migrants and stayers,} .

\bibitem[\protect\citeauthoryear{Berman, Jonides, and Kaplan}{Berman
  et~al.}{2008}]{berman08}
\textsc{Berman, M.~G., J.~Jonides, and S.~Kaplan} (2008): \enquote{The
  cognitive benefits of interacting with nature,} \emph{Psychological Science},
  19, 1207--1212.

\bibitem[\protect\citeauthoryear{Berman, Kross, Krpan, Askren, Burson, Deldin,
  Kaplan, Sherdell, Gotlib, and Jonides}{Berman et~al.}{2012}]{berman12}
\textsc{Berman, M.~G., E.~Kross, K.~M. Krpan, M.~K. Askren, A.~Burson, P.~J.
  Deldin, S.~Kaplan, L.~Sherdell, I.~H. Gotlib, and J.~Jonides} (2012):
  \enquote{Interacting with nature improves cognition and affect for
  individuals with depression,} \emph{Journal of affective disorders}, 140,
  300--305.

\bibitem[\protect\citeauthoryear{Berry and Okulicz-Kozaryn}{Berry and
  Okulicz-Kozaryn}{2011}]{aok11a}
\textsc{Berry, B.~J. and A.~Okulicz-Kozaryn} (2011): \enquote{An Urban-Rural
  Happiness Gradient,} \emph{Urban Geography}, 32, 871--883.

\bibitem[\protect\citeauthoryear{Berry and Okulicz-Kozaryn}{Berry and
  Okulicz-Kozaryn}{2009}]{aokcities}
\textsc{Berry, B. J.~L. and A.~Okulicz-Kozaryn} (2009):
  \enquote{Dissatisfaction with City Life: A New Look at Some Old Questions,}
  \emph{Cities}, 26, 117--124.

\bibitem[\protect\citeauthoryear{Blanchflower and Oswald}{Blanchflower and
  Oswald}{2011}]{blanchflower11}
\textsc{Blanchflower, D.~G. and A.~J. Oswald} (2011): \enquote{International
  happiness: A new view on the measure of performance,} \emph{The Academy of
  Management Perspectives}, 25, 6--22.

\bibitem[\protect\citeauthoryear{{Central Intelligence Agency}}{{Central
  Intelligence Agency}}{2017}]{cia}
\textsc{{Central Intelligence Agency}} (2017): \enquote{COUNTRY COMPARISON ::
  LIFE EXPECTANCY AT BIRTH,} \emph{The World Factbook}.

\bibitem[\protect\citeauthoryear{Chandran}{Chandran}{2017}]{chandranCNBC17oct24}
\textsc{Chandran, N.} (2017): \enquote{Global passport rankings: US loses
  power, tiny Southeast Asian country takes top spot,} \emph{CNBC}.

\bibitem[\protect\citeauthoryear{Chatterji}{Chatterji}{2013}]{ibt13}
\textsc{Chatterji, A.} (2013): \enquote{London is the Unhappiest Place to Live
  in Britain,} \emph{International Business Times}.

\bibitem[\protect\citeauthoryear{Chen, Davis, Wu, and Dai}{Chen
  et~al.}{2015}]{chen15}
\textsc{Chen, J., D.~S. Davis, K.~Wu, and H.~Dai} (2015): \enquote{Life
  satisfaction in urbanizing China: The effect of city size and pathways to
  urban residency,} \emph{Cities}, 49, 88--97.

\bibitem[\protect\citeauthoryear{Coughlan}{Coughlan}{2017}]{coughlanBBC17dec6}
\textsc{Coughlan, S.} (2017): \enquote{Pisa tests: Singapore top in global
  education rankings,} \emph{BBC}.

\bibitem[\protect\citeauthoryear{De~Crevecoeur}{De~Crevecoeur}{1981}]{deCrevecoeur81}
\textsc{De~Crevecoeur, J. H. S.~J.} (1981): \emph{Letters from an American
  Farmer; And, Sketches of Eighteenth-century America}, Penguin.

\bibitem[\protect\citeauthoryear{Diener}{Diener}{2015}]{diener15}
\textsc{Diener, E.} (2015): \enquote{Advances in the Science of Subjective
  Well-Being,} \emph{2015 ISQOLS Keynote}.

\bibitem[\protect\citeauthoryear{Diener, Pressman, Hunter, and
  Delgadillo-Chase}{Diener et~al.}{2017}]{diener17}
\textsc{Diener, E., S.~D. Pressman, J.~Hunter, and D.~Delgadillo-Chase} (2017):
  \enquote{If, Why, and When Subjective Well-Being Influences Health, and
  Future Needed Research,} \emph{Applied Psychology: Health and Well-Being}, 9,
  133--167.

\bibitem[\protect\citeauthoryear{Eagly and Wood}{Eagly and
  Wood}{2010}]{eagly10}
\textsc{Eagly, A.~H. and W.~Wood} (2010): \enquote{Gender,} in \emph{Handbook
  of social psychology}, ed. by S.~T. Fiske, Wiley Online Library, 629--667.

\bibitem[\protect\citeauthoryear{Engels}{Engels}{[1845] 1987}]{engels87}
\textsc{Engels, F.} ([1845] 1987): \emph{The condition of the working class in
  England}, Penguin, New York NY.

\bibitem[\protect\citeauthoryear{{Farmer Talk}}{{Farmer
  Talk}}{2014{\natexlab{a}}}]{FFP14oct20}
\textsc{{Farmer Talk}} (2014{\natexlab{a}}): \enquote{8 Things Farm Equipment
  Can Teach You About Yourself,} \emph{Fastline Front Page}.

\bibitem[\protect\citeauthoryear{{Farmer Talk}}{{Farmer
  Talk}}{2014{\natexlab{b}}}]{FFP14dec16}
---\hspace{-.1pt}---\hspace{-.1pt}--- (2014{\natexlab{b}}): \enquote{9 Life
  Lessons Everyone Can Learn from a Farmer,} \emph{Fastline Front Page}.

\bibitem[\protect\citeauthoryear{{Farmer Talk}}{{Farmer
  Talk}}{2014{\natexlab{c}}}]{FFP14aug13}
---\hspace{-.1pt}---\hspace{-.1pt}--- (2014{\natexlab{c}}): \enquote{Ten Thing
  Things You Learn Growing Up A Farm Kid,} \emph{Fastline Front Page}.

\bibitem[\protect\citeauthoryear{{Ferrer-i-Carbonell} and
  Frijters}{{Ferrer-i-Carbonell} and Frijters}{2004}]{carbonell04}
\textsc{{Ferrer-i-Carbonell}, A. and P.~Frijters} (2004): \enquote{How
  Important is Methodology for the Estimates of the Determinants of Happiness?}
  \emph{Economic Journal}, 114, 641--659.

\bibitem[\protect\citeauthoryear{Fischer}{Fischer}{1972}]{fischer72}
\textsc{Fischer, C.~S.} (1972): \enquote{Urbanism as a Way of Life (A Review
  and an Agenda),} \emph{Sociological Methods and Research}, 1, 187--242.

\bibitem[\protect\citeauthoryear{Fischer}{Fischer}{1973}]{fischer73}
---\hspace{-.1pt}---\hspace{-.1pt}--- (1973): \enquote{Urban malaise,}
  \emph{Social Forces}, 52, 221--235.

\bibitem[\protect\citeauthoryear{Fischer}{Fischer}{1975}]{fischer75}
---\hspace{-.1pt}---\hspace{-.1pt}--- (1975): \enquote{Toward a subcultural
  theory of urbanism,} \emph{American Journal of Sociology}, 80, 1319--1341.

\bibitem[\protect\citeauthoryear{Fischer}{Fischer}{1982}]{fischer82}
---\hspace{-.1pt}---\hspace{-.1pt}--- (1982): \emph{To dwell among friends:
  Personal networks in town and city}, University of Chicago Press, Chicago IL.

\bibitem[\protect\citeauthoryear{Fischer}{Fischer}{1991}]{fischer91}
\textsc{Fischer, D.} (1991): \emph{Albion's seed: Four British folkways in
  America}, vol.~1, Oxford University Press, New York NY.

\bibitem[\protect\citeauthoryear{Florida}{Florida}{2016}]{floridaCL16apr28}
\textsc{Florida, R.} (2016): \enquote{Where Millennials and the Working Class
  Can No Longer Afford to Live,} \emph{City Lab}.

\bibitem[\protect\citeauthoryear{Frumkin}{Frumkin}{2001}]{frumkin01b}
\textsc{Frumkin, H.} (2001): \enquote{Beyond toxicity: human health and the
  natural environment,} \emph{American journal of preventive medicine}, 20,
  234--240.

\bibitem[\protect\citeauthoryear{Fuguitt and Brown}{Fuguitt and
  Brown}{1990}]{fuguitt90}
\textsc{Fuguitt, G.~V. and D.~L. Brown} (1990): \enquote{Residential
  Preferences and Population Redistribution,} \emph{Demography}, 27, 589--600.

\bibitem[\protect\citeauthoryear{Fuguitt and Zuiches}{Fuguitt and
  Zuiches}{1975}]{fuguitt75}
\textsc{Fuguitt, G.~V. and J.~J. Zuiches} (1975): \enquote{Residential
  Preferences and Population Distribution,} \emph{Demography}, 12, 491--504.

\bibitem[\protect\citeauthoryear{Fuller}{Fuller}{2017}]{fullerNYT17monD}
\textsc{Fuller, T.} (2017): \enquote{California's Far North Deplores Tyranny of
  the Urban Majority,} \emph{The New York Times}.

\bibitem[\protect\citeauthoryear{Fullman, Barber, Abajobir, Abate, Abbafati,
  Abbas, Abd-Allah, Abdulkader, Abdulle, Abera et~al.}{Fullman
  et~al.}{2017}]{fullman2017measuring}
\textsc{Fullman, N., R.~M. Barber, A.~A. Abajobir, K.~H. Abate, C.~Abbafati,
  K.~M. Abbas, F.~Abd-Allah, R.~S. Abdulkader, A.~M. Abdulle, S.~F. Abera,
  et~al.} (2017): \enquote{Measuring progress and projecting attainment on the
  basis of past trends of the health-related Sustainable Development Goals in
  188 countries: an analysis from the Global Burden of Disease Study 2016,}
  \emph{The Lancet}, 390, 1423--1459.

\bibitem[\protect\citeauthoryear{Glaeser}{Glaeser}{2011}]{glaeser11}
\textsc{Glaeser, E.} (2011): \emph{Triumph of the City: How Our Greatest
  Invention Makes Us Richer, Smarter, Greener, Healthier, and Happier}, Penguin
  Press, New York NY.

\bibitem[\protect\citeauthoryear{Hanson}{Hanson}{2015}]{hansonCityJournalautumn15}
\textsc{Hanson, V.~D.} (2015): \enquote{The Oldest Divide. With roots dating
  back to our Founding, America's urban-rural split is wider than ever.}
  \emph{City Journal}, Autumn 2015.

\bibitem[\protect\citeauthoryear{Harvey}{Harvey}{2014}]{harvey14}
\textsc{Harvey, D.} (2014): \emph{Seventeen contradictions and the end of
  capitalism}, Oxford University Press, New York NY.

\bibitem[\protect\citeauthoryear{Harvey}{Harvey}{2016}]{harvey16}
---\hspace{-.1pt}---\hspace{-.1pt}--- (2016): \enquote{Senior Loeb Scholar
  lecture,} \emph{Harvard GSD}.

\bibitem[\protect\citeauthoryear{Hendriks, Ludwigs, and Veenhoven}{Hendriks
  et~al.}{2014}]{hendriks14}
\textsc{Hendriks, M., K.~Ludwigs, and R.~Veenhoven} (2014): \enquote{Why are
  Locals Happier than Internal Migrants? The Role of Daily Life,} \emph{Social
  Indicators Research}, 1--28.

\bibitem[\protect\citeauthoryear{Herbst and Lucio}{Herbst and
  Lucio}{2014}]{herbst14}
\textsc{Herbst, C. and J.~Lucio} (2014): \enquote{Happy in the Hood? The Impact
  of Residential Segregation on Self-Reported Happiness,} \emph{IZA Discussion
  Paper}.

\bibitem[\protect\citeauthoryear{Heriatge}{Heriatge}{2017}]{heritage}
\textsc{Heriatge} (2017): \enquote{2017 Index of Economic Freedom,}
  \emph{Heritage}.

\bibitem[\protect\citeauthoryear{Hymowitz}{Hymowitz}{2017}]{hymowitzCJ17autumn}
\textsc{Hymowitz, K.~S.} (2017): \enquote{Why America Can't Lower Child-Poverty
  Rates. Allowing millions of low-skilled immigrants into the U.S. every year
  swells the ranks of the poor.} \emph{City Journal}.

\bibitem[\protect\citeauthoryear{IMF}{IMF}{2017}]{authorNYT17monD}
\textsc{IMF} (2017): \enquote{World Economic Outlook Database,}
  \emph{International Monetary Fund}.

\bibitem[\protect\citeauthoryear{Irwin}{Irwin}{2017}]{irwinNYT17sep3}
\textsc{Irwin, N.} (2017): \enquote{To Understand Rising Inequality, Consider
  the Janitors at Two Top Companies, Then and Now,} \emph{The New York Times}.

\bibitem[\protect\citeauthoryear{Jokela}{Jokela}{2014}]{jokela14}
\textsc{Jokela, M.} (2014): \enquote{Flow of cognitive capital across rural and
  urban United States,} \emph{Intelligence}, 46, 47--53.

\bibitem[\protect\citeauthoryear{Jong, Chamratrithirong, and Tran}{Jong
  et~al.}{2002}]{jong02}
\textsc{Jong, G.~F., A.~Chamratrithirong, and Q.-G. Tran} (2002): \enquote{For
  better, for worse: Life satisfaction consequences of migration,}
  \emph{International Migration Review}, 36, 838--863.

\bibitem[\protect\citeauthoryear{Kallis, Kerschner, and Martinez-Alier}{Kallis
  et~al.}{2012}]{kallis12}
\textsc{Kallis, G., C.~Kerschner, and J.~Martinez-Alier} (2012): \enquote{The
  economics of degrowth,} \emph{Ecological Economics}, 84, 172--180.

\bibitem[\protect\citeauthoryear{Kasser}{Kasser}{2003}]{kasser13}
\textsc{Kasser, T.} (2003): \emph{The high price of materialism}, MIT press.

\bibitem[\protect\citeauthoryear{Kasser and Ryan}{Kasser and
  Ryan}{1993}]{kasser93}
\textsc{Kasser, T. and R.~Ryan} (1993): \enquote{A dark side of the American
  dream: correlates of financial success as a central life aspiration.}
  \emph{Journal of personality and social psychology}, 65, 410.

\bibitem[\protect\citeauthoryear{Kettlewell}{Kettlewell}{2010}]{kettlewell10}
\textsc{Kettlewell, N.} (2010): \enquote{The impact of rural to urban migration
  on wellbeing in Australia,} \emph{Australasian Journal of Regional Studies},
  16, 187.

\bibitem[\protect\citeauthoryear{Klein}{Klein}{2014}]{klein14}
\textsc{Klein, N.} (2014): \emph{This changes everything: capitalism vs. the
  climate}, Simon and Schuster, New York NY.

\bibitem[\protect\citeauthoryear{Knight and Gunatilaka}{Knight and
  Gunatilaka}{2010}]{knight10}
\textsc{Knight, J. and R.~Gunatilaka} (2010): \enquote{Great expectations? The
  subjective well-being of rural--urban migrants in China,} \emph{World
  Development}, 38, 113--124.

\bibitem[\protect\citeauthoryear{Lall, Selod, and Shalizi}{Lall
  et~al.}{2006}]{lall06al}
\textsc{Lall, S.~V., H.~Selod, and Z.~Shalizi} (2006): \enquote{{Rural-Urban
  Migration In Developing Countries. 'A Survey Of Theoretical Predictions And
  Empirical Findings'},} \emph{World Bank Policy Research Working Paper}, 15,
  1--63.

\bibitem[\protect\citeauthoryear{LaMothe}{LaMothe}{2016}]{lamothe16}
\textsc{LaMothe, R.} (2016): \enquote{Neoliberal capitalism and the corruption
  of society: A pastoral political analysis,} \emph{Pastoral Psychology}, 65,
  5.

\bibitem[\protect\citeauthoryear{Lane}{Lane}{2000}]{lane00}
\textsc{Lane, R.~E.} (2000): \emph{The loss of happiness in market
  democracies}, New Haven CT: Yale University Press.

\bibitem[\protect\citeauthoryear{Lederbogen, Kirsch, Haddad, Streit, Tost,
  Schuch, Wust, Pruessner, Rietschel, Deuschle, and
  {Meyer-Lindenberg}}{Lederbogen et~al.}{2011}]{lederbogen11}
\textsc{Lederbogen, F., P.~Kirsch, L.~Haddad, F.~Streit, H.~Tost, P.~Schuch,
  S.~Wust, J.~C. Pruessner, M.~Rietschel, M.~Deuschle, and
  A.~{Meyer-Lindenberg}} (2011): \enquote{City living and urban upbringing
  affect neural social stress processing in humans,} \emph{Nature}, 474.

\bibitem[\protect\citeauthoryear{Lenzi and Perucca}{Lenzi and
  Perucca}{2016}]{lenzi16D}
\textsc{Lenzi, C. and G.~Perucca} (2016): \enquote{The Easterlin paradox and
  the urban-rural divide in life satisfaction: Evidence from Romania,}
  \emph{Unpublished; \url{http://www.grupposervizioambiente.it}}.

\bibitem[\protect\citeauthoryear{Liu, Floud, Pirie, Green, Peto, Beral,
  Collaborators et~al.}{Liu et~al.}{2016}]{liu16}
\textsc{Liu, B., S.~Floud, K.~Pirie, J.~Green, R.~Peto, V.~Beral, M.~W.~S.
  Collaborators, et~al.} (2016): \enquote{Does happiness itself directly affect
  mortality? The prospective UK Million Women Study,} \emph{The Lancet}, 387,
  874--881.

\bibitem[\protect\citeauthoryear{Lu, Schellenberg, Hou, and Helliwell}{Lu
  et~al.}{2015}]{lu15}
\textsc{Lu, C., G.~Schellenberg, F.~Hou, and J.~F. Helliwell} (2015):
  \enquote{How's Life in the City? Life Satisfaction Across Census Metropolitan
  Areas and Economic Regions in Canada,} \emph{Economic Insights}, 11-626-X.

\bibitem[\protect\citeauthoryear{Malanga}{Malanga}{2006}]{malangaCJ06summer}
\textsc{Malanga, S.} (2006): \enquote{How Unskilled Immigrants Hurt Our
  Economy. A handful of industries get low-cost labor, and the taxpayers foot
  the bill.} \emph{City Journal}.

\bibitem[\protect\citeauthoryear{Maller, Townsend, Pryor, Brown, and
  St~Leger}{Maller et~al.}{2006}]{maller06}
\textsc{Maller, C., M.~Townsend, A.~Pryor, P.~Brown, and L.~St~Leger} (2006):
  \enquote{Healthy nature healthy people:'contact with nature'as an upstream
  health promotion intervention for populations,} \emph{Health promotion
  international}, 21, 45--54.

\bibitem[\protect\citeauthoryear{Marcuse}{Marcuse}{2015}]{marcuse15}
\textsc{Marcuse, H.} (2015): \emph{Eros and civilization: A philosophical
  inquiry into Freud}, Boston MA: Beacon Press.

\bibitem[\protect\citeauthoryear{McSpadden}{McSpadden}{2015}]{mcspaddenNYT17monD}
\textsc{McSpadden, K.} (2015): \enquote{Singapore has the world's fastest
  Internet: Akamai,} \emph{E27}.

\bibitem[\protect\citeauthoryear{Morrison}{Morrison}{2015}]{morrison15}
\textsc{Morrison, P.} (2015): \enquote{Capturing effects of cities on
  subjective wellbeing,} \emph{European Regional Science Association
  Conference, Lisbon}.

\bibitem[\protect\citeauthoryear{Morrison}{Morrison}{2011}]{morrison11}
\textsc{Morrison, P.~S.} (2011): \enquote{Local expressions of subjective
  well-being: The New Zealand experience,} \emph{Regional studies}, 45,
  1039--1058.

\bibitem[\protect\citeauthoryear{Morrison and Weckroth}{Morrison and
  Weckroth}{2017}]{morrison17}
\textsc{Morrison, P.~S. and M.~Weckroth} (2017): \enquote{Human values,
  subjective well-being and the metropolitan region,} \emph{Regional Studies},
  1--13.

\bibitem[\protect\citeauthoryear{Ng}{Ng}{1996}]{ng96}
\textsc{Ng, Y.-K.} (1996): \enquote{Happiness surveys: Some comparability
  issues and an exploratory survey based on just perceivable increments,}
  \emph{Social Indicators Research}, 38, 1--27.

\bibitem[\protect\citeauthoryear{Ng}{Ng}{1997}]{ng97}
---\hspace{-.1pt}---\hspace{-.1pt}--- (1997): \enquote{A case for happiness,
  cardinalism, and interpersonal comparability,} \emph{The Economic Journal},
  107, 1848--1858.

\bibitem[\protect\citeauthoryear{Ng}{Ng}{2011}]{ng11}
---\hspace{-.1pt}---\hspace{-.1pt}--- (2011): \enquote{Happiness is absolute,
  universal, ultimate, unidimensional, cardinally measurable and
  interpersonally comparable: A basis for the environmentally responsible Happy
  Nation Index,} Tech. rep., Monash University, Department of Economics.

\bibitem[\protect\citeauthoryear{{Office for National Statistics}}{{Office for
  National Statistics}}{2011}]{ons11}
\textsc{{Office for National Statistics}} (2011): \enquote{Analysis of
  Experimental Subjective Well-being Data from the Annual Population Survey,}
  \emph{The National Archives}.

\bibitem[\protect\citeauthoryear{Okulicz-Kozaryn}{Okulicz-Kozaryn}{2010}]{aokrel}
\textsc{Okulicz-Kozaryn, A.} (2010): \enquote{Religiosity and life satisfaction
  across nations,} \emph{Mental Health, Religion \& Culture}, 13, 155--169.

\bibitem[\protect\citeauthoryear{Okulicz-Kozaryn}{Okulicz-Kozaryn}{2011}]{aokRel2}
---\hspace{-.1pt}---\hspace{-.1pt}--- (2011): \enquote{Does religious diversity
  make us unhappy?} \emph{Mental Health, Religion \& Culture}, 14, 1063--1076.

\bibitem[\protect\citeauthoryear{Okulicz-Kozaryn}{Okulicz-Kozaryn}{2015}]{aokCityBook15}
---\hspace{-.1pt}---\hspace{-.1pt}--- (2015): \emph{Happiness and Place. Why
  Life is Better Outside of the City.}, Palgrave Macmillan, New York NY.

\bibitem[\protect\citeauthoryear{Okulicz-Kozaryn}{Okulicz-Kozaryn}{2016}]{aok-ls_fisher16}
---\hspace{-.1pt}---\hspace{-.1pt}--- (2016): \enquote{Unhappy metropolis (when
  American city is too big),} \emph{Cities}.

\bibitem[\protect\citeauthoryear{Okulicz-Kozaryn}{Okulicz-Kozaryn}{2018{\natexlab{a}}}]{aok-swbGenYcity18}
---\hspace{-.1pt}---\hspace{-.1pt}--- (2018{\natexlab{a}}): \enquote{No Urban
  Malaise for Millennials,} \emph{Regional Studies}.

\bibitem[\protect\citeauthoryear{Okulicz-Kozaryn}{Okulicz-Kozaryn}{2018{\natexlab{b}}}]{aok-misanthropy-trustCity}
---\hspace{-.1pt}---\hspace{-.1pt}--- (2018{\natexlab{b}}): \enquote{Urban
  Misanthropy: Cities and Dislike of Humankind,} \emph{Unpublished}.

\bibitem[\protect\citeauthoryear{Okulicz-Kozaryn and
  da~Rocha~Valente}{Okulicz-Kozaryn and da~Rocha~Valente}{2017}]{okulicz17B}
\textsc{Okulicz-Kozaryn, A. and R.~da~Rocha~Valente} (2017): \enquote{Life
  Satisfaction of Career Women and Housewives,} \emph{Applied Research in
  Quality of Life}, 1--30.

\bibitem[\protect\citeauthoryear{Okulicz-Kozaryn and Mazelis}{Okulicz-Kozaryn
  and Mazelis}{2016}]{aok_brfss_city_cize16}
\textsc{Okulicz-Kozaryn, A. and J.~M. Mazelis} (2016): \enquote{Urbanism and
  Happiness: A Test of Wirth's Theory on Urban Life,} \emph{Urban Studies}.

\bibitem[\protect\citeauthoryear{Okulicz-Kozaryn and Valente}{Okulicz-Kozaryn
  and Valente}{2017}]{aok-sizeFetish17}
\textsc{Okulicz-Kozaryn, A. and R.~R. Valente} (2017): \enquote{The Unconscious
  Size Fetish: Glorification and Desire of the City,} in \emph{Psychoanalysis
  and the GlObal}, ed. by I.~Kapoor, University of Nebraska Press.

\bibitem[\protect\citeauthoryear{Oswald and Wu}{Oswald and
  Wu}{2009}]{oswald09w}
\textsc{Oswald, A.~J. and S.~Wu} (2009): \enquote{{Objective Confirmation of
  Subjective Measures of Human Well-Being: Evidence from the U.S.A.}}
  \emph{Science}, 327, 576--579.

\bibitem[\protect\citeauthoryear{Park}{Park}{1915}]{park15}
\textsc{Park, R.~E.} (1915): \enquote{The city: Suggestions for the
  investigation of human behavior in the city environment,} \emph{The American
  Journal of Sociology}, 20, 577--612.

\bibitem[\protect\citeauthoryear{Park}{Park}{1928}]{park28}
---\hspace{-.1pt}---\hspace{-.1pt}--- (1928): \enquote{Human migration and the
  marginal man,} \emph{American journal of sociology}, 881--893.

\bibitem[\protect\citeauthoryear{Park, Burgess, and Mac~Kenzie}{Park
  et~al.}{[1925] 1984}]{park84}
\textsc{Park, R.~E., E.~W. Burgess, and R.~D. Mac~Kenzie} ([1925] 1984):
  \emph{The city}, University of Chicago Press, Chicago IL.

\bibitem[\protect\citeauthoryear{Postmes and Branscombe}{Postmes and
  Branscombe}{2002}]{postmes02}
\textsc{Postmes, T. and N.~R. Branscombe} (2002): \enquote{Influence of
  long-term racial environmental composition on subjective well-being in
  African Americans.} \emph{Journal of personality and social psychology}, 83,
  735.

\bibitem[\protect\citeauthoryear{Pretty}{Pretty}{2012}]{pretty12}
\textsc{Pretty, J.} (2012): \emph{The earth only endures: On reconnecting with
  nature and our place in it}, Routledge, New York NY.

\bibitem[\protect\citeauthoryear{Putnam}{Putnam}{2007}]{putnam07}
\textsc{Putnam, R.} (2007): \enquote{E pluribus unum: Diversity and community
  in the twenty-first century,} \emph{Scandinavian Political Studies}, 30,
  137--174.

\bibitem[\protect\citeauthoryear{Ross, Reynolds, and Geis}{Ross
  et~al.}{2000}]{ross00}
\textsc{Ross, C.~E., J.~R. Reynolds, and K.~J. Geis} (2000): \enquote{The
  contingent meaning of neighborhood stability for residents' psychological
  well-being,} \emph{American Sociological Review}, 581--597.

\bibitem[\protect\citeauthoryear{Schkade and Kahneman}{Schkade and
  Kahneman}{1998}]{schkade98k}
\textsc{Schkade, D. and D.~Kahneman} (1998): \enquote{{Does living in
  California make people happy? A focusing illusion in judgments of life
  satisfaction},} \emph{Psychological Science}, 9, 340--346.

\bibitem[\protect\citeauthoryear{Schmuck, Kasser, and Ryan}{Schmuck
  et~al.}{2000}]{schmuck00}
\textsc{Schmuck, P., T.~Kasser, and R.~M. Ryan} (2000): \enquote{Intrinsic and
  extrinsic goals: Their structure and relationship to well-being in German and
  US college students,} \emph{Social Indicators Research}, 50, 225--241.

\bibitem[\protect\citeauthoryear{Schoenbaum}{Schoenbaum}{2017}]{schoenbaum17}
\textsc{Schoenbaum, N.} (2017): \enquote{Stuck or Rooted? The Costs of Mobility
  and the Value of Place,} .

\bibitem[\protect\citeauthoryear{Schor}{Schor}{2008}]{schor08}
\textsc{Schor, J.} (2008): \emph{The overworked American: The unexpected
  decline of leisure}, Basic books, New York NY.

\bibitem[\protect\citeauthoryear{Scitovsky}{Scitovsky}{1976}]{scitovsky76}
\textsc{Scitovsky, T.} (1976): \emph{The joyless economy: An inquiry into human
  satisfaction and consumer dissatisfaction.}, Oxford U Press, New York NY.

\bibitem[\protect\citeauthoryear{Senior}{Senior}{2006}]{senior_ny_sep16_14}
\textsc{Senior, J.} (2006): \enquote{Some Dark Thoughts on Happiness,}
  \emph{New York Magazine}.

\bibitem[\protect\citeauthoryear{Shafir, Diamond, and Tversky}{Shafir
  et~al.}{1997}]{shafir97al}
\textsc{Shafir, E., P.~Diamond, and A.~Tversky} (1997): \enquote{money
  illusion.} \emph{Quarterly Journal of Economics}, 112, 341 -- 374.

\bibitem[\protect\citeauthoryear{Simmel}{Simmel}{1903}]{simmel03}
\textsc{Simmel, G.} (1903): \enquote{The metropolis and mental life,} \emph{The
  Urban Sociology Reader}, 23--31.

\bibitem[\protect\citeauthoryear{Smith}{Smith}{2017}]{smithNYT17aug9}
\textsc{Smith, B.} (2017): \enquote{Don't Let Your Children Grow Up to Be
  Farmers,} \emph{The New York Times}.

\bibitem[\protect\citeauthoryear{Sorensen}{Sorensen}{2012}]{sorensen12}
\textsc{Sorensen, J.~B.} (2012): \enquote{Endogeneity is a fancy word for a
  simple problem,} \emph{Unpublished}.

\bibitem[\protect\citeauthoryear{Stark and Taylor}{Stark and
  Taylor}{1991}]{stark91t}
\textsc{Stark, O. and J.~Taylor} (1991): \enquote{{Migration incentives, mig
  types: The role of relative deprivation},} \emph{Economic Journal}, 101,
  63--1178.

\bibitem[\protect\citeauthoryear{Stefan}{Stefan}{2010}]{stefanSS10may}
\textsc{Stefan} (2010): \enquote{Are You a Wage Slave?} \emph{Socialist
  Standard}.

\bibitem[\protect\citeauthoryear{Stephan and McMullin}{Stephan and
  McMullin}{1982}]{stephan82}
\textsc{Stephan, G.~E. and D.~R. McMullin} (1982): \enquote{Tolerance of sexual
  nonconformity: City size as a situational and early learning determinant,}
  \emph{American Sociological Review}, 411--415.

\bibitem[\protect\citeauthoryear{Tesson}{Tesson}{2013}]{tesson13}
\textsc{Tesson, S.} (2013): \emph{Consolations of the Forest: Alone in a Cabin
  in the Middle Taiga}, Penguin, London UK.

\bibitem[\protect\citeauthoryear{T{\"o}nnies}{T{\"o}nnies}{[1887]
  2002}]{tonnies57}
\textsc{T{\"o}nnies, F.} ([1887] 2002): \emph{Community and society},
  DoverPublications.com, Mineola NY.

\bibitem[\protect\citeauthoryear{Tuch}{Tuch}{1987}]{tuch87}
\textsc{Tuch, S.~A.} (1987): \enquote{Urbanism, region, and tolerance
  revisited: The case of racial prejudice,} \emph{American Sociological
  Review}, 504--510.

\bibitem[\protect\citeauthoryear{Veenhoven}{Veenhoven}{1995}]{veenhoven95b}
\textsc{Veenhoven, R.} (1995): \enquote{World database of happiness,}
  \emph{Social Indicators Research}, 34, 299--313.

\bibitem[\protect\citeauthoryear{Vogt~Yuan}{Vogt~Yuan}{2007}]{vogt07}
\textsc{Vogt~Yuan, A.~S.} (2007): \enquote{Racial composition of neighborhood
  and emotional well-being,} \emph{Sociological Spectrum}, 28, 105--129.

\bibitem[\protect\citeauthoryear{Vohs, Mead, and Goode}{Vohs
  et~al.}{2006}]{vohs06}
\textsc{Vohs, K., N.~Mead, and M.~Goode} (2006): \enquote{The psychological
  consequences of money,} \emph{science}, 314, 1154--1156.

\bibitem[\protect\citeauthoryear{Wheeler, White, Stahl-Timmins, and
  Depledge}{Wheeler et~al.}{2012}]{wheeler12}
\textsc{Wheeler, B.~W., M.~White, W.~Stahl-Timmins, and M.~H. Depledge} (2012):
  \enquote{Does living by the coast improve health and wellbeing?} \emph{Health
  \& Place}.

\bibitem[\protect\citeauthoryear{White and White}{White and
  White}{1977}]{white77}
\textsc{White, M.~G. and L.~White} (1977): \emph{The intellectual versus the
  city: from Thomas Jefferson to Frank Lloyd Wright}, Oxford University Press,
  Oxford UK.

\bibitem[\protect\citeauthoryear{White, Alcock, Wheeler, and Depledge}{White
  et~al.}{2013{\natexlab{a}}}]{white13b}
\textsc{White, M.~P., I.~Alcock, B.~W. Wheeler, and M.~H. Depledge}
  (2013{\natexlab{a}}): \enquote{Coastal proximity, health and well-being:
  Results from a longitudinal panel survey,} \emph{Health \& Place}.

\bibitem[\protect\citeauthoryear{White, Alcock, Wheeler, and Depledge}{White
  et~al.}{2013{\natexlab{b}}}]{white13}
---\hspace{-.1pt}---\hspace{-.1pt}--- (2013{\natexlab{b}}): \enquote{Would You
  Be Happier Living in a Greener Urban Area? A Fixed-Effects Analysis of Panel
  Data,} \emph{Psychological science}, 24, 920--928.

\bibitem[\protect\citeauthoryear{Wirth}{Wirth}{1938}]{wirth38}
\textsc{Wirth, L.} (1938): \enquote{Urbanism as a Way of Life,} \emph{American
  Journal of Sociology}, 44, 1--24.

\bibitem[\protect\citeauthoryear{YouGov}{YouGov}{2012}]{YouGov12-place}
\textsc{YouGov} (2012): \enquote{The suburban dream: Suburbs are most popular
  place to live,} \emph{YouGov}.

\end{thebibliography}


\end{spacing}
\end{document}
