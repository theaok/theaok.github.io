%to have line numbers
%\RequirePackage{lineno}
\documentclass[12pt, letterpaper]{article}      
\usepackage[margin=.1cm,font=small,labelfont=bf]{caption}[2007/03/09]
%\usepackage{endnotes}
\usepackage{setspace}
\usepackage{longtable}                        
\usepackage{anysize}                          
\usepackage{natbib}                           
%\bibpunct{(}{)}{,}{a}{,}{,}                   
\bibpunct{(}{)}{,}{a}{}{,}                   
\usepackage{amsmath}
\usepackage[pdftex]{graphicx}                
\usepackage{epstopdf}
\usepackage[hidelinks]{hyperref}                             % For creating hyperlinks in cross references


% \usepackage[margins]{trackchanges}

% \note[editor]{The note}
% \annote[editor]{Text to annotate}{The note}
%    \add[editor]{Text to add}
% \remove[editor]{Text to remove}
% \change[editor]{Text to remove}{Text to add}



\marginsize{2.5cm}{2.5cm}{.99cm}{.99cm}%{left}{right}{top}{bottom}   
					          % Helps LaTeX put figures where YOU want
 \renewcommand{\topfraction}{1}	                  % 90% of page top can be a float
 \renewcommand{\bottomfraction}{1}	          % 90% of page bottom can be a float
 \renewcommand{\textfraction}{0.0}	          % only 10% of page must to be text

 \usepackage{float}                               %latex will not complain to include float after float

\usepackage[table]{xcolor}                        %for table shading
\definecolor{gray90}{gray}{0.90}
\definecolor{orange}{RGB}{255,128,0}

\renewcommand\arraystretch{.9}                    %for spacing of arrays like tabular

%-------------------- my commands -----------------------------------------
\newenvironment{ig}[1]{
\begin{center}
 %\includegraphics[height=5.0in]{#1} 
 \includegraphics[height=3.3in]{#1} 
\end{center}}

 \newcommand{\cc}[1]{
\hspace{-.13in}$\bullet$\marginpar{\begin{spacing}{.6}\begin{footnotesize}{#1}\end{footnotesize}\end{spacing}}
\hspace{-.13in} }

%-------------------- END my commands -----------------------------------------



%-------------------- extra options -----------------------------------------

%%%%%%%%%%%%%
% footnotes %
%%%%%%%%%%%%%

%\long\def\symbolfootnote[#1]#2{\begingroup% %these can be used to make footnote  nonnumeric asterick, dagger etc
%\def\thefootnote{\fnsymbol{footnote}}\footnote[#1]{#2}\endgroup}	%see: http://help-csli.stanford.edu/tex/latex-footnotes.shtml

%%%%%%%%%%%
% spacing %
%%%%%%%%%%%

% \abovecaptionskip: space above caption
% \belowcaptionskip: space below caption
%\oddsidemargin 0cm
%\evensidemargin 0cm

%%%%%%%%%
% style %
%%%%%%%%%

%\pagestyle{myheadings}         % Option to put page headers
                               % Needed \documentclass[a4paper,twoside]{article}
%\markboth{{\small\it Politics and Life Satisfaction }}
%{{\small\it Adam Okulicz-Kozaryn} }

%\headsep 1.5cm
% \pagestyle{empty}			% no page numbers
% \parindent  15.mm			% indent paragraph by this much
% \parskip     2.mm			% space between paragraphs
% \mathindent 20.mm			% indent math equations by this much

%%%%%%%%%%%%%%%%%%
% extra packages %
%%%%%%%%%%%%%%%%%%

\usepackage{datetime}


\usepackage[latin1]{inputenc}
\usepackage{tikz}
\usetikzlibrary{shapes,arrows,backgrounds}


%\usepackage{color}					% For creating coloured text and background
%\usepackage{float}
\usepackage{subfig}                                     % for combined figures

\renewcommand{\ss}[1]{{\colorbox{blue}{\bf \color{white}{#1}}}}
\newcommand{\ee}[1]{\endnote{\vspace{-.10in}\begin{spacing}{1.0}{\normalsize #1}\end{spacing}\vspace{.20in}}}




% \usepackage{sectsty}
% \allsectionsfont{\normalfont\sffamily}



% \usepackage{sectsty}
% \allsectionsfont{\normalfont\sffamily}
% %\usepackage[margins]{trackchanges}

% \renewcommand\familydefault{\sfdefault}


\usepackage{rotating}
%\usepackage{catchfilebetweentags}
%-------------------- END extra options -----------------------------------------
\date{Draft: {}\today}
\title{
%  Urban Malaise Revisited% \\ (Exactly When  Big is Too Big)
%When Place is Too Big:\\ Happy Town and Unhappy Metropolis\\
Unhappy Metropolis\\ (When American City Is Too Big)
}
\author{
% Adam Okulicz-Kozaryn\thanks{EMAIL: adam.okulicz.kozaryn@gmail.com
%   \hfill I thank Claude S Fischer and Michael Timberlake (urban
%   studies editor) for useful comments.  All mistakes are mine.} \\
% {\small Rutgers - Camden}
}

\begin{document}

%%\setpagewiselinenumbers
%\modulolinenumbers[1]
%\linenumbers

\bibliographystyle{ecta}
\maketitle
\vspace{-.4in}
\begin{center}

\end{center}


\begin{abstract}
\noindent  
 Most scholars in urban studies and public policy/administration
support city living, that is,  they  suggest
that people are happy in cities or at least they focus on how to make
people happy in cities. Planners also largely focus on making
cities  happy places. Economists emphasize 
agglomeration economies. Urbanism is
 popular and fashionable. % , although it used to be several
% decades ago.
 The goal of this study is to challenge this common wisdom and
stimulate discussion. 
I use the General Social Survey to calculate subjective
 wellbeing or happiness by size of a place  to find out when a place
 is too big.  
 %LESS CLEAR WHEN CONTROLLING
                                %FOR RACE, but malaise develops already at around 50 thousand.
Malaise or unhappiness increases with size of a place (with a bump around 10k
people) and reaches a significant level when population exceeds several hundred thousand. 
 Results are robust to the
operationalization of an urban area, and to the elaboration of the model
with multiple  controls known to predict happiness. 
This study concerns only the US, and results should not be generalized to other countries or historical contexts. Directions for future research are discussed. 
\end{abstract}
\vspace{.15in} 
\noindent{\sc keywords:  urbanism, urban, rural, cites, suburbs, size
  of a place, population density, happiness, life satisfaction,  Subjective
  Wellbeing (SWB), General Social Survey}
%\vspace{-.25in} 

\begin{spacing}{1.9}
\rowcolors{1}{white}{gray90}

%  \ExecuteMetaData[../out/tex]{ginipov}

\section*{Introduction}

Urban scholars, regional scientists, and planners study Quality Of
Life (QOL), which is usually defined in a narrow sense as
quality of transportation, housing, or some other
domain. Psychologists, on the other
hand, study Subjective Wellbeing (SWB),\footnote{SWB is, roughly
  speaking, 
synonymous with happiness and life satisfaction. I will use these terms
interchangeably. In laboratory settings
using small samples with many
measures, it is possible to differentiate between the concepts, but it is not
possible in large scale surveys as used here.
Happiness measurement is discussed later. 
% Definitions are provided in the data section.
} which is usually measured
 with surveys asking respondents about their happiness. SWB is
subjective, self-reported, cognitive, and affective evaluation of one's
life. SWB can be used to evaluate and direct policy and planning. 
Ultimately, public policy 
should  make people happy. This idea, to make people
happy through policies and planning, is not only the author's or Jeremy
Bentham's idea,\footnote{Jeremy Bentham (1748-1832), a British  philosopher, is
  a founder of moral utilitarianism--an idea that what  makes us happy is the
  right thing to do. It follows, according to  this doctrine,  that the role of
  the public policy should be to maximize the happiness, that is,  governments
  should produce  the  greatest happiness for the greatest number.} but it is
also advocated  by leading social scientists such as  Amartya Sen
\citep{stiglitz09al}.   There is a  need to
study happiness simply because
it is happiness and not  income or consumption that
is the ultimate goal of broadly understood development \citep[e.g.,][]{stiglitz09al,diener12, easterlin13}. 
%a010
 % Social scientists take an interdisciplinary perspective on happiness \citep{ashkanasy11,
 %  blanchflower11, judge11,ballas13,schwanen15}.
This study draws on sociology,
psychology, and geography to 
%a010
  investigate the link between size of a place and happiness. 



Claude S Fischer, an urban sociologist, asked in  \citeyear{fischer73}, ``Does the likelihood
of an individual  expressing malaise increase with an  increase in
the urbanism of his place of residence (indexed by size of
community)?'' For over 40 years nobody has answered this
question, that is, no study has investigated the effect of ``size of
  community'' (number of people) on happiness. There were only indirect and
  imprecise  answers \citep{fischer73,fischer82,veenhoven94}, often 
  limited to specific groups of people or
  geographic areas \citep{amato92,adams92,adams00s,balducci09, evans09, lu15}.\footnote{Researchers either did not measure happiness, but related concepts (health, income, etc);  or they used small-area or unrepresentative  samples; or crude measures of
    urbanicity, either binary or few categories, e.g., cities, towns, and smaller areas.}  No study has operationalized urbanism with
  population size as in Fischer's question. 
%a9
Likewise, a recent review of literature about 
  happiness and cities \citep{ballas13} does not provide the answer.
%a9
I have also recently started answering Fischer's question indirectly  \citep{aok11a}, but this study is more comprehensive:  
 it uses multiple and elaborated measures of size of a place, explores how
 exactly happiness declines when size of a place increases, and tests directly
 when a place is too big.  
   %TODO comment out
                                %this footnote and send to editor for
                                %blind review
% BLIND FOR  PEER REVIEW \footnote{Excluding
                        % my recent research.} 
%     uses a  continuous measure of 
%   size of the community--and this is precisely  what \citet{fischer73}
% was asking future research to accomplish.
% Also, it is worth noting that there is recent neurological evidence that cities have bad impact
% on our brains \citep{lederbogen11}.
 % I am answering the question here:
The goal of this study is to call attention to the finding that cities are
least happy places, challenge contemporary urbanism and stimulate
further discussion. Results suggest that people are least happy in cities bigger
than hundreds of thousands of people, which  may appear as a very
imprecise answer. This is an approximate range that is estimated from different
regression models, and I do not attempt to narrow it down. I want to be able to
provide a statement about relative happiness across places of different size in
the US in general. American cities are, of course, very different in about everything,
including size at which unhappiness develops. %  American cities have also something in common--all
% large cities are less happy than small towns $<$blind for peer review$>$. %brfsscity cize


\section*{The Concept of Happiness}

% Before discussing the dataset used in this study, a brief overview of
% concept of happiness is provided.
For simplicity,  the terms happiness, life satisfaction, and
Subjective Wellbeing (SWB) are used interchangeably. 
% There are many definitions of happiness across
% different fields.
% Happiness and life satisfaction are  conceptually
% different. Life satisfaction refers to cognition and happiness refers
% to affect. Then, there is also Subjective Well-Being (SWB) that
% encompasses both happiness and life satisfaction.
Ed Diener  defined
SWB as % ``people's
% cognitive and affective evaluations of their lives,'' or in
% little more elaborate words,
``both cognitive judgments of one's life
 satisfaction in addition to affective evaluations of mood and emotions''
\citep[][p. 213]{diener1999_C}. This is almost 
 the same as the definition by Ruut Veenhoven (\citeyear[p. 2]{veenhoven08}), another key happiness
 scholar: ``overall judgment of life that draws on two sources of information: cognitive comparison with standards of the good life (contentment) and affective information from how one feels most of the time (hedonic level of affect).'' 
 Some scholars use 'life satisfaction' to  refer to cognition and 'happiness' to
 refer to affect \citep[e.g.,][]{dorahy98etal}. This dichotomy is not pursued here, because
   there is only one survey item, which likely captures
   mostly life satisfaction but also happiness to some
   degree. Therefore the SWB definition by 
 \citet{diener1999_C} and \citet{veenhoven08} seems most appropriate, and
 again, SWB is used interchangeably   with ``happiness.''

The  happiness measure, even though self-reported and subjective,
 is  reliable  (precision varies),
 % per ruut asa13 slides
 valid \citep{ditella06m,myers00}, % . The survey-based life
 and  correlated with similar objective
 measures  of wellbeing  such as brain waves \citep{layard05}.
  Unhappiness strongly
  correlates with suicide incidence and mental health problems
  \citep{bray06g}. 
Finally, to avoid confusion, this 
  study investigates general/overall happiness, not a domain-specific happiness
  such as neighborhood or community satisfaction. 
% \citep{krueger07s}
% ``While reliability figures for subjective well-being measures are lower than those typically found
% for education, income and many other microeconomic variables, they are probably sufficiently high to
% spport much of the research that is currently being undertaken on subjective well-being,
% particularly in studies where group means are compared (e.g., across activities or demographic
% groups).''
% \citep{clark08al}
%  ``A  point  worth  making  is  that  when  asked  to  report  their  level  of  happiness,  life  satisfaction  or  well-being  in 
% surveys, only a small minority of respondents do not provide an answer (less than 1\% of respondents in the BHPS 
% or GSOEP). The concept of happiness is intuitively understood by almost everyone.''
% \citep{clark08}
% ``An  early  example  is  Freeman  (1978),  which 
% showed that current job satisfaction is a better predictor of future quits than are current wages. 
% More  generally,  satisfaction  scores  have  been  shown  to  predict:  life  expectancy,  morbidity, 
% productivity,  quits,  absenteeism,  unemployment  duration,  and  marriage  duration.  In  this  case, 
% subjective  well-being  must  be  to  some  extent  comparable  between  individuals.  Were  each 
% individual  to  answer  idiosyncratically,  it  would  not  be  possible  to  predict  the  distribution  of 
% future observable behaviours across individuals.'' For review see \citep{clark08al}

% Still, many scholars voice their skepticism about happiness. This is especially
% troubling if such criticism comes from scholars who are only just beginning to
% study happiness themselves, such as Angus Deaton,\footnote{For instance, see this
%   interview \url{https://www.youtube.com/watch?v=tz3D-36RuLo}. Also, at a recent
% conference I have heard happiness researchers voicing concerns over usefulness
% of happiness research just because Angus Deaton was not convinced to it. Just
% because Angus Deaton is very knowledgeable in some areas of economics does not
% mean that he knows much about happiness.} %hamilton conference
%  while happiness
% have been studies scientifically for decades. This is just one example, but in
% general it is popular among
% economists to disparage happiness research. Curiously, economists do study
% happiness at the same time, and they
% even sometimes use term ``happiness economics,'' yet most of them do not treat it
% seriously. % I propose then not to treat seriously people who do not treat
% % seriously what they study.
%  There are of course exceptions--notably, Richard
% Easterlin and Andrew Oswald are both economists and serious happiness
% researchers. On the other hand, topic of happiness is treated seriously in psychology, where it
% naturally belongs.

Happiness, as  any measure, has some limitations. Much of happiness is hereditary or due to genes \citep{lykken96t}.   We are on so called ``hedonic
treadmill''--we adapt or get used to both fortune and misfortune, even very major
events such as winning millions in a lottery or loosing limbs in an accident \citep{brickman78cj}. 
 Happiness is affected by various comparisons
\citep{michalos85}--whatever happens to other people (and whatever happened to
ourselves in the past) affect  our current happiness. These issues, however,
are not critical. 
Recently, \citet{diener09} has provided a good
discussion of why potential problems with happiness are not serious
enough to make it unusable for interventions, planning, and public policy. % --see especially  chapter 6
 % More support for validity, reliability and precision is provided by 
 %  \citet{myers00,ditella06m,layard05,bray06g,sandvik93ds,clark08al}.    
%LATER can say as per that philosophy lady as i dd in ls_car that happiness from %burger and from child birth is not the same! and also carver that happiness %simply signals completion of a task!


\section*{Urbanism: Happy or Unhappy?}

 Social scientists say or imply that happiness has its place in big cities. While there is
no evidence to support it, the proposition that people are happy in the
city has been assumed by many to be a self-evident truth,
an axiom. 
 Notable enthusiasts of happy city living are Jane Jacobs in her classic
``The Death and Life of Great American Cities'' (\citeyear{jacobs93}),
and more recently Ed Glaeser in  ``Triumph of the City: How Our Greatest
Invention Makes Us Richer, Smarter, Greener, Healthier, and
Happier'' (\citeyear{glaeser11}):  

\begin{quote}
There is a myth that even if cities enhance prosperity, they will make
people miserable. But people report being happier in those countries
that are more urban. In those countries where more than half of the
population is urban, 30 percent of people say they are very happy and
17 percent say they are not very or not at all happy. [...] Across
countries, reported life satisfaction rises with the share of the
population that lives in cities, even when controlling for the
countries' income and education.
\end{quote}
This is  ecological fallacy.  People are
happier in more urbanized countries than in less urbanized countries, but it  
 does not mean that people are happier in cities than in smaller
 areas. More urbanized countries are simply richer, healthier, better governed,
 etc, than less urbanized
 countries. This is one of the most agreed upon findings in
 happiness literature: In a cross-section of countries, people are
 happier in more developed areas \citep[e.g.,][]{aok13liavbility}. 
 Urbanization leads to economic growth, but  economic growth does
 not lead to much happiness over time, especially in developed countries \citep{easterlin13}. 
 I have discussed these issues in depth elsewhere \citep{aokCityBook15}.
   \citet{glaeser11} continues:
\begin{quote}
Cities and urbanization are not only associated with greater material
prosperity. In poorer countries, people in cities also say that they
are happier. Throughout a sample of twenty-five poorer countries,
where per capita GDP levels are below \$10,000, where I had access to
self-reported happiness surveys for urban and non-urban populations, I
found that the share of urban people saying that they were very happy
was higher in eighteen countries and lower in seven. The share of
people saying that they were not at all happy was higher in the
non-urban areas in sixteen countries and lower in nine.
\end{quote}
This statement is either due to  unhappy sampling or cherry picking. 
%a1
Indeed, people are
happier in cities in developing countries as shown by
\citet{aokcities}, but in  rich countries, it is the other way
round--the bigger the area, the lower the happiness \citep{aokCityBook15}. The fact that people
are happy in cities in poor countries is arguably not due to cities' ``greatness.'' It may  be simply that life outside of the city in a
poor country is
unbearable and lacking the necessities, such as food, shelter, sanitation,
and transportation. Quality of life or so called ``livability'' differs greatly between urban and
rural areas in developing countries. For instance, urbanites enjoyed three times
higher income and consumption than rural dwellers in China in 2000
\citep{knight06}. Simply, the urban happiness in developing countries
is rather due to unfavorable conditions outside of cities, not due to virtues of
cities. Cities have few virtues, but many vices \citep{wirth38,park15}.
%a1

In addition to the positive side, the affirmation of city life, there is
a negative side, a condemnation of suburban life--contemporary
scholars also build their argument in favor of city living by arguing
against suburban living. There are many studies dedicated to condemnation of
suburban sprawl \citep{kay97,duany01,dreier05,kunstler12,ewing97, frumkin02,
  ewing03}. There are problems associated with sprawl, but scholars usually 
overlook that people are least happy in cities. There is a clear discord--residents
prefer \citep{fuguitt90,fuguitt75} and  are happier in
small areas, but academics, policy makers, and planners promote cities as
``better'' places. In addition,  
 enthusiasts of city living, proponents and opponents of suburban living  miss
the point that people are happiest neither in cities nor in suburbs, but
in small towns and villages. 

Of course, the negative side of the city living has been noticed long time ago--it was succinctly summarized by \citet{wirth38}
over 70 years ago. According to \citet{campbell76etal}, the first major
quantitative study finding happiness to be lower in cities was \citet{gurin60}.
Many studies followed,  notably by  Claude S Fischer
(\citeyear{fischer82,fischer76,fischer75,fischer73,fischer72}).
%a8
 The literature has argued many city problems. 
Cities exemplify a mechanical society without much community \citep{tonnies57},
they overstimulate \citep{simmel03} and are unhealthy to  the brain
\citep{lederbogen11}. Cities intensify vice, crime, and conspicuous consumption;
labor specialization and industrialization that accompany urbanization kill
spontaneity and joy \citep{park15,park84,veblen05a,veblen05b,wirth38}. Cities
are 
full of pollution, dirt, noise, crowding, poverty, beggary, and monotony of the
buildings (senseless chunks) \citep{wirth38, white77}.
In short, we know that cities are in many ways incompatible with human flourishing and wellbeing.   
%a8
This study is the first, however, to show precisely how subjective well being
declines with size of a place.   

%recent data supports  the urban malaise hypothesis. % BLIND FOR  PEER
                                % REVIEW \footnote{Again, Excluding my recent research.}

Sometime ago, there was a discussion about an optimal size for a place--the idea
being that it is efficient to have many people living together, but beyond some
point, further concentration does not make sense. As a place
grows, so do benefits and they grow faster than costs.  At some point, however,
 costs start to grow faster and there is a point when costs outweigh
benefits. This line of research
\citep[e.g.,][]{singell74,elgin75} discontinued a few decades ago. For more recent review see \citet{capello00}, which
concluded that it is difficult to calculate an optimal city size because every
city is different. This study agrees with such perspective, and hence, a wide
range offered in conclusion: city is too big when it exceeds hundreds of
thousands people. A dataset containing both happiness and detailed size of a place
variables is needed to investigate the relationship.


\section*{Data}

This study uses 1972-2012 US General Social Survey (GSS) from \url{http://www3.norc.org/gss+website}. For definitions of size of a  place and happiness, frequency figures and coding of all variables see appendix A.
The outcome of interest (dependent variable) is
 happiness.  
 The main explanatory (independent) variable
is size of a place. 
Size of a place is defined  in three ways to show that the
results are robust to the definition. 
First, it is  deciles of population size (\textsc{size}). Deciles are used to
investigate if there are any nonlinear effects on {happiness}.  Two other variables are used under their original GSS names: \textsc{xnorcsiz} and
\textsc{srcbelt}. Both variables categorize places  into metropolitan
areas, big cities, suburbs, and  unincorporated areas.
 The advantage of
\textsc{size} is that it allows to calculate a happiness gradient by
exact size of settlement. \textsc{Xnorcsiz} and \textsc{srcbelt} take
into account the fact that populations cluster at different densities, e.g., suburbs are less dense than cities. GSS does not provide 
  density variable.

 The choice of control variables for regressions is based on the literature
 (see extended discussion in the  appendices). % However, I do not intend to paraphrazie the voluminous happines
 % literature here--again, for a good introduction to the topic see: \citet{diener00,stiglitz09al,diener12, helliwell12, easterlin13}.
Those variables are controlled for,  because they predict happiness as
  shown in the literature, but they are not of direct interest to this study and
  hence they are only  discussed briefly. 
% 
What makes people happy? Young and old people are happy
 \citep[e.g.,][]{teksoz}--large cities may attract the young and
repel the old. Income boosts happiness and
unemployment depresses it
\citep[e.g.,][]{ditella01moa,ditella01mob,ditella06m}. Being married
 boosts happiness \citep[e.g.,][]{myers00,diener04s}.
 Blacks are less happy than whites 
\citep[e.g.,][]{aokcities,aok11a}, and they are traditionally
concentrated in cities \citep{jargowsky97}.   Cities are
predominantly Democrat \citep{hansonCityJournalautumn15}, and 
 Democrats are less healthy \citep{subramanian09a} and less happy
\citep{jost09al,napier08,jost03al}. % {For instance see
%   \url{http://www.forbes.com/sites/markhendrickson/2012/11/15/what-explains-the-partisan-divide-between-urban-and-non-urban-areas/},
% % \url{http://m.theatlanticcities.com/politics/2013/02/what-makes-some-cities-vote-democratic/4598/},
% % \url{http://www.theatlanticcities.com/politics/2012/11/political-map-weve-been-waiting/3908/},
%  and \url{http://www.theatlantic.com/politics/archive/2012/11/red-state-blue-city-how-the-urban-rural-divide-is-splitting-america/265686/}}
  To better account for ideology and political values,   liberalism is also
 controlled for. 
 There are a few other important variables, such as health  and social
 capital. They are missing for many respondents in GSS,  and their
 discussion is postponed to appendix B, where robustness checks are
 covered.  

\section*{Results and Discussion}

This empirical section explores how happiness declines when  place
 becomes bigger, and it attempts to find when a city is too large. There is likely to be a point at which advantages of size are overcome by
costs. 
%  Again, this
% research is inspired by urban sociologist
% Claude S Fischer, who asked future research to do the following: ``The very largest cities may be too
% large and may contribute modestly to unhappiness. Future research
% should be directed at replicating this finding and establishing a
% point of inflection in the size of continuum.'' 
 % The following graphs show  happiness by
 % size of a place using three operationalizations.
 Figure \ref{desSta} shows 
average happiness by categories of three size variables. Panel (a) shows
that there is a happiness gradient by
size of the settlement. As \citet{fischer73}
suggested, it is clearly the biggest cities that are much less happy
than all smaller places.  In other words, the largest decline
in happiness is observed for the largest cities. The gradient is smooth (monotonic) except a bump at  3rd decile.
  Panel (b) shows the same
pattern:  the largest cities ($>250k$) are
least happy, and there is a small happiness gradient for other areas, except a
bump at 2.5-50k. Sizes  on y axis
  are not necessarily ordered in ascending order and unincorporated areas (both
  medium and large) are quite happy.
 Panel (c) confirms the pattern 
using yet another definition of size: 
 the 12 largest MSA (Metropolitan Statistical Areas),  are least happy, followed by the largest 13-100 MSA 
and there is less difference in happiness among smaller areas.
 These figures show that unhappiness intensifies at
 hundreds of thousands of people. There are only about 60 cities in the US with
 a population larger than 300 thousand. These unhappy cities are large, much larger than census definition of an urban area (2.5 thousand), and
  larger than  a central place (50 thousand).  A
  person does not have to give up city living to be happy, she just
  needs to avoid the biggest cities. There are usually only one or two
  such cities per state.% \footnote{Maybe that's the suburbs appeal: escape big
 % city malaise, yet stay close to its amenities.}


\begin{figure}%
    \centering
    \subfloat[]{{\includegraphics[width=2.2in]{happy_d_size.pdf} }}%
%    \qquad
    \subfloat[]{{\includegraphics[width=2.2in]{happy_xnorcsiz.pdf} }}%
    \subfloat[]{{\includegraphics[width=2.2in]{happy_srcbelt.pdf} }}%
    \caption{Average happiness by categories of size variables. 95\% confidence
      intervals shown.}%
    \label{desSta}%
\end{figure}

\citet{fischer73} was correct by suggesting that at some point the city
may be too big. The biggest cities are clearly the least happy, and there
is less difference among smaller areas, although the happiness
gradient persists. In general, the smaller the place, the happier the people.  The patterns from the above
figures hold when controlling for other relevant predictors of happiness.
Figures \ref{postgr3_d_size}, \ref{postgr3_xnorc_d_size}, and
\ref{postgr3_srcbelt_d_size} show predicted probabilities for happiness categories.\footnote{ The corresponding regressions  are in appendix B in
columns marked a1. The subsequent columns are the robustness checks
using more covariates.}
%
%LATER think more abt concaivity...
Apart from an interesting dip in happiness at around 2.5-10 thousand
people (figures \ref{postgr3_d_size} and \ref{postgr3_xnorc_d_size}),
 the happiness gradient persists. Perhaps,  such
  places already strip residents of their contact with nature that is plentiful
in villages and open country. On the other hand, they are not big
enough to provide residents with basic city amenities. 

  
 % He was not right taht it is ``modest'' as i will show later. And it
% happens earluier than he suggested, and biiger drop little later than
% his 100k in teh cantril pictires, but not his fault--there were no data...

% fisher was right that it is the very biggest cities. And here is the
% inflection point; otherwise it is bumpy at the beginning. in regression with
% d\_siz categories 9 and 10 (and later d20\_siz) categories 17-20;
% and both correspond to the same of cities bigger than 200k where a
% substantial drop happens. But while bumpy (postgr3) the drop strats
% happening earlier --as early as 6, which is bigger than 25k or 7
% bigger than 40k--and the the same was already clear from figure \ref{happy_d_size}.

% so there is no inflection point--the bigger the more unhappy--the
% quadratic regression may give an impression as if there is an
% inflection point between 4 and 5 mil, and so the postgr3 predicted
% values; but that's an artifact of the fact that people in the very
% biggect areas (over 5 mil) may be very slightly happier than those in
% little smaller areas; but the standard errors are wide (qfit) and in
% addition these are not reliable because there are very few cities with
% over 5 mil of population; and the two sets of regressiosn with 20 and
% 50 dummies (about 1,000 people per group(actualy fewer than that due
% to missing obs)) show that indeed the bigger the area the less happy
% are the people all along.

% very interesting \ref{postgr3_xnorc_d_size}  here the biggest jump
% happens in 1 to 2 250k vs 50-250k; but also, which corresponds to the
% same (just unincorporated) 5 vs 6 large vs medium unincorporatyed big
% city! very cool--consistent! 

% The other thing is that in both graps it is clear that by one
% definition the very smallest areas are happy, or by other definition,
% the country; and then also in both cases there is not that much
% deifference for the middle categorires.

\begin{figure}[H]
  \includegraphics[height=3.5in]{mm-d_size.pdf}\centering
  \caption{Predicted happiness probabilities by deciles of city size. X denotes
    a significant (.05 level)  reverse Helmert contrast. Appendix B provides details.}\label{postgr3_d_size}
\end{figure}

\begin{figure}[H]
  \includegraphics[height=3.5in]{mm-xnorcsiz.pdf}\centering
  \caption{Predicted happiness probabilities by {\textsc{xnorcsiz.}} X denotes a
    significant (.05 level) reverse Helmert contrast. Appendix B provides details.}\label{postgr3_xnorc_d_size}
\end{figure}

\begin{figure}[H]
 \includegraphics[height=3.5in]{mm-srcbelt.pdf}\centering
\caption{Predicted happiness probabilities by {\textsc{srcbelt}.} X denotes a
  significant (.05 level) reverse Helmert contrast. Appendix B provides details.}\label{postgr3_srcbelt_d_size}
\end{figure}

Importantly, it is only the largest places, where hundreds
of thousands of people live, that are significantly less happy than smaller
places.\footnote{5-8k category in figure \ref{postgr3_d_size} is an exception.} Places with a
significant reverse Helmert contrasts are denoted with ``X'' in figures above:
they are significantly less happy than all smaller places on average--for
details see appendix B.   
%
% I DONT KNOW ABOUT THIS!!! I RAN WITH BASE CASE 8th cat and not significant!
%  Is the magnitude of the effect  practically substantial? It is easy to
%  interpret plotted predicted probabilities. If we compare a person from a large
%  city to a person from slightly smaller city, say 200 v 100 thousand, using a  conservative estimate of 0.25\% difference in happiness probability, every 400th
% person would be ``pretty happy'' in smaller place as opposed to being ``not too happy'' in larger place. Say, that 1\% of Americans (3.2m) were in a slightly smaller city 
% (by one decile), then about 8,000 would be ``pretty happy'' instead of ``not too
% happy.''  
% %
% This is not something we should disregard.
 While the differences across categories are not very large,
even a finding of no effect would be worth reporting. Again, for some  reason, it is
 fashionable\footnote{I am using a non-scholarly word ``fashionable''
   on purpose, to show that a view that people are happy in cities
   is non-scholarly.
%
 I can speculate that public policy and public
   administration scholars imply that cities are happy because that is where
   their work is focused. Most people live in cities, and most
   policies are crafted for urban areas. % Also, that is where scholars
   % work--most  US universities are located in urban areas. % Also, they are
   % % upset that people flee to suburbs, and maybe that is why they want to argue
   % %  that life in a city is happy.

} in social science to imply that people are happier in the cities than elsewhere
\citep[e.g.,][]{jacobs93, glaeser11}.  

% Of course, probabilities differ across size categories and size
% variables as shown in graphs above, and the above scenario is only approximate
% and ``on average.'' 


%  \citet{fischer73} says ``all these
% differences are slight, however, and the general picture that emerges
% is one of minimal relationship.'' It is not minimal. The difference of
% XXX is equivalent to making XXX people  ``not too happy'' from
% ``very happy'' for an average American city of XXX if those same
% people could live in city of XXX. Another example: if all Americans moved
% to a place smaller by 100,000 people, then everyone's happiness would
% increase by $.0000136*100$ coefficient times 100\footnote{It is based on
%   coefficient on size from OLS regression like one in column a1} on scale from 1 to 3, which is a big
% effect--it is equivalent to making  $314000000*.0000136*100$ number of
% americans times the coeff times 100 for a decreasy by 100,000 in size:
% almost half a million people from ``not too happy'' to ``pretty
% happy.''
% TODO think about this again if this is right


\section*{Conclusion}

What is the message of this study? % First, it is not only that
% people prefer smaller areas as \citet{fuguitt90,fuguitt75} have shown,
% but  people are also happier in smaller areas. Second, it supports
% Fischer's hypothesis (\citeyear{fischer73}) that 
The big cities are too big: probability of being unhappy increases significantly
 when city size exceeds
 hundreds of thousands of people.  This is close to estimate by \citet{chen15} for China. In China,
 people become less happy in places larger than 500 thousand (Chinese are
happiest in places of 200-500 thousand, not in the very smallest places as in the US).



Importantly, these results counter the contemporary common
wisdom in academia. Much of public policy/administration  and
economics assumes or argues that cities are ``better'' than  smaller areas, but
people are happiest in smallest areas despite  that these places seem
largely forgotten by academics, policy makers, and businesses. It is the largest
cities that are considered ``the best.''
 \citet{savitch10} proposed the ``4 C'' measure of city greatness: Currency conveys the
 unique attributes of a city's fundamental values and its ability to form, lead
 or dictate the temper of the times. Cosmopolitanism entails an ability to
 embrace international, multicultural or polyethnic features. Concentration is
 defined by demographic density and productive mass. Charisma is based on a
 magical appeal that generates mass enthusiasm, admiration or reverence. By this
 definition largest cities are greatest, and counterintuitively, as this study
 argues, these places are least happy of all.

By condemning cities, this study does not
defend suburbs. On the contrary, results show that both cities and suburbs are much
 less happy than  towns and villages.  This finding needs to be stressed because
 residential debate is usually only about
 cities v suburbs. Smaller areas are forgotten. The omission of
 smaller places from residential debate is arguably due to the
 benefits of
 agglomeration economies and economies of scale found in metropolitan
 areas that make smaller places irrelevant. But this may change
 soon. The information and creative sectors are becoming increasingly
 important--jobs that are not creative or extremely complex will be
 automated \citep{brynjolfsson14}. Much of the creative and complex tasks are done on a
 computer. Thanks to the  internet these tasks can be done from anywhere,
 including small towns and villages. Especially in rich countries,
 transportation networks and online communication enable people to live in
 smaller areas.   Still, it
   is not entirely true that ``the world is
   flat''\citep{freidman05}. Indeed, ``world is spiky''
   \citep{florida08}--place matters and it will matter in the
   foreseeable future. Yet having to live in the
   metropolitan area (city or suburb) or having to commute
   everyday to one's workplace will probably be less necessary in the
   future. 
% : people want to be
 % close to nature.\footnote{Being close to friends and family is of
 %   course important, too. I proxied social capital with trust in my models--and
 %   the bigger the area the less trust--it appears that as
 %   \citet{wirth38} was saying that human contact in the city is
 %   superficial and transitory.}

 Findings of this study can be used by policymakers in the spirit of
 \citet{stiglitz09al}, who  urged policymakers to pay attention
 to happiness.  America is (sub)urbanizing,  yet people
 are unhappy in cities (and in suburbs) as compared to smaller areas.
 Tax/subsidy incentives to promote what makes people happy and
 healthy make sense. City living is not only unhappy but also unhealthy \citep[][]{lederbogen11}.\footnote{To be fair, while city living is more
     unhealthy than rural living, it is rural living that generates more
     pollution and uses more resources \citep{meyer13}. Still, much of it is
     due to overconsumption--we could consume less and live in less dense areas
     more sustainably.} Sprawl, however, is not a solution. Ideally, people should
   live in villages, towns, and small cities as they used to for millennia, but
   unfortunately due to overpopulation, metropolis will stay with us for the
   foreseeable future.
 % On
   % the contrary, people should live in smaller locations only if they
   % can work there or work remotely only with occasional commute to a
   % large city.
   % The assumption is that people are happy in small areas
   % because there is plenty of nature, and it should stay that way,
   % that is, there cannot be too much residential development because
   % it defeats the purpose. Houses should be as small as possible and
   % roads narrow.
 Fundamentally, results from this study are important for everyone: if you want to be
happy, avoid large cities. % Again, most Americans intuitively know
% that--they prefer smaller areas \citep{fuguitt90}, and they should
% trust their instincts.  

% Arguably, although not directly tested in the present study, people want to be
% close to nature and are happy in natural settings.  Adam Smith observed:``The beauty of the country, besides, the pleasures
% of a country life, the tranquility of mind which it promises, and
% wherever the injustice of human laws does not disturb it, the
% independency which it really affords, have charms that more or less
% attract everybody'' \citep[][:IIIi]{smith76}.  Animals, plants,
% landscapes, and wilderness benefit our wellbeing
% \citep{frumkin01b}. Exposure to nature produces positive emotions and
% positive affect \citep{mayer09}. Nature helps recover from pre-existing stress, immunizes and protects
% from future problems \citep{pretty12}. 
% For a literature review of nature's benefits see
% \citet{maller06, pretty12, aokCityBook15}.
% %  Suburb-lovers miss that
% % point, too.
% % LATER may google around that suburbs have fake nature 
% % This is one of the possible mechanisms explaining urban unhappiness--there is
% % simply less nature in cities than elsewhere. 
% % %a7
% % This hypothesis is not directly tested,
% % however. There is simply no variable to measure nature in the dataset used
% % here. An indirect test is conducted using city variables--it is assumed that the
% % larger the place, the less nature. Of course, in such case lack of nature
% % confounds with other city attributes, but still based on the above review, it is
% % reasonable to argue that lack of nature contributes to city
% % unhappiness.  

% % %a7


There are several limitations or directions for  future research. First, this is an
observational or correlational study, and as any such study, it cannot
claim causality. 
%a5
Testing and in-depth discussion of the underlying causal mechanism is beyond the
scope of this study. Furthermore, a true experimental design usually is
 not possible--one cannot assign randomly people to cities and villages. Last
 but not least, experimental designs typically suffer from low or non-existent external
 validity, and hence, they are not an absolute and obvious improvement over
 correlational studies \citep[e.g.,][]{pawson97}. % In what follows, some other
 % limitations are discussed; and also a case is made why these other limitations
 % are not critical either.

This study is about relative happiness in American cities, and specifically
about relative happiness in big cities compared to happiness in alternative settlements.
As mentioned earlier, specificity is sacrificed 
 to gain generality. Another study could do the
opposite and focus on a set of specific cities, or a specific
region. That could
be achieved using restricted use/geocoded version of GSS or the Behavioral Risk
Factor Surveillance System.
%
Exploring interactions of happiness predictors with city size may be an interesting topic for the future
  research, but it is beyond scope of this study, which focuses on elaboration of city measurement.
% 
And it would be worthwhile to explore why there is
a happiness dip at around 2.5-10 thousand of people. Is it, as speculated
earlier, that these places are big enough to kill the contact with nature, and
not big enough to provide the city amenities?  Another interesting topic
would be to explore how a distance from large city affects a
person. We know that Americans want to live in sparsely populated
areas that are close to a major city \citep{fuguitt90,fuguitt75}.
%
Finally, larger cities are likely to have  more social polarization:
income/wealth inequality, residential segregation, and so
forth. % These are  important factors to consider; however it is the
% limitation of the (publicly available) GSS dataset: geographical
% location of respondent cannot be identified.
 Use of geocoded
data for the study of urban malaise will be an important contribution 
and a great topic for future research. % It would allow for the
% separation of  the effects of city size per se from  the effects of compositional differences of bigger cities.



% \begin{figure}[H]
%  \includegraphics[height=2in]{../out/gov_res_trust.pdf}\centering\label{gov_res_trust}
% \caption{woo}
% \end{figure}



% %table centered on decimal points:)
% \begin{table}[H]\centering\footnotesize
% \caption{\label{freq_im_god} importance of God}
% \begin{tabular} {@{} lrrrr @{}}   \hline 
% Item& Number & Per cent   \\ \hline
% 1(not at all)&    9,285&  9\\
% 2&    3,555&        3\\
% 3&    3,937&        4\\
% 4&    2,888&        3\\
% 5&    7,519&        7\\
% 6&    5,175&        5\\
% 7&    6,050&        6\\
% 8&    8,067&        8\\
% 9&    8,463&        8\\
% 10&   52,385&       49\\
% Total&  107,324&      100\\ \hline
% \end{tabular}\end{table}


% % Define block styles
% \tikzstyle{block} = [rectangle, draw, fill=black!20, 
%     text width=10em, text centered, rounded corners, minimum height=4em]
% \tikzstyle{b} = [rectangle, draw,  
%     text width=6em, text centered, rounded corners, minimum height=4em]
% \tikzstyle{line} = [draw, -latex']
% \tikzstyle{cloud} = [draw, ellipse,fill=black!20, node distance = 5cm,
%     minimum height=2em]
    
% \begin{tikzpicture}[node distance = 2cm, auto]
%     % Place nodes
%     \node [block] (lib) {liberalism, egalitarianism, welfare};
%     \node [block, below of=lib] (con) {conservatism, competition, individualism};
%     \node [cloud, right of=con] (ls) {well-being};
%     \node [block, below of=ls] (cul) {genes, culture};
%     \node [b, left of =lib, node distance = 4cm] (c) {country-level};
%     \node [b, left of =con,  node distance = 4cm] (c) {person-level};
%     % Draw edges
%     \path [line] (lib) -- (ls);
%     \path [line] (con) -- (ls);
%     \path [line,dashed] (cul) -- (ls);
% \end{tikzpicture}


\section*{Appendix A: GSS survey items: size of settlement and
happiness.}
\label{app_a}

Dataset: General Social Surveys, 1972-2006 [Cumulative File]

{\scriptsize
\begin{verbatim}
Variable size : SIZE OF PLACE IN 1000S
Literal Question
Size of Place in thousands
A 4-digit number which provides actual size of place of interview
(Cols. 166-169). Remember when using this code to add 3 zeros. Listed
below are the frequencies for gross population categories.

Descriptive Text
This code is the population to the nearest 1,000 of the smallest civil
division listed by the U.S. Census (city, town, other incorporated
area over 1,000 in population, township, division, etc.) which
encompasses the segment. If a segment falls into more than one
locality, the following rules apply in determining the locality for
which the rounded population figure is coded.
If the predominance of the listings for any segment are in one of the
localities, the rounded population of that locality is coded.
If the listings are distributed equally over localities in the
segment, and the localities are all cities, towns, or villages, the
rounded population of the larger city or town is coded. The same is
true if the localities are all rural townships or divisions.
If the listings are distributed equally over localities in the segment
and the localities include a town or village and a rural township or
division, the rounded population of the town or village is coded.
The source of the data is the 1970 U.S. Census population figures
published in the PC (1) -A series, Tables 6 and 10. For cases from the
1980 and 1990 frames analogous tables from the 1980 and 1990 Censuses
were used. See Appendix N for changes across surveys.

\end{verbatim}

\begin{verbatim}
Variable xnorcsiz : EXPANDED N.O.R.C. SIZE CODE
Literal Question
NORC SIZE OF PLACE
PostQuestion Text
a A suburb is defined as any incorporated area or unincorporated area
of 1,000+ (or listed as such in the U.S. Census PC (1)-A books) within
the boundaries of an SMSA but not within the limits of a central city
of the SMSA. Some SMSAs have more than one central city, e.g.,
Minneapolis-St. Paul. In these cases, both cities are coded as central
cities.
b If such an instance were to arise, a city of 50,000 or over which is
not part of an SMSA would be coded '7'.
c Unincorporated areas of over 2,499 are treated as incorporated areas
of the same size. Unincorporated areas under 1,000 are not listed by
the Census and are treated here as part of the next larger civil
division, usually the township.
The source of the data is the 1970 U.S. Census population figures
published in the PC (1) -A series, Tables 6 and 10. Practically, the
codes '6' and '10' are localities not listed in Table 6 (Population of
Incorporated Places and Unincorporated Places over 1,000). For the
1980 frame cases analogous tables from the 1980 Census were used.

Descriptive Text
See Appendix T, GSS Methodological Report No. 4.
\end{verbatim}

\begin{verbatim}
Variable srcbelt : SRC BELTCODE
Literal Question
SRC (SURVEY RESEARCH CENTER, UNIVERSITY OF MICHIGAN) NEW BELT CODE
Descriptive Text
The SRC belt code is described in Appendix D: Recodes. See Appendix N
for changes across surveys. See Appendix T, GSS Methodological Report
No. 4.

Intent of Recode

The SRC belt code (a coding system originally devised to describe
rings around a metropolitan area and to categorize places by size
and type simultaneously) first appeared in an article written by
Bernard Laserwitz (American Sociological Review, v. 25, no. 2, 1960),
and has been used subsequently in several SRC surveys.
Its use was discontinued in 1971 because of difficulties particularly
evident in the operationalization of "adjacent and outlying areas."
For this study, however, we have revised the SRC belt code for users
who might find such a variable useful. The new SRC belt code utilizes
"name of place" information contained in the sampling units
of the NORC Field Department.

Method of Recode

This recode assigns codes to the place of interview. City
characteristics were determined by reference to the
rank ordering of SMSAs in the Statistical Abstract of the United
States, 1972, Table 20. Suburb characteristics
were determined by reference to the urbanized map in the U.S. Bureau
of the Census, 1970 Census ofPopulation, Number of Inhabitants, Series
PC (1) -A. The "other urban" codes were assigned on the basis of
county characteristics found in Table 10 of the 1970 Census of
Population, Number of Inhabitants. For cases
from the 1980, 1990, and 2000 frames analogous tables from the 1980 or
1990 Census were used.
\end{verbatim}


\begin{verbatim}
Variable happy : GENERAL HAPPINESS
Literal Question
157. Taken all together, how would you say things are these
days--would you say that you are very happy, pretty happy, or not too
happy?
\end{verbatim}

}

%PUT THIS NOTE, polish and put to /root/author_what_data --ALWAYS
%stick here stuff as i run it!!! maybe comment out later...

\input{hist}

\section*{Appendix B: Regression results and robustness checks.}

\begin{table}[h!]\centering
\caption{Multinomial survey weighted regressions of happiness on size of a
  place. Base category is the middle one, ``pretty happy.'' Odds ratios shown. Size variables are defined
in appendix A.} \label{reg_d_size}
\begin{scriptsize} \begin{tabular}{p{2in}p{.55in}p{.55in}p{.55in}p{.55in}p{.55in}p{.55in}p{.55in}p{.55in}p{.55in}p{.55 in}}\hline
\input{DS2-DS10.tex}
\hline  *** p$<$0.01, ** p$<$0.05, * p$<$0.1
\end{tabular}\end{scriptsize}\end{table}

\begin{table}[h!]\centering
\caption{Multinomial survey weighted regressions of happiness on
  \textsc{xnorcsiz}. Base category is the middle one, ``pretty happy.'' Odds ratios shown.  Size variables are defined
in appendix A.} \label{reg_xnorcsiz}
\begin{scriptsize} \begin{tabular}{p{2in}p{.55in}p{.55in}p{.55in}p{.55in}p{.55in}p{.55in}p{.55in}p{.55in}p{.55in}p{.55 in}}\hline
\input{XN2-XN10.tex}
\hline  *** p$<$0.01, ** p$<$0.05, * p$<$0.1
\end{tabular}\end{scriptsize}\end{table}


\begin{table}[h!]\centering
\caption{Multinomial survey weighted regressions of happiness on
  \textsc{srcbelt}. Odds ratios shown.  Base category is the middle one,
  ``pretty happy.'' Size variables are defined
in appendix A.} \label{reg_srcbelt}
\begin{scriptsize} \begin{tabular}{p{2in}p{.55in}p{.55in}p{.55in}p{.55in}p{.55in}p{.55in}p{.55in}p{.55in}p{.55in}p{.55 in}}\hline
\input{SC2-SC6.tex}
\hline  *** p$<$0.01, ** p$<$0.05, * p$<$0.1
\end{tabular}\end{scriptsize}\end{table}

All models are survey weighted.\footnote{Stata's syntax is: svyset vpsu
  [pw=wtssall], strata(vstrat) singleunit(centered)} Base models (columns a1) include usual happiness predictors. There is an extended discussion of control variables in my earlier research on this topic \citep{aok11a,aokcities}. 
 Columns a1 also include race dummies, which attenuate urban-rural happiness gradient because more
minorities live in bigger areas and they are less happy than
whites. Still, the gradient persists. Columns a2 and a3 add additional controls,
which unfortunately have many values missing. 
%a2
Cities have economic profiles that determine labor market and demographic
characteristics of the population, which may impact happiness.\footnote{I am
  grateful to an anonymous reviewer for this point.} In an effort to
account for this, column a4 adds  dummies for major occupation categories
(International Standard Classification codes): professional,
administrative/managerial, clerical, sales, service, agriculture, production and
transport, craft and technical. The results persist. 
Still, future research may improve on this by using geocoded data and controlling for actual economic profile of each place.
%a2

\citet{fischer73} and \citet{campbell76etal} suggested that it may not be  that the city size
 by itself produces unhappiness, but it is the state of the
American cities,  their current problems (crime, congestion, etc). 
I elaborate models to account for many city problems: crime, lack of
trust, and potential health problems due to urban stress, and the
happiness gradient still persists. The  point of this robustness exercise is to show that 
\citet{wirth38} was right saying that city unhappiness happens because of the
size, not because of other negative things that happen in city (e.g., crime). 

  
% Trust, a proxy for social capital,  is key for happiness. One of the reasons
% why people could be less happy in bigger cities is because as
% \citet{wirth38} argued that human relationships are superficial, but also
% because city is a subcultural mosaic \citep{fischer75}, and that may
% be a reason why generalized trust is lower in cities as I found
% here.
%  Hence,  lack of trust would bias the coefficient on size of  a settlement.
Social support is important for
happiness \citep{diener12}, and it seems that people in cities lack it
\citep{wirth38}. Ideally,  it should be controlled for 
directly, but the trust variable used in this study should pick up some of it.
%a6
Urbanites are less trusting than others. In largest cities, about 32\% of
respondents think that most people can be trusted, while in areas smaller than
190 thousand, 39\% of respondents think that most people can be trusted.
%a6
%
Trust variable attenuates only slightly the negative effect of city on happiness. 

% The columns are in this particular order becaue education does not
% have many vars missingb; trust, health and fear have some; timefrnd
% has only 4000 obs and commute 1,300
How about unhappiness in suburbs?
 Note that  odds ratios on suburbs in first panels of tables \ref{reg_xnorcsiz}
 and \ref{reg_srcbelt} are bigger than 1: suburbanites are about as unhappy as
 urbanites.  

It could be argued that it may not be the size of the cities, but pollution and noise
in cities. Both pollution \citep{mackerron09} and noise \citep{weinhold12}  make
people unhappy. GSS does not measure pollution and noise, and it remains for the future
research to control for them. Yet, it can be also argued that both pollution and
noise are defining characteristics of cities, and hence they are accounted for
in this study by variables measuring city size. To be sure, cities differ in
terms of noise and pollution, but in general the larger the settlement, the more
noise and pollution.

Finally, I have recoded ordinal happiness into a binary
variable by coding  ``not too happy'' as ``0'' and collapsing together
``pretty happy'' and ``very happy'' into ``1''. Results were similar. % , and
% if anything urban unhappiness developed at smaller sizes and with greater
% magnitude than in the ordinal model. Results are available upon request.


%-------------CONTRASTS-------------

Tables \ref{HE-siz}, \ref{HE-xno}, and \ref{HE-src} present reverse Helmert
contrasts.\footnote{I thank anonymous reviewer for bringing this procedure to my
attention. More elaboration can be found in \citet[][p. 187]{mitchell12}.} The idea behind these contrasts is to test whether each successive
 size of a place is less happy than the average for smaller places. That is,
 this procedure provides a test for a threshold effect--to use wording from the
 title--at which point a place is too big. Place is too big when it reaches
 hundreds of thousands of people--such conclusion is based on   tables \ref{HE-siz},  \ref{HE-xno}, and \ref{HE-src}.

\begin{table}[H]\centering\footnotesize
\caption{\label{HE-siz}  Reverse Helmert contrasts for size of a place
  based on specification a1 (base case pretty happy).}
\begin{tabular}{llll|lllll}   \hline 
&\multicolumn{3}{c}{not too happy}&\multicolumn{3}{c}{very happy}\\\hline
 category&  contrast&   p value&     95\% CI& contrast&   p value&     95\% CI\\
        3-4k vs  0-2k  & 0.09&0.33&-0.09, 0.28&-0.10&  0.06& -0.20, 0.01\\
        5-8k vs $<$5-8k& 0.20&0.01& 0.05, 0.36&-0.04&  0.49& -0.14, 0.07\\
      9-14k vs $<$9-14k& 0.03&0.74&-0.12, 0.17& 0.06&  0.25& -0.04, 0.15\\
    15-23k vs $<$15-23k&-0.08&0.29&-0.24, 0.07&-0.04&  0.44& -0.14, 0.06\\
    24-40k vs $<$24-40k&-0.09&0.27&-0.24, 0.07&-0.04&  0.36& -0.13, 0.05\\
    41-78k vs $<$41-78k& 0.08&0.26&-0.06, 0.22& 0.01&  0.82& -0.08, 0.11\\
  79-188k vs $<$79-188k& 0.15&0.02& 0.02, 0.27& 0.03&  0.48& -0.06, 0.12\\
190-622k vs $<$190-622k& 0.15&0.02& 0.02, 0.27&-0.06&  0.20& -0.15, 0.03\\
      .6-8m vs $<$.6-8m& 0.21&0.00& 0.09, 0.33&-0.01&  0.87& -0.11, 0.10\\\hline
\end{tabular}\end{table}

\begin{table}[H]\centering\footnotesize
\caption{\label{HE-xno}  Reverse Helmert contrasts for \textsc{xnorcsiz}
  based on specification a1 (base case: pretty happy).}
\begin{tabular}{llll|lllll}   \hline 
&\multicolumn{3}{c}{not too happy}&\multicolumn{3}{c}{very happy}\\\hline
 category&  contrast&   p value&     95\% CI& contrast&   p value&     95\% CI\\
    lt 2.5k vs  country& 0.08&0.53&-0.17, 0.34& 0.08&0.20& -0.04, 0.21\\             
    2.5-10k vs $<$2.5-10k& 0.01&0.94&-0.21, 0.22&-0.13&0.06& -0.26, 0.01\\             
      10-50k vs $<$10-50k&-0.08&0.45&-0.27, 0.12& 0.03&0.62& -0.09, 0.15\\             
uninc med vs $<$uninc med& 0.12&0.20&-0.06, 0.31&-0.01&0.90& -0.12, 0.10\\             
uninc lrg vs $<$uninc lrg& 0.06&0.59&-0.14, 0.25&-0.07&0.20& -0.18, 0.04\\             
    med sub vs $<$med sub& 0.10&0.14&-0.03, 0.24&-0.04&0.42& -0.13, 0.06\\             
    lrg sub vs $<$lrg sub& 0.17&0.00& 0.06, 0.27&-0.02&0.64& -0.09, 0.05\\             
    50-250k vs $<$50-250k& 0.03&0.65&-0.08, 0.13&-0.00&0.96& -0.08, 0.08\\             
  gt 250k vs $<$gt 250k& 0.21&0.00& 0.11, 0.31&-0.05&0.26& -0.12, 0.03\\\hline        
\end{tabular}\end{table}


\begin{table}[H]\centering\footnotesize
\caption{\label{HE-src}  Reverse Helmert contrasts for \textsc{srcbelt}
  based on specification a1 (base case: pretty happy).}
\begin{tabular}{llll|lllll}   \hline 
&\multicolumn{3}{c}{not too happy}&\multicolumn{3}{c}{very happy}\\\hline
 category&  contrast&   p value&     95\% CI& contrast&   p value&     95\% CI\\
    small urb vs  small   rur &0.07&0.25&-0.05, 0.20& 0.07&0.11&-0.02, 0.15\\
  13-100 sub vs $<$13-100 sub &0.09&0.17&-0.04, 0.21&-0.01&0.86&-0.09, 0.08\\
      1-12 sub vs $<$1-12 sub &0.27&0.00& 0.14, 0.40& 0.02&0.60&-0.06, 0.11\\
  13-100 msa vs $<$13-100 msa &0.10&0.07&-0.01, 0.21&-0.06&0.13&-0.14, 0.02\\
      1-12 msa vs $<$1-12 msa &0.17&0.01& 0.04, 0.30& 0.06&0.33&-0.06, 0.17\\ \hline
\end{tabular}\end{table}
  

  
Size of a place is related to residents' unhappiness. 
 There are, however, several alternative
explanations, factors that correlate with size of a place and affect
happiness, and may bias results. Regression models do not control for them, because
GSS does not contain appropriate variables. 
People in big cities may have higher expectations than people
elsewhere--they may be the so-called ``over-achievers'' who never get
completely satisfied.\footnote{This idea comes from a friend of mine,
  who works for one of the ``Big Four''  business consulting firms in
  a big city and that's what she has observed among her colleagues.} On
the other hand, there are many poor people either stuck (cannot afford
to move) in the cities, or many poor who came to cities looking for a
better life. Much of their misery, however, should be picked up by
income, race, and other variables. 
%a4
Last, but not least, some of the arguably most unhappy urban dwellers are
unaccounted for, that is, cities are in fact even less happy than argued
here. These dwellers include homeless people, addicts, criminals, prostitutes,
and so forth.\footnote{I am grateful to an anonymous reviewer for this point.}
%a4

% Ideally, density of population should be controlled for--it
% correlates with size of a place (but not exactly). 
% Higher density predicts lower trust \citep{helliwell10w}, and  trust is a good predictor of mental health
% \citep[e.g.,][]{putnam01}--this study controls for trust. Third,
  
  
%make sure i have [H] or h! ???
% \begin{table}[H]
% \caption{}
% \centering
% \label{}
% \begin{scriptsize}
% \input{../out/reg_c.tex}
% \end{scriptsize}
% \end{table}

\newpage
%\bibliography{ebib.bib}
%\bibliography{/home/aok/papers/root/tex/ebib.bib}
\begin{thebibliography}{77}
\newcommand{\enquote}[1]{``#1''}
\expandafter\ifx\csname natexlab\endcsname\relax\def\natexlab#1{#1}\fi

\bibitem[\protect\citeauthoryear{Adams}{Adams}{1992}]{adams92}
\textsc{Adams, R.~E.} (1992): \enquote{Is Happiness a Home in the Suburbs?: The
  Influence of Urban Versus Suburban Neighborhoods on Psychological Health.}
  \emph{Journal of Community Psychology}, 20, 353--372.

\bibitem[\protect\citeauthoryear{Adams and Serpe}{Adams and
  Serpe}{2000}]{adams00s}
\textsc{Adams, R.~E. and R.~T. Serpe} (2000): \enquote{Social Integration, Fear
  of Crime, and Life Satisfaction,} \emph{Sociological Perspectives}, 43,
  605--629.

\bibitem[\protect\citeauthoryear{Amato and Zuo}{Amato and Zuo}{1992}]{amato92}
\textsc{Amato, P.~R. and J.~Zuo} (1992): \enquote{Rural Poverty, Urban Poverty,
  and Psychological Well-being,} \emph{Sociological Quarterly}, 33, 229--240.

\bibitem[\protect\citeauthoryear{Balducci and Checchi}{Balducci and
  Checchi}{2009}]{balducci09}
\textsc{Balducci, A. and D.~Checchi} (2009): \enquote{Happiness and Quality of
  City Life: The Case of Milan, the Richest Italian City.} \emph{International
  Planning Studies}, 14, 25--64.

\bibitem[\protect\citeauthoryear{Ballas}{Ballas}{2013}]{ballas13}
\textsc{Ballas, D.} (2013): \enquote{What makes a 'happy city'?} \emph{Cities},
  32, S39--S50.

\bibitem[\protect\citeauthoryear{Berry and Okulicz-Kozaryn}{Berry and
  Okulicz-Kozaryn}{2011}]{aok11a}
\textsc{Berry, B.~J. and A.~Okulicz-Kozaryn} (2011): \enquote{An Urban-Rural
  Happiness Gradient,} \emph{Urban Geography}, 32, 871--883.

\bibitem[\protect\citeauthoryear{Berry and Okulicz-Kozaryn}{Berry and
  Okulicz-Kozaryn}{2009}]{aokcities}
\textsc{Berry, B. J.~L. and A.~Okulicz-Kozaryn} (2009):
  \enquote{Dissatisfaction with City Life: A New Look at Some Old Questions,}
  \emph{Cities}, 26, 117--124.

\bibitem[\protect\citeauthoryear{Bray and Gunnell}{Bray and
  Gunnell}{2006}]{bray06g}
\textsc{Bray, I. and D.~Gunnell} (2006): \enquote{Suicide rates, life
  satisfaction and happiness as markers for population mental health,}
  \emph{Social Psychiatry and Psychiatric Epidemiology}, 41, 333--337.

\bibitem[\protect\citeauthoryear{Brickman, Coates, and Janoff-Buman}{Brickman
  et~al.}{1978}]{brickman78cj}
\textsc{Brickman, P., D.~Coates, and R.~Janoff-Buman} (1978): \enquote{Lottery
  winners and accident victims: Is happiness relative?} \emph{Journal of
  Personality and Social Psychology}, 36, 917--927.

\bibitem[\protect\citeauthoryear{Brynjolfsson and McAfee}{Brynjolfsson and
  McAfee}{2014}]{brynjolfsson14}
\textsc{Brynjolfsson, E. and A.~McAfee} (2014): \emph{The Second Machine Age:
  Work, Progress, and Prosperity in a Time of Brilliant Technologies}, WW
  Norton \& Company, New York NY.

\bibitem[\protect\citeauthoryear{Campbell, Converse, and Rodgers}{Campbell
  et~al.}{1976}]{campbell76etal}
\textsc{Campbell, A., P.~E. Converse, and W.~L. Rodgers} (1976): \emph{The
  quality of American life: perceptions, evaluations, and satisfactions},
  Russell Sage Foundation, New York NY.

\bibitem[\protect\citeauthoryear{Capello and Camagni}{Capello and
  Camagni}{2000}]{capello00}
\textsc{Capello, R. and R.~Camagni} (2000): \enquote{Beyond optimal city size:
  an evaluation of alternative urban growth patterns,} \emph{Urban Studies},
  37, 1479--1496.

\bibitem[\protect\citeauthoryear{Chen, Davis, Wu, and Dai}{Chen
  et~al.}{2015}]{chen15}
\textsc{Chen, J., D.~S. Davis, K.~Wu, and H.~Dai} (2015): \enquote{Life
  satisfaction in urbanizing China: The effect of city size and pathways to
  urban residency,} \emph{Cities}, 49, 88--97.

\bibitem[\protect\citeauthoryear{{Di Tella} and MacCulloch}{{Di Tella} and
  MacCulloch}{2006}]{ditella06m}
\textsc{{Di Tella}, R. and R.~MacCulloch} (2006): \enquote{Some Uses of
  Happiness Data in Economics,} \emph{The Journal of Economic Perspectives},
  20, 25--46.

\bibitem[\protect\citeauthoryear{{Di Tella}, MacCulloch, and Oswald}{{Di Tella}
  et~al.}{2001{\natexlab{a}}}]{ditella01mob}
\textsc{{Di Tella}, R., R.~J. MacCulloch, and A.~J. Oswald}
  (2001{\natexlab{a}}): \enquote{The macroeconomics of happiness,} Warwick
  Economic Research Papers No 615.

\bibitem[\protect\citeauthoryear{{Di Tella}, MacCulloch, and Oswald}{{Di Tella}
  et~al.}{2001{\natexlab{b}}}]{ditella01moa}
---\hspace{-.1pt}---\hspace{-.1pt}--- (2001{\natexlab{b}}):
  \enquote{Preferences over inflation and unemployment: Evidence from surveys
  of happiness,} \emph{American Economic Review}, 91, 335--341.

\bibitem[\protect\citeauthoryear{Diener}{Diener}{2009}]{diener09}
\textsc{Diener, E.} (2009): \emph{Well-being for public policy}, Oxford
  University Press, New York NY.

\bibitem[\protect\citeauthoryear{Diener}{Diener}{2012}]{diener12}
---\hspace{-.1pt}---\hspace{-.1pt}--- (2012): \enquote{New findings and future
  directions for subjective well-being research,} \emph{American Psychologist},
  67, 590--597.

\bibitem[\protect\citeauthoryear{Diener and Lucas}{Diener and
  Lucas}{1999}]{diener1999_C}
\textsc{Diener, E. and R.~E. Lucas} (1999): \enquote{Personality and Subjective
  Well-Being,} in \emph{Well-Being: Foundations of Hedonic Psychology:
  Foundations of Hedonic Psychology}, Russell Sage Foundation, New York NY,
  213--229.

\bibitem[\protect\citeauthoryear{Diener and Seligman}{Diener and
  Seligman}{2004}]{diener04s}
\textsc{Diener, E. and M.~E.~P. Seligman} (2004): \enquote{Beyond Money: Toward
  an Economy of Well-being,} \emph{Psychological Science}, 5, 1--31.

\bibitem[\protect\citeauthoryear{Dorahy, Lewis, Schumaker, Akuamoah-Boateng,
  Duze, and Sibiya}{Dorahy et~al.}{1998}]{dorahy98etal}
\textsc{Dorahy, M.~J., C.~A. Lewis, J.~F. Schumaker, R.~Akuamoah-Boateng,
  M.~Duze, and T.~E. Sibiya} (1998): \enquote{A cross-cultural analysis of
  religion and life satisfaction.} \emph{Mental Health, Religion \& Culture},
  1, 37--43.

\bibitem[\protect\citeauthoryear{Dreier, Swanstrom, and Mollenkopf}{Dreier
  et~al.}{2005}]{dreier05}
\textsc{Dreier, P., T.~Swanstrom, and J.~Mollenkopf} (2005): \emph{Place
  Matters: Metropolitics For The Twenty-First Century(Studies In Government \&
  Public Policy) Author: Peter Dreie}, University Press of Kansas, Lawrence KS.

\bibitem[\protect\citeauthoryear{Duany, Plater-Zyberk, and Speck}{Duany
  et~al.}{2001}]{duany01}
\textsc{Duany, A., E.~Plater-Zyberk, and J.~Speck} (2001): \emph{Suburban
  nation: The rise of sprawl and the decline of the American dream}, North
  Point Press, New York NY.

\bibitem[\protect\citeauthoryear{Easterlin}{Easterlin}{2013}]{easterlin13}
\textsc{Easterlin, R.} (2013): \enquote{Happiness, Growth, and Public Policy,}
  \emph{Economic Inquiry}, 51, 1--15.

\bibitem[\protect\citeauthoryear{Elgin}{Elgin}{1975}]{elgin75}
\textsc{Elgin, D.} (1975): \emph{City size and the quality of life}, US
  Government Printing Office.

\bibitem[\protect\citeauthoryear{Evans}{Evans}{2009}]{evans09}
\textsc{Evans, R.~J.} (2009): \enquote{A Comparison of Rural and Urban Older
  Adults in Iowa on Specific Markers of Successful Aging,} \emph{Journal of
  Gerontological Social Work}, 52, 423--438.

\bibitem[\protect\citeauthoryear{Ewing}{Ewing}{1997}]{ewing97}
\textsc{Ewing, R.} (1997): \enquote{Is Los Angeles-style sprawl desirable?}
  \emph{Journal of the American planning association}, 63, 107--126.

\bibitem[\protect\citeauthoryear{Ewing, Schmid, Killingsworth, Zlot, and
  Raudenbush}{Ewing et~al.}{2003}]{ewing03}
\textsc{Ewing, R., T.~Schmid, R.~Killingsworth, A.~Zlot, and S.~Raudenbush}
  (2003): \enquote{Relationship between urban sprawl and physical activity,
  obesity, and morbidity,} \emph{American Journal of Health Promotion}, 18,
  47--57.

\bibitem[\protect\citeauthoryear{Fischer}{Fischer}{1972}]{fischer72}
\textsc{Fischer, C.~S.} (1972): \enquote{Urbanism as a Way of Life (A Review
  and an Agenda),} \emph{Sociological Methods and Research}, 1, 187--242.

\bibitem[\protect\citeauthoryear{Fischer}{Fischer}{1973}]{fischer73}
---\hspace{-.1pt}---\hspace{-.1pt}--- (1973): \enquote{Urban malaise,}
  \emph{Social Forces}, 52, 221--235.

\bibitem[\protect\citeauthoryear{Fischer}{Fischer}{1975}]{fischer75}
---\hspace{-.1pt}---\hspace{-.1pt}--- (1975): \enquote{Toward a subcultural
  theory of urbanism,} \emph{American Journal of Sociology}, 80, 1319--1341.

\bibitem[\protect\citeauthoryear{Fischer}{Fischer}{1982}]{fischer82}
---\hspace{-.1pt}---\hspace{-.1pt}--- (1982): \emph{To dwell among friends:
  Personal networks in town and city}, University of Chicago Press, Chicago IL.

\bibitem[\protect\citeauthoryear{Fischer and Merton}{Fischer and
  Merton}{1976}]{fischer76}
\textsc{Fischer, C.~S. and R.~K. Merton} (1976): \emph{The urban experience},
  Harcourt Brace Jovanovich New York.

\bibitem[\protect\citeauthoryear{Florida}{Florida}{2008}]{florida08}
\textsc{Florida, R.} (2008): \emph{Who's your city?}, Basic Books, New York NY.

\bibitem[\protect\citeauthoryear{Freidman}{Freidman}{2005}]{freidman05}
\textsc{Freidman, T.} (2005): \enquote{The world is flat,} \emph{Farrar, Straus
  and Giroux, New York NY}.

\bibitem[\protect\citeauthoryear{Frumkin}{Frumkin}{2002}]{frumkin02}
\textsc{Frumkin, H.} (2002): \enquote{Urban sprawl and public health,}
  \emph{Public health reports}, 117, 201--217.

\bibitem[\protect\citeauthoryear{Fuguitt and Brown}{Fuguitt and
  Brown}{1990}]{fuguitt90}
\textsc{Fuguitt, G.~V. and D.~L. Brown} (1990): \enquote{Residential
  Preferences and Population Redistribution,} \emph{Demography}, 27, 589--600.

\bibitem[\protect\citeauthoryear{Fuguitt and Zuiches}{Fuguitt and
  Zuiches}{1975}]{fuguitt75}
\textsc{Fuguitt, G.~V. and J.~J. Zuiches} (1975): \enquote{Residential
  Preferences and Population Distribution,} \emph{Demography}, 12, 491--504.

\bibitem[\protect\citeauthoryear{Glaeser}{Glaeser}{2011}]{glaeser11}
\textsc{Glaeser, E.} (2011): \emph{Triumph of the City: How Our Greatest
  Invention Makes Us Richer, Smarter, Greener, Healthier, and Happier}, Penguin
  Press, New York NY.

\bibitem[\protect\citeauthoryear{Gurin, Veroff, and Feld}{Gurin
  et~al.}{1960}]{gurin60}
\textsc{Gurin, G., J.~Veroff, and S.~Feld} (1960): \emph{Americans view their
  mental health: A nationwide interview survey.}, Basic Books, New York NY.

\bibitem[\protect\citeauthoryear{Hanson}{Hanson}{2015}]{hansonCityJournalautumn15}
\textsc{Hanson, V.~D.} (2015): \enquote{The Oldest Divide. With roots dating
  back to our Founding, America's urban-rural split is wider than ever.}
  \emph{City Journal}, Autumn 2015.

\bibitem[\protect\citeauthoryear{Jacobs}{Jacobs}{[1961] 1993}]{jacobs93}
\textsc{Jacobs, J.} ([1961] 1993): \emph{The death and life of great American
  cities}, Random House, New York NY.

\bibitem[\protect\citeauthoryear{Jargowsky}{Jargowsky}{1997}]{jargowsky97}
\textsc{Jargowsky, P.~A.} (1997): \emph{Poverty and place: Ghettos, barrios,
  and the American city}, Russell Sage Foundation, New York NY.

\bibitem[\protect\citeauthoryear{Jost, Federico, and Napier}{Jost
  et~al.}{2009}]{jost09al}
\textsc{Jost, J.~T., C.~M. Federico, and J.~L. Napier} (2009):
  \enquote{Political Ideology: Its Structure, Functions, and Elective
  Affinities,} \emph{Annual Review of Psychology}, 60, 307--337.

\bibitem[\protect\citeauthoryear{Jost, Kruglanski, Glaser, and Sulloway}{Jost
  et~al.}{2003}]{jost03al}
\textsc{Jost, J.~T., A.~W. Kruglanski, J.~Glaser, and F.~J. Sulloway} (2003):
  \enquote{Political Conservatism as Motivated Social Cognition.}
  \emph{Psychological Bulletin}, 129, 339--375.

\bibitem[\protect\citeauthoryear{Kay}{Kay}{1997}]{kay97}
\textsc{Kay, J.~H.} (1997): \emph{Asphalt nation: how the automobile took over
  America, and how we can take it back}, University of California Press,
  Berkeley CA.

\bibitem[\protect\citeauthoryear{Knight, Shi, and Song}{Knight
  et~al.}{2006}]{knight06}
\textsc{Knight, J., L.~Shi, and L.~Song} (2006): \enquote{The rural-urban
  divide and the evolution of political economy in China,} in \emph{Human
  Development in the Era of Globalization: Essays in Honor of Keith B.
  Griffin}, ed. by P.~K.~P. James K.~Boyce, Stephen~Cullenberg and R.~Pollin,
  Edward Elgar Publishing, Northampton MA, 44--64.

\bibitem[\protect\citeauthoryear{Kunstler}{Kunstler}{2012}]{kunstler12}
\textsc{Kunstler, J.~H.} (2012): \emph{The geography of nowhere}, Simon and
  Schuster, New York NY.

\bibitem[\protect\citeauthoryear{Layard}{Layard}{2005}]{layard05}
\textsc{Layard, R.} (2005): \emph{Happiness. Lessons from a new science.}, The
  Penguin Press, New York NY.

\bibitem[\protect\citeauthoryear{Lederbogen, Kirsch, Haddad, Streit, Tost,
  Schuch, Wust, Pruessner, Rietschel, Deuschle, and
  {Meyer-Lindenberg}}{Lederbogen et~al.}{2011}]{lederbogen11}
\textsc{Lederbogen, F., P.~Kirsch, L.~Haddad, F.~Streit, H.~Tost, P.~Schuch,
  S.~Wust, J.~C. Pruessner, M.~Rietschel, M.~Deuschle, and
  A.~{Meyer-Lindenberg}} (2011): \enquote{City living and urban upbringing
  affect neural social stress processing in humans,} \emph{Nature}, 474.

\bibitem[\protect\citeauthoryear{Lu, Schellenberg, Hou, and Helliwell}{Lu
  et~al.}{2015}]{lu15}
\textsc{Lu, C., G.~Schellenberg, F.~Hou, and J.~F. Helliwell} (2015):
  \enquote{How's Life in the City? Life Satisfaction Across Census Metropolitan
  Areas and Economic Regions in Canada,} \emph{Economic Insights}, 11-626-X.

\bibitem[\protect\citeauthoryear{Lykken and Tellegen}{Lykken and
  Tellegen}{1996}]{lykken96t}
\textsc{Lykken, D. and A.~Tellegen} (1996): \enquote{Happiness is a Stochastic
  Phenomenon,} \emph{Psychological Science}, 7, 186--189.

\bibitem[\protect\citeauthoryear{MacKerron and Mourato}{MacKerron and
  Mourato}{2009}]{mackerron09}
\textsc{MacKerron, G. and S.~Mourato} (2009): \enquote{Life Satisfaction and
  Air Quality in London,} \emph{Ecological Economics}, 68, 1441--1453.

\bibitem[\protect\citeauthoryear{Meyer}{Meyer}{2013}]{meyer13}
\textsc{Meyer, W.~B.} (2013): \emph{The Environmental Advantages of Cities:
  Countering Commonsense Antiurbanism}, MIT Press, Cambridge MA.

\bibitem[\protect\citeauthoryear{Michalos}{Michalos}{1985}]{michalos85}
\textsc{Michalos, A.} (1985): \enquote{Multiple discrepancies theory (MDT),}
  \emph{Social Indicators Research}, 16, 347--413.

\bibitem[\protect\citeauthoryear{Mitchell}{Mitchell}{2012}]{mitchell12}
\textsc{Mitchell, M.~N.} (2012): \emph{Interpreting and
  visualizing regression models using Stata}, Stata Press, College Station TX.

\bibitem[\protect\citeauthoryear{Myers}{Myers}{2000}]{myers00}
\textsc{Myers, D.~G.} (2000): \enquote{The Funds, Friends, and Faith of Happy
  People,} \emph{American Psychologist}, 55, 56--67.

\bibitem[\protect\citeauthoryear{Napier and Jost}{Napier and
  Jost}{2008}]{napier08}
\textsc{Napier, J.~L. and J.~T. Jost} (2008): \enquote{{Why are Conservatives
  Happier than Liberals?}} \emph{Psychological Science}, 19, 565--72.

\bibitem[\protect\citeauthoryear{Okulicz-Kozaryn}{Okulicz-Kozaryn}{2011}]{aok13liavbility}
\textsc{Okulicz-Kozaryn, A.} (2011): \enquote{City Life: Rankings (Livability)
  Versus Perceptions (Satisfaction),} \emph{Social Indicators Research}, 110,
  433--451.

\bibitem[\protect\citeauthoryear{Okulicz-Kozaryn}{Okulicz-Kozaryn}{2015}]{aokCityBook15}
---\hspace{-.1pt}---\hspace{-.1pt}--- (2015): \emph{Happiness and Place. Why
  Life is Better Outside of the City.}, Palgrave Macmillan, New York NY.

\bibitem[\protect\citeauthoryear{Park}{Park}{1915}]{park15}
\textsc{Park, R.~E.} (1915): \enquote{The city: Suggestions for the
  investigation of human behavior in the city environment,} \emph{The American
  Journal of Sociology}, 20, 577--612.

\bibitem[\protect\citeauthoryear{Park, Burgess, and Mac~Kenzie}{Park
  et~al.}{[1925] 1984}]{park84}
\textsc{Park, R.~E., E.~W. Burgess, and R.~D. Mac~Kenzie} ([1925] 1984):
  \emph{The city}, University of Chicago Press, Chicago IL.

\bibitem[\protect\citeauthoryear{Pawson and Tilley}{Pawson and
  Tilley}{1997}]{pawson97}
\textsc{Pawson, R. and N.~Tilley} (1997): \emph{Realistic evaluation}, Sage,
  Beverly Hills CA.

\bibitem[\protect\citeauthoryear{Sanfey and Teksoz}{Sanfey and
  Teksoz}{2005}]{teksoz}
\textsc{Sanfey, P. and U.~Teksoz} (2005): \enquote{Does Transition Make You
  Happy?} EBRD Working Paper 58.

\bibitem[\protect\citeauthoryear{Savitch}{Savitch}{2010}]{savitch10}
\textsc{Savitch, H.} (2010): \enquote{What makes a great city great? An
  American perspective,} \emph{Cities}, 27, 42--49.

\bibitem[\protect\citeauthoryear{Simmel}{Simmel}{1903}]{simmel03}
\textsc{Simmel, G.} (1903): \enquote{The metropolis and mental life,} \emph{The
  urban sociology reader}, 23--31.

\bibitem[\protect\citeauthoryear{Singell}{Singell}{1974}]{singell74}
\textsc{Singell, L.~D.} (1974): \enquote{Optimum city size: Some thoughts on
  theory and policy,} \emph{Land Economics}, 207--212.

\bibitem[\protect\citeauthoryear{Stiglitz, Sen, and Fitoussi}{Stiglitz
  et~al.}{2009}]{stiglitz09al}
\textsc{Stiglitz, J., A.~Sen, and J.~Fitoussi} (2009): \enquote{Report by the
  Commission on the measurement of economic performance and social progress,}
  \emph{Available at www.stiglitz-sen-fitoussi.fr}.

\bibitem[\protect\citeauthoryear{Subramanian and Perkins}{Subramanian and
  Perkins}{2009}]{subramanian09a}
\textsc{Subramanian, S.~V. and J.~M. Perkins} (2009): \enquote{Are republicans
  healthier than democrats?} \emph{International Journal of Epidemiology}, 39,
  930--931.

\bibitem[\protect\citeauthoryear{T{\"o}nnies}{T{\"o}nnies}{[1887]
  2002}]{tonnies57}
\textsc{T{\"o}nnies, F.} ([1887] 2002): \emph{Community and society},
  DoverPublications.com, Mineola NY.

\bibitem[\protect\citeauthoryear{Veblen}{Veblen}{2005{\natexlab{a}}}]{veblen05a}
\textsc{Veblen, T.} (2005{\natexlab{a}}): \emph{Conspicuous consumption},
  vol.~38, ePenguin, New York NY.

\bibitem[\protect\citeauthoryear{Veblen}{Veblen}{2005{\natexlab{b}}}]{veblen05b}
---\hspace{-.1pt}---\hspace{-.1pt}--- (2005{\natexlab{b}}): \emph{The theory of
  the leisure class; an economic study of institutions}, Aakar Books, New York
  NY.

\bibitem[\protect\citeauthoryear{Veenhoven}{Veenhoven}{1994}]{veenhoven94}
\textsc{Veenhoven, R.} (1994): \enquote{How Satisfying is Rural Life?: Fact and
  Value,} in \emph{Changing Values and Attitudes in Family Households with
  Rural Peer Groups, Social Networks, and Action Spaces: Implications of
  Institutional Transition in East and West for Value Formation and
  Transmission}, ed. by J.~Cecora, Society for Agricultural Policy Research and
  Rural Sociology (FAA).

\bibitem[\protect\citeauthoryear{Veenhoven}{Veenhoven}{2008}]{veenhoven08}
---\hspace{-.1pt}---\hspace{-.1pt}--- (2008): \enquote{Sociological theories of
  subjective well-being,} in \emph{The Science of Subjective Well-being: A
  tribute to Ed Diener}, ed. by M.~Eid and R.~Larsen, The Guilford Press, New
  York NY, 44--61.

\bibitem[\protect\citeauthoryear{Weinhold}{Weinhold}{2013}]{weinhold12}
\textsc{Weinhold, D.} (2013): \enquote{The Happiness-Reducing Costs of Noise
  Pollution,} \emph{Journal of regional science}, 53, 292--303.

\bibitem[\protect\citeauthoryear{White and White}{White and
  White}{1977}]{white77}
\textsc{White, M.~G. and L.~White} (1977): \emph{The intellectual versus the
  city: from Thomas Jefferson to Frank Lloyd Wright}, Oxford University Press,
  Oxford UK.

\bibitem[\protect\citeauthoryear{Wirth}{Wirth}{1938}]{wirth38}
\textsc{Wirth, L.} (1938): \enquote{Urbanism as a Way of Life,} \emph{American
  Journal of Sociology}, 44, 1--24.

\end{thebibliography}


\end{spacing}
\end{document}

